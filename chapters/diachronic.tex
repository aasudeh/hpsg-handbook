\documentclass[output=paper
                ,modfonts
                ,nonflat
	        ,collection
	        ,collectionchapter
	        ,collectiontoclongg
 	        ,biblatex
                ,babelshorthands
                ,newtxmath
                ,draftmode
                ,colorlinks, citecolor=brown
]{./langsci/langscibook}

\IfFileExists{../localcommands.tex}{%hack to check whether this is being compiled as part of a collection or standalone
  % add all extra packages you need to load to this file 

\usepackage{graphicx}
\usepackage{tabularx}
\usepackage{amsmath} 
\usepackage{tipa}      % Davis Koenig
\usepackage{multicol}
\usepackage{lipsum}


\usepackage{./langsci/styles/langsci-optional} 
\usepackage{./langsci/styles/langsci-lgr}
%\usepackage{./styles/forest/forest}
\usepackage{./langsci/styles/langsci-forest-setup}
\usepackage{morewrites}

\usepackage{tikz-cd}

\usepackage{./styles/tikz-grid}
\usetikzlibrary{shadows}


%\usepackage{pgfplots} % for data/theory figure in minimalism.tex
% fix some issue with Mod https://tex.stackexchange.com/a/330076
\makeatletter
\let\pgfmathModX=\pgfmathMod@
\usepackage{pgfplots}%
\let\pgfmathMod@=\pgfmathModX
\makeatother

\usepackage{subcaption}

% Stefan Müller's styles
\usepackage{./styles/merkmalstruktur,german,./styles/makros.2e,./styles/my-xspace,./styles/article-ex,
./styles/eng-date}

\selectlanguage{USenglish}

\usepackage{./styles/abbrev}

\usepackage{./langsci/styles/jambox}

% Has to be loaded late since otherwise footnotes will not work

%%%%%%%%%%%%%%%%%%%%%%%%%%%%%%%%%%%%%%%%%%%%%%%%%%%%
%%%                                              %%%
%%%           Examples                           %%%
%%%                                              %%%
%%%%%%%%%%%%%%%%%%%%%%%%%%%%%%%%%%%%%%%%%%%%%%%%%%%%
% remove the percentage signs in the following lines
% if your book makes use of linguistic examples
\usepackage{./langsci/styles/langsci-gb4e} 

% Crossing out text
% uncomment when needed
%\usepackage{ulem}

\usepackage{./styles/additional-langsci-index-shortcuts}

%\usepackage{./langsci/styles/langsci-avm}
\usepackage{./styles/avm+}


\renewcommand{\tpv}[1]{{\avmjvalfont\itshape #1}}

% no small caps please
\renewcommand{\phonshape}[0]{\normalfont\itshape}

\regAvmFonts

\usepackage{theorem}

\newtheorem{mydefinition}{Def.}
\newtheorem{principle}{Principle}

{\theoremstyle{break}
%\newtheorem{schema}{Schema}
\newtheorem{mydefinition-break}[mydefinition]{Def.}
\newtheorem{principle-break}[principle]{Principle}
}

% This avoids linebreaks in the Schema
\newcounter{schema}
\newenvironment{schema}[1][]
  {% \begin{Beispiel}[<title>]
  \goodbreak%
  \refstepcounter{schema}%
  \begin{list}{}{\setlength{\labelwidth}{0pt}\setlength{\labelsep}{0pt}\setlength{\rightmargin}{0pt}\setlength{\leftmargin}{0pt}}%
    \item[{\textbf{Schema~\theschema}}]\hspace{.5em}\textbf{(#1)}\nopagebreak[4]\par\nobreak}%
  {\end{list}}% \end{Beispiel}

%% \newcommand{schema}[2]{
%% \begin{minipage}{\textwidth}
%% {\textbf{Schema~\theschema}}]\hspace{.5em}\textbf{(#1)}\\
%% #2
%% \end{minipage}}

%\usepackage{subfig}





% Davis Koenig Lexikon

\usepackage{tikz-qtree,tikz-qtree-compat} % Davis Koenig remove

\usepackage{shadow}




\usepackage[english]{isodate} % Andy Lücking
\usepackage[autostyle]{csquotes} % Andy
%\usepackage[autolanguage]{numprint}

%\defaultfontfeatures{
%    Path = /usr/local/texlive/2017/texmf-dist/fonts/opentype/public/fontawesome/ }

%% https://tex.stackexchange.com/a/316948/18561
%\defaultfontfeatures{Extension = .otf}% adds .otf to end of path when font loaded without ext parameter e.g. \newfontfamily{\FA}{FontAwesome} > \newfontfamily{\FA}{FontAwesome.otf}
%\usepackage{fontawesome} % Andy Lücking
\usepackage{pifont} % Andy Lücking -> hand

\usetikzlibrary{decorations.pathreplacing} % Andy Lücking
\usetikzlibrary{matrix} % Andy 
\usetikzlibrary{positioning} % Andy
\usepackage{tikz-3dplot} % Andy

% pragmatics
\usepackage{eqparbox} % Andy
\usepackage{enumitem} % Andy
\usepackage{longtable} % Andy
\usepackage{tabu} % Andy


% Manfred's packages

%\usepackage{shadow}

\usepackage{tabularx}
\newcolumntype{L}[1]{>{\raggedright\arraybackslash}p{#1}} % linksbündig mit Breitenangabe


% Jong-Bok

%\usepackage{xytree}

\newcommand{\xytree}[2][dummy]{Let's do the tree!}

% seems evil, get rid of it
% defines \ex is incompatible with gb4e
%\usepackage{lingmacros}

% taken from lingmacros:
\makeatletter
% \evnup is used to line up the enumsentence number and an entry along
% the top.  It can take an argument to improve lining up.
\def\evnup{\@ifnextchar[{\@evnup}{\@evnup[0pt]}}

\def\@evnup[#1]#2{\setbox1=\hbox{#2}%
\dimen1=\ht1 \advance\dimen1 by -.5\baselineskip%
\advance\dimen1 by -#1%
\leavevmode\lower\dimen1\box1}
\makeatother


% YK -- CG chapter

%\usepackage{xspace}
\usepackage{bm}
\usepackage{bussproofs}


% Antonio Branco, remove this
\usepackage{epsfig}

% now unicode
%\usepackage{alphabeta}



% Berthold udc
%\usepackage{qtree}
%\usepackage{rtrees}

\usepackage{pst-node}

  %add all your local new commands to this file

\makeatletter
\def\blx@maxline{77}
\makeatother


\newcommand{\page}{}



\newcommand{\todostefan}[1]{\todo[color=orange!80]{\footnotesize #1}\xspace}
\newcommand{\todosatz}[1]{\todo[color=red!40]{\footnotesize #1}\xspace}

\newcommand{\inlinetodostefan}[1]{\todo[color=green!40,inline]{\footnotesize #1}\xspace}


\newcommand{\spacebr}{\hspaceThis{[}}

\newcommand{\danish}{\jambox{(\ili{Danish})}}
\newcommand{\english}{\jambox{(\ili{English})}}
\newcommand{\german}{\jambox{(\ili{German})}}
\newcommand{\yiddish}{\jambox{(\ili{Yiddish})}}
\newcommand{\welsh}{\jambox{(\ili{Welsh})}}

% Cite and cross-reference other chapters
\newcommand{\crossrefchaptert}[2][]{\citet*[#1]{chapters/#2}, Chapter~\ref{chap-#2} of this volume} 
\newcommand{\crossrefchapterp}[2][]{(\citealp*[#1][]{chapters/#2}, Chapter~\ref{chap-#2} of this volume)}
% example of optional argument:
% \crossrefchapterp[for something, see:]{name}
% gives: (for something, see: Author 2018, Chapter~X of this volume)

\let\crossrefchapterw\crossrefchaptert



% Davis Koenig

\let\ig=\textsc
\let\tc=\textcolor

% evolution, Flickinger, Pollard, Wasow

\let\citeNP\citet

% Adam P

%\newcommand{\toappear}{Forthcoming}
\newcommand{\pg}[1]{p.#1}
\renewcommand{\implies}{\rightarrow}

\newcommand*{\rref}[1]{(\ref{#1})}
\newcommand*{\aref}[1]{(\ref{#1}a)}
\newcommand*{\bref}[1]{(\ref{#1}b)}
\newcommand*{\cref}[1]{(\ref{#1}c)}

\newcommand{\msadam}{.}
\newcommand{\morsyn}[1]{\textsc{#1}}

\newcommand{\nom}{\morsyn{nom}}
\newcommand{\acc}{\morsyn{acc}}
\newcommand{\dat}{\morsyn{dat}}
\newcommand{\gen}{\morsyn{gen}}
\newcommand{\ins}{\morsyn{ins}}
\newcommand{\loc}{\morsyn{loc}}
\newcommand{\voc}{\morsyn{voc}}
\newcommand{\ill}{\morsyn{ill}}
\renewcommand{\inf}{\morsyn{inf}}
\newcommand{\passprc}{\morsyn{passp}}

%\newcommand{\Nom}{\msadam\nom}
%\newcommand{\Acc}{\msadam\acc}
%\newcommand{\Dat}{\msadam\dat}
%\newcommand{\Gen}{\msadam\gen}
\newcommand{\Ins}{\msadam\ins}
\newcommand{\Loc}{\msadam\loc}
\newcommand{\Voc}{\msadam\voc}
\newcommand{\Ill}{\msadam\ill}
\newcommand{\INF}{\msadam\inf}
\newcommand{\PassP}{\msadam\passprc}

\newcommand{\Aux}{\textsc{aux}}

\newcommand{\princ}[1]{\textnormal{\textsc{#1}}} % for constraint names
\newcommand{\notion}[1]{\emph{#1}}
\renewcommand{\path}[1]{\textnormal{\textsc{#1}}}
\newcommand{\ftype}[1]{\textit{#1}}
\newcommand{\fftype}[1]{{\scriptsize\textit{#1}}}
\newcommand{\la}{$\langle$}
\newcommand{\ra}{$\rangle$}
%\newcommand{\argst}{\path{arg-st}}
\newcommand{\phtm}[1]{\setbox0=\hbox{#1}\hspace{\wd0}}
\newcommand{\prep}[1]{\setbox0=\hbox{#1}\hspace{-1\wd0}#1}

%%%%%%%%%%%%%%%%%%%%%%%%%%%%%%%%%%%%%%%%%%%%%%%%%%%%%%%%%%%%%%%%%%%%%%%%%%%

% FROM FS.STY:

%%%
%%% Feature structures
%%%

% \fs         To print a feature structure by itself, type for example
%             \fs{case:nom \\ person:P}
%             or (better, for true italics),
%             \fs{\it case:nom \\ \it person:P}
%
% \lfs        To print the same feature structure with the category
%             label N at the top, type:
%             \lfs{N}{\it case:nom \\ \it person:P}

%    Modified 1990 Dec 5 so that features are left aligned.
\newcommand{\fs}[1]%
{\mbox{\small%
$
\!
\left[
  \!\!
  \begin{tabular}{l}
    #1
  \end{tabular}
  \!\!
\right]
\!
$}}

%     Modified 1990 Dec 5 so that features are left aligned.
%\newcommand{\lfs}[2]
%   {
%     \mbox{$
%           \!\!
%           \begin{tabular}{c}
%           \it #1
%           \\
%           \mbox{\small%
%                 $
%                 \left[
%                 \!\!
%                 \it
%                 \begin{tabular}{l}
%                 #2
%                 \end{tabular}
%                 \!\!
%                 \right]
%                 $}
%           \end{tabular}
%           \!\!
%           $}
%   }

\newcommand{\ft}[2]{\path{#1}\hspace{1ex}\ftype{#2}}
\newcommand{\fsl}[2]{\fs{{\fftype{#1}} \\ #2}}

\newcommand{\fslt}[2]
 {\fst{
       {\fftype{#1}} \\
       #2 
     }
 }

\newcommand{\fsltt}[2]
 {\fstt{
       {\fftype{#1}} \\
       #2 
     }
 }

\newcommand{\fslttt}[2]
 {\fsttt{
       {\fftype{#1}} \\
       #2 
     }
 }


% jak \ft, \fs i \fsl tylko nieco ciasniejsze

\newcommand{\ftt}[2]
% {{\sc #1}\/{\rm #2}}
 {\textsc{#1}\/{\rm #2}}

\newcommand{\fst}[1]
  {
    \mbox{\small%
          $
          \left[
          \!\!\!
%          \sc
          \begin{tabular}{l} #1
          \end{tabular}
          \!\!\!\!\!\!\!
          \right]
          $
          }
   }

%\newcommand{\fslt}[2]
% {\fst{#2\\
%       {\scriptsize\it #1}
%      }
% }


% superciasne

\newcommand{\fstt}[1]
  {
    \mbox{\small%
          $
          \left[
          \!\!\!
%          \sc
          \begin{tabular}{l} #1
          \end{tabular}
          \!\!\!\!\!\!\!\!\!\!\!
          \right]
          $
          }
   }

%\newcommand{\fsltt}[2]
% {\fstt{#2\\
%       {\scriptsize\it #1}
%      }
% }

\newcommand{\fsttt}[1]
  {
    \mbox{\small%
          $
          \left[
          \!\!\!
%          \sc
          \begin{tabular}{l} #1
          \end{tabular}
          \!\!\!\!\!\!\!\!\!\!\!\!\!\!\!\!
          \right]
          $
          }
   }



% %add all your local new commands to this file

% \newcommand{\smiley}{:)}

% you are not supposed to mess with hardcore stuff, St.Mü. 22.08.2018
%% \renewbibmacro*{index:name}[5]{%
%%   \usebibmacro{index:entry}{#1}
%%     {\iffieldundef{usera}{}{\thefield{usera}\actualoperator}\mkbibindexname{#2}{#3}{#4}{#5}}}

% % \newcommand{\noop}[1]{}



% Rui

\newcommand{\spc}[0]{\hspace{-1pt}\underline{\hspace{6pt}}\,}
\newcommand{\spcs}[0]{\hspace{-1pt}\underline{\hspace{6pt}}\,\,}
\newcommand{\bad}[1]{\leavevmode\llap{#1}}
\newcommand{\COMMENT}[1]{}


% Andy Lücking gesture.tex
\newcommand{\Pointing}{\ding{43}}
% Giotto: "Meeting of Joachim and Anne at the Golden Gate" - 1305-10 
\definecolor{GoldenGate1}{rgb}{.13,.09,.13} % Dress of woman in black
\definecolor{GoldenGate2}{rgb}{.94,.94,.91} % Bridge
\definecolor{GoldenGate3}{rgb}{.06,.09,.22} % Blue sky
\definecolor{GoldenGate4}{rgb}{.94,.91,.87} % Dress of woman with shawl
\definecolor{GoldenGate5}{rgb}{.52,.26,.26} % Joachim's robe
\definecolor{GoldenGate6}{rgb}{.65,.35,.16} % Anne's robe
\definecolor{GoldenGate7}{rgb}{.91,.84,.42} % Joachim's halo

\makeatletter
\newcommand{\@Depth}{1} % x-dimension, to front
\newcommand{\@Height}{1} % z-dimension, up
\newcommand{\@Width}{1} % y-dimension, rightwards
%\GGS{<x-start>}{<y-start>}{<z-top>}{<z-bottom>}{<Farbe>}{<x-width>}{<y-depth>}{<opacity>}
\newcommand{\GGS}[9][]{%
\coordinate (O) at (#2-1,#3-1,#5);
\coordinate (A) at (#2-1,#3-1+#7,#5);
\coordinate (B) at (#2-1,#3-1+#7,#4);
\coordinate (C) at (#2-1,#3-1,#4);
\coordinate (D) at (#2-1+#8,#3-1,#5);
\coordinate (E) at (#2-1+#8,#3-1+#7,#5);
\coordinate (F) at (#2-1+#8,#3-1+#7,#4);
\coordinate (G) at (#2-1+#8,#3-1,#4);
\draw[draw=black, fill=#6, fill opacity=#9] (D) -- (E) -- (F) -- (G) -- cycle;% Front
\draw[draw=black, fill=#6, fill opacity=#9] (C) -- (B) -- (F) -- (G) -- cycle;% Top
\draw[draw=black, fill=#6, fill opacity=#9] (A) -- (B) -- (F) -- (E) -- cycle;% Right
}
\makeatother


% pragmatics
\newcommand{\speaking}[1]{\eqparbox{name}{\textsc{\lowercase{#1}\space}}}
\newcommand{\name}[1]{\eqparbox{name}{\textsc{\lowercase{#1}}}}
\newcommand{\HPSGTTR}{HPSG$_{\text{TTR}}$\xspace}

\newcommand{\ttrtype}[1]{\textit{#1}}
% \newcommand{\avmel}{\q<\quad\q>} %% shortcut for empty lists in AVM
\newcommand{\ttrmerge}{\ensuremath{\wedge_{\textit{merge}}}}
\newcommand{\Cat}[2][0.1pt]{%
  \begin{scope}[y=#1,x=#1,yscale=-1, inner sep=0pt, outer sep=0pt]
   \path[fill=#2,line join=miter,line cap=butt,even odd rule,line width=0.8pt]
  (151.3490,307.2045) -- (264.3490,307.2045) .. controls (264.3490,291.1410) and (263.2021,287.9545) .. (236.5990,287.9545) .. controls (240.8490,275.2045) and (258.1242,244.3581) .. (267.7240,244.3581) .. controls (276.2171,244.3581) and (286.3490,244.8259) .. (286.3490,264.2045) .. controls (286.3490,286.2045) and (323.3717,321.6755) .. (332.3490,307.2045) .. controls (345.7277,285.6390) and (309.3490,292.2151) .. (309.3490,240.2046) .. controls (309.3490,169.0514) and (350.8742,179.1807) .. (350.8742,139.2046) .. controls (350.8742,119.2045) and (345.3490,116.5037) .. (345.3490,102.2045) .. controls (345.3490,83.3070) and (361.9972,84.4036) .. (358.7581,68.7349) .. controls (356.5206,57.9117) and (354.7696,49.2320) .. (353.4652,36.1439) .. controls (352.5396,26.8573) and (352.2445,16.9594) .. (342.5985,17.3574) .. controls (331.2650,17.8250) and (326.9655,37.7742) .. (309.3490,39.2045) .. controls (291.7685,40.6320) and (276.7783,24.2380) .. (269.9740,26.5795) .. controls (263.2271,28.9013) and (265.3490,47.2045) .. (269.3490,60.2045) .. controls (275.6359,80.6368) and (289.3490,107.2045) .. (264.3490,111.2045) .. controls (239.3490,115.2045) and (196.3490,119.2045) .. (165.3490,160.2046) .. controls (134.3490,201.2046) and (135.4934,249.3212) .. (123.3490,264.2045) .. controls (82.5907,314.1553) and (40.8239,293.6463) .. (40.8239,335.2045) .. controls (40.8239,353.8102) and (72.3490,367.2045) .. (77.3490,361.2045) .. controls (82.3490,355.2045) and (34.8638,337.3259) .. (87.9955,316.2045) .. controls (133.3871,298.1601) and   (137.4391,294.4766) .. (151.3490,307.2045) -- cycle;
\end{scope}%
}


% KdK
\newcommand{\smiley}{:)}

\renewbibmacro*{index:name}[5]{%
  \usebibmacro{index:entry}{#1}
    {\iffieldundef{usera}{}{\thefield{usera}\actualoperator}\mkbibindexname{#2}{#3}{#4}{#5}}}

% \newcommand{\noop}[1]{}

% chngcntr.sty otherwise gives error that these are already defined
%\let\counterwithin\relax
%\let\counterwithout\relax

% the space of a left bracket for glossings
\newcommand{\LB}{\hspaceThis{[}}

\newcommand{\LF}{\mbox{$[\![$}}

\newcommand{\RF}{\mbox{$]\!]_F$}}

\newcommand{\RT}{\mbox{$]\!]_T$}}





% Manfred's

\newcommand{\kommentar}[1]{}

\newcommand{\bsp}[1]{\emph{#1}}
\newcommand{\bspT}[2]{\bsp{#1} `#2'}
\newcommand{\bspTL}[3]{\bsp{#1} (lit.: #2) `#3'}

\newcommand{\noidi}{§}

\newcommand{\refer}[1]{(\ref{#1})}

%\newcommand{\avmtype}[1]{\multicolumn{2}{l}{\type{#1}}}
\newcommand{\attr}[1]{\textsc{#1}}

\newcommand{\srdefault}{\mbox{\begin{tabular}{c}{\large <}\\[-1.5ex]$\sqcap$\end{tabular}}}

%% \newcommand{\myappcolumn}[2]{
%% \begin{minipage}[t]{#1}#2\end{minipage}
%% }

%% \newcommand{\appc}[1]{\myappcolumn{3.7cm}{#1}}


% Jong-Bok


% clean that up and do not use \def (killing other stuff defined before)
%\if 0
\def\DEL{\textsc{del}}
\def\del{\textsc{del}}

\def\conn{\textsc{conn}}
\def\CONN{\textsc{conn}}
\def\CONJ{\textsc{conj}}
\def\LITE{\textsc{lex}}
\def\lite{\textsc{lex}}
\def\HON{\textsc{hon}}

\def\CAUS{\textsc{caus}}
\def\PASS{\textsc{pass}}
\def\NPST{\textsc{npst}}
\def\COND{\textsc{cond}}



\def\hd-lite{\textsc{head-lex construction}}
\def\NFORM{\textsc{nform}}

\def\RELS{\textsc{rels}}
\def\TENSE{\textsc{tense}}


%\def\ARG{\textsc{arg}}
\def\ARGs{\textsc{arg0}}
\def\ARGa{\textsc{arg}}
\def\ARGb{\textsc{arg2}}
\def\TPC{\textsc{top}}
\def\PROG{\textsc{prog}}

\def\pst{\textsc{pst}}
\def\PAST{\textsc{pst}}
\def\DAT{\textsc{dat}}
\def\CONJ{\textsc{conj}}
\def\nominal{\textsc{nominal}}
\def\NOMINAL{\textsc{nominal}}
\def\VAL{\textsc{val}}
\def\val{\textsc{val}}
\def\MODE{\textsc{mode}}
\def\RESTR{\textsc{restr}}
\def\SIT{\textsc{sit}}
\def\ARG{\textsc{arg}}
\def\RELN{\textsc{rel}}
\def\REL{\textsc{rel}}
\def\RELS{\textsc{rels}}
\def\arg-st{\textsc{arg-st}}
\def\xdel{\textsc{xdel}}
\def\zdel{\textsc{zdel}}
\def\sug{\textsc{sug}}
\def\IMP{\textsc{imp}}
\def\conn{\textsc{conn}}
\def\CONJ{\textsc{conj}}
\def\HON{\textsc{hon}}
\def\BN{\textsc{bn}}
\def\bn{\textsc{bn}}
\def\pres{\textsc{pres}}
\def\PRES{\textsc{pres}}
\def\prs{\textsc{pres}}
\def\PRS{\textsc{pres}}
\def\agt{\textsc{agt}}
\def\DEL{\textsc{del}}
\def\PRED{\textsc{pred}}
\def\AGENT{\textsc{agent}}
\def\THEME{\textsc{theme}}
\def\AUX{\textsc{aux}}
\def\THEME{\textsc{theme}}
\def\PL{\textsc{pl}}
\def\SRC{\textsc{src}}
\def\src{\textsc{src}}
\def\FORM{\textsc{form}}
\def\form{\textsc{form}}
\def\GCASE{\textsc{gcase}}
\def\gcase{\textsc{gcase}}
\def\SCASE{\textsc{scase}}
\def\PHON{\textsc{phon}}
\def\SS{\textsc{ss}}
\def\SYN{\textsc{syn}}
\def\LOC{\textsc{loc}}
\def\MOD{\textsc{mod}}
\def\INV{\textsc{inv}}
\def\L{\textsc{l}}
\def\CASE{\textsc{case}}
\def\SPR{\textsc{spr}}
\def\COMPS{\textsc{comps}}
%\def\comps{\textsc{comps}}
\def\SEM{\textsc{sem}}
\def\CONT{\textsc{cont}}
\def\SUBCAT{\textsc{subcat}}
\def\CAT{\textsc{cat}}
\def\C{\textsc{c}}
\def\SUBJ{\textsc{subj}}
\def\subj{\textsc{subj}}
\def\SLASH{\textsc{slash}}
\def\LOCAL{\textsc{local}}
\def\ARG-ST{\textsc{arg-st}}
\def\AGR{\textsc{agr}}
\def\PER{\textsc{per}}
\def\NUM{\textsc{num}}
\def\IND{\textsc{ind}}
\def\VFORM{\textsc{vform}}
\def\PFORM{\textsc{pform}}
\def\decl{\textsc{decl}}
\def\loc{\textsc{loc   }}
% \def\   {\textsc{  }}

\def\NEG{\textsc{neg}}
\def\FRAMES{\textsc{frames}}
\def\REFL{\textsc{refl}}

\def\MKG{\textsc{mkg}}

\def\BN{\textsc{bn}}
\def\HD{\textsc{hd}}
\def\NP{\textsc{np}}
\def\PF{\textsc{pf}}
\def\PL{\textsc{pl}}
\def\PP{\textsc{pp}}
\def\SS{\textsc{ss}}
\def\VF{\textsc{vf}}
\def\VP{\textsc{vp}}
\def\bn{\textsc{bn}}
\def\cl{\textsc{cl}}
\def\pl{\textsc{pl}}
\def\Wh{\ital{Wh}}
\def\ng{\textsc{neg}}
\def\wh{\ital{wh}}
\def\ACC{\textsc{acc}}
\def\AGR{\textsc{agr}}
\def\AGT{\textsc{agt}}
\def\ARC{\textsc{arc}}
\def\ARG{\textsc{arg}}
\def\ARP{\textsc{arc}}
\def\AUX{\textsc{aux}}
\def\CAT{\textsc{cat}}
\def\COP{\textsc{cop}}
\def\DAT{\textsc{dat}}
\def\DEF{\textsc{def}}
\def\DEL{\textsc{del}}
\def\DOM{\textsc{dom}}
\def\DTR{\textsc{dtr}}
\def\FUT{\textsc{fut}}
\def\GAP{\textsc{gap}}
\def\GEN{\textsc{gen}}
\def\HON{\textsc{hon}}
\def\IMP{\textsc{imp}}
\def\IND{\textsc{ind}}
\def\INV{\textsc{inv}}
\def\LEX{\textsc{lex}}
\def\Lex{\textsc{lex}}
\def\LOC{\textsc{loc}}
\def\MOD{\textsc{mod}}
\def\MRK{{\nr MRK}}
\def\NEG{\textsc{neg}}
\def\NEW{\textsc{new}}
\def\NOM{\textsc{nom}}
\def\NUM{\textsc{num}}
\def\PER{\textsc{per}}
\def\PST{\textsc{pst}}
\def\QUE{\textsc{que}}
\def\REL{\textsc{rel}}
\def\SEL{\textsc{sel}}
\def\SEM{\textsc{sem}}
\def\SIT{\textsc{arg0}}
\def\SPR{\textsc{spr}}
\def\SRC{\textsc{src}}
\def\SUG{\textsc{sug}}
\def\SYN{\textsc{syn}}
\def\TPC{\textsc{top}}
\def\VAL{\textsc{val}}
\def\acc{\textsc{acc}}
\def\agt{\textsc{agt}}
\def\cop{\textsc{cop}}
\def\dat{\textsc{dat}}
\def\foc{\textsc{focus}}
\def\FOC{\textsc{focus}}
\def\fut{\textsc{fut}}
\def\hon{\textsc{hon}}
\def\imp{\textsc{imp}}
\def\kes{\textsc{kes}}
\def\lex{\textsc{lex}}
\def\loc{\textsc{loc}}
\def\mrk{{\nr MRK}}
\def\nom{\textsc{nom}}
\def\num{\textsc{num}}
\def\plu{\textsc{plu}}
\def\pne{\textsc{pne}}
\def\pst{\textsc{pst}}
\def\pur{\textsc{pur}}
\def\que{\textsc{que}}
\def\src{\textsc{src}}
\def\sug{\textsc{sug}}
\def\tpc{\textsc{top}}
\def\utt{\textsc{utt}}
\def\val{\textsc{val}}
\def\LITE{\textsc{lex}}
\def\PAST{\textsc{pst}}
\def\POSP{\textsc{pos}}
\def\PRS{\textsc{pres}}
\def\mod{\textsc{mod}}%
\def\newuse{{`kes'}}
\def\posp{\textsc{pos}}
\def\prs{\textsc{pres}}
\def\psp{{\it en\/}}
\def\skes{\textsc{kes}}
\def\CASE{\textsc{case}}
\def\CASE{\textsc{case}}
\def\COMP{\textsc{comp}}
\def\CONJ{\textsc{conj}}
\def\CONN{\textsc{conn}}
\def\CONT{\textsc{cont}}
\def\DECL{\textsc{decl}}
\def\FOCUS{\textsc{focus}}
\def\FORM{\textsc{form}}
\def\FREL{\textsc{frel}}
\def\GOAL{\textsc{goal}}
\def\HEAD{\textsc{head}}
\def\INDEX{\textsc{ind}}
\def\INST{\textsc{inst}}
\def\MODE{\textsc{mode}}
\def\MOOD{\textsc{mood}}
\def\NMLZ{\textsc{nmlz}}
\def\PHON{\textsc{phon}}
\def\PRED{\textsc{pred}}
%\def\PRES{\textsc{pres}}
\def\PROM{\textsc{prom}}
\def\RELN{\textsc{pred}}
\def\RELS{\textsc{rels}}
\def\STEM{\textsc{stem}}
\def\SUBJ{\textsc{subj}}
\def\XARG{\textsc{xarg}}
\def\bse{{\it bse\/}}
\def\case{\textsc{case}}
\def\caus{\textsc{caus}}
\def\comp{\textsc{comp}}
\def\conj{\textsc{conj}}
\def\conn{\textsc{conn}}
\def\decl{\textsc{decl}}
\def\fin{{\it fin\/}}
\def\form{\textsc{form}}
\def\gend{\textsc{gend}}
\def\inf{{\it inf\/}}
\def\mood{\textsc{mood}}
\def\nmlz{\textsc{nmlz}}
\def\pass{\textsc{pass}}
\def\past{\textsc{past}}
\def\perf{\textsc{perf}}
\def\pln{{\it pln\/}}
\def\pred{\textsc{pred}}


%\def\pres{\textsc{pres}}
\def\proc{\textsc{proc}}
\def\nonfin{{\it nonfin\/}}
\def\AGENT{\textsc{agent}}
\def\CFORM{\textsc{cform}}
%\def\COMPS{\textsc{comps}}
\def\COORD{\textsc{coord}}
\def\COUNT{\textsc{count}}
\def\EXTRA{\textsc{extra}}
\def\GCASE{\textsc{gcase}}
\def\GIVEN{\textsc{given}}
\def\LOCAL{\textsc{local}}
\def\NFORM{\textsc{nform}}
\def\PFORM{\textsc{pform}}
\def\SCASE{\textsc{scase}}
\def\SLASH{\textsc{slash}}
\def\SLASH{\textsc{slash}}
\def\THEME{\textsc{theme}}
\def\TOPIC{\textsc{topic}}
\def\VFORM{\textsc{vform}}
\def\cause{\textsc{cause}}
%\def\comps{\textsc{comps}}
\def\gcase{\textsc{gcase}}
\def\itkes{{\it kes\/}}
\def\pass{{\it pass\/}}
\def\vform{\textsc{vform}}
\def\CCONT{\textsc{c-cont}}
\def\GN{\textsc{given-new}}
\def\INFO{\textsc{info-st}}
\def\ARG-ST{\textsc{arg-st}}
\def\SUBCAT{\textsc{subcat}}
\def\SYNSEM{\textsc{synsem}}
\def\VERBAL{\textsc{verbal}}
\def\arg-st{\textsc{arg-st}}
\def\plain{{\it plain}\/}
\def\propos{\textsc{propos}}
\def\ADVERBIAL{\textsc{advl}}
\def\HIGHLIGHT{\textsc{prom}}
\def\NOMINAL{\textsc{nominal}}

\newenvironment{myavm}{\begingroup\avmvskip{.1ex}
  \selectfont\begin{avm}}%
{\end{avm}\endgroup\medskip}
\def\pfix{\vspace{-5pt}}


\def\jbsub#1{\lower4pt\hbox{\small #1}}
\def\jbssub#1{\lower4pt\hbox{\small #1}}
\def\jbtr{\underbar{\ \ \ }\ }


%\fi

  %% hyphenation points for line breaks
%% Normally, automatic hyphenation in LaTeX is very good
%% If a word is mis-hyphenated, add it to this file
%%
%% add information to TeX file before \begin{document} with:
%% %% hyphenation points for line breaks
%% Normally, automatic hyphenation in LaTeX is very good
%% If a word is mis-hyphenated, add it to this file
%%
%% add information to TeX file before \begin{document} with:
%% %% hyphenation points for line breaks
%% Normally, automatic hyphenation in LaTeX is very good
%% If a word is mis-hyphenated, add it to this file
%%
%% add information to TeX file before \begin{document} with:
%% \include{localhyphenation}
\hyphenation{
A-la-hver-dzhie-va
anaph-o-ra
affri-ca-te
affri-ca-tes
Atha-bas-kan
Chi-che-ŵa
com-ple-ments
Da-ge-stan
Dor-drecht
er-klä-ren-de
Ginz-burg
Gro-ning-en
Jon-a-than
Ka-tho-lie-ke
Ko-bon
krie-gen
Le-Sourd
moth-er
Mül-ler
Nie-mey-er
Prze-piór-kow-ski
phe-nom-e-non
re-nowned
Rie-he-mann
un-bound-ed
}

% why has "erklärende" be listed here? I specified langid in bibtex item. Something is still not working with hyphenation.


% to do: check
%  Alahverdzhieva

\hyphenation{
A-la-hver-dzhie-va
anaph-o-ra
affri-ca-te
affri-ca-tes
Atha-bas-kan
Chi-che-ŵa
com-ple-ments
Da-ge-stan
Dor-drecht
er-klä-ren-de
Ginz-burg
Gro-ning-en
Jon-a-than
Ka-tho-lie-ke
Ko-bon
krie-gen
Le-Sourd
moth-er
Mül-ler
Nie-mey-er
Prze-piór-kow-ski
phe-nom-e-non
re-nowned
Rie-he-mann
un-bound-ed
}

% why has "erklärende" be listed here? I specified langid in bibtex item. Something is still not working with hyphenation.


% to do: check
%  Alahverdzhieva

\hyphenation{
A-la-hver-dzhie-va
anaph-o-ra
affri-ca-te
affri-ca-tes
Atha-bas-kan
Chi-che-ŵa
com-ple-ments
Da-ge-stan
Dor-drecht
er-klä-ren-de
Ginz-burg
Gro-ning-en
Jon-a-than
Ka-tho-lie-ke
Ko-bon
krie-gen
Le-Sourd
moth-er
Mül-ler
Nie-mey-er
Prze-piór-kow-ski
phe-nom-e-non
re-nowned
Rie-he-mann
un-bound-ed
}

% why has "erklärende" be listed here? I specified langid in bibtex item. Something is still not working with hyphenation.


% to do: check
%  Alahverdzhieva

  \bibliography{../Bibliographies/stmue,
                ../localbibliography,
../Bibliographies/formal-background,
../Bibliographies/understudied-languages,
../Bibliographies/phonology,
../Bibliographies/case,
../Bibliographies/evolution,
../Bibliographies/agreement,
../Bibliographies/lexicon,
../Bibliographies/np,
../Bibliographies/negation,
../Bibliographies/argst,
../Bibliographies/binding,
../Bibliographies/complex-predicates,
../Bibliographies/coordination,
../Bibliographies/relative-clauses,
../Bibliographies/udc,
../Bibliographies/processing,
../Bibliographies/cl,
../Bibliographies/dg,
../Bibliographies/islands,
../Bibliographies/diachronic,
../Bibliographies/gesture,
../Bibliographies/semantics,
../Bibliographies/pragmatics,
../Bibliographies/information-structure,
../Bibliographies/idioms,
../Bibliographies/cg,
../Bibliographies/udc}

  \togglepaper[25]
}{}



%%%%%%%%%%%%%%%%%%%%%%%%%%%%%%%%%%%%%%%%%
\title{Diachronic syntax} 
\author{%
 Ulrike Demske\affiliation{Universität Potsdam}
}
% \chapterDOI{} %will be filled in at production

% \epigram{}

\abstract{Basic questions of language change include the what, how and why of a change. The present paper focusses on the first question, raising the issue how a representational framework as \hpsg can be used to model syntactic change. Taking the history of the \ili{German} language as a showcase, different types of morpho-syntactic changes are considered using three case studies. The paper only briefly touches upon the last question, \ie why a particular syntactic change happens.
}

%%%%%%%%%%%%%%%%%%%%%%%%%%%%%%%%%%%%%%%%%%%%
\begin{document}
\label{chap-diachronic}
\maketitle

\section{Dimensions of syntactic change} 

Syntactic change links the language of two speech communities which are chronologically related such as Old  and Modern \ili{Norwegian}. Only the grammar of the first speech community allows null subjects in restricted contexts, while referential subjects may not be omitted in Modern Norwegian irrespective from the context. According to \cite{kinn2015}, Norwegian changes from a partial null-subject language to a language disallowing null subjects as exemplified by her examples in (\ref{onor}) and (\ref{mnor}).

\eal 
\ex \label{onor}
\gll Sægir hann þat  at   æigi  man \textit{pro} satt vera.\\ 
     says  he   that that not   can [it] true be \\\hfill(\ili{Old Norwegian})
\glt `He says that it cannot be true.'  
\ex \label{mnor}
\gll Han sier at $*$(han) ikke kan komme.\\ 
     he says that \hspaceThis{$*$(}he not can come \\ \hfill (\ili{Modern Norwegian}) 
\glt `He says that he cannot come.' 
\zl
The reinterpretation of non-subjects as subjects in the history of \ili{English} provides a second example \citep{denison1993}: A verb like \textit{langian} `long for' selects an accusative and not a nominative noun phrase in \ili{Old English} as exemplified in (\ref{imp}). 
\ea \label{imp}
\gll \th a ongan hine eft langian on his cy\th \th e  \\ then began him.\textsc{acc} again long for his {native land} \\ \hfill (\ili{Old English})
\glt `Then he began again to long for his native land.' 
\z
As shown by the translation of the Old English example, the experiencer argument of the verb \textit{long} has been promoted from object to subject in Modern English. 

Linking (\ref{onor}) and (\ref{mnor}) as well as the English versions of (\ref{imp}) in terms of syntactic change would amount to the linking of surface manifestations instead of the underlying grammars. And it is in fact the underlying grammar that is affected by syntactic change triggered by internal and external factors. Internal factors belong to the linguistic system and may be part either of the syntax or any other module of the grammar as the morphology or the lexicon. As regards the syntactic change in the history of English, illustrated above, it may be taken as a lexical change, modifying the argument structure of verbs with experiencer objects, as suggested by \cite{denison1993}. \cite{lightfoot1979} on the other hand considers the change as a syntactic one, \ie as an effect of the word order change from OV to VO order in the history of English. 

The impact of morphological changes on the syntactic component of the grammar has always been of particular importance in diachronic syntax. Strong correlations are supposed to hold between a rich morphology on the one hand and the possibility of for instance word order variation or the occurrence of null subjects. Thus an attested collapse of case distinctions may be made up by a less variable word order with structural positions conveying the information formerly provided by case markers. The loss of morphology may also be compensated by replacing case marked noun phrases by prepositional phrases with prepositions, now providing information regarding thematic roles. A pertinent example from the history of English concerns the loss of adnominal genitives as a consequence of the loss of the genitive as a morphological case. While the postnominal genitive phrase is superseded by the \textit{of}-phrase, the prenominal genitive marker is reanalyzed as a clitic \citep{allen2006}.\footnote{Considering the prenominal genitive marker in Present-day English as a clitic is disputable. \cite{zwicky1987}, for instance, argues that it is an inflectional marker.} In Old English, adnominal genitives are still attested in pre- and post-head position, respectively. 
\eal
\ex
\gll \& \th ær wæs Kola \dh æs cyning-es heahgereafa \\ and there was Kola the.\textsc{gen} king-\textsc{gen} {high reeve}  \\ \hfill (Old English)
\glt 'And there was Kola, the king's high reeve.' 
\ex
\gll \th æt wæron \th a ærestan scipu deniscra monna \th e \dots \\ that were the first ships Danish.\textsc{gen} men.\textsc{gen} which \\ \hfill (OE)
\glt 'those were the first ships of Danish men which \dots'
\zl

Syntactic change may also be driven by factors external to the language system as different as language processing, information packaging or language contact: According to \cite{hawkins2004}, grammars conventionalize syntactic structures depending on their degree of preference in performance. And the rise of VO patterns in the German variety of Cimbrian is due to its contact with the Italian VO order in the North Eastern part of Italy as pointed out by \cite{GrPo2005}. The following example illustrates the verb-object order attested in Cimbrian in contrast to the reversed order in Present-day German.
\eal
\ex
\gll Haütte die Mome hat gebläscht di Piattn.  \\ today the mother has washed the plates \\ \hfill (\ili{Cimbrian})
\glt 'Today, mother has washed the plates.' 
\ex 
\gll Heute hat die Mutter die Teller gespült.  \\ today has the mother the plates washed \\ \hfill (\ili{Present-day German})
\glt 'Today, mother has washed the plates.' 
\zl

Derivational approaches such as Government and Binding (= GB) and Minimalism assume that language change happens in the course of first language acquisition. Individual language learners reanalyze the linguistic input they get during the acquisition process (due to the sometimes ambiguous nature of the input), resulting in the resetting of a parameter value \citep{lightfoot1979}. The nature of parameters has substantially changed from GB to Minimalism: Resetting of a parameter value in terms of GB meant for the loss of null subjects, as illustrated in (\ref{onor}) and (\ref{mnor}), that the value from the null-subject parameter changed from [+] to [-] throughout the history of Norwegian, given appropriate input. The Minimalist view of parameters restricts parametric variation to lexical items in general and to the formal features of functional heads in particular (Borer-Chomsky Conjecture according to \cite{baker2008}). The loss of null subjects in Norwegian is then triggered by the absence of particular formal features in Modern Norwegian. Depending on the range of specific formal features, \cite{BiRo2017} distinguish four types of parameters according to their size: macro-, meso-, micro-, and nanoparameters, with the first type being the most stable one in the development of a language (e.g. rigid head-final or head-initial order). They further claim that formal features are not pre-specified in Universal Grammar but emerge from the interaction of Universal Grammar, primary linguistic data, and general cognitive optimization strategies in the sense of \cite{chomsky2005}. Two principles in particular are suggested by \cite{BiRo2017}: (i) feature economy, restricting the acquisition of features to those with robust evidence in the input data, thereby minimizing computation; and (ii) input generalization, requiring to make maximal use of the acquired features. A case in point for the latter strategy are languages with harmonic head-final or head-initial word orders triggered by the generalization of the head parameter. The resetting of parameters in a Minimalist framework is therefore no longer restricted to the linguistic system, but includes external factors such as acquisition strategies. Since robust evidence in the primary linguistic data is required to allow a particular formal feature to be present in the underlying grammar, its frequency in the linguistic input will play a prominent role. And here further external factors come into play: processing ease as well as language contact might affect the frequency of individual variants in the lifespan of a speaker, thus modifying her output and consequently the input of language learners. A case in point is the continuous form in Pennsylvania German \citep{louden1988}: while Common \ili{Pennsylvania German} (1850--1950) behaves like Standard German and uses the simple present to convey that an action is ongoing at the moemnt of speaking. Plain Pennsylvania German (since 1950), on the other hand, uses the present continuous, motivated by the contact with its \ili{English} counterpart.\footnote{The present continous is also attested in German dialects spoken in the south of Germany: 
\ea
\gll Sie ist das Buch am Lesen. \\ she is the book at read \\
\glt `She is reading the book.'
\z
The form is probably part of the Common Pennsylvania German grammar as well, since many speakers of this German variety originate from the south-western part of Germany. Assuming this being true, then the contact situation contributes to the increasing use of the continuous form in Pennsylvania German.
}      
\eal
\ex
\gll er lasst de Hund los \\ he lets the dog loose \\  \hfill (Common Pennsylvania German)
\glt `He is letting the dog loose.'
\ex
\gll er is de Hund an los lasse \\ he is the dog \textsc{prep} loose let \\  \hfill (Plain Pennsylvania German)
\glt `He is letting the dog loose.'
\zl
Another instance is the contact between \ili{Low German} and \ili{Swedish}. As shown by \cite[149]{petzell2016}, Swedish scribes of Low German frequently use a VO instead of an OV order, motivated by their L1 language. Such changes might result in changes of the underlying grammar of subsequent generations. Language internal and language external motivations for syntactic change probably hold across frameworks. Since the present paper focusses on the question how syntactic change can be captured in terms of \hpsg, I will not further explore how different frameworks motivate attested changes in the grammar of a language. 

Under the assumption that syntactic change is in fact lexical change triggered by the reanalysis of linguistic input, as \cite{BiWa2015} actually suggest for changes accounted for in terms of Minimalism, we would expect that lexicalist approaches to syntactic change such as \hpsg or \lfg are at least equally well suited as derivational approaches to model syntactic change, cf. \cite{vincent2001} and \cite{BoVi2017} for proposals couched in \lfg. The remainder of the paper will provide some case studies, each representing a particular type of syntactic change. The goal is to show how \hpsg can be used to model the way structure can change over time. All case studies are taken from the history of German.
 
The outline of the chapter is as follows: grammaticalization processes are exemplified by the rise of auxiliary verbs in Section \ref{GR}. Changes with respect to word order are addressed in Sections \ref{VC} regarding verb clusters and \ref{NP} with respect to constituent order changes within the noun phrase. Overall, Section \ref{NP} is devoted to various changes affecting the left periphery of noun phrases and briefly touches upon the issue why particular changes happen.  

%%%%%%%%%%%%%%%%%%%%%%%%%%%%%%%%%%%%%%%%%%
%%%%%%%%%%%%%%%%%%%%%%%%%%%%%%%%%%%%%%%%%%
\section{Case studies}

The case studies presented in this section are supposed to exemplify different types of syntactic change. Grammaticalization processes typically give rise to grammatical markers with the development of auxiliary verbs from main verbs figuring as a prominent example. The next section will hence focus on the emergence of the passive auxiliaries \textit{kriegen} `get', \textit{bekommen} `get' and \textit{geben} `give' in the history of German. A second type of change are word order changes, exemplified by changes affecting the order of verbs at the right clausal periphery in German. This particular change has attracted a lot of interest from a descriptive as well as a more formal perspective, cf. Section \ref{VC}. The third case study addresses several changes affecting the left periphery of noun phrases, including the grammaticalization of the definite determiner as well as the word order change of adnominal genitives. 

%%%%%%%%%%%%%%%%%%%%%%%%%%%%%%%%%%%%%%%%%%%
\subsection{Grammaticalization: Rise of auxiliary verbs \label{GR}} 

Grammaticalization processes have an impact on the way grammatical information is marked in a language. Grammatical information such as verbal mood may be expressed either by morphological means as in (\ref{gram}a) or by syntactic means as in (\ref{gram}b) which uses the auxiliary verb \textit{werden} to convey a modal meaning. With the grammatical meaning being alike, a morphological marker such as \textit{-e} in (\ref{gram}) is taken to be further down on the grammaticalization cline as the corresponding auxiliary verb because of its morphological boundedness.  
\eal \label{gram}
\ex 
\gll Fred ging-e       ins Kino.\\  
     Fred went-\SBJV{} to.the movies\\  \jambox*{(Present-day German)}
\glt `Fred would go to the movies.'
\ex
\gll  Fred würde ins Kino gehen.\\ 
      Fred \SBJV{} to.the movies go\\  \jambox{(Present-day German)}
\glt `Fred would go to the movies.'
\zl

Further prominent examples of grammaticalization processes involve the rise of \textit{n}-words in \ili{French} such as  \textit{pas} from the \ili{Latin} noun \textit{passus} `step' or \textit{personne} `no-one' which both were originally restricted to a positive meaning. A well-known German example for a grammaticalization process is the subordinating conjunction \textit{weil} `because' evolving from the noun \textit{Weile} `while'. Grammaticalization typically consists of a reduction in meaning (lexical >> grammatical) and in form (syntactic marker >> morphological marker).\footnote{Cf. \cite{lehmann2015} for his seminal work on issues of grammaticalization.}

As regards the class of auxiliary verbs in Present-day German, the verbs \textit{kriegen} and \textit{bekommen}, both meaning `get', are fairly recent members of this class \citep{reis1976}. They pattern with the passive auxiliary \textit{werden} `be', triggering classic diagnostics for passive constructions: an object of the  active counterpart figures as subject of the finite verb in the passive clause, and the subject of the active clause is optionally realized as \textit{von}-PP in its passive equal. A crucial difference between passive constructions with the auxiliary \textit{werden} `be' and auxiliaries like \textit{kriegen} `get` and \textit{bekommen} `get' concerns the object affected by the passive transformation: in contrast to the canonical passive construction with \textit{werden} `be', it is the dative object which becomes the subject of passive clauses with \textit{kriegen} `get` and \textit{bekommen} `get`, hence the term \textit{dative passive}.\footnote{The focus here is on the passive meaning of the verbs in question. Resultative uses of \textit{bekommen} `get`and \textit{kriegen} `get' as exemplified below are disregarded in the present context. 
\ea
\gll Selma kriegt die Gleichung gelöst. \\ Selma manages the equation solved  \\
\glt  `Selma manages to solve the equation.' 
\z
The example means that \textit{Selma} is an agent argument reaching a particular goal with some effort, with the auxiliary conveying an active meaning. This interpretation of \textit{kriegen} `get' is clearly different from its interpretation in a dative passive construction like \textit{Selma kriegt die Gleichung von Fred gelöst.} `The equation was solved for Selma by Fred.' Cf. \cite{reis1985a} for a discussion of both types of constructions.}  A characteristic example is provided by (\ref{bek}) with the dative object of \textit{jemandem die Leviten lesen} `read somebody the Riot Act'  being promoted to the grammatical subject of \textit{bekommen}. 
\ea \label{bek} 
\gll  Er bekommt die Leviten gelesen.  \\  he was the Riot.Act read \\
\glt `He was read the Riot Act.'
\z
Both verbs \textit{kriegen} `get' as well as \textit{bekommen} `get' have developed a grammatical meaning alongside their lexical meaning only in the recent history of German \citep{glaser2005, lenz2012}. Early examples for their use as auxiliary verbs come from the 16th and the 17th century with both verbs typically combining with past participles of lexical verbs which select for recipient arguments in object position (\textit{schenken} `grant', \textit{schicken} `send'), which may also be interpreted as recipient arguments of the governing predicate. The same holds for the theme argument which may have been selected either by the lexical verb or the emerging auxiliary. Examples as in (\ref{get1}) therefore represent the beginning of the grammaticalization process, \ie stage 1 according to \cite[63]{ebert78}. Diagnostics for the passive construction include the \textit{von}-phrase in (\ref{get1}a) as well as the valency alternation turning the dative object of the active clause into the subject of the corresponding passive clause. Both examples originate in the second half of the 17th century and are taken from sources included in \textit{Deutsches Textarchiv}. 
\eal \label{get1}
\ex
\gll und die Schuh hat sie von einem Mannsbild geschenkt bekommen / \\  and the shoes has she by a man given got  \\ 
\glt `and the shoes were given to her by some man.' 
\ex
\gll  Es verdrüst sie / daß du Wein hast geschickt bekommen / und sie keinen.  \\  \textsc{expl} annoys her {} that you wine have send got {} and she none  \\ 
\glt `She is annoyed, because you got wine, but she has not.'
\zl
Examples like (\ref{rp}) are instantiations of stage 2: \textit{bekommen} `get' has preserved its meaning as a verb of obtaining, the complement clause, however, can only be interpreted as argument of \textit{sagen} `tell', providing evidence for the rise of the auxiliary \textit{bekommen} `get'.\footnote{These examples provide evidence against the claim that the direct object is assigned a thematic role by both the finite and the nonfinite verb, cf. \cite{haider1986}.} 
\ea \label{rp} 
\gll  Ich habe nur gesagt bekommen, dass er Probleme mit den Hinterreifen hat.  \\  I have only told got that he problems with the rear.tires has \\ \hfill  (BRZ13/APR.07807)
\glt `I was just told that he has problems with his rear tires.'
\z
Stage 3 on the grammaticalization cline includes verbs with a privative semantics, indicating that the emerging auxiliary is grammaticalized to such a degree that dative objects are no longer restricted to recipient arguments. The use of intransitive verbs like \textit{helfen} `help' are considered to represent stage 4 on the grammaticalization path \citep[64]{ebert78}.
\eal \label{intrans}
\ex
\gll Aber nach einer Woche hatte sie noch nicht einmal die Fäden gezogen bekommen.  \\ but after one week had she.\nom {} still not again the stitches removed got \\ \hfill (RHZ03/DEZ.09011 RZ)
\glt `After one week, she did not even had removed the stitches.'
\ex
\gll  Sie wollen konkret geholfen bekommen.  \\ they.\nom {} want definitely helped get  \\ \hfill  (PHE/W18.00094)
\glt `They definitely want to get help.'
\zl
The use of \textit{kriegen} `get', \textit{bekommen} `get' as passive auxiliaries is attested in all German varieties with stylistic differences between the two verbs.\footnote{While \textit{bekommen} is used in the standard variety of German, \textit{kriegen} is confined to less formal registers. The verb \textit{erhalten} `get' is only rarely used as an auxiliary with a passive meaning:
\vspace{-0.2cm}
\ea Von diesem Gelde sollten Sie 10000 Mark bei der Hochzeit, das andere Geld in vier Raten ausbezahlt erhalten.
\glt `They are supposed to receive 10000 Mark at the occasion of their wedding, more money should be paid in four instalments.'
\z
\vspace{-0.2cm}
} 
This does not hold for the verb \textit{geben} `give' which may be used as a passive auxiliary only in certain dialects. Like \textit{kriegen} and \textit{bekommen}, the verb \textit{geben} developed into a passive auxiliary in the recent history of German \citep{lenz2007}. As (\ref{gebb}) shows, the experiencer argument of \textit{mitnehmen} `give a lift' appears as the subject of the auxiliary \textit{geben} `give', while the agent argument is realized by a \textit{von}-phrase as expected in a passive construction.\footnote{Thanks go to Christian Ramelli for his native speaker judgment as regards the West Central German variety spoken in the Eastern part of the Saarland.}
\ea \label{gebb}
\gll De Tobi gebbt vom Yannick mitgehol. \\
the Tobi gives by.the Yannick given.a.lift   \\ \hfill (\ili{West Central German})
\glt `Tobi is given a lift by Yannik.'
\z

In a Minimalistic framework, grammaticalization processes are modeled as a categorial reanalysis of lexical categories as functional categories (or of functional categories low in the functional structure of a clause into functional categories occupying higher positions in the tree structure) in line with the Borer-Chomsky Conjecture \citep[17]{RobRou2003}. In many instances, grammaticalization involves the loss of features, as predicted by a cognitive optimization strategy captured in terms of feature economy. How do declarative frameworks such as \lfg or \hpsg account for this type of syntactic change? 

An \hpsg analysis of the dative passive in Present-day German includes a lexical entry for the passive auxiliary \textit{bekommen} and a lexical rule deriving the participle from the verb stem of a main verb. According to \cite[288]{mueller2018}, the lexical entry of participles specifies the form of the participle as \textit{ppp} for \textit{perfect passive participle} and takes care of suppressing the most prominent argument with structural case,\footnote{The assumption that there is only one participle form implies that \textit{von} `by'-phrases in passive constructions have to be analyzed as adjuncts.} which is also called the designated argument.\footnote{The designated argument has been introduced into the analysis of complex verbs by \cite{haider1986} to account for the different syntactic behavior of ergative and unergative verbs. Depending on the particular auxiliary verb, the blocked designated argument may be deblocked, a case in point being the perfect auxiliary verb \textit{haben} `have'.} The feature descriptions of the rule's input and output for the ditransitive verb \textit{verordnen} `prescribe' illustrate the effects (adapted from \cite[149]{mueller2002} and \cite[285]{mueller2018}.)
\ea 
\label{verordnen-active}
Input \textit{verordn-} `prescribe': \\*
\ms{
phon \phonliste{ verordn- } \\
synsem|loc
\ms{
cat
\ms{
  head & verb \\
  arg-st & \sliste{ NP[\type{str}]$_{\ibox{1}}$, NP[\type{ldat}]$_{\ibox{2}}$, NP[\type{str}]$_{\ibox{3}}$}
  }\\
  cont
  \ms{
  ind & \ibox{4} event \\
  rels & \sliste{ \ms[verordnen]{
                               event & \ibox{4} \\
                               agent & \ibox{1}\\
                               recipient & \ibox{2} \\
                               theme & \ibox{3}\\
  }} \\
  }}}
\z 

\ea \label{verordnet-passive}
Output \textit{verordnet} `prescribed' \\
\ms{
phon \phonliste{ verordnet } \\
synsem|loc
\ms{
cat
\ms{
  head &  \ms[verb]{
    vform & ppp } \\
  arg-st & \sliste{ NP[\type{ldat}]$_{\ibox{2}}$, NP[\type{str}]$_{\ibox{3}}$}
  }\\
  cont
  \ms{
  ind & \ibox{4} event \\
  rels & \sliste{ \ms[verordnen]{
                               event & \ibox{4}\\
                               agent & \ibox{1}\\
                               recipient & \ibox{2}\\
                               theme & \ibox{3}\\
  }} \\
  }}}
\z 
Example (\ref{promotion}a) provides a further illustration of the observation that it is the recipient argument that is promoted to subject in a dative passive construction, while it is the theme argument in an agentive passive (\ref{promotion}b). The realization of the agent argument is blocked in both passive constructions. 
\eal \label{promotion}
\ex
\gll Er bekommt strenge Bettruhe verordnet. \\ he.\nom {} gets strict bed.rest.\acc {} prescribed \\
\glt `Strict bed rest was prescribed to him.'
\ex
\gll Ihm wird strenge Bettruhe verordnet \\ him.\dat {} gets strict bed.rest.\nom {} prescribed \\
\glt  `Strict bed rest was prescribed to him.'
\zl
The above contrast between dative and agentive passive suggests that the respective auxiliary
determines whether the theme or the recipient argument of the participle figures as subject of the
passive clause. In the lexical entry for the passive auxiliary \textit{bekommen} (simplified from
\cite[313]{mueller2013}), the subject of the auxiliary is coindexed with the dative argument of the
embedded participle. The lexical entry in (\ref{get2}) illustrates furthermore that the auxiliary
attracts the remaining arguments of the embedded verb. The Case Principle\is{principle!case} (see \crossrefchaptert{case}
on case) controls how structural case is assigned.\footnote{\emph{Case Principle} according to \cite[287]{mueller2018}:
\begin{itemize}
\setlength{\itemsep}{0pt}
\item The first element with structural case in the argument structure list receives nominative.
\item All other elements in the list with structural case receive accusative.
\end{itemize}}
\ea \label{get2}
\textit{bekomm-} `get' (auxiliary, dative passive): \\
\ms{
phon \phonliste{ bekomm-} \\
synsem|loc|cat
\ms{
  head &  \ms[verb]{
   aux & + }\\
   arg-st & \sliste{ NP[\type{str}]$_{\ibox{1}}$}  $\oplus$ \ibox{2}  $\oplus$ \ibox{3}\\
          & $\oplus$ \sliste { V [\type{ppp}, {arg-st \ibox{2}  $\oplus$ \sliste{ NP[\type{ldat}]$_{\ibox{1}}$}  $\oplus$ \ibox{3}}]}} \\
 }
\z 
The lexical entry requires that the dative passive is restricted to verbs with dative arguments, including ditransitive verbs like \textit{verordnen} `prescribe', \textit{schenken} `grant' and \textit{schicken} `send' as well as intransitive verbs like \textit{helfen} `help' and \textit{danken} `thank'. The dative passive is excluded with ergative verbs \citep[298]{mueller2013}, a restriction which is not reflected in the feature description above.\footnote{To this effect, \cite[298, 313]{mueller2013} includes the feature \da in the argument structure of the participle, based on examples such as:
\ea[*]{
\gll Die Gewerkschaft kriegt beigetreten. \\ 
     the union        gets   joined\\
}
\z
}

In view of the analysis of the dative passive in Present-day German, its rise in the history of German may be modeled in terms of a change modifying the \textsc{content} feature of the verb \textit{bekommen} `get'. Semantic bleaching of the lexical verb in the course of its grammaticalization process affects above all its argument structure: the lexical verb \textit{bekommen} assigns the thematic role recipient to its subject argument and a theme role to its direct object: 
\ea \label{active_voll} 
\textit{bekomm-} `get': \\
\ms{
phon \phonliste{ bekomm- } \\
synsem|loc
\ms{
cat
\ms{
  head &  \ms[verb]{
   aux & $-$ }\\
  arg-st & \sliste{ NP[\type{str}]$_{\ibox{1}}$, NP[\type{str}]$_{\ibox{2}}$}
  }\\
  cont
  \ms{
  index & \ibox{3} event \\
  rels & \sliste{ \ms[bekommen]{
                               event & \ibox{3}\\
                               recipient & \ibox{1}\\
                               theme & \ibox{2}\\
  }} \\
  }}}
\z 
The auxiliary \textit{bekommen} `get', on the other hand, has no lexical meaning and does not assign thematic roles. Thematic roles come instead from the embedded non-finite verb, cf. (\ref{get2}). Its rise in the history of German can be thought of as taking place in the following way: when the verb \textit{bekommen} enters the class of auxiliary verbs in the 16th century, it selects participles of ditransitive verbs denoting a change of possession. The recipient and the theme argument of the participle match the thematic roles associated with the lexical verb \textit{bekommen} `get' which belongs to the verbs of obtaining. Matching of thematic roles is certainly a prerequisite for the career of \textit{bekommen} as an auxiliary. Its increasing grammaticalization is testified by the occurence of verbal complements which are less and less restricted with respect to argument structure: Their argument structure may include a second argument which does not match, however, an argument of the lexical verb \textit{bekommen}, cf. the clausal argument of \textit{sagen} `say' in example (\ref{rp}). And their argument structure may lack a second argument altogether as with the intransitive verb \textit{helfen} `helfen`, cf. (\ref{intrans}b). The increasing grammaticalization of \textit{bekommen} `get' is further attested by the observation that its distribution is no longer governed by a semantic restriction to verbs denoting change of possession, but only by a syntactic one, because it requires verbs with a dative object in its argument structure (\ref{get2}). The difference between the passive auxiliary \textit{bekommen} `get' and its lexical verb version is hence captured by their feature descriptions, \ie the \cat feature as well as the \content feature. The diachronic scenario sketched for \textit{bekommen} `get' is supposed to hold for other passive auxiliaries as well, cf. \textit{kriegen, erhalten} `get' and \textit{geben} `give'. In much greater detail, \cite{warner1995} shows how changes affecting the English auxiliary system, such as the rise of the progressive passive, can be modeled in \hpsg in a straightforward way.\footnote{More examples regarding the English auxiliary system and an account in terms of \hpsg are discussed in \cite{warner1993}. Further studies on the rise of auxiliary verbs in representational theories of grammar include \cite{schwarze2001} who models this change within the framework of \lfg. He suggests to capture the rise of auxiliary verbs in Romance by a difference with respect to the f-structure features in the lexical entries between lexical and auxiliary verbs. Thus the f-structure of a passive clause is headed by the non-finite verb in actual Romance, while the modern Romance counterparts of  \textit{esse} `be' may no longer function as heads of an f-structure with their transition to auxiliaries.} Since derivational approaches model grammaticalization processes such as the one described above in terms of feature loss, this type of syntactic change may be equally well modeled in representational and derivational approaches.

%%%%%%%%%%%%%%%%%%%%%%%%%%%%%%%%%%%%%%%%%%
%%%%%%%%%%%%%%%%%%%%%%%%%%%%%%%%%%%%%%%%%%
\subsection{Word order changes in the verbal complex \label{VC}} 

Our second case study is concerned with a word order change affecting the order of verbs at the right clausal periphery throughout the period of \ili{Early New High German} (1350--1650). 

\ili{Present-day German} is an OV language with the (finite) verb occuring at the right edge of a clause. In case more than one verb appears in final position, the canoncial order is descending, \ie V$_3$V$_2$V$_1$, with the governing verb following the governed verb as illustrated by the three-place verb cluster in (\ref{canon}). 
\ea \label{canon}
\gll auch wenn Selma die Noten gefunden$_3$ haben$_2$ wird$_1$, \\ even if Selma the notes found have will \\
\glt `even if Selma will have found the music'
\z
The occurrence of non-canonical orders like V$_1$V$_3$V$_2$ is restricted in Standard German with respect to the number of verbs in the cluster (at least three verbs) and the type of auxiliary. While the auxiliary verb \textit{haben} `have' requires the non-canonical word order, cf. (\ref{noncanon}), the tense auxiliary \textit{werden} `will' and the modal verbs may occur with both orders. As regards the auxiliary \textit{sein} `be', the non-canonical order is not available (\ref{noncanon_wrong}).  
\eal
\ex[]{
\label{noncanon}
\gll auch wenn Fred das hätte$_1$ wissen$_3$ müssen$_2$, \\ even if Fred this had know must \\
\glt `even if Fred had been required to know this'
}
\ex[*]{
\label{noncanon_wrong}
\gll auch wenn er gestern in der Vorlesung ist$_1$ gesehen$_3$ worden$_2$, \\ 
     even if he yesterday in the lecture is seen been   \\
}
\zl
Neither of these restrictions holds for non-standard varieties of German: Upper German dialects provide ample evidence for both descending and ascending verb orders, even if the verb cluster includes only two verbs \citep{dubenion2010}. Likewise historical stages of German witness a wide variety of word orders regardless of the number of verbs appearing at the right periphery of the clause and independent of the nature of the auxiliary \citep{ebert1981,haerd1981,sapp2011}. The Early New High German examples in (\ref{vc_dia}) render attestations for a two-place verbal complex with \textit{haben} `have' preceding a past participle and a three-place verbal complex with the auxiliary verb \textit{sein} `be' preceding two past participles. Both patterns are ruled out in the standard varieties of Present-day German.
\eal \label{vc_dia}
\ex
\gll uns ist ein Abentüer widerfaren underwegen, daz uns ein Wolff vil Leids hat$_1$ gethon$_2$ \\ us is an adventure happened on.the.way that us a wolf much harm has done  \\  \hfill (Early New High German)
\glt `An adventure happened to us on the way: a wolf has done much harm to us.'
\ex
\gll so schreibt man auch aus Holl. das newlich in Frießlandt ein fewriger fliegender Trach sey$_1$ gesehen$_3$ worden$_2$ \\ so writes one also from Holland that recently in Friesland a fiery flying dragon were seen been \\  
\glt `News come from the Netherlands that a fiery flying dragon has been seen in Friesland.' \hfill (Early New High German)
\zl
According to \cite{ebert1981,haerd1981} and \cite{sapp2011}, the restrictions, effective in Present-day German, arise in a two-step process:  the order V$_2$V$_1$ becomes fixed with two-place verbal complexes troughout the 16th century, while it took about a hundred more years for the order V$_3$V$_2$V$_1$ to become the canonical order for three-place verbal complexes.

How can this change modeled in a representational framework such as \hpsg? Auxiliary verbs and their
verbal complements as given in (\ref{canon}) through (\ref{vc_dia}) are supposed to build verb
clusters with the arguments of the respective verbal complement being attracted by the auxiliary
\citep{HiNa94,pollard1994,kiss1995,kathol2000,Meurers2000,mueller2002,mueller2013}. The verb cluster
consequently requires the same arguments as the embedded verb. Accordingly, the structure of a verb
cluster exemplifying the canonical descending order can be represented as in the passive verb
cluster \textit{dass sie die Fäden gezogen bekam} `that she had removed the stitches' shown in Figure~\ref{fig-faeden-gezogen-bekam}.
\begin{figure}
\begin{forest}
sm edges
[{V[\subj \eliste, \comps \eliste]}
  [ \ibox{1} {NP [\type{nom}]} [sie;she] ] [{V[\subj \sliste{ \ibox{1} }, \comps \eliste]}
  [ \ibox{2} {NP [\type{acc}]} [die Fäden;the stitches,roof] ] [{V[\subj \sliste{ \ibox{1} }, \comps \sliste{ \ibox{2} } ] } 
  [\ibox{3} {V[\subj \sliste{ \ibox{1} }, \comps \sliste{ \ibox{2} }] } [gezogen;removed] ] [ {V[\subj \sliste{ \ibox{1} }, \comps \sliste{ \ibox{3}, \ibox{2} }]} [bekam;got] ] ]
  ] ] ]
\end{forest}
\caption{\label{fig-faeden-gezogen-bekam}Passive verb cluster in \emph{dass sie die Fäden gezogen$_2$ bekam$_1$} `that she had
  removed the stitches'}
\end{figure}

The variation regarding the order of auxiliary and lexical verb in verb clusters is currently addressed from two perspectives: (i) non-canonical patterns of three-place verb clusters in Present-day German as in (\ref{noncanon}) which figure under the notion of \textit{Oberfeldbildung} since \cite{bech1983}, and (ii) the canonical order in \ili{Dutch} verb clusters which is ascending instead of descending, \ie V$_1$V$_2$V$_3$. Building on previous proposals, one way to account for the word order change affecting the verbal complex in the history of German would include the assumption of an appropriate head feature as advocated by \cite{HiNa94}. They emphasize that a lexical approach to the word order variation within verb clusters would also account for the variation on the level of individual speakers.

Recent work suggests the head feature \textsc{gvor} which indicates the direction of government of non-finite verbs and was proposed to capture synchronic variation in German and Dutch \citep{BoNo1996,kathol2000,augustinus2015}.\footnote{The head feature suggested by \cite{HiNa94} is \textsc{flip} which indicates government to the right when exhibiting a positive value.} From a diachronic perspective, one would either assume that infinitival as well as participial complements carry a feature \textsc{gvor} which is underspecified in earlier stages of German (= \textsc{gvor} \textit{dir}) allowing for the attested variation of word orders, cf. Figure \ref{gvor}, or one would go on the assumption that the head feature \textsc{gvor} arises only later in the history of German. In the context of example (\ref{vc_dia}b), the value of \textsc{gvor} may be determined as follows: the participle \textit{gesehen} `seen' is governed by the auxiliary \textit{worden} `been', appearing on its right side and the \textsc{gvor} value of the verb cluster \textit{gesehen worden} `seen been' is feature shared with its head daughter \textit{worden} `been' which is governed by the auxiliary \textit{sey} `be'.  

\begin{figure} 
\begin{forest} 
sm edges
[{V}
[{V} [sey;be]]
[{V[ \textsc{gvor} $\leftarrow$]}
[{V[ \textsc{gvor} $\rightarrow$]} [gesehen;seen]][{V[ \textsc{gvor} $\leftarrow$]}[worden;been]]]
]
\end{forest}
\caption{Three-place verb cluster in \ili{Early New High German} \label{gvor}}
\end{figure}
\noindent
In Present-day German, a three-place verb cluster including the auxiliary \textit{sein} `be' exhibits the canonical word order V$_3$V$_2$V$_1$. In contrast to \ili{Early New High German}, the feature value is provided by the lexical entry of the respective verb. A partial lexical description of the passive participle \textit{worden} `been' is illustrated below, indicating that its governing auxiliary has to appear on the right side. 
\ea \label{worden}
\textit{worden} `been': \\
\ms{
phon \phonliste{ worden } \\
synsem|loc|cat|head
  \ms[verb]{
    vform & ppp \\
    aux & $+$ \\
    gvor & $\rightarrow$ }\\
      } \\
\z 

As has been suggested in work on synchronic variation as regards verb cluster in German and Dutch,
diachronic variation can be modeled in a straightforward way by building on lexical entries of
verbs. The analysis of the word order change sketched above makes use of the possibility that a
lexical feature may be underspecified in a particular variety of a language. Cf.\ also
\crossrefchaptert[Section~\ref{sec-id-lp}]{order} on constituent order and underspecification.

%%%%%%%%%%%%%%%%%%%%%%%%%%%%%%%%%%%%%%%%%%
%%%%%%%%%%%%%%%%%%%%%%%%%%%%%%%%%%%%%%%%%%
\subsection{Left periphery of noun phrases \label{NP}}

The third case study presents a set of changes affecting the left periphery of noun phrases in the history of German. All changes might be due to a single change as regards the relationship between nominal and determiner \citep{demske2001}. 

As other Germanic languages, German distinguishes two types of adjectival declension, \ie the weak and the strong declension \citep{VeSlPe2014}. What governs the distribution of both types changes throughout the history of German. The historical record indicates that the distribution of adjectival inflection types is semantically governed in \ili{Old High German}, \ie definite determiners trigger weak adjectival inflection, whereas indefinite determiners call for strong adjectival inflection. In (\ref{ohg_decl1}), the weak declension type is triggered by the demonstrative and the possessive determiner, respectively, while the strong declension type in (\ref{ohg_decl2}) is motivated by the indefinite \textit{ein} `ein'. The strong declension type is used irrespective of the morphology of the indefinite determiner (cf. \textit{ein} `a' vs. \textit{einemo} `a' in (\ref{ohg_decl2})). Old High German behaves in this respect as Modern \ili{Icelandic}.\footnote{In Modern Icelandic, a definite noun phrase requires a weakly inflected adjective, while an indefinite noun phrase requires a strongly inflected one: \textit{þessi raud-i hestur} `this red horse' vs. \textit{raud-ur hestur} `a red horse'.} 

      
\eal \label{ohg_decl1}
\ex
\gll thiz irdisg-a dal \\  this worldly-\textsc{weak}{} valley \\ \hfill (Old High German)
\glt `this worldly valley'
\ex
\gll min liob-o sun \\ my good-\textsc{weak} sun \\  \hfill (Old High German)
\glt `my good sun'
\zl

\eal \label{ohg_decl2}
\ex 
\gll ein arm-az uuîb \\ a poor-\textsc{strong} woman \\ \hfill (Old High German)
\glt `a poor woman'
\ex 
\gll einemo diur-emo merigrioze \\ a valuable-\textsc{strong} pearl \\ \hfill (Old High German)
\glt `a valuable pearl'

\zl
In Present-day German the distribution of adjectival inflection types is morphologically governed: if grammatical features of the noun phrase are overtly marked by the determiner, the following adjective instantiates the weak inflection type, otherwise the adjective exhibits strong inflection (cf. \textit{ein} `a' vs. \textit{einem} `a').
       
\eal
\ex 
\gll ein herausragend-er Cellist  \\ an.\textsc{m.nom} outstanding-\textsc{strong} cellist\\  \hfill (Present-day German)
\glt `an outstanding cellist'
\ex
\gll ein-em herausragend-en Cellisten \\ an-\textsc{m.dat} outstanding-\textsc{weak} {cellist}\\ \hfill (Present-day German)
\glt `an outstanding cellist'
\zl
The changing nature of the relationship between determiner and nominal may be captured by a change affecting the selectional requirements of the determiner: in Old High German, it is the \content feature of the determiner that drives the distribution of adjectival declension types, in Present-day German, on the other hand, the distribution is driven by its \cat feature. 

The feature description of the indefinite determiner in Old High German includes as \AGR {} features \textsc{case}, \textsc{number} and \textsc{gender}. The feature \textsc{spec} indicates that the determiner requires a nominal expression lacking a specifier, \ie \nom {} according to \cite[64]{SaWaBe2003}. The set of possible nominal expressions is restricted by the \decl feature, whose value is semantically driven in Old High German, with strong adjectival declension signaling indefiniteness.  
\ea 
\ms{
phon \phonliste{ einemo } \\
synsem|loc|cat
\ms{
  head & \ms[det]{
          agr & \ibox{1} \ms{
  case & dat \\
  num & sg \\
  gend & masc $\vee$ neut
  } } \\ 
  spec & \sliste{ \nom {} [\AGR {} \ibox{1} {},  \decl {} \type{strong}] }
       } \\
}
\z  
In Present-day German, the \decl value of \nom {} no longer conveys information about its definiteness. The determiner selects for \nom according to categorial features: in case the determiner provides information on the \AGR {} value of the noun phrase, it asks for a weakly inflected \nom {} as in (\ref{decl-pdg}). Otherwise the information in question has to be provided by a strongly inflected \nom.\footnote{Using data provided by Andreas Kathol, \cite[373]{PoSa94} point out that nouns like \textit{Verwandter} `relative' support the assumption that \decl {} is a feature not only of adjectives but also of nouns, because the declension class of these nouns is governed by the respective determiner. Cf. also \cite[64]{kiss1995} for the example \textit{Beamter} `public official'.} Note that the \decl feature now appears in the feature description of \cat \citep[65]{kiss1995}.
\ea \label{decl-pdg}
\ms{
phon \phonliste{ einem } \\
synsem|loc|cat
\ms{
  head & \ms[det]{
           agr & \ibox{1} \ms{
  case & dat \\
  num & sg \\
  gend & masc $\vee$ neut } \\
  \decl & \type{strong}
    } \\ 
  spec & \sliste{ \nom {} [\AGR {} \ibox{1} {}, \decl {} \type{weak}] }
       } \\
}
\z 

A possible motivation for this change comes from the increasing grammaticalization of the definite determiner: while the determiner is attested above all with sortal concepts in Old High German testifying to its use as a demonstrative, it lacks in cases where the head noun refers to functional concepts that are inherently unambiguous \citep{demske2001}. The examples in (\ref{sortal}) illustrate this distribution: a head noun like \textit{figboum} `figtree' has a sortal meaning, \ie its unique referent is specified by the context in (\ref{sortal}a), while the noun \textit{erda} `earth' denotes a functional concept which refers unambiguously irrespective of particular situations (\ref{sortal}b). A demonstrative determiner as \textit{ther} `this one', denoting uniqueness, is hence not necessary in such a context.\footnote{Demonstrative pronouns are excluded with functional nouns, because they require at least two entites of a kind.}
\eal \label{sortal}
\ex Inti quad Imo, niomer fon thir uuahsmo arboran uuerde zi éiuuidu thô sâr sliumo arthorr\&{a} \textbf{ther figboum}. \hfill (Old High German)
\glt `and he saith unto it, Let there be no fruit from thee henceforward for ever. And immediately the fig tree withered away.'
\ex Inti \textbf{erda} giruorit uuas Inti steina gislizane uuarun \hfill (OHG)
\glt `And the earth shook and the rocks were split.'  
\zl
The rise of weak definites in the course of Early New High German represents a further step in the grammaticalization process of the definite determiner: relational nouns such as \textit{Sohn} `son', \textit{Zahn} `tooth' or \textit{Stelle} `position' with their argument in postnominal position may lack any determiner (\ref{weakdef}a) or they exhibit either the indefinite or the definite determiner (\ref{weakdef}b, \ref{weakdef}c). The third pattern is the default pattern in Present-day German \citep{demske2020}.
\eal \label{weakdef}
\ex 
\gll Vnd wie.wol ich bin \textbf{sone} \textbf{eins} \textbf{konigs} \\  and although I am son a king's \\  \hfill (Early New High German)
\glt `And although I am the son of a king.'
\ex 
\gll Sie namen \textbf{einen} \textbf{Zahn} \textbf{eines} \textbf{Thiers}/ welches so groß ist wie eine Ratte \\ they took a tooth an animal's  which as big is as a rat  \\  \hfill (Early New High German)
\glt `They took the tooth of an animal which was as big as a rat.'
\ex ich verstunde gleich aus ihrem Diskurs (...) daß ihr Mann beim Senat wäre, und ohngezweifelte Hoffnung hätte, denselben Tag \textbf{die Stell eines Landvogts oder Landamtmanns} zu bekommen \hfill (ENHG)
\glt `From her conversation I came to discover that her husband was in the senate (...) He was also supposed to have had good expectations of receiving the position of a district governor or a bailiff that very day.'
\zl
In Present-day German, definite determiners are used with sortal, functional and relational nouns, cf. the functional noun \textit{Mond} `moon' and the relational noun \textit{Tochter} `daughter' in (\ref{def_PDG}), suggesting that the definite determiner is no longer licensed on semantic grounds as in Old High German, but on morphosyntactic grounds. The changing distribution of the definite determiner fits nicely the assumption that the specifier relation between determiner and \nom {} originally is semantically based and then turns into a morphosyntactically based relationship.

\eal \label{def_PDG}
\ex 
\gll Der Mond ist aufgegangen. \\ the moon has risen \\ \hfill (Present-day German)
\glt `The moon has risen.'
\ex 
\gll Sie ist die Tochter eines Unternehmers. \\  she is the daughter an entrepreneur's    \\ \hfill (Present-day German)
\glt `She is the daughter of an entrepreneur.'  
\zl

A significant role in the history of the nominal left periphery is played by the adnominal genitive. Three stages have to be distinguished in its development: in the first stage, genitive noun phrases systematically appear in prenominal position as attested in Old High German sources. The genitive may even be preceded by a determiner and an adjective as illustrated by the second example. Note that the determiner and the adjective are marked here for dative case as required by the governing preposition \textit{in} `in'.
\eal
\ex 
\gll scouuot thes accares lilia uuvo sie uuahsen \\ observe the field's lilies how they grow \\  \hfill (Old High German)
\glt `Observe how the lilies of the field grow.'
\ex 
\gll In dhemu heilegin daniheles chiscribe \\ in the holy Daniel's scripture  \\  \hfill (Old High German)
\glt `in the holy scripture of Daniel'
\zl
The adnominal genitive of stage two also occurs in prenominal position, provided that it denotes humans or animals:
\eal \label{animate}
\ex 
\gll Der Frawen [zu vnseren zeiten] kunst weyßheit vnd tugende ist nit not zu erzelen \\  the women at our times art wisdom and virtue is not necessary to tell  \\  \hfill (Early New High German)
\glt `The art, wisdom and virtue of women in our days does not need to be recounted'
\ex 
\gll Dieser Tagen seyn allhie der Evangel. Fürsten vnnd Städt/ [so zu Schwäbschen Hall jüngst beysamen gewest/] Abgesandte alher komen \\ these days are here the Protestamt sovereigns' and cities' who at Schwäbisch Hall recently together been envoys here come  \\\hfill (Early New High German)
\glt `These days envoys of the Protestant sovereigns and cities have come to this location, after they had met at Schwäbisch Hall recently.'	
\zl
All adnominal genitives marked [$-$animate] are now restricted to postnominal position as testified by historical data from the Early New High German period \citep{ebert88}. The prenominal genitive is still a full noun phrase in stage two, allowing not only for pre-head, but also for post-head dependents as shown by the postnominal modifiers in the examples above, which can be phrasal (\ref{animate}a) or sentential (\ref{animate}b).

The final stage in the development of the adnominal genitive is represented by Present-day German: the prenominal position is confined to proper names and kinship terms disfavoring pre-head or post-head modifiers. In the unmarked case, genitive phrases headed by individual nouns appear in postnominal position irrespective of the feature [$\pm$animate]:\footnote{Since this word order change happened only recently in the history of German, \ie the fast decrease of prenominal genitives is attested for the 17th century, we expect relics of the once productive pattern to occur in Present-day German. This expectation is borne out, as exemplified by the following titles representing works of fiction published in the 20th century.
\eal
\ex
\gll Des Teufels General \\ the devil's general \\ \hfill (Drama by Carl Zuckmayer)
\glt `The Devil's General'
\ex
\gll Des Mauren letzter Seufzer \\ the moor's last sigh  \\ \hfill (Novel by Salman Rushdie)
\glt  `The Moor's Last Sigh'  
\zl
}

\eal
\ex[]{
\gll Selmas/Vaters altes Fahrrad \\ Selma's/Daddy's old bike\\
}
\ex[*]{
\gll der Menschheit ältester Traum \\ 
     the mankind's oldest dream \\
}
\zl

\eal
\ex[]{
\gll das Gartenhaus des alten Goethe \\ 
     the summer.house of.the old Goethe \\
\glt 'the summer house of late Goethe'
}
\ex[*]{ 
\gll des alten Goethe Gartenhaus \\ 
     the old Goethe summer.house \\
}
\zl

The historical scenario sketched for the development of the adnominal genitive fits in well with other changes affecting the left periphery of noun phrases in German: while the adnominal genitive is a full noun phrase in Old High German, it functions as a possessive determiner in Present-day German just as the possessive pronoun, disallowing pre- and post-head dependents. Again the change can be modeled as a change affecting the relation between a prenominal constituent and the nominal head. The Old High German demonstrative pronoun \textit{ther} `this one' develops from a modifier into a specifier in the history of German. Likewiese, the prenominal genitive develops into a specifier of the nominal, provided that it can contribute a possessive meaning. Adnominal genitives with the feature [$-$animate] are consequently postponed and they retain their grammatical function as modifiers of the head noun. Prenominal genitives on the other hand become possessive determiners establishing a \textsc{spec} relation to nominals with the \decl feature \textit{strong} (adapted from \cite[54]{PoSa94}):
\ea 
\ms{
phon \phonliste{ Selmas } \\
synsem|loc
\ms{
cat  
\ms{
  head & det \\
  spec & \sliste{ \nom {} [\decl {} \type{strong}, \textsc{ind} \ibox{1} , \textsc{restr} \ibox{2} ] }
       } \\
content
\ms{
  det & the \\
  restind &
  \ms{
  ind \ibox{1} \\
  restr & \sliste{ 
  \ms{
  reln & poss \\
  possessor & \ibox{3} \\
  possessed & \ibox{1} 
 }}
 $\oplus$ \ibox{2}
 }} \\
context|backgr \sliste{
\ms{
reln & naming \\
bearer & \ibox{3} \\
name & Selma
}}}}
\z 
The reanalysis of the relation between prenominal possessive and nominal in the history of German not only affects the pre-head genitive but also the possessive pronoun. In Middle High German and still in Early New High German, the possessive pronoun patterns with adjectives considering its co-occurrence patterns: It may follow a definite determiner and may even agree with another adjective as regards its declension type as in (\ref{agreement}). 
\eal
\ex \label{agreement}
\gll  die iuwer-n scoen-en tohter \\ the your-\textsc{weak} beautiful-\textsc{weak} daughter  \\ \hfill (Middle High German)
\glt `your beautiful daughter'
\ex \label{postadjective}
\gll mit gross-em jhr-em Rhum vnd Lob \\ with big-\textsc{strong} their-\textsc{strong} glory and praise   \\ \hfill (ENHG)
\glt `with their big glory and praise'
\zl
Here possessive pronoun and adjective both exhibit weak declension triggered by the preceding definite determiner. At this stage in the history of German, the possessive pronoun is still a constituent of the nominal. The Early New High German example in (\ref{postadjective}) shows that a possessive pronoun may also follow a prenominal adjective suggesting that it functions as a modifier itself (note also the agreement with respect to declencion type between adjective and possessive pronoun). This word order is excluded in Present-day German.

Once again, it is the relation between prenominal element and head noun which is subject to change: the possessive pronoun behaves as an adjectival modifier in earlier stages of German, before it becomes a possessive determiner (\ie \textsc{mod} relation develops into \textsc{spec} relation). According to \cite[54]{PoSa94}, its lexical description looks very much like the description of the prenominal genitive given above (disregarding the \content and the \textsc{context} value).

Changes at the left periphery of noun phrases start with the grammaticalization of the definite determiner throughout the period of Old High German, testified by the extension of its distribution. In Present-day German, not only sortal, but also functional and relational nouns combine with the definite determiner. The steady increase in the use of the determiner triggers a reanalysis of the relation between determiner and nominal: a semantically driven relation turns into a relation that is also morphosyntactically based as evidenced by the changing motivation for the adjectival declension type. The determiner is consequently licensed by the \cat feature in the noun's feature structure as shown for the functional noun \textit{Mond} `moon' which subcategorizes for its determiner in Present-day German but has not done so in Old High German.
\ea \label{noun}
\ms{
phon \phonliste{ Mond } \\
synsem|loc|cat
\ms{
  head & \ms[noun] {
  agr & \ibox{1}
  } \\
  spec & \sliste{  loc|cat|head \ms[det]{
                                   agr & \ibox{1}
                                   }} \\
  arg-st & \eliste
  }\\
}
\z 
The reanalysis of the relation between determiner and nominal has consequences for the interpretation of other prenominal constituents: possessive pronouns as well as possessor phrases are likewise taken to instantiate a specifier relation to the nominal in question, thus augmenting the class of determiners in German. In addition, pre-head constituents precluding a specifier interpretation are postponed, cf. genitive complements with [$-$animate], and pre-head constituents ambiguous between a specifier and a modifier reading are limited to one interpretation, cf. the demonstrative \textit{solch} `such' behaving as a demonstrative adjective in Present-day German \citep{demske2005}. All changes can be modeled in a straightforward way as lexical changes.  

%%%%%%%%%%%%%%%%%%%%%%%%%%%%%%%%%%%%%%%%%
%%%%%%%%%%%%%%%%%%%%%%%%%%%%%%%%%%%%%%%%%%%
\section{Summary} 

Recent years witnessed a growing consensus that quite a few  instances of syntactic change may be accounted for in the lexicon of a language. The consensus holds across frameworks: \cite{BiWa2015} highlight the role of the lexicon in Minimalism, the volume by \cite{BuKi2001} presents case studies of syntactic change in the representational framework of \lfg. The present contribution set out to show how the typed feature structures of \hpsg can be used to model the way syntactic structures change over time. Different types of morpho-syntactic change have been considered in the history of German: the grammaticalization of auxiliary verbs and of demonstrative pronouns, word order changes affecting verb and noun phrases and changing relations between prenominal constituents and the respective nominal. In all cases, the change in question can be modeled in terms of feature structures in the lexicon of a language and hence in terms of a framework as \hpsg.  


%%%%%%%%%%%%%%%%%%%%%%%%%%%%%%%%%%%%
\section*{Acknowledgements}
Many thanks go to three anonymous reviewers for their critical questions and comments regarding an earlier version of the paper. Most of all I would like to thank Stefan Müller, who encouraged me to contribute to this handbook. With gratitude, I acknowledge the training I received in Peter Suchsland's group in Jena, where I learned to remain open to more than one framework in generative linguistics.



\nocite{Schoene1939a,Lindo1966a,DeReKo-dia,Simplicissimus}
{\sloppy
\printbibliography[heading=diachrony-sources,keyword=diachrony-source]
}

%% \begin{description}
%% \setlength{\itemsep}{0pt}
%% \item \textit{Aviso 1609}. Edited by Walter Schöne. Facsimile edition. Leipzig: Harrassowitz 1939.
%% \item \textit{Ein kurtzweilig Lesen von Dil Ulenspiegel.} According to the print of 1515. Edited by Wolfgang Lindo. Stuttgart: Philipp Reclam jun. 1966.
%% \item \textit{DeReKo. Das Deutsche Referenzkorpus.} URL: http://www.ids-mannheim.de/kl/\\projekte/korpora [retrieved July 4, 2017].
%% \item von Grimmelshausen, Hans Jacob Christoffel. \textit{Der Abentheurliche Simplicissimus Teutsch und Continuatio des abentheurlichen Simplicissimi} [1669]. Edited by Rolf Tarot, Tübingen: Max Niemeyer 1984.
%% \item \textit{Pontus und Sidonia in der Verdeutschung eines Ungenannten aus dem 15. Jahrhundert}. Edited by Karin Schneider. Berlin: Erich Schmidt 1961. (Texte des späten Mittelalters 14).
%% \item Ralegh, Walter. \textit{Americæ achter Theil / in welchem erstlich beschrieben wirt das maechtige vnd goldtreiche Koenigreich Guiana (\dots) durch (\dots) Walthern Ralegh Rittern und Hauptmann vber jrer koen. mayest. auß Engellandt Leibs Guardi (\dots). Alles erstlich in engellændischer Sprach außgangen / jetzt aber auß der ollændischen Translation in die hochteutsche Sprache gebracht / durch Avgvstinum Cassiodorvm Reinivm (\dots) An Tag gegeben durch Dieterschen von Bryseligen hinderlassenen Erben}. Frankfurt 1599.
%% \item \textit{Otfrids Evangelienbuch}. Edited by Oskar Erdmann, 6th edition by Ludwig Wolff. Tübingen: Max Niemeyer 1973.
%% \item \textit{Die lateinisch-althochdeutsche Tatianbilingue}. Stiftsbibliothek St. Gallen Cod. 56. Edited by Achim Masser. Göttingen: Vandenhoeck \& Ruprecht 1994.
%% \end{description}

%%%%%%%%%%%%%%%%%%%%%%%%%%%%%%%%%%%%%%%%%%%%%%%
{\sloppy
\printbibliography[heading=subbibliography,notkeyword=this]
}
\end{document}
