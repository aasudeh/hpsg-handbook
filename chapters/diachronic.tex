\documentclass[output=paper]{langsci/langscibook} 
\author{Ulrike Demske\affiliation{Universität Potsdam}}
\title{Diachronic syntax}

% \chapterDOI{} %will be filled in at production

%\epigram{Change epigram in chapters/03.tex or remove it there }
\abstract{Basic questions of language change concern the what, how, why of a change. The present paper focusses on syntactic change and addresses the question how a representational framework as \hpsg can be used to model syntactic change. Taking the history of German as a showcase, different types of morpho-syntactic changes are considered including changes within the verb phrase as well as the noun phrase.
}
\maketitle

\begin{document}
\label{chap-diachronic}


\section{Dimensions of syntactic change} 

Syntactic change links the language of two speech communities which are chronologically related such as Old  and Modern Norwegian. Only the grammar of the first speech community allows null subjects in restricted contexts, while referential subjects are obligatory in Modern Norwegian. As shown by \cite{kinn2015}, Norwegian changes from a partial null-subject language to a language disallowing null subjects.

\eal 
\ex \label{onor}
\gll Sægir hann þat  at   æigi  man \textit{pro} satt vera.\\ 
     says  he   that that not   can [it] true be \\\hfill(Old Norwegian)
\glt 'He says that it cannot be true.'  
\ex \label{mnor}
\gll Han sier at $*$(han) ikke kan komme.\\ 
     he says that \hspaceThis{$*$(}he not can come \\ \hfill (Modern Norwegian) 
\glt 'He says that he cannot come.' 
\zl
The reinterpretation of non-subjects as subjects in the history of English provides a second example \citep{denison1993}: A verb like \textit{langian} 'long for' selects an accusative but no nominative noun phrase in Old English as exemplified in (\ref{imp}). As shown by the translation, the experiencer argument of the verb \textit{long} has been promoted from object to subject in Modern English. 
\ea \label{imp}
\gll \th a ongan hine eft langian on his cy\th \th e  \\ then began him.\textsc{acc} again long for his {native land} \\ \hfill (Old English)
\glt 'Then he began again to long for his native land.' 
\z
Linking (\ref{onor}) and (\ref{mnor}) as well as the English versions of (\ref{imp}) in terms of syntactic change would amount to the linking of surface manifestations instead of the underlying grammars. And it is in fact the underlying grammar that is affected by syntactic change triggered by internal and external factors. Internal factors belong to the linguistic system and may be part either of the syntax or any other module of the grammar as the morphology or the lexicon. As regards the syntactic change in the history of English, illustrated above, it may be taken as lexically driven, modifying the argument structure of verbs with experiencer objects, as suggested by \citep{denison1993}. \cite{lightfoot1979} on the other hand considers the change to be syntactically driven, i.e. as an effect of the word order change from OV to VO orders in the history of English. 

The impact of morphological changes on the syntactic component of the grammar has always been of particular importance in diachronic syntax. Strong correlations are supposed to hold between a rich morphology on the one hand and the possibility of for instance word order variation or null subjects. Thus an attested collapse of case distinctions may be made up by a less variable word order with structural positions conveying the information formerly provided by case markers. The loss of morphology may also be compensated by replacing case marked noun phrases by prepositional phrases with prepositions now providing the relevant information. A pertinent example from English concerns the loss of adnominal genitives as a consequence of the loss of the genitive as a morphological case. While the postnominal genitive phrase is superseded by the \textit{of}-phrase, the prenominal genitive marker is reanalyzed as a clitic \citep{allen2006}. In Old English, adnominal genitives are attested in pre- and post-head position.       
\eal
\ex
\gll \& \th ær wæs Kola \dh æs cyning-es heahgereafa \\ and there was Kola the.\textsc{gen} king-\textsc{gen} {high reeve}  \\ \hfill (Old English)
\glt 'And there was Kola, the king's high reeve.' 
\ex
\gll \th æt wæron \th a ærestan scipu deniscra monna \th e \dots \\ that were the first ships Danish.\textsc{gen} men.\textsc{gen} which \dots \\ \hfill (OE)
\glt 'those were the first ships of Danish men which \dots'
\zl

Syntactic change may also be driven by external factors such as language processing, information packaging or language contact: According to \cite{hawkins2004}, grammars conventionalize syntactic structures depending on their degree of preference in performance. And the rise of VO patterns in the German variety of Cimbrian is due to its contact with the Italian VO order  in the North Eastern part of Italy \citep{GrPo2005}. 
\eal
\ex
\gll Haütte die Mome hat gebläscht di Piattn.  \\ today the mother has washed the plates \\ \hfill (Cimbrian)
\glt 'Today, mother has washed the plates.' 
\ex 
\gll Heute hat die Mutter die Teller gespült.  \\ today has the mother the plates washed \\ \hfill (Present-day German)
\zl

Derivational approaches such as Government and Binding and Minimalism assume that language change happens in the course of first language acquisition. Individual language learners reanalyze the linguistic input they get during the acquisition process, resulting in the resetting of a parameter value \citep{lightfoot1979}. The nature of parameters has substantially changed from GB to Minimalism: Resetting of a parameter value in terms of GB meant for the loss of null subjects, as illustrated in (\ref{onor}) and (\ref{mnor}), for instance that the value from the null-subject parameter changed from [+] to [-] in the history of Norwegian. The minimalist view of parameters restricts parametric variation to lexical items in general and to the formal features of functional heads in particular (Borer-Chomsky Conjecture according to \cite{baker2008}). The null-subject parameter will then be phrased in terms of a feature bundle associated with a functional verbal head with the loss of null-subjects triggered by the absence of particular features. Depending on the range of specific formal features, \cite{BiRo2017} distinguish four types of parameters according to their size: macro-, meso-, micro-, and nanoparameters, with the first type being the most stable one in the development of a language (e.g. rigid head-final or head-initial order). They further claim that formal features are not pre-specified in Universal Grammar but emerge from the interaction of Universal Grammar, primary linguistic data, and general cognitive optimization strategies in the sense of \cite{chomsky2005}. Two principles in particular are suggested by \cite{BiRo2017}: (i) feature economy, restricting the acquisition of features to those with robust evidence in the input data, thereby minimizing computation; and (ii) input generalization, requiring to make maximal use of the acquired features. A case in point for the latter strategy are languages with harmonic head-final or head-initial word orders triggered by the generalization of the head parameter. The resetting of parameters in a Minimalist framework is therefore no longer restricted to the linguistic system, but includes external factors such as acquisition strategies. Since robust evidence in the primary linguistic data is required to allow a particular formal feature to be present in the underlying grammar, its frequency in the linguistic input will play a prominent role. And here further external factors come into play: Processing ease as well as language contact might affect the frequency of individual variants in the lifespan of a speaker, thus modifying her output and consequently the input of language learners. A case in point is the continuous form in Pennsylvania German \citep{louden1988}: While the pattern is still restricted in Common Pennsylvania German (1850--1950), it is widely spread in Plain Pennsylvania German (since 1950) motivated by the contact with its English counterpart. Another instance is the contact between Low German and Swedish. As shown by \cite{petzell2016}, Swedish scribes of Low German frequently use a VO instead of an OV order, motivated by their L1 language. Such changes might result in changes of the underlying grammar of subsequent generations.     
\eal
\ex
\gll er lasst de Hund los \\ he lets the dog loose \\  \hfill (Common Pennsylvania German)
\ex
\gll er is de Hund an los lasse \\ he is the dog \textsc{prep} loose let \\  \hfill (Plain Pennsylvania German)
\glt 'He is letting the dog loose.'
\zl
Under the assumption that syntactic change is triggered by changes affecting the lexicon, one might conclude that there is no such thing as syntactically triggered change, as \cite{BiWa2015} actually suggest. We would then expect that lexicalist approaches to syntactic change such as \hpsg or \lfg are at least equally well suited as derivational approaches to model syntactic change, cf. \cite{vincent2001} and \cite{BoVi2017} for \lfg. The remainder of the paper will hence provide some case studies to illustrate how \hpsg can be used to model the way structure can change over time. All case studies are taken from the history of German.
 
The outline is as follows: Grammaticalization processes are exemplified by the rise auf auxiliary verbs in section \ref{GR}. Changes with respect to word order are addressed in sections \ref{VC} regarding verb clusters and \ref{NP} with respect to word order changes within the noun phrase. Overall, section \ref{NP} is devoted to various changes affecting the left periphery of noun phrases and also touches upon the issue why particular changes happen.  

%%%%%%%%%%%%%%%%%%%%%%%%%%%%%%%%%%%%%%%%%%
%%%%%%%%%%%%%%%%%%%%%%%%%%%%%%%%%%%%%%%%%%
\section{Case studies}

The case studies presented in this section are supposed to exemplify different types of syntactic change. Grammaticalization processes typically give rise to grammatical markers with the development of auxiliary verbs from main verbs figuring as a prominent example. The next section will therefore deal with the emergence of the passive auxiliaries \textit{kriegen, bekommen} 'get' and \textit{geben} 'give' in the history of German. A second type of change are word order changes which are the topic of section \ref{VC}, focussing on the order of verbs at the right clausal periphery in German, a syntactic change which has attracted a lot of interest from a descriptive as well as a more formal perspective. The third case study addresses several changes affecting the left periphery of noun phrases, including the grammaticalization of the definite determiner as well as the word order change of adnominal genitives. 

%%%%%%%%%%%%%%%%%%%%%%%%%%%%%%%%%%%%%%%%%%%
\subsection{Grammaticalization: rise of auxiliary verbs \label{GR}} 

Grammaticalization processes have an impact on the way grammatical information is marked in a language. Grammatical information such as verbal mood may be expressed either by morphological means as in (\ref{gram}a) or by syntactic means as in (\ref{gram}b) which uses the auxiliary verb \textit{werden} to convey a modal meaning. With the grammatical meaning being alike, a morphological marker such as \textit{-e} in (\ref{gram}) is taken to be further down on the grammaticalization cline as the corresponding auxiliary verb because of its morphological boundedness.  
\eal \label{gram}
\ex 
\gll Fred ging-e ins Kino. \\  Fred went-\SBJV{} {to the} movies \\  \hfill (Present-day German)
\ex
\gll  Fred würde ins Kino gehen. \\ Fred \SBJV{} {to the} movies go \\  \hfill (Present-day German)
\glt 'Fred would go to the movies.'
\zl

Further prominent examples of grammaticalization processes involve the rise of negation markers in French such as  \textit{pas} from the Latin noun \textit{passus} 'step' or \textit{personne} 'no-one' which both were originally restricted to a positive meaning. A well-known German example for a grammaticalization process is the subordinating conjunction \textit{weil} 'because' evolving from the noun \textit{Weile} 'while'. Grammaticalization typically consists of a reduction in meaning (lexical >> grammatical) and in form (syntactic marker >> morphological marker).\footnote{Cf. \cite{lehmann2015} for his seminal work on issues of grammaticalization.}

As regards the class of auxiliary verbs in Present-day German, the verbs \textit{kriegen} and \textit{bekommen}, both meaning 'get', are fairly recent members of this class \citep{reis1976}. They pattern with passive auxiliaries such as \textit{werden} 'get' and \textit{sein} 'be' in German, triggering classic diagnostics for passive constructions: An object of the  active counterpart figures as subject of the finite verb in the passive clause, and the subject of the active clause is optionally realized as \textit{von}-PP in its passive equal. A crucial difference to the canonical passive construction concerns the status of the object: It is not the direct but the indirect object which becomes the subject of the passive clause, hence the term \textit{dative passive}.\footnote{The focus here is on the passive meaning of the verbs in question. Resultative uses of \textit{bekommen, kriegen} 'get' are disregarded in the present context. \textit{Selma kriegt die Gleichung gelöst} 'Selma manages to solve the equation', means that \textit{Selma} is an agent argument reaching a particular goal with some effort with the auxiliary conveying an active meaning. Cf. \cite{reis1985a} for a discussion of both types of constructions.}  In (\ref{rp}), the indirect object of \textit{sagen} 'tell'  was promoted to grammatical subject of \textit{bekommen},favored by the fact that both verbs select for experciencer arguments although in different slots of their argument structure. The \textit{dass} 'that'-clause can only be understood as an argument of \textit{sagen} 'tell', providing evidence for an analysis of \textit{bekommen} 'get' as auxiliary.
  
\ea \label{rp} 
\gll  Ich habe nur gesagt bekommen, dass er Probleme mit den Hinterreifen hat.  \\  I have only told got that he problems with the {rear tires} has \\ \hfill  (BRZ13/APR.07807)
\glt 'I was just told that he has problems with his rear tires.'
\z
Both verbs \textit{kriegen} 'get' as well as \textit{bekommen} 'get' have developed a grammatical meaning alongside their lexical meaning in the recent history of German \citep{glaser2005, lenz2012}. Early examples for their use as auxiliary verbs come from the 16th and the 17th century with both auxiliaries typically combining with past participles of lexical verbs which select for experiencer arguments in object position (\textit{schenken} 'grant', \textit{schicken} 'send'), which may be interpreted as experiencer arguments of the governing predicate. The same holds for the theme argument which may be selected either by the lexical verb or the emerging auxiliary. Examples as in (\ref{get1}) therefore represent the beginning of the grammaticalization process. Diagnostics for the passive construction include the \textit{von}-phrase in (\ref{get1}a) as well as the valency alternation turning the indirect objects of the corresponding active clause into subjects of the passive clause.

\eal \label{get1}
\ex
\gll da hatte ich eben ein paar Ducaten vom Herrn geschenckt kriegt \\  \expl{} had I just a few ducats {by the} master granted got   \\   \hfill (1672: Weise)
\glt 'I just got a few ducats by the master.' 
\ex
\gll  Es verdrüst sie/ daß du Wein hast geschickt bekommen/ und sie keinen.  \\  \textsc{expl} annoys her that you wine have send got and she none  \\ \hfill (1695: Reuter)
\glt 'She is annoyed, because you got wine, but she has not.'
\zl
Examples like (\ref{get1}) are taken to represent stage 1 in the grammaticalization process, while examples like (\ref{rp}) are instantiations of stage 2, because the direct object can only be interpreted as argument of the non-finite verb.\footnote{These examples provide evidence against the claim that the direct object is assigned a thematic role by both the finite and the nonfinite verb, cf. \cite{haider1986}.} Stage 3 on the grammaticalization cline includes verbs with a privative semantics, indicating that the emerging auxiliary is grammaticalized to such a degree that indirect objects are no longer restricted to recipient arguments. The use of intransitive verbs like \textit{helfen} 'help' are considered to represent stage 4 on the grammaticalization path \citep{ebert78}.
\eal \label{intrans}
\ex
\gll Aber nach einer Woche hatte sie {noch nicht einmal} die Fäden gezogen bekommen.  \\ but after one week had she.\nom {} {not again} the stitches removed got \\ \hfill (RHZ03/DEZ.09011 RZ)
\glt 'After one week, she had not again removed the stitches.'
\ex
\gll  Sie wollen konkret geholfen bekommen.  \\ sie.\nom {} want definitely helped get  \\ \hfill  (PHE/W18.00094)
\glt 'They definitely want to get help.'
\zl
The use of \textit{kriegen, bekommen} 'get' as passive auxiliaries is attested in all German varieties with stilistic differences between the two verbs.\footnote{While \textit{bekommen} is used in the standard variety of German, \textit{kriegen} is confined to less formal registers. The verb \textit{erhalten} 'get' is only rarely used as an auxiliary with a passive meaning.} This does not hold for the verb \textit{geben} 'give' which may be used as a passive auxiliary only in certain dialects. Like \textit{kriegen} and \textit{bekommen}, the verb \textit{geben} developed into a passive auxiliary in the recent history of German \citep{lenz2007}. As (\ref{gebb}) shows, the experiencer argument of \textit{mitnehmen} 'give a lift' appears as subject of the auxiliary \textit{geben} 'give', while the agent argument is realized by a \textit{von}-phrase as expected in a passive construction.\footnote{Thanks go to Christian Ramelli for his native speaker judgment as regards the West Central German variety spoken in the Eastern part of the Saarland.}
\ea \label{gebb}
\gll De Tobi gebbt vom Yannick mitgehol. \\
the Tobi gives {by the} Yannick {given a lift}   \\ \hfill (West Central German)
\glt 'Tobi is given a lift by Yannik.'
\z

In a minimalistic framework, grammaticalization processes are modeled as a categorial reanalysis of lexical categories as functional categories or of hierarchically lower functional categories as higher ones in line with the Borer-Chomsky Conjecture \citep{RobRou2003}. In many instances, grammaticalization means the loss of features, as predicted by a cognitive optimization strategy as feature economy. How do declarative frameworks such as \lfg or \hpsg account for this type of change? 

An \hpsg analysis of the dative passive in Present-day German includes a lexical entry for the passive auxiliary \textit{bekommen} and a lexical rule deriving the lexical entry for a passive verb from its active counterpart. The feature description for the passive auxiliary in (\ref{get2}) captures the fact that auxiliaries are underspecified for their argument structure, i.e. \textit{become} takes as its complement the passive participle and all the complements the passive verb requires. In contrast to the passive participle in the standard passive, participles in dative passive constructions may assign accusative case when an appropriate argument is present. Examples as (\ref{intrans}b) have shown that the dative passive is not restricted to ditransitive verbs in Present-day German as has been claimed by \cite{KoNo2009}.
\ea \label{get2}
\textit{bekommt}-\textsc{aux} 'gets': \\
\ms{
phon \phonliste{ bekommt } \\
synsem|loc|cat
\ms{
  head &  \ms[verb]{
    vform & fin \\
    aux & + }\\
   arg-st & \sliste{ \ms{
   head & passive-part \\
   arg-st & \sliste { NP[\type{nom}]$_{\ibox{1}}$, (NP[\type{acc}]$_{\ibox{2}}$) }
   }
      } }\\
 }
\z 
The passive lexical rule derives the passive participle subcategorized by the auxiliary \textit{bekommen} 'get', illustrated by way of its input and output feature description in (\ref{active}) and (\ref{passive}), cf. \cite[285]{mueller2018} for the standard passive construction in German. The dative passive rule suppresses the most prominent argument, the agent, of the active verb and promotes the argument marked for dative case by the active verb, the beneficiary, to subject of the passive verb \textit{verordnet} 'prescribed'. The theme argument is not affected by the dative passive rule.\footnote{Assuming a  lexical rule to derive a passive participle that fits the requirements of the passive auxiliary \textit{bekommen}, results in three different participles, i.e. perfect participle, passive participle for standard passive constructions and passive participle for dative passive constructions. \cite{mueller2018} argues that only one lexical rule is needed to derive the past participle. According to him, the respective auxiliary determines which arguments are realized in a perfect or a passive context. Whether his analysis is better suited to fit the participle data in German does not touch upon the issue of the grammaticalization process illustrated here.}
\ea
\gll Er bekommt strenge Bettruhe verordnet. \\ he.\nom {} gets strict {bed rest}.\nom {} prescribed \\
\glt ' Strict bed rest was prescribed to him.'
\z

\ea \label{active}
Input \textit{verordn-} 'prescribe' \\
\ms{
phon \phonliste{ verordn } \\
synsem|loc
\ms{
cat
\ms{
  head & verb \\
  arg-st & \sliste{ NP[\type{nom}]$_{\ibox{1}}$, NP[\type{dat}]$_{\ibox{2}}$, NP[\type{acc}]$_{\ibox{3}}$}
  }\\
  cont
  \ms{
  index & \ibox{4} event \\
  rels & \sliste{ \ms[verordnen]{
                               event \ibox{4} \\
                               agent \ibox{1}\\
                               benefac \ibox{2} \\
                               theme \ibox{3}\\
  }} \\
  }}}
\z 

\ea \label{passive}
Output \textit{verordnet} 'prescribed' \\
\ms{
phon \phonliste{ verordnet } \\
synsem|loc
\ms{
cat
\ms{
  head &  \ms[verb]{
    vform & passive-part } \\
  arg-st & \sliste{ NP[\type{nom}]$_{\ibox{2}}$, NP[\type{acc}]$_{\ibox{3}}$}
  }\\
  cont
  \ms{
  index & \ibox{4} event \\
  rels & \sliste{ \ms[verordnen]{
                               event \ibox{4}\\
                               agent \ibox{1}\\
                               benefac \ibox{2}\\
                               theme \ibox{3}\\
  }} \\
  }}}
\z 

In view of the analysis of the dative passive in Present-day German, its rise in the history of German may be modeled in terms of a change modifying the \textsc{content} feature of the verb \textit{bekommen}. Semantic bleaching of the main verb in the course of its grammaticalization process affects above all its argument structure: The lexical verb \textit{bekommen} assigns the thematic role beneficiary to its subject argument and a theme role to its direct object, cf. (\ref{active}). The auxiliary \textit{bekommen} on the other hand has no lexical meaning and does not assign thematic roles. The rise of the passive auxiliary \textit{bekommen} can be thought of as taking place in two steps:
Starting with the lexical verb \textit{bekommen}, the first step includes its semantic bleaching such that the verb admits verbal complements headed by ditransitive verbs denoting a change of possession. The second step is best characterized by the further bleaching of the auxiliary-to-be: Its distribution is no longer governed by a semantic restriction to verbs denoting a change of possession, but only by a syntactic one, because it requires verbs with a dative object in its active counterpart, cf. (\ref{intrans}a). Alongside its bleached version, the verb \textit{bekommen} retains its original status as a lexical verb. The diachronic scenario sketched for \textit{bekommen} 'get' is supposed to hold for the other passive auxiliaries as well. 
\ea \label{active} 
\textit{bekommt} 'gets': \\
\ms{
phon \phonliste{ gets } \\
synsem|loc
\ms{
cat
\ms{
  head & verb \\
  arg-st & \sliste{ NP[\type{nom}]$_{\ibox{1}}$, NP[\type{acc}]$_{\ibox{2}}$}
  }\\
  cont
  \ms{
  index & \ibox{3} event \\
  rels & \sliste{ \ms[bekommen]{
                               event \ibox{3}\\
                               benefac \ibox{1}\\
                               theme \ibox{2}\\
  }} \\
  }}}
\z 

Other studies on the rise of auxiliary verbs in representational theories of grammar include \cite{schwarze2001} who models this change within the framework of lexical functional grammar. He suggests to capture the rise of auxiliary verbs in Romance by a difference with respect to the f-structure features in the lexical entries between lexical and auxiliary verbs. In actual Romance, the f-structure of a passive clause is headed by the non-finite verb, while the modern Romance counterparts of  \textit{esse} 'be' may no longer function as heads of an f-structure with their transition to auxiliary status.

%%%%%%%%%%%%%%%%%%%%%%%%%%%%%%%%%%%%%%%%%%
%%%%%%%%%%%%%%%%%%%%%%%%%%%%%%%%%%%%%%%%%%
\subsection{Word Order Changes in the Verbal Complex \label{VC}} 

Present-day German is an OV language with the finite verb occuring at the right edge of subordinate clauses. In case more than one verb appears in final position, the canoncial order is descending, i.e. V$_3$V$_2$V$_1$ with the governing verb following the governed verb as illustrated by the three-place verb cluster in (\ref{canon}). The occurrence of non-canonical orders like V$_1$V$_3$V$_2$ is restricted in Standard German with respect to the number of verbs in the cluster (at least three verbs) and the type of auxiliary. While the auxiliary verb \textit{haben} 'have' requires the non-canonical word order, cf. (\ref{noncanon}), the tense auxiliary \textit{werden} 'will' and the modal verbs may occur with both orders. As regards the auxiliary \textit{sein} 'be', the non-canonical order is not available (\ref{noncanon_wrong}).  

\ea \label{canon}
\gll Auch wenn Selma die Noten gefunden$_3$ haben$_2$ wird$_1$, \\ even if Selma the notes found have will \\
\glt 'Even if Selma will have found the music'
\z

\eal
\ex \label{noncanon}
\gll Auch wenn Fred das hätte$_1$ wissen$_3$ müssen$_2$, \\ even if Fred this had know must \\
\glt 'even if Fred should have known this’
\ex \label{noncanon_wrong}
\gll $*$Auch wenn er gestern in der Vorlesung ist$_1$ gesehen$_3$ worden$_2$, \\ \hspaceThis{$*$(}even if he yesterday in the lecture is seen been   \\
\glt 'Even if he has been seen in the lecture yesterday '
\zl
Neither of these restrictions holds for non-standard varieties of German: Upper German dialects provide ample evidence for both descending and ascending verb orders, even if the verb cluster includes only two verbs \citep{dubenion2010}. Likewise historical stages of German witness a wide variety of word orders regardless of the number of verbs appearing in final position and independent of the nature of the auxiliary \citep{ebert1981,haerd1981,sapp2011}. The Early New High German examples in (\ref{vc_dia}) render attestations for a two-place verbal complex with \textit{haben} preceding a past participle and a three-place verbal complex with the auxiliary verb \textit{sein} preceding two past participles. Both patterns are ruled out in the standard varieties of Present-day German.
\eal \label{vc_dia}
\ex
\gll uns ist ein Abentüer widerfaren underwegen, daz uns ein Wolff vil Leids hat$_1$ gethon$_2$ \\ us is an adventure happened {on the way} that us a wolf much harm has done  \\  \hfill (Ulenspiegel 226.4)
\glt 'An adventure happened to us on the way: a wolf has done much harm to us.'
\ex
\gll so schreibt man auch aus Holl. das newlich in Frießlandt ein fewriger fliegender Trach sey$_1$ gesehen$_3$ worden$_2$ \\ so writes one also from Holland that recently in Friesland a fiery flying dragon were seen been \\  \hfill (Aviso 35.14)
\glt 'News come from the Netherlands that a fiery flying dragon has been seen in Friesland.'
\zl
The restrictions effective in Present-day German arise in a two-step process: The order V$_2$V$_1$ becomes fixed with two-place verbal complexes troughout the 16th century, while it took about a hundred more years for the order V$_3$V$_2$V$_1$ to become the canonical order for three-place verbal complexes \citep{ebert1981,haerd1981,sapp2011}.

How is this change modeled in a representational framework such as \hpsg? Auxiliary verbs and their verbal complements as given in (\ref{canon}) through (\ref{vc_dia}) are supposed to build verb clusters with the arguments of the respective verbal complement being attracted by the auxiliary \citep{HiNa94,pollard1994,kathol2000,mueller2002,mueller2008}.\footnote{\cite{AbGo2002} argue that verbal complements of auxiliary verbs in French are part of a flat VP when the auxiliary conveys tense information, while the passive auxiliary \textit{être} 'be' takes a VP complement.} Accordingly, the structure of a verb cluster exemplifying the canonical descending order can be represented as in the passive verb cluster \textit{dass sie die Fäden gezogen bekam} 'that she had removed the stitches':
\begin{figure}
\begin{forest}
[{V[\subj \eliste, \comps \eliste]}
  [ \ibox{1} {NP [\type{nom}]} [sie] ] [{V[\subj \sliste{ \ibox{1} }, \comps \eliste]}
  [ \ibox{2} {NP [\type{acc}]} [die Fäden] ] [{V[\subj \sliste{ \ibox{1} }, \comps \sliste{ \ibox{2} } ] } 
  [\ibox{3} {V[\subj \sliste{ \ibox{1} }, \comps \sliste{ \ibox{2} }] } [gezogen] ] [ {V[\subj \sliste{ \ibox{1} }, \comps \sliste{ \ibox{2}, \ibox{3} }]} [bekam] ] ]
  ] ] ]
\end{forest}
\caption{Passive verb cluster in \emph{dass sie die Fäden gezogen$_2$ bekam$_1$}}
\end{figure}

The variation regarding the order of auxiliary and lexical verb in verb clusters is currently addressed from two perspectives: (i) non-canonical patterns of three-place verb clusters in Present-day German as in (\ref{noncanon}) which figure under the notion of \textit{Oberfeldbildung} since \cite{bech55}, and (ii) the canonical order in Dutch verb clusters which is ascending instead of descending, i.e. V$_1$V$_2$V$_3$. Building on previous proposals, one way to account for the word order change affecting the verbal complex in the history of German would include the assumption of an appropriate head feature as advocated by \cite{HiNa94}. They emphasize that a lexical approach to the word order within the verb cluster would account also for  the variation on the level of individual speakers.

Recent work suggests the head feature \textsc{govr} which indicates the direction of government of non-finite verbs and was proposed to capture synchronic variation in German and Dutch \citep{BoNo1996,kathol2000,augustinus2015}.\footnote{The head feature suggested by \cite{HiNa94} is \textsc{flip} which indicates government to the right when exhibiting a positive value as in example (\ref{noncanon}) above.} From a diachronic perspective, one would either assume that infinitival as well as participial complements carry a feature \textsc{govr} which is underspecified in earlier stages of German (= \textsc{govr} \textit{dir}) allowing for the attested variation of word orders, cf. figure \ref{govr}, or one would go on the assumption that the head feature \textsc{govr} arises only later in the history of German. In the context of example (\ref{vc_dia}b), the value of \textsc{govr} may be determined as follows: The participle \textit{gesehen} 'seen' is governed by the auxiliary \textit{worden} 'been' appearing on its right side and the \textsc{govr} value of the verb cluster \textit{gesehen worden} is feature shared with its head daughter \textit{worden} which is governed by the auxiliary \textit{worden}.  

\begin{figure} 
\begin{forest} 
[{V}
[{V} [sey]]
[{V[ \textsc{govr} $\leftarrow$]}
[{V[ \textsc{govr} $\rightarrow$]} [gesehen]][{V[ \textsc{govr} $\leftarrow$]}[worden]]]
]
\end{forest}
\caption{Three-place verb cluster in Early New High German \label{govr}}
\end{figure}
\noindent
In Present-day German, a three-place verb cluster including the auxiliary \textit{sein} 'be' exhibits the canonical word order V$_3$V$_2$V$_1$. In contrast to Early New High German, the feature value is provided by the lexical entry of the respective verb. A partial lexical description of the passive participle \textit{worden} 'been' is illustrated below, indicating that its governing auxiliary has to appear on the right side. 
\ea \label{worden}
\textit{worden}-\textsc{aux} 'been': \\
\ms{
phon \phonliste{ worden } \\
synsem|loc|cat
\ms{
  head &  \ms[verb]{
    vform & passive-part \\
    aux & + \\
    govr & $\rightarrow$ }\\
      } \\
 }
\z 

As has been suggested in work on synchronic variation as regards verb cluster in German and Dutch, diachronic variation can be modeled in a straightforward way by building on lexical entries of verbs. The analysis of the word order change sketched above makes use of the possibility that a lexical feature may be underspecified in a particular variety of a language.

%%%%%%%%%%%%%%%%%%%%%%%%%%%%%%%%%%%%%%%%%%
%%%%%%%%%%%%%%%%%%%%%%%%%%%%%%%%%%%%%%%%%%
\subsection{Left Periphery of Noun Phrases \label{NP}}

The third case study presents a bundle of changes affecting the left periphery of noun phrases in the history of German. All changes might be due to a single change as regards the relationship between nominal and determiner \citep{demske2001}.

The historical record indicates that the distribution of adjectival inflection types is semantically governed in Old High German, i.e. definite determiners trigger weak adjectival inflection, whereas indefinite determiners call for strong adjectival inflection. In (\ref{ohg_decl1}), the weak declension type is triggered by the demonstrative and the possessive determiner, respectively, while the strong declension type in (\ref{ohg_decl2}) is motivated by the indefinite pronoun. In contrast to Present-day German the strong declension type is used irrespective of the morphology of the indefinite determiner (cf. \textit{ein} vs. \textit{einemo}). Old High German behaves in this respect as Modern Icelandic.\footnote{In Modern Icelandic, a definite noun phrase requires a weakly inflected adjective and an indefinite noun phrase a strongly inflected one: \textit{þessi raud-i hestur} 'this red horse' vs. \textit{raud-ur hestur} ' a red horse'.} 

      
\eal \label{ohg_decl1}
\ex
\gll thiz irdisg-a dal \\  this worldly-\textsc{weak}{} valley \\ \hfill (OHG)
\glt 'this worldly valley'
\ex
\gll min liob-o sun \\ my good-\textsc{weak} sun \\  \hfill (OHG)
\glt 'my good sun'
\zl

\eal \label{ohg_decl2}
\ex 
\gll ein arm-az uuîb \\ a poor-\textsc{strong} woman \\ \hfill (OHG)
\glt 'a poor woman'
\ex 
\gll einemo diur-emo merigrioze \\ a valuable-\textsc{strong} pearl \\ \hfill (OHG)
\glt 'a valuable pearl'

\zl
In Present-Day German the distribution of adjectival inflection types is morphologically governed: If grammatical features are overtly marked by the determiner, the following adjective instantiates the weak inflection type, otherwise the adjective exhibits strong inflection (cf. \textit{ein} vs. \textit{einem}).
       
\eal
\ex 
\gll ein herausragend-er Cellist  \\ an outstanding-\textsc{strong} cellist\\  \hfill (PDG)
\ex
\gll einem herausragend-en Cellisten \\ an outstanding-\textsc{weak} {cellist}\\ \hfill (PDG)
\glt 'an outstanding cellist'
\zl
The changing nature of the relationship between determiner and nominal may be captured by a change in the feature structure of the determiner: In Old High German, it is the \content feature of the determiner that drives the distribution of adjectival declension types, in Present-day German on the other hand, the distribution is driven by its \cat feature. The feature description of the indefinite determiner Old High German includes as \AGR features \textsc{case}, \textsc{number} and \textsc{gender}\footnote{Disregarding the context in (\ref{ohg_decl2}), the determiner \textit{einemo} may also agree with a neuter noun.}. The feature \spr indicates that the determiner requires a nominal expression lacking a specifier, i.e. \nom \citep{SaWaBe2003}. The index \ibox{1} signals the structure sharing of the determiner's \content feature with the \decl feature of the nominal expression. Determiner and nominal agree in the following example with respect to indefiniteness.
\ea 
\ms{
phon \phonliste{ einemo } \\
synsem|loc
\ms{
cat  
\ms{
  head & \ms[det]{
  agr & \ms{
  case & dat \\
  num & sg \\
  gend & masc
  } } \\ 
  spr & \sliste{ \nom {} [\decl {} \ibox{1} ] }
       } \\
content
\ms{
  det & exists \\
  restind \ibox{1}  
 }}}
\z 
In Present-day German, the \decl value of \nom no longer conveys information about its definiteness. The determiner selects for \nom according to categorial features: In case the determiner provides information on the \AGR value of the noun phrase, it asks for a weakly inflected \nom as in (\ref{decl-pdg}).\footnote{In my view, \decl is not a \head feature of the determiner, since  declension type is an inherent feature of determiners. Cf. however \cite[72]{kiss1995}.} Otherwise the information in question has to be provided by a strongly inflected \nom.\footnote{\cite[373]{PoSa94} point out that nouns like \textit{Verwandter} 'relative' support the assumption that \decl is a feature not only of adjectives but also of nouns, because the declension class of these nouns is governed by the respective determiner.} 
\ea \label{decl-pdg}
\ms{
phon \phonliste{ einem } \\
synsem|loc|cat
\ms{
  head & \ms[det]{
  agr & \ms{
  case & dat \\
  num & sg \\
  gend & masc
  } } \\ 
  spr & \sliste{ \nom {} [\decl {} \type{weak}] }
       } \\
}
\z 

The changing nature of the relationship between determiner and \nom can be modeled by modifying the feature structure of \spr in the lexical description of the determiner. A possible motivation for this change comes from the increasing grammaticalization of the definite determiner: While the determiner is attested above all with sortal concepts in Old High German testifying to its use as a demonstrative, it lacks in cases where the head noun refers to functional concepts that are inherently unambiguous \citep{demske2001}. The examples in (\ref{sortal}) illustrate this distribution: A head noun like \textit{figboum} 'figtree' has a sortal meaning, i.e. its unique referent is specified by the context in (\ref{sortal}a), while the noun \textit{erda} 'earth' denotes a functional concept which refers unambiguously irrespective of particular situations (\ref{sortal}b). A demonstrative determiner as \textit{ther} 'this one' is hence excluded in such a context.
\eal \label{sortal}
\ex Inti quad Imo, niomer fon thir uuahsmo arboran uuerde zi éiuuidu thô sâr sliumo arthorr\&{a} \textbf{ther figboum}. \hfill (OHG)
\glt 'and he saith unto it, Let there be no fruit from thee henceforward for ever. And immediately the fig tree withered away.'
\ex Inti \textbf{erda} giruorit uuas Inti steina gislizane uuarun \hfill (OHG)
\glt 'And the earth shook and the rocks were split.'  
\zl
The rise of weak definites in the course of Early New High German represents a further step in the grammaticalization process of the definite determiner: Relational nouns with their argument in postnominal position may lack any determiner (\ref{weakdef}a) or they exhibit either the indefinite or the definite determiner (\ref{weakdef}b, \ref{weakdef}c). The third pattern is the default pattern in Present-Day German \citep{demske2019}.
\eal \label{weakdef}
\ex Vnd wie wol ich bin \textbf{sone eins konigs} \hfill (ENHG)
\glt 'And though I am the son of a king.'
\ex Sie namen \textbf{einen Zahn eines Thiers} / welches so groß ist wie eine Ratte  \hfill (ENHG)
\glt 'They took the tooth of an animal which was as big as a rat.'
\ex ich verstunde gleich aus ihrem Diskurs (...) daß ihr Mann beim Senat wäre, und ohngezweifelte Hoffnung hätte, denselben Tag \textbf{die Stell eines Landvogts oder Landamtmanns} zu bekommen \hfill (ENHG)
\glt 'From her conversation I came to discover that her husband was in the senate (...) He was also supposed to have had good expectations of receiving the position of a district governor or a bailiff that very day.'
\zl
In Present-Day German, definite determiners are used with sortal, functional and relational nouns, cf. (\ref{def_PDG}), suggesting that the definite determiner is no longer licensed on semantic grounds as in Old High German, but on morphological grounds. The changing distribution of the definite determiner fits nicely the assumption that the specifier relation originally is semantically based and then turns into a morphologically based relationship.

\eal \label{def_PDG}
\ex Der Mond ist aufgegangen. \hfill (PDG)
\glt 'The moon has risen.'
\ex Sie ist die Tochter eines Unternehmers. \hfill (PDG)
\glt 'She is the daughter of an entrepreneur'  
\zl

The increasing grammaticalization of the nominal left periphery has further effects on pre-head constituents. The demonstrative pronoun \textit{solch} 'such' has either a sortal or an individual reading in older stages of German allowing for singular count nouns to occur without another determiner (\ref{such}). In Present-Day German, \textit{solch} is restricted to a sortal interpretation and  exhibits classical diagnostics for adjectivehood including the requirement of a determiner with singular count nouns \citep{demske2005}. The developing restrictions governing the use of the demonstrative pronoun \textit{solch} 'such' are easily modeled within a lexical approach, going on the assumption that there are either two lexical entries for the demonstrative in earlier stages of German or that the \head feature allows both a \spr as well as a \textsc{mod} relation between the demonstrative and the nominal head. With the development of the demonstrative \textit{ther} 'this one' into the definite determiner, the demonstrative \textit{solch} conventionalizes its use as a demonstrative adjective while losing the one as a determiner. 
\ea \label{such}
derselbe Landherr hatt an jetzo \textbf{solche Türckin} / weil jhr man nicht zur hand / sondern verreist gewest / wider vnder die Türcken vmb etliche Türckische Teppich vnd andere sachen verkaufft /  \hfill (ENHG) 
\glt 'this overlord has sold now this Turkish woman to other Turkish people for several Turkish tapestry and other things, because her husband has been abroad.' 
\z

A significant role in the history of the left periphery is played by the adnominal genitive. Three stages have to be distinguished in its development: In the first stage, genitive noun phrases systematically appear in prenominal position as attested in OHG sources (\ref{unanimate}). The genitive is a prenominal complement at this stage indicated by the preceding determiner and adjective as illustrated by the second example. Note that the determiner and the adjective are marked for dative case as required by the initial preposition \textit{in} 'in'.
\eal \label{unanimate}
\ex 
\gll scouuot thes accares lilia uuvo sie uuahsen \\ observe the field's lilies how they grow \\  \hfill (OHG)
\glt 'Observe how the lilies of the field grow.'
\ex 
\gll In dhemu heilegin daniheles chiscribe \\ in the holy Daniel's scripture  \\  \hfill (OHG)
\glt 'in the holy scripture of Daniel'
\zl
The adnominal genitive of stage two also occurs in prenominal position, provided that it denotes humans or animals, cf. (\ref{animate}). All adnominal genitives marked [-animate] are now limited to postnominal position as testified by historical data from the Early New High German period \citep{ebert88}. The prenominal genitive is still a full noun phrase in stage two, allowing not only for pre-head, but also for post-head dependents as shown by the postnominal modifiers in (\ref{animate}), which can be phrasal or sentential.
\eal \label{animate}
\ex \textbf{Der Frawen zu vnseren zeiten} kunst weyßheit vnd tugende ist nit not zu erzelen \hfill (ENHG)
\glt 'The art, wisdom and virtue of women in our days does not need to be recounted'
\ex Dieser Tagen seyn allhie \textbf{der Evangel. Fürsten vnnd Städt/ so zu Schwäbschen Hall jüngst beysamen gewest/} Abgesandte alher komen \hfill (ENHG)
\glt 'These days envoys of the Protestant sovereigns and cities have come here, after they have met at Schwäbisch Hall recently.'	
\zl
The final stage in the development of the adnominal genitive is represented by Present-day German: The prenominal position is restricted to proper names and kinship terms disallowing  any pre-head or post-head modifier. Genitive phrases headed by individual nouns appear in postnominal position irrespective of the feature [$\pm$animate].
\eal
\ex
\gll Selmas/Vaters altes Fahrrad \\ Selma's/Daddy's old bike\\
\ex
\gll *der Menschheit ältester Traum \\ \hspaceThis{$* $}the mankind's oldest dream \\
\zl

\eal
\ex
\gll das Gartenhaus des alten Goethe \\ the {summer house} {of the} old Goethe \\
\ex 
\gll *des alten Goethe Gartenhaus \\ \hspaceThis{$* $}the old Goethe {summer house} \\
\glt 'the summer house of late Goethe'
\zl

The historical scenario sketched for the development of the adnominal genitive fits in well with other changes affecting the left periphery of noun phrases: While the adnominal genitive is a full noun phrase in Old High German functioning as a complement of the head noun, the adnominal genitive is used as a possessive determiner in Present-day German. Again the change can be modeled as a change affecting the relation between a prenominal constituent and the nominal head. The Old High German demonstrative pronouns \textit{ther} 'this one' and \textit{sulîh} 'such' conventionalize either an individual or a sortal reading in the history of German. Patterning with the former, the prenominal genitive conventionalizes a determiner relation to the nominal, provided that it can contribute a possessive meaning. Adnominal genitives with the feature [-animate] are consequently postponed and they retain their grammatical function as complements of the head noun. Prenominal genitives on the other hand become possessive determiners   establishing a \spr relation to nominals with strong inflection \citep[54]{PoSa94}:
\ea 
\ms{
phon \phonliste{ Selmas } \\
synsem|loc
\ms{
cat  
\ms{
  head & det \\
  spr & \sliste{ \nom {} [\decl {} \type{strong}, \textsc{index} \ibox{1} , \textsc{restr} \ibox{2} ] }
       } \\
content
\ms{
  det & the \\
  restind &
  \ms{
  index \ibox{1} \\
  restr & \sliste{ 
  \ms{
  reln & poss \\
  possessor & \ibox{3} \\
  possessed & \ibox{1} 
 }}
 $\oplus$ \ibox{2}
 }} \\
context|backgr \sliste{
\ms{
reln & naming \\
bearer & \ibox{3} \\
name & Selma
}}}}
\z 
The reanalysis of the relation between prenominal possessive and nominal in the history of German not only affects the pre-head genitive but also the possessive pronoun. In Middle High German and still in Early New High German, the possessive pronoun patterns with adjectives considering its co-occurrence patterns: It may follow a definite determiner and may even agree with another adjective as regards its declension type as in (\ref{agreement}). Here possessive pronoun and adjective both exhibit weak declension triggered by the preceding definite determiner. At this stage in the history of German, the possessive pronoun ist still a constituent of the nominal. The Early New High German example in (\ref{postadjective}) shows that a possessive pronoun may also follow a prenominal adjective suggesting that it functions as a modifier itself (note also the agreement with respect to declencion type between adjective and possessive pronoun). This word order is excluded in Present-day German.
\eal
\ex \label{agreement}
\gll  die iuwer-n scoen-en tohter \\ the your-\textsc{weak} beautiful-\textsc{weak} daughter  \\ \hfill (MHG)
\glt 'your beautiful daughter'
\ex \label{postadjective}
\gll mit gross-em jhr-em Rhum vnd Lob \\ with big-\textsc{strong} their-\textsc{strong} glory and praise   \\ \hfill (ENHG)
\glt 'with their big glory and praise'
\zl
Once again, it is the relation between prenominal element and head noun which is subject to change: The possessive pronoun behaves as an adjectival modifier in earlier stages of German, before it becomes a possessive determiner (i.e. \textsc{mod} relation develops into \spr relation). According to \cite[54]{PoSa94}, its lexical description looks very much like the description of the prenominal genitive given above (disregarding the \content and the \textsc{context} value).

Changes at the left periphery of noun phrases start with the grammaticalization of the definite determiner throughout the period of Old High German, testified by a widening of its distribution. In Present-day German, not only sortal, but also functional and relational nouns combine with the definite determiner. The steady increase in the use of the determiner triggers a reanalysis of the relation between determiner and nominal: A semantically driven relation turns into a relation that is also morphologically based as evidenced by the changing trigger for the adjectival declension type. The determiner is consequently licensed by the \cat feature in the noun's feature structure as shown for the functional noun \textit{Mond} 'moon' which subcategorizes for its determiner in Present-day German but has not done so in Old High German.
\ea \label{noun}
\ms{
phon \phonliste{ Mond } \\
synsem|loc|cat
\ms{
  head & \ms[noun] {
  agr \ibox{1} 
  } \\
  spr & \sliste{  
  \ms{
   head & \ms[det]{
  agr & \ibox{1}
  }}} \\
  arg-st & \eliste
  }\\
}
\z 
The reanalysis of the relation between determiner and nominal has consequences for the interpretation of other prenominal constituents: Possessive pronouns as well as possessor phrases are likewise taken to instantiate a specifier relation to the nominal in question, thus augmenting the class of determiners in German. In addition, pre-head constituents precluding a specifier interpretation are postponed, cf. genitive complements with [-animate], and pre-head constituents ambiguous between a specifier and a modifier reading are limited to one interpretation, cf. \textit{solch} 'such' behaving as a demonstrative adjective in Present-day German. All changes can be modeled in a straightforward way as lexical changes.  

%%%%%%%%%%%%%%%%%%%%%%%%%%%%%%%%%%%%%%%%%
%%%%%%%%%%%%%%%%%%%%%%%%%%%%%%%%%%%%%%%%%%%
\section{Summary} 

Recent years witnessed a growing consensus that syntactic change is best accounted for in the lexicon of a language. The consensus holds across frameworks: \cite{BiWa2015} highlight the role of the lexicon in Minimalism, the volume by \cite{BuKi2001} exhibits case studies of syntactic change in the representational framework of \lfg. The present contribution set out to show how the typed feature structures of \hpsg can be used to model the way syntactic structures change over time. Different types of morpho-syntactic change have been considered in the history of German: the grammaticalization of auxiliary verbs and of demonstrative pronouns, word order changes affecting verb and noun phrases and changing relations between prenominal constituents and the respective nominal. In all cases, the change in question can be modeled in terms of feature structures in the lexicon of a language.  


%%%%%%%%%%%%%%%%%%%%%%%%%%%%%%%%%%%%
\section*{Abbreviations}
\section*{Acknowledgements}


\section*{Sources}

Include a list the historical sources used in the paper?

%%%%%%%%%%%%%%%%%%%%%%%%%%%%%%%%%%%%%%%%%%%%%%%
{\sloppy
\printbibliography[heading=subbibliography,notkeyword=this]
}



\end{document}
