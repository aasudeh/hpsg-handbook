%% -*- coding:utf-8 -*-

% Forest management
% \forestset{external/readonly}

\documentclass[output=paper
                ,modfonts
                ,nonflat
	        ,collection
	        ,collectionchapter
	        ,collectiontoclongg
 	        ,biblatex
                ,babelshorthands
                ,newtxmath
                ,draftmode
                ,colorlinks, citecolor=brown
]{./langsci/langscibook}


\IfFileExists{../localcommands.tex}{%hack to check whether this is being compiled as part of a collection or standalone
  \input{../localpackages}
  \input{../localcommands}
  \tikzset{external/prefix={hpsg-handbook.for.dir/}}
  \forestset{external/master dir={./}}
%  \tikzexternalize
  \input{../localhyphenation}
  \bibliography{../Bibliographies/stmue,
                ../localbibliography,
../Bibliographies/formal-background,
../Bibliographies/understudied-languages,
../Bibliographies/phonology,
../Bibliographies/case,
../Bibliographies/evolution,
../Bibliographies/agreement,
../Bibliographies/lexicon,
../Bibliographies/np,
../Bibliographies/negation,
../Bibliographies/argst,
../Bibliographies/binding,
../Bibliographies/complex-predicates,
../Bibliographies/coordination,
../Bibliographies/relative-clauses,
../Bibliographies/udc,
../Bibliographies/processing,
../Bibliographies/cl,
../Bibliographies/dg,
../Bibliographies/islands,
../Bibliographies/diachronic,
../Bibliographies/gesture,
../Bibliographies/semantics,
../Bibliographies/pragmatics,
../Bibliographies/information-structure,
../Bibliographies/idioms,
../Bibliographies/cg,
../Bibliographies/lfg,
../Bibliographies/udc,
collection.bib}

  \togglepaper[32]
}{}

\author{Stephen Wechsler\affiliation{The University of Texas}%
\lastand Ash Asudeh\affiliation{University of Rochester \& Carleton University}}
\title{HPSG and Lexical Functional Grammar}

%not needed \epigram{Here is the epigram:  more people have been to Berlin than I have}
\abstract{This chapter compares two closely related grammatical frameworks,
Head-Driven Phrase Structure Grammar (HPSG) and Lexical Functional
Grammar (LFG).   Among the similarities: both frameworks  draw a
lexicalist distinction between morphology and syntax; both frameworks
associate lexical argument structures with certain words; both employ semantic
theories based on underspecification; and both are fully explicit and
computationally implemented.   At the same time, each theory makes
available certain representational resources that the other lacks.
Many differences in the analyses proffered under the two frameworks
can be understood as consequences of the slightly different design
orientations underlying the frameworks: while HPSG focuses on
syntactic locality conditions,  LFG identifies functional equivalence
classes within grammatical structure, and across languages.  The
comparison of the two theories emphasizes contrasts between the two
that arise due to differing design principles.   After a point by
point syntactic comparison, we turn to an exposition of Glue
Semantics, a theory of semantic composition closely associated with LFG.}


\IfFileExists{../localcommands.tex}{
\input lfg-include.tex}{\input chapters/lfg-include.tex}
