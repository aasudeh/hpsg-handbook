\documentclass[output=paper
                ,modfonts
                ,nonflat
	        ,collection
	        ,collectionchapter
	        ,collectiontoclongg
 	        ,biblatex
                ,babelshorthands
                ,newtxmath
                ,draftmode
                ,colorlinks, citecolor=brown
]{./langsci/langscibook}

\IfFileExists{../localcommands.tex}{%hack to check whether this is being compiled as part of a collection or standalone
  % add all extra packages you need to load to this file 

\usepackage{graphicx}
\usepackage{tabularx}
\usepackage{amsmath} 
\usepackage{tipa}      % Davis Koenig
\usepackage{multicol}
\usepackage{lipsum}


\usepackage{./langsci/styles/langsci-optional} 
\usepackage{./langsci/styles/langsci-lgr}
%\usepackage{./styles/forest/forest}
\usepackage{./langsci/styles/langsci-forest-setup}
\usepackage{morewrites}

\usepackage{tikz-cd}

\usepackage{./styles/tikz-grid}
\usetikzlibrary{shadows}


%\usepackage{pgfplots} % for data/theory figure in minimalism.tex
% fix some issue with Mod https://tex.stackexchange.com/a/330076
\makeatletter
\let\pgfmathModX=\pgfmathMod@
\usepackage{pgfplots}%
\let\pgfmathMod@=\pgfmathModX
\makeatother

\usepackage{subcaption}

% Stefan Müller's styles
\usepackage{./styles/merkmalstruktur,german,./styles/makros.2e,./styles/my-xspace,./styles/article-ex,
./styles/eng-date}

\selectlanguage{USenglish}

\usepackage{./styles/abbrev}

\usepackage{./langsci/styles/jambox}

% Has to be loaded late since otherwise footnotes will not work

%%%%%%%%%%%%%%%%%%%%%%%%%%%%%%%%%%%%%%%%%%%%%%%%%%%%
%%%                                              %%%
%%%           Examples                           %%%
%%%                                              %%%
%%%%%%%%%%%%%%%%%%%%%%%%%%%%%%%%%%%%%%%%%%%%%%%%%%%%
% remove the percentage signs in the following lines
% if your book makes use of linguistic examples
\usepackage{./langsci/styles/langsci-gb4e} 

% Crossing out text
% uncomment when needed
%\usepackage{ulem}

\usepackage{./styles/additional-langsci-index-shortcuts}

%\usepackage{./langsci/styles/langsci-avm}
\usepackage{./styles/avm+}


\renewcommand{\tpv}[1]{{\avmjvalfont\itshape #1}}

% no small caps please
\renewcommand{\phonshape}[0]{\normalfont\itshape}

\regAvmFonts

\usepackage{theorem}

\newtheorem{mydefinition}{Def.}
\newtheorem{principle}{Principle}

{\theoremstyle{break}
%\newtheorem{schema}{Schema}
\newtheorem{mydefinition-break}[mydefinition]{Def.}
\newtheorem{principle-break}[principle]{Principle}
}

% This avoids linebreaks in the Schema
\newcounter{schema}
\newenvironment{schema}[1][]
  {% \begin{Beispiel}[<title>]
  \goodbreak%
  \refstepcounter{schema}%
  \begin{list}{}{\setlength{\labelwidth}{0pt}\setlength{\labelsep}{0pt}\setlength{\rightmargin}{0pt}\setlength{\leftmargin}{0pt}}%
    \item[{\textbf{Schema~\theschema}}]\hspace{.5em}\textbf{(#1)}\nopagebreak[4]\par\nobreak}%
  {\end{list}}% \end{Beispiel}

%% \newcommand{schema}[2]{
%% \begin{minipage}{\textwidth}
%% {\textbf{Schema~\theschema}}]\hspace{.5em}\textbf{(#1)}\\
%% #2
%% \end{minipage}}

%\usepackage{subfig}





% Davis Koenig Lexikon

\usepackage{tikz-qtree,tikz-qtree-compat} % Davis Koenig remove

\usepackage{shadow}




\usepackage[english]{isodate} % Andy Lücking
\usepackage[autostyle]{csquotes} % Andy
%\usepackage[autolanguage]{numprint}

%\defaultfontfeatures{
%    Path = /usr/local/texlive/2017/texmf-dist/fonts/opentype/public/fontawesome/ }

%% https://tex.stackexchange.com/a/316948/18561
%\defaultfontfeatures{Extension = .otf}% adds .otf to end of path when font loaded without ext parameter e.g. \newfontfamily{\FA}{FontAwesome} > \newfontfamily{\FA}{FontAwesome.otf}
%\usepackage{fontawesome} % Andy Lücking
\usepackage{pifont} % Andy Lücking -> hand

\usetikzlibrary{decorations.pathreplacing} % Andy Lücking
\usetikzlibrary{matrix} % Andy 
\usetikzlibrary{positioning} % Andy
\usepackage{tikz-3dplot} % Andy

% pragmatics
\usepackage{eqparbox} % Andy
\usepackage{enumitem} % Andy
\usepackage{longtable} % Andy
\usepackage{tabu} % Andy


% Manfred's packages

%\usepackage{shadow}

\usepackage{tabularx}
\newcolumntype{L}[1]{>{\raggedright\arraybackslash}p{#1}} % linksbündig mit Breitenangabe


% Jong-Bok

%\usepackage{xytree}

\newcommand{\xytree}[2][dummy]{Let's do the tree!}

% seems evil, get rid of it
% defines \ex is incompatible with gb4e
%\usepackage{lingmacros}

% taken from lingmacros:
\makeatletter
% \evnup is used to line up the enumsentence number and an entry along
% the top.  It can take an argument to improve lining up.
\def\evnup{\@ifnextchar[{\@evnup}{\@evnup[0pt]}}

\def\@evnup[#1]#2{\setbox1=\hbox{#2}%
\dimen1=\ht1 \advance\dimen1 by -.5\baselineskip%
\advance\dimen1 by -#1%
\leavevmode\lower\dimen1\box1}
\makeatother


% YK -- CG chapter

%\usepackage{xspace}
\usepackage{bm}
\usepackage{bussproofs}


% Antonio Branco, remove this
\usepackage{epsfig}

% now unicode
%\usepackage{alphabeta}



% Berthold udc
%\usepackage{qtree}
%\usepackage{rtrees}

\usepackage{pst-node}

  %add all your local new commands to this file

\makeatletter
\def\blx@maxline{77}
\makeatother


\newcommand{\page}{}



\newcommand{\todostefan}[1]{\todo[color=orange!80]{\footnotesize #1}\xspace}
\newcommand{\todosatz}[1]{\todo[color=red!40]{\footnotesize #1}\xspace}

\newcommand{\inlinetodostefan}[1]{\todo[color=green!40,inline]{\footnotesize #1}\xspace}


\newcommand{\spacebr}{\hspaceThis{[}}

\newcommand{\danish}{\jambox{(\ili{Danish})}}
\newcommand{\english}{\jambox{(\ili{English})}}
\newcommand{\german}{\jambox{(\ili{German})}}
\newcommand{\yiddish}{\jambox{(\ili{Yiddish})}}
\newcommand{\welsh}{\jambox{(\ili{Welsh})}}

% Cite and cross-reference other chapters
\newcommand{\crossrefchaptert}[2][]{\citet*[#1]{chapters/#2}, Chapter~\ref{chap-#2} of this volume} 
\newcommand{\crossrefchapterp}[2][]{(\citealp*[#1][]{chapters/#2}, Chapter~\ref{chap-#2} of this volume)}
% example of optional argument:
% \crossrefchapterp[for something, see:]{name}
% gives: (for something, see: Author 2018, Chapter~X of this volume)

\let\crossrefchapterw\crossrefchaptert



% Davis Koenig

\let\ig=\textsc
\let\tc=\textcolor

% evolution, Flickinger, Pollard, Wasow

\let\citeNP\citet

% Adam P

%\newcommand{\toappear}{Forthcoming}
\newcommand{\pg}[1]{p.#1}
\renewcommand{\implies}{\rightarrow}

\newcommand*{\rref}[1]{(\ref{#1})}
\newcommand*{\aref}[1]{(\ref{#1}a)}
\newcommand*{\bref}[1]{(\ref{#1}b)}
\newcommand*{\cref}[1]{(\ref{#1}c)}

\newcommand{\msadam}{.}
\newcommand{\morsyn}[1]{\textsc{#1}}

\newcommand{\nom}{\morsyn{nom}}
\newcommand{\acc}{\morsyn{acc}}
\newcommand{\dat}{\morsyn{dat}}
\newcommand{\gen}{\morsyn{gen}}
\newcommand{\ins}{\morsyn{ins}}
\newcommand{\loc}{\morsyn{loc}}
\newcommand{\voc}{\morsyn{voc}}
\newcommand{\ill}{\morsyn{ill}}
\renewcommand{\inf}{\morsyn{inf}}
\newcommand{\passprc}{\morsyn{passp}}

%\newcommand{\Nom}{\msadam\nom}
%\newcommand{\Acc}{\msadam\acc}
%\newcommand{\Dat}{\msadam\dat}
%\newcommand{\Gen}{\msadam\gen}
\newcommand{\Ins}{\msadam\ins}
\newcommand{\Loc}{\msadam\loc}
\newcommand{\Voc}{\msadam\voc}
\newcommand{\Ill}{\msadam\ill}
\newcommand{\INF}{\msadam\inf}
\newcommand{\PassP}{\msadam\passprc}

\newcommand{\Aux}{\textsc{aux}}

\newcommand{\princ}[1]{\textnormal{\textsc{#1}}} % for constraint names
\newcommand{\notion}[1]{\emph{#1}}
\renewcommand{\path}[1]{\textnormal{\textsc{#1}}}
\newcommand{\ftype}[1]{\textit{#1}}
\newcommand{\fftype}[1]{{\scriptsize\textit{#1}}}
\newcommand{\la}{$\langle$}
\newcommand{\ra}{$\rangle$}
%\newcommand{\argst}{\path{arg-st}}
\newcommand{\phtm}[1]{\setbox0=\hbox{#1}\hspace{\wd0}}
\newcommand{\prep}[1]{\setbox0=\hbox{#1}\hspace{-1\wd0}#1}

%%%%%%%%%%%%%%%%%%%%%%%%%%%%%%%%%%%%%%%%%%%%%%%%%%%%%%%%%%%%%%%%%%%%%%%%%%%

% FROM FS.STY:

%%%
%%% Feature structures
%%%

% \fs         To print a feature structure by itself, type for example
%             \fs{case:nom \\ person:P}
%             or (better, for true italics),
%             \fs{\it case:nom \\ \it person:P}
%
% \lfs        To print the same feature structure with the category
%             label N at the top, type:
%             \lfs{N}{\it case:nom \\ \it person:P}

%    Modified 1990 Dec 5 so that features are left aligned.
\newcommand{\fs}[1]%
{\mbox{\small%
$
\!
\left[
  \!\!
  \begin{tabular}{l}
    #1
  \end{tabular}
  \!\!
\right]
\!
$}}

%     Modified 1990 Dec 5 so that features are left aligned.
%\newcommand{\lfs}[2]
%   {
%     \mbox{$
%           \!\!
%           \begin{tabular}{c}
%           \it #1
%           \\
%           \mbox{\small%
%                 $
%                 \left[
%                 \!\!
%                 \it
%                 \begin{tabular}{l}
%                 #2
%                 \end{tabular}
%                 \!\!
%                 \right]
%                 $}
%           \end{tabular}
%           \!\!
%           $}
%   }

\newcommand{\ft}[2]{\path{#1}\hspace{1ex}\ftype{#2}}
\newcommand{\fsl}[2]{\fs{{\fftype{#1}} \\ #2}}

\newcommand{\fslt}[2]
 {\fst{
       {\fftype{#1}} \\
       #2 
     }
 }

\newcommand{\fsltt}[2]
 {\fstt{
       {\fftype{#1}} \\
       #2 
     }
 }

\newcommand{\fslttt}[2]
 {\fsttt{
       {\fftype{#1}} \\
       #2 
     }
 }


% jak \ft, \fs i \fsl tylko nieco ciasniejsze

\newcommand{\ftt}[2]
% {{\sc #1}\/{\rm #2}}
 {\textsc{#1}\/{\rm #2}}

\newcommand{\fst}[1]
  {
    \mbox{\small%
          $
          \left[
          \!\!\!
%          \sc
          \begin{tabular}{l} #1
          \end{tabular}
          \!\!\!\!\!\!\!
          \right]
          $
          }
   }

%\newcommand{\fslt}[2]
% {\fst{#2\\
%       {\scriptsize\it #1}
%      }
% }


% superciasne

\newcommand{\fstt}[1]
  {
    \mbox{\small%
          $
          \left[
          \!\!\!
%          \sc
          \begin{tabular}{l} #1
          \end{tabular}
          \!\!\!\!\!\!\!\!\!\!\!
          \right]
          $
          }
   }

%\newcommand{\fsltt}[2]
% {\fstt{#2\\
%       {\scriptsize\it #1}
%      }
% }

\newcommand{\fsttt}[1]
  {
    \mbox{\small%
          $
          \left[
          \!\!\!
%          \sc
          \begin{tabular}{l} #1
          \end{tabular}
          \!\!\!\!\!\!\!\!\!\!\!\!\!\!\!\!
          \right]
          $
          }
   }



% %add all your local new commands to this file

% \newcommand{\smiley}{:)}

% you are not supposed to mess with hardcore stuff, St.Mü. 22.08.2018
%% \renewbibmacro*{index:name}[5]{%
%%   \usebibmacro{index:entry}{#1}
%%     {\iffieldundef{usera}{}{\thefield{usera}\actualoperator}\mkbibindexname{#2}{#3}{#4}{#5}}}

% % \newcommand{\noop}[1]{}



% Rui

\newcommand{\spc}[0]{\hspace{-1pt}\underline{\hspace{6pt}}\,}
\newcommand{\spcs}[0]{\hspace{-1pt}\underline{\hspace{6pt}}\,\,}
\newcommand{\bad}[1]{\leavevmode\llap{#1}}
\newcommand{\COMMENT}[1]{}


% Andy Lücking gesture.tex
\newcommand{\Pointing}{\ding{43}}
% Giotto: "Meeting of Joachim and Anne at the Golden Gate" - 1305-10 
\definecolor{GoldenGate1}{rgb}{.13,.09,.13} % Dress of woman in black
\definecolor{GoldenGate2}{rgb}{.94,.94,.91} % Bridge
\definecolor{GoldenGate3}{rgb}{.06,.09,.22} % Blue sky
\definecolor{GoldenGate4}{rgb}{.94,.91,.87} % Dress of woman with shawl
\definecolor{GoldenGate5}{rgb}{.52,.26,.26} % Joachim's robe
\definecolor{GoldenGate6}{rgb}{.65,.35,.16} % Anne's robe
\definecolor{GoldenGate7}{rgb}{.91,.84,.42} % Joachim's halo

\makeatletter
\newcommand{\@Depth}{1} % x-dimension, to front
\newcommand{\@Height}{1} % z-dimension, up
\newcommand{\@Width}{1} % y-dimension, rightwards
%\GGS{<x-start>}{<y-start>}{<z-top>}{<z-bottom>}{<Farbe>}{<x-width>}{<y-depth>}{<opacity>}
\newcommand{\GGS}[9][]{%
\coordinate (O) at (#2-1,#3-1,#5);
\coordinate (A) at (#2-1,#3-1+#7,#5);
\coordinate (B) at (#2-1,#3-1+#7,#4);
\coordinate (C) at (#2-1,#3-1,#4);
\coordinate (D) at (#2-1+#8,#3-1,#5);
\coordinate (E) at (#2-1+#8,#3-1+#7,#5);
\coordinate (F) at (#2-1+#8,#3-1+#7,#4);
\coordinate (G) at (#2-1+#8,#3-1,#4);
\draw[draw=black, fill=#6, fill opacity=#9] (D) -- (E) -- (F) -- (G) -- cycle;% Front
\draw[draw=black, fill=#6, fill opacity=#9] (C) -- (B) -- (F) -- (G) -- cycle;% Top
\draw[draw=black, fill=#6, fill opacity=#9] (A) -- (B) -- (F) -- (E) -- cycle;% Right
}
\makeatother


% pragmatics
\newcommand{\speaking}[1]{\eqparbox{name}{\textsc{\lowercase{#1}\space}}}
\newcommand{\name}[1]{\eqparbox{name}{\textsc{\lowercase{#1}}}}
\newcommand{\HPSGTTR}{HPSG$_{\text{TTR}}$\xspace}

\newcommand{\ttrtype}[1]{\textit{#1}}
% \newcommand{\avmel}{\q<\quad\q>} %% shortcut for empty lists in AVM
\newcommand{\ttrmerge}{\ensuremath{\wedge_{\textit{merge}}}}
\newcommand{\Cat}[2][0.1pt]{%
  \begin{scope}[y=#1,x=#1,yscale=-1, inner sep=0pt, outer sep=0pt]
   \path[fill=#2,line join=miter,line cap=butt,even odd rule,line width=0.8pt]
  (151.3490,307.2045) -- (264.3490,307.2045) .. controls (264.3490,291.1410) and (263.2021,287.9545) .. (236.5990,287.9545) .. controls (240.8490,275.2045) and (258.1242,244.3581) .. (267.7240,244.3581) .. controls (276.2171,244.3581) and (286.3490,244.8259) .. (286.3490,264.2045) .. controls (286.3490,286.2045) and (323.3717,321.6755) .. (332.3490,307.2045) .. controls (345.7277,285.6390) and (309.3490,292.2151) .. (309.3490,240.2046) .. controls (309.3490,169.0514) and (350.8742,179.1807) .. (350.8742,139.2046) .. controls (350.8742,119.2045) and (345.3490,116.5037) .. (345.3490,102.2045) .. controls (345.3490,83.3070) and (361.9972,84.4036) .. (358.7581,68.7349) .. controls (356.5206,57.9117) and (354.7696,49.2320) .. (353.4652,36.1439) .. controls (352.5396,26.8573) and (352.2445,16.9594) .. (342.5985,17.3574) .. controls (331.2650,17.8250) and (326.9655,37.7742) .. (309.3490,39.2045) .. controls (291.7685,40.6320) and (276.7783,24.2380) .. (269.9740,26.5795) .. controls (263.2271,28.9013) and (265.3490,47.2045) .. (269.3490,60.2045) .. controls (275.6359,80.6368) and (289.3490,107.2045) .. (264.3490,111.2045) .. controls (239.3490,115.2045) and (196.3490,119.2045) .. (165.3490,160.2046) .. controls (134.3490,201.2046) and (135.4934,249.3212) .. (123.3490,264.2045) .. controls (82.5907,314.1553) and (40.8239,293.6463) .. (40.8239,335.2045) .. controls (40.8239,353.8102) and (72.3490,367.2045) .. (77.3490,361.2045) .. controls (82.3490,355.2045) and (34.8638,337.3259) .. (87.9955,316.2045) .. controls (133.3871,298.1601) and   (137.4391,294.4766) .. (151.3490,307.2045) -- cycle;
\end{scope}%
}


% KdK
\newcommand{\smiley}{:)}

\renewbibmacro*{index:name}[5]{%
  \usebibmacro{index:entry}{#1}
    {\iffieldundef{usera}{}{\thefield{usera}\actualoperator}\mkbibindexname{#2}{#3}{#4}{#5}}}

% \newcommand{\noop}[1]{}

% chngcntr.sty otherwise gives error that these are already defined
%\let\counterwithin\relax
%\let\counterwithout\relax

% the space of a left bracket for glossings
\newcommand{\LB}{\hspaceThis{[}}

\newcommand{\LF}{\mbox{$[\![$}}

\newcommand{\RF}{\mbox{$]\!]_F$}}

\newcommand{\RT}{\mbox{$]\!]_T$}}





% Manfred's

\newcommand{\kommentar}[1]{}

\newcommand{\bsp}[1]{\emph{#1}}
\newcommand{\bspT}[2]{\bsp{#1} `#2'}
\newcommand{\bspTL}[3]{\bsp{#1} (lit.: #2) `#3'}

\newcommand{\noidi}{§}

\newcommand{\refer}[1]{(\ref{#1})}

%\newcommand{\avmtype}[1]{\multicolumn{2}{l}{\type{#1}}}
\newcommand{\attr}[1]{\textsc{#1}}

\newcommand{\srdefault}{\mbox{\begin{tabular}{c}{\large <}\\[-1.5ex]$\sqcap$\end{tabular}}}

%% \newcommand{\myappcolumn}[2]{
%% \begin{minipage}[t]{#1}#2\end{minipage}
%% }

%% \newcommand{\appc}[1]{\myappcolumn{3.7cm}{#1}}


% Jong-Bok


% clean that up and do not use \def (killing other stuff defined before)
%\if 0
\def\DEL{\textsc{del}}
\def\del{\textsc{del}}

\def\conn{\textsc{conn}}
\def\CONN{\textsc{conn}}
\def\CONJ{\textsc{conj}}
\def\LITE{\textsc{lex}}
\def\lite{\textsc{lex}}
\def\HON{\textsc{hon}}

\def\CAUS{\textsc{caus}}
\def\PASS{\textsc{pass}}
\def\NPST{\textsc{npst}}
\def\COND{\textsc{cond}}



\def\hd-lite{\textsc{head-lex construction}}
\def\NFORM{\textsc{nform}}

\def\RELS{\textsc{rels}}
\def\TENSE{\textsc{tense}}


%\def\ARG{\textsc{arg}}
\def\ARGs{\textsc{arg0}}
\def\ARGa{\textsc{arg}}
\def\ARGb{\textsc{arg2}}
\def\TPC{\textsc{top}}
\def\PROG{\textsc{prog}}

\def\pst{\textsc{pst}}
\def\PAST{\textsc{pst}}
\def\DAT{\textsc{dat}}
\def\CONJ{\textsc{conj}}
\def\nominal{\textsc{nominal}}
\def\NOMINAL{\textsc{nominal}}
\def\VAL{\textsc{val}}
\def\val{\textsc{val}}
\def\MODE{\textsc{mode}}
\def\RESTR{\textsc{restr}}
\def\SIT{\textsc{sit}}
\def\ARG{\textsc{arg}}
\def\RELN{\textsc{rel}}
\def\REL{\textsc{rel}}
\def\RELS{\textsc{rels}}
\def\arg-st{\textsc{arg-st}}
\def\xdel{\textsc{xdel}}
\def\zdel{\textsc{zdel}}
\def\sug{\textsc{sug}}
\def\IMP{\textsc{imp}}
\def\conn{\textsc{conn}}
\def\CONJ{\textsc{conj}}
\def\HON{\textsc{hon}}
\def\BN{\textsc{bn}}
\def\bn{\textsc{bn}}
\def\pres{\textsc{pres}}
\def\PRES{\textsc{pres}}
\def\prs{\textsc{pres}}
\def\PRS{\textsc{pres}}
\def\agt{\textsc{agt}}
\def\DEL{\textsc{del}}
\def\PRED{\textsc{pred}}
\def\AGENT{\textsc{agent}}
\def\THEME{\textsc{theme}}
\def\AUX{\textsc{aux}}
\def\THEME{\textsc{theme}}
\def\PL{\textsc{pl}}
\def\SRC{\textsc{src}}
\def\src{\textsc{src}}
\def\FORM{\textsc{form}}
\def\form{\textsc{form}}
\def\GCASE{\textsc{gcase}}
\def\gcase{\textsc{gcase}}
\def\SCASE{\textsc{scase}}
\def\PHON{\textsc{phon}}
\def\SS{\textsc{ss}}
\def\SYN{\textsc{syn}}
\def\LOC{\textsc{loc}}
\def\MOD{\textsc{mod}}
\def\INV{\textsc{inv}}
\def\L{\textsc{l}}
\def\CASE{\textsc{case}}
\def\SPR{\textsc{spr}}
\def\COMPS{\textsc{comps}}
%\def\comps{\textsc{comps}}
\def\SEM{\textsc{sem}}
\def\CONT{\textsc{cont}}
\def\SUBCAT{\textsc{subcat}}
\def\CAT{\textsc{cat}}
\def\C{\textsc{c}}
\def\SUBJ{\textsc{subj}}
\def\subj{\textsc{subj}}
\def\SLASH{\textsc{slash}}
\def\LOCAL{\textsc{local}}
\def\ARG-ST{\textsc{arg-st}}
\def\AGR{\textsc{agr}}
\def\PER{\textsc{per}}
\def\NUM{\textsc{num}}
\def\IND{\textsc{ind}}
\def\VFORM{\textsc{vform}}
\def\PFORM{\textsc{pform}}
\def\decl{\textsc{decl}}
\def\loc{\textsc{loc   }}
% \def\   {\textsc{  }}

\def\NEG{\textsc{neg}}
\def\FRAMES{\textsc{frames}}
\def\REFL{\textsc{refl}}

\def\MKG{\textsc{mkg}}

\def\BN{\textsc{bn}}
\def\HD{\textsc{hd}}
\def\NP{\textsc{np}}
\def\PF{\textsc{pf}}
\def\PL{\textsc{pl}}
\def\PP{\textsc{pp}}
\def\SS{\textsc{ss}}
\def\VF{\textsc{vf}}
\def\VP{\textsc{vp}}
\def\bn{\textsc{bn}}
\def\cl{\textsc{cl}}
\def\pl{\textsc{pl}}
\def\Wh{\ital{Wh}}
\def\ng{\textsc{neg}}
\def\wh{\ital{wh}}
\def\ACC{\textsc{acc}}
\def\AGR{\textsc{agr}}
\def\AGT{\textsc{agt}}
\def\ARC{\textsc{arc}}
\def\ARG{\textsc{arg}}
\def\ARP{\textsc{arc}}
\def\AUX{\textsc{aux}}
\def\CAT{\textsc{cat}}
\def\COP{\textsc{cop}}
\def\DAT{\textsc{dat}}
\def\DEF{\textsc{def}}
\def\DEL{\textsc{del}}
\def\DOM{\textsc{dom}}
\def\DTR{\textsc{dtr}}
\def\FUT{\textsc{fut}}
\def\GAP{\textsc{gap}}
\def\GEN{\textsc{gen}}
\def\HON{\textsc{hon}}
\def\IMP{\textsc{imp}}
\def\IND{\textsc{ind}}
\def\INV{\textsc{inv}}
\def\LEX{\textsc{lex}}
\def\Lex{\textsc{lex}}
\def\LOC{\textsc{loc}}
\def\MOD{\textsc{mod}}
\def\MRK{{\nr MRK}}
\def\NEG{\textsc{neg}}
\def\NEW{\textsc{new}}
\def\NOM{\textsc{nom}}
\def\NUM{\textsc{num}}
\def\PER{\textsc{per}}
\def\PST{\textsc{pst}}
\def\QUE{\textsc{que}}
\def\REL{\textsc{rel}}
\def\SEL{\textsc{sel}}
\def\SEM{\textsc{sem}}
\def\SIT{\textsc{arg0}}
\def\SPR{\textsc{spr}}
\def\SRC{\textsc{src}}
\def\SUG{\textsc{sug}}
\def\SYN{\textsc{syn}}
\def\TPC{\textsc{top}}
\def\VAL{\textsc{val}}
\def\acc{\textsc{acc}}
\def\agt{\textsc{agt}}
\def\cop{\textsc{cop}}
\def\dat{\textsc{dat}}
\def\foc{\textsc{focus}}
\def\FOC{\textsc{focus}}
\def\fut{\textsc{fut}}
\def\hon{\textsc{hon}}
\def\imp{\textsc{imp}}
\def\kes{\textsc{kes}}
\def\lex{\textsc{lex}}
\def\loc{\textsc{loc}}
\def\mrk{{\nr MRK}}
\def\nom{\textsc{nom}}
\def\num{\textsc{num}}
\def\plu{\textsc{plu}}
\def\pne{\textsc{pne}}
\def\pst{\textsc{pst}}
\def\pur{\textsc{pur}}
\def\que{\textsc{que}}
\def\src{\textsc{src}}
\def\sug{\textsc{sug}}
\def\tpc{\textsc{top}}
\def\utt{\textsc{utt}}
\def\val{\textsc{val}}
\def\LITE{\textsc{lex}}
\def\PAST{\textsc{pst}}
\def\POSP{\textsc{pos}}
\def\PRS{\textsc{pres}}
\def\mod{\textsc{mod}}%
\def\newuse{{`kes'}}
\def\posp{\textsc{pos}}
\def\prs{\textsc{pres}}
\def\psp{{\it en\/}}
\def\skes{\textsc{kes}}
\def\CASE{\textsc{case}}
\def\CASE{\textsc{case}}
\def\COMP{\textsc{comp}}
\def\CONJ{\textsc{conj}}
\def\CONN{\textsc{conn}}
\def\CONT{\textsc{cont}}
\def\DECL{\textsc{decl}}
\def\FOCUS{\textsc{focus}}
\def\FORM{\textsc{form}}
\def\FREL{\textsc{frel}}
\def\GOAL{\textsc{goal}}
\def\HEAD{\textsc{head}}
\def\INDEX{\textsc{ind}}
\def\INST{\textsc{inst}}
\def\MODE{\textsc{mode}}
\def\MOOD{\textsc{mood}}
\def\NMLZ{\textsc{nmlz}}
\def\PHON{\textsc{phon}}
\def\PRED{\textsc{pred}}
%\def\PRES{\textsc{pres}}
\def\PROM{\textsc{prom}}
\def\RELN{\textsc{pred}}
\def\RELS{\textsc{rels}}
\def\STEM{\textsc{stem}}
\def\SUBJ{\textsc{subj}}
\def\XARG{\textsc{xarg}}
\def\bse{{\it bse\/}}
\def\case{\textsc{case}}
\def\caus{\textsc{caus}}
\def\comp{\textsc{comp}}
\def\conj{\textsc{conj}}
\def\conn{\textsc{conn}}
\def\decl{\textsc{decl}}
\def\fin{{\it fin\/}}
\def\form{\textsc{form}}
\def\gend{\textsc{gend}}
\def\inf{{\it inf\/}}
\def\mood{\textsc{mood}}
\def\nmlz{\textsc{nmlz}}
\def\pass{\textsc{pass}}
\def\past{\textsc{past}}
\def\perf{\textsc{perf}}
\def\pln{{\it pln\/}}
\def\pred{\textsc{pred}}


%\def\pres{\textsc{pres}}
\def\proc{\textsc{proc}}
\def\nonfin{{\it nonfin\/}}
\def\AGENT{\textsc{agent}}
\def\CFORM{\textsc{cform}}
%\def\COMPS{\textsc{comps}}
\def\COORD{\textsc{coord}}
\def\COUNT{\textsc{count}}
\def\EXTRA{\textsc{extra}}
\def\GCASE{\textsc{gcase}}
\def\GIVEN{\textsc{given}}
\def\LOCAL{\textsc{local}}
\def\NFORM{\textsc{nform}}
\def\PFORM{\textsc{pform}}
\def\SCASE{\textsc{scase}}
\def\SLASH{\textsc{slash}}
\def\SLASH{\textsc{slash}}
\def\THEME{\textsc{theme}}
\def\TOPIC{\textsc{topic}}
\def\VFORM{\textsc{vform}}
\def\cause{\textsc{cause}}
%\def\comps{\textsc{comps}}
\def\gcase{\textsc{gcase}}
\def\itkes{{\it kes\/}}
\def\pass{{\it pass\/}}
\def\vform{\textsc{vform}}
\def\CCONT{\textsc{c-cont}}
\def\GN{\textsc{given-new}}
\def\INFO{\textsc{info-st}}
\def\ARG-ST{\textsc{arg-st}}
\def\SUBCAT{\textsc{subcat}}
\def\SYNSEM{\textsc{synsem}}
\def\VERBAL{\textsc{verbal}}
\def\arg-st{\textsc{arg-st}}
\def\plain{{\it plain}\/}
\def\propos{\textsc{propos}}
\def\ADVERBIAL{\textsc{advl}}
\def\HIGHLIGHT{\textsc{prom}}
\def\NOMINAL{\textsc{nominal}}

\newenvironment{myavm}{\begingroup\avmvskip{.1ex}
  \selectfont\begin{avm}}%
{\end{avm}\endgroup\medskip}
\def\pfix{\vspace{-5pt}}


\def\jbsub#1{\lower4pt\hbox{\small #1}}
\def\jbssub#1{\lower4pt\hbox{\small #1}}
\def\jbtr{\underbar{\ \ \ }\ }


%\fi

  %% hyphenation points for line breaks
%% Normally, automatic hyphenation in LaTeX is very good
%% If a word is mis-hyphenated, add it to this file
%%
%% add information to TeX file before \begin{document} with:
%% %% hyphenation points for line breaks
%% Normally, automatic hyphenation in LaTeX is very good
%% If a word is mis-hyphenated, add it to this file
%%
%% add information to TeX file before \begin{document} with:
%% %% hyphenation points for line breaks
%% Normally, automatic hyphenation in LaTeX is very good
%% If a word is mis-hyphenated, add it to this file
%%
%% add information to TeX file before \begin{document} with:
%% \include{localhyphenation}
\hyphenation{
A-la-hver-dzhie-va
anaph-o-ra
affri-ca-te
affri-ca-tes
Atha-bas-kan
Chi-che-ŵa
com-ple-ments
Da-ge-stan
Dor-drecht
er-klä-ren-de
Ginz-burg
Gro-ning-en
Jon-a-than
Ka-tho-lie-ke
Ko-bon
krie-gen
Le-Sourd
moth-er
Mül-ler
Nie-mey-er
Prze-piór-kow-ski
phe-nom-e-non
re-nowned
Rie-he-mann
un-bound-ed
}

% why has "erklärende" be listed here? I specified langid in bibtex item. Something is still not working with hyphenation.


% to do: check
%  Alahverdzhieva

\hyphenation{
A-la-hver-dzhie-va
anaph-o-ra
affri-ca-te
affri-ca-tes
Atha-bas-kan
Chi-che-ŵa
com-ple-ments
Da-ge-stan
Dor-drecht
er-klä-ren-de
Ginz-burg
Gro-ning-en
Jon-a-than
Ka-tho-lie-ke
Ko-bon
krie-gen
Le-Sourd
moth-er
Mül-ler
Nie-mey-er
Prze-piór-kow-ski
phe-nom-e-non
re-nowned
Rie-he-mann
un-bound-ed
}

% why has "erklärende" be listed here? I specified langid in bibtex item. Something is still not working with hyphenation.


% to do: check
%  Alahverdzhieva

\hyphenation{
A-la-hver-dzhie-va
anaph-o-ra
affri-ca-te
affri-ca-tes
Atha-bas-kan
Chi-che-ŵa
com-ple-ments
Da-ge-stan
Dor-drecht
er-klä-ren-de
Ginz-burg
Gro-ning-en
Jon-a-than
Ka-tho-lie-ke
Ko-bon
krie-gen
Le-Sourd
moth-er
Mül-ler
Nie-mey-er
Prze-piór-kow-ski
phe-nom-e-non
re-nowned
Rie-he-mann
un-bound-ed
}

% why has "erklärende" be listed here? I specified langid in bibtex item. Something is still not working with hyphenation.


% to do: check
%  Alahverdzhieva

  \bibliography{../Bibliographies/stmue,
                ../localbibliography,
../Bibliographies/formal-background,
../Bibliographies/understudied-languages,
../Bibliographies/phonology,
../Bibliographies/case,
../Bibliographies/evolution,
../Bibliographies/agreement,
../Bibliographies/lexicon,
../Bibliographies/np,
../Bibliographies/negation,
../Bibliographies/argst,
../Bibliographies/binding,
../Bibliographies/complex-predicates,
../Bibliographies/coordination,
../Bibliographies/relative-clauses,
../Bibliographies/udc,
../Bibliographies/processing,
../Bibliographies/cl,
../Bibliographies/dg,
../Bibliographies/islands,
../Bibliographies/diachronic,
../Bibliographies/gesture,
../Bibliographies/semantics,
../Bibliographies/pragmatics,
../Bibliographies/information-structure,
../Bibliographies/idioms,
../Bibliographies/cg,
../Bibliographies/udc,
collection.bib}

  \togglepaper[17]
}{}


\author{Anne Abeillé\affiliation{Laboratoire de Linguistique Formelle, University of Paris} \lastand Rui P. Chaves\affiliation{Linguistics Department, University at Buffalo, The State University of New York}}
\title{Coordination} 



% \chapterDOI{} %will be filled in at production

%\epigram{Change epigram in chapters/03.tex or remove it there }

\abstract{Coordination is a central topic in theoretical linguistics. Following GPSG, which provided the first formal analysis of unlike coordination, HPSG has developed detailed analyses of different coordination constructions in a variety of unrelated languages. Central to the HPSG analyses are two main ideas: (i) coordination structures are non-headed phrases, (ii) coordinate daughters display some kind of parallelism, which is captured by feature sharing; from these ideas, specific properties can be derived, regarding extraction and agreement for instance. Many HPSG analyses also agree that coordination is a cover term for a wide variety of  different constructions which can be viewed as different subtypes of coordinate phrases, which can be cross-classified with other subtypes of the grammar (nominal or not, with ellipsis or not etc.) We present the description of various coordination phenomena and show that HPSG can account for their subtle properties, while integrating them in the general organization of the grammar.}



\begin{document}
\maketitle
\label{chap-coordination}

{
\avmoptions{inactive}

%\if0

\section{Introduction} 

A great deal of research has been dedicated to the topic of coordination structures in the last  70 years, spanning a multitude of different approaches in many different theoretical frameworks.  With regard to the linguistic problems, research questions abound. In the realm of syntax there is much debate concerning the role of coordination lexemes, the existence of null coordinators, the syntactic relationship between conjuncts, the peculiar extraction phenomena that certain coordination structures exhibit, the necessary properties that allow two different structures to be coordinated, the relation between coordination structures and comparative and subordination structures, peculiar ellipsis phenomena that can optionally occur, the various patterns of agreement that obtain in nominal coordination structures, the distribution and syntactic realization of the lexemes \emph{either} and \emph{or}, etc. In the realm of semantics, the issues are no less complex, and the debate no less lively. There are many questions pertaining to how exactly the meaning of coordination structures is construed. 

Among the first attempts to offer a precise formalization of the syntax and semantics of coordination was the seminal work of \citet{gazdarc}. Other seminal work soon followed, including the demonstration that phrase structure grammar offered a way to model filler-gap dependencies and certain island constraints \citep{gazdar}. In particular, Gazdar's account showed how long-distance dependencies involving multiple gaps linked to the same filler phrase could be modeled straightforwardly, something that mainstream movement-based models still struggle with to this day. Finally, there was also 
 in-depth examinations of a number of complex empirical phenomena, i.e.\ in  \citet{gazd1982}, which  proved highly influential in the genesis of Generalized Phrase-Structure Grammar, and later, of HPSG. Coordination thus has a special place in the history of HPSG, and still figures  in many theoretical arguments within generative grammar  given the extremely challenging phenomena it poses for linguistic theory. 
Nevertheless, there is no clear consensus, even within HPSG, about how to analyze coordination. For example, in some accounts the coordinator
expression is a weak head, whereas in others it is a marker. Coordinate structures are binary branching in some accounts but not so in others. Finally, in  some accounts non-constituent coordination involves some form of deletion, but in others no deletion operation is assumed.  
In this chapter we survey the empirical arguments and formal accounts of coordination, with special focus on its morphosyntax.

\section{Headedness}

The head of a construction is traditionally defined as the constituent which determines the syntactic distribution and the meaning of the whole, and it also often the case that a dependent can be omitted, fronted, or extraposed while the head cannot \citep{zwicky85}. In coordination constructions something very different occurs. First, the syntactic category and the distribution of a coordinate phrase is collectively determined by the conjuncts, not by one particular conjunct nor by the coordination particle. Thus, an S coordination yields an S, a VP coordination yields a VP and so on, for virtually all categories.\footnote{The exceptions include coordinator expressions themselves, e.g.\ \emph{*You ordered a coffee and or or a tea?} The oddness of the former is presumably due to the fact that expletives are devoid of any meaning
\citep{MuellerGT-Eng1}, and the oddness of the latter may be due to the conjuncts being of the wrong semantic type. See Section \ref{lexcoord} for more on lexical coordination.}
This is perhaps clearer in cases like (\ref{c0}), where
expressions such as \emph{simultaneously}, \emph{both}, 
\emph{together} can be used to show that the entire bracketed string
is interpreted as a complex unit denoting a plurality.


\begin{exe}
\ex
\begin{xlista}
\ex{} [\subl{s} [\subl{s} Tom sang] and [\subl{s} Mia danced]] simultaneously.

\ex{} Often [\subl{vp} [\subl{vp}Kim goes to the beach] and [\subl{vp}Sue goes to the city]].

\ex{} Sue [\subl{vp} [\subl{vp} read the instructions] and [\subl{vp} dried her hair]], in twenty seconds.

\ex{} You can't simultaneously [\subl{vp} [\subl{vp} drive a car] and [\subl{vp} talk on the phone]].

\ex{} Simultaneously [\subl{vp} [\subl{vp} shocked] and [\subl{vp} saddened]], Robin decided to go home.

\ex Robin is both [\subl{a} [\subl{a} tall] and [\subl{a} thin]].

\ex{} [\subl{np} [\subl{np} Tom] and [\subl{np} Mia]] agreed to jump into the water together.

\end{xlista}\label{c0}
\end{exe}


Generally, a coordinate structure has the same grammatical function and category as the conjuncts: given a number of conjuncts of category $X$, the distribution of the coordinate constituent that is obtained is again the same as of an $X$ constituent, what \citet{pullumzwicky} refer as \isi{Wasow's Generalization}.
In particular, this is what allows coordination to apply recursively:

\begin{exe}
\ex
\begin{xlista}
\ex {}[[Tom and Mary]\subl{np} or [Mia and Sue]\subl{np}]\subl{np} got married.
\ex I can either [[sing and dance]\subl{vp} or [sing and play the guitar]\subl{vp}]\subl{vp}.
\ex Either [[John went to Paris and Kim went to Brussels]\subl{s} or
[none of them ever left home]\subl{s}]\subl{s}.
\end{xlista}
\end{exe}

Another piece of evidence in favor of a non"=headed analysis comes from the fact that there is no typological correlation between the position of the coordinator and the head directionality \citep{zwart}. For example, in Zwart's  survey of 136 languages where half are verb-final and half
are verb-initial languages,  verb-final languages overwhelmingly employ initial conjunction strategies.
In particular, 119 of these languages have exclusively initial conjunctions, 12 languages exhibit both initial
and final conjunctions, and only 4 have exclusively final conjunctions. 


Finally, Coordination is also special in that the relationship between conjuncts is unlike adjunction \citep{levinepostal}.
Whereas adjuncts can in principle be displaced, conjuncts do not have any mobility, as (\ref{c2}) illustrates.

\begin{exe}
\ex
\begin{xlista}
\ex[] {Because/Since Jane likes music, Tom learned to play the piano.}
\ex[*] {And Jane likes music, Tom learned to play the piano.}
\end{xlista}\label{c2}
\end{exe}


\noindent
Thus, no conjunct can usually be said to be a dependent. For example,  reversing the order of the conjuncts in (\ref{c1}) causes no major change in meaning. Neither daughter can be said to be the head because no subordination dependency is established between conjuncts.

\begin{exe}
\ex
\begin{xlista}
\ex Sam ordered a burger and Robin ordered a pizza.
\ex Robin ordered a pizza and Sam ordered a burger.
\end{xlista}\label{c1}
\end{exe}

\noindent
To be sure, there are certain coordination structures which do not have such symmetric 
interpretations, as noted by \citet{ross67}; see also
\citet{goldsmith}, \citet{lakoff86}, and \citet{levinprince86}.
Regardless, such constructions retain many of the properties that characterize coordinate structures, and therefore are likely 
coordinate just the same \citep[Chapter 5]{kehler}.

\begin{exe}
\ex
\begin{xlista}
\ex Robin jumped on a horse and rode into the sunset.
\ex Robin rode into the sunset and jumped on a horse.
\end{xlista}
\end{exe}
%the b example could be marked as ? or #

For these reasons, HPSG adopts a rather traditional non"=headed analysis of coordination, an approach  going back
to \citet[195]{bloom} and  \citet{ross67}, and later adopted in many other frameworks such as \citet{pesetsky}, \citet{gazdarc},  \citet[1275]{rodney}, among many others. 
See \citet{borsley94}, \citet{Borsley:05} and 
\citet[Chaves 2]{chavesthesis} for more discussion about previous claims in the literature that coordination structures are headed.
Finally, we note that the HPSG account is in agreement with \citet[196]{chom65}, who argued against postulating complex syntactic representations without direct empirical evidence:\footnote{In more recent times Chomskyan theorizing has assumed that all structures should be binary branching purely on conceptual economy grounds; see \citet{Johnson:Lappin:99} for criticism.}

\begin{quote}
It has sometimes been claimed that the traditional coordinate structures are necessarily right-recursive (Yngve, 1960) or left-recursive (Harman, 1963, p.\,613, rule 3i). These conclusions seem to me equally unacceptable. Thus to assume (with Harman) that the phrase “a tall, young, handsome, intelligent man” has the structure [[[[tall young] handsome] intelligent] man] seems to me no more justifiable than to assume that it has
the structure [tall [young [handsome [intelligent man]]]]. In fact, there is no grammatical
motivation for any internal structure, [\ldots] The burden of proof rests on one who claims additional
structure beyond this \citep[196--197]{chom65}
\end{quote}

\noindent
As we shall see, the empirical evidence suggests that
the simplest and most parsimonious structure for coordination is neither left- nor right-recursive.


\section{On the Syntax of Coordinate Structures}

In this chapter we refer to expressions like \emph{and}, \emph{either},  \emph{or}, \emph{but}, 
\emph{let alone}, etc.\ as \emph{coordinators}\is{Coordinators} and the phrases that a coordinator can combine with as  \emph{coordinands}\is{Coordinands}.
Thus,  in ``A or B'' both A and B are coordinands and \emph{or} is the coordinator. 

There are a wide range of coordination strategies in the languages of the world \citet{haspelmath}. In some languages no coordinand is accompanied by any coordinator (syndenton coordination; as in \emph{We came, we saw, we conquered}), or one of the conjuncts is accompanied by a coordinator (monosyndenton coordination,  as in \emph{We came, we saw, and we conquered}. Other strategies involve marking multiple coordinands with a coordinator (polysindenton coordination;
\emph{We came, and we saw, and we conquered}, or all coordinands (omnysyndenton coordination;
\emph{Either you come or you go}.
All of these are schematically depicted in (\ref{types}); see
 \citet{Drellishak:Bender:05} for more discussion about how to accommodate such typological patterns in a computational HPSG platform.

\eal
\label{types}
\settowidth\jamwidth{(monosyndenton)}
\ex A, B, C \jambox{(asyndenton)}
\ex A, B, \emph{coord} C \jambox{(monosyndenton)}
\ex A \emph{coord} B \emph{coord} C \jambox{(polysyndenton)}
\ex \emph{coord} A \emph{coord} B \emph{coord} C \jambox{(omnisyndenton)}
\zl


\noindent
 Finally, a single coordination strategy often serves to coordinate all types of constituent phrases, but in many languages different coordination strategies only cover a subset of the types of phrases in the language. For example, in
Japanese the suffix \emph{to} is used for nominal coordination
and \emph{te} is used for other coordinations.

In what follows we start by focusing on monosyndenton coordination. There are three possible structures one can assign to such coordinations, as Figure~\ref{f1} illustrates. The binary branching approach goes back to \citet{yngve}, and is used in HPSG work such as
\citet{pollardsag}, \citet{Yatabe:03}, \citet{berthold03},
\citet{Beavers}, \citet{Drellishak:Bender:05}, \citet{Abeille:05}, and
\ \citet{chavesthesis}, \citet{chavesextr}, among others.\addpages The flat branching approach has also been  assumed in HPSG, albeit less frequently. See for example
 \citet{sagwasowbender} and  \citet{Sag:03}.\footnote{See \citet{Borsley:05} for criticism of 
ConjP and of binary branching analysis of coordinate structures with three conjuncts.}

 
\begin{figure}
\hfill
    \Tree[.X X [.X X [.X {Coord}  X ] ] ]
\hfill
    \Tree[.X X  X  [.X {Coord} X ] ]
\hfill
    \Tree[.X X  X {Coord}  X ] 
\hfill\mbox{}
\caption{Three possible headless analyses of coordination}\label{f1}
\end{figure}


The binary branching analysis requires two different rules, informally depicted in (\ref{bin}), and a special feature to prevent the coordinator to recursively apply to the last coordinand, e.g.\ *\emph{Robin and and and Kim}. Otherwise, the two rules are unremarkable and are handled by the grammar like any other immediate dominance schema. See for example \citet{Beavers}
for a formalization.

\begin{exe}
\ex
\begin{xlista}
\ex X$_{crd+}$ $\rightarrow$ \, \emph{Coord} \, X$_{crd-}$
 
\ex X $\rightarrow$ \, X$_{crd-}$  \,\, X$_{crd+}$
\end{xlista}\label{bin}
\end{exe}

\noindent
\citet{Kayne:94}\addpages and  \citet{johann}  argue that coordination follows X-bar theory and that the conjunction is the head of the construction; see \crossrefchaptert{minimalism}. But in HPSG, even though one of the conjuncts (or more) may combine with a conjunction, this subconstituent is not the head of the construction, which is considered as unheaded.
The two analyses are contrasted in Figure~\ref{f10}.

\begin{figure}
\hfill
    \Tree[.ConjP NP1 [.Conj$'$  Coord NP2 ] ]
\hfill
    \Tree[.NP NP1 [.NP  Coord NP2 ] ]
\hfill\mbox{}
\caption{Binary-branching analyses of coordination, headed and non"=headed}\label{f10}
\end{figure}


Similarly, the flat analysis where the coordinator and the coordinand attach to each other  requires two  
rules as well (where $n \geq 1$):

\begin{exe}
\ex
\begin{xlista}
\ex X$_{crd+}$ $\rightarrow$ \, \emph{Coord} \, X$_{crd-}$
 
\ex X $\rightarrow$ \, X$^1_{crd-}$  \,\, \ldots \,\, X$^n_{crd-}$ \,\, X$_{crd+}$
\end{xlista}
\end{exe}\label{ok}

\noindent
However, the flat analysis requires only one rule, and no
special features at all, as (\ref{flat}) illustrates. 

\begin{exe}
\ex X  $\rightarrow$ X$^1$ \ldots{} X$^n$ \emph{Coord} \, X$_{n+1}$
\end{exe}\label{flat}

That said, there are some reasons for assuming that the coordinator does in fact combine with the coordinand, as in (\ref{ok}). First, in some  languages of the world the coordinator is a bound morpheme instead of a free morpheme. For example,  verbs are coordinated by adding one of a set of suffixes to one of the coordinands in \ili{Abelam} (Papua New Guinea),  usually the prior one.  Similarly, in Kanuri (Nilo-Saharan), verb phrases are coordinated by marking the first verb with a conjunctive form affix, and
 in languages like Telugu (Dravidian), the coordination of proper names is marked  by the lengthening of their final vowels \citep{Bender05}. The latter is illustrated in (\ref{telugu}).

\begin{exe}
\ex \gll kamalaa wimalaa poDugu \\ 
 Kamala Vimala tall\\
\glt `Kamala and Vimala are tall'\label{telugu}
\end{exe}



Second, as \citet[165]{ross67} originally noted, the natural intonation break occurs before the coordination lexeme rather than between the coordinator and the coordinand, so that a  prosodic constituent is formed.
Although prosodic phrasing is not generally believed to always align with syntactic phrasing, the fact that the coordinator prosodifies with the  coordinand suggests that it forms a unit with it.

 The analysis in Figure~\ref{ok} can be formalized in HPSG as
 shown in  (\ref{coordparam2}),  using parametric lists  \citep{pollardsag}\addpages to enforce
 that all conjuncts structure-share the morphosyntactic information. The type \textit{n$($on$)$e$($mpty$)$-list} corresponds
 to a list that has at least one member, and when used parametrically as in (\ref{coordparam2}) it in addition requires that
 every member of the list must bear the features $[$\textsc{synsem}$|$""\textsc{loc}$|$""\textsc{cat}  \avmbox{1}$]$.

\begin{exe}
\ex \textsc{Coordination Construction} (preliminary)

\begin{avm} \type{coord-phr} \impl
\[synsem   \| cat \@{1}\\
dtrs \<\[synsem \| loc \| cat \@{1}\]\>$\oplus$ 
\type{ne-list}\(\[synsem \| loc \| cat  \@{1}\]\)\]\end{avm}\label{coordparam2}
\end{exe}

\noindent
The constraint forcing all daughters to be of the same category is excessive, as we shall see below, 
and this will have to undergo a revision. For now, we are focusing on standard coordinations.

In order to  account for the fact that different kinds of coordination strategies are possible, \citet{Mouret:05} 
defines three subtypes of \type{coord-phr}, assuming a lexical feature \textsc{coord} to distinguish between   coordination types:\footnote{Mouret's formulation is slightly different in that the relevant feature is instead called \textsc{conj}, and a slightly different type hierarchy is assumed, with negative constraints like  \textsc{conj} $\not=$ \type{none} are employed instead of \textsc{coord} \type{crd}. The current formulation   avoids negative constraints, though nothing much hinges on this. Similar liberty is taken in subsequent constraints, for exposition purposes.}

\begin{exe}
\ex
 \begin{avm}\type{simple-coord-phr} \impl
 \[dtrs &  \type{ne-list}\(\<\[coord &  none\]\>\) \, $\oplus$ \type{ne-list}\(\[coord \type{crd}\]\)\]\end{avm}

 \begin{avm}\type{omnisyndetic-coord}  \impl 
 \[dtrs  \type{ne-list}\(\[coord \type{crd} \]\)\]\end{avm}\label{omni}

 \begin{avm}\type{asyndetic-coord}  \impl 
\[dtrs   \type{ne-list}\(\[coord &  none\]\)\]\end{avm}
\end{exe}

\noindent
Here, we assume that the value of \textsc{coord} must be typed as \type{coord},
and that the latter has various sub-types as shown in Figure~\ref{fig:mlabelc}.

\begin{figure}
    \centering
    \Tree[.\type{coord} \type{none} [.\type{crd} \type{and} \type{or} \type{but} {\ldots{}} ] ]
    \caption{Coordinator sub-types}\label{fig:mlabelc}
\end{figure}

\noindent
Thus, simple (monosyndenton and polysyndenton) coordinations are those where all but the first coordinand are allowed to combine with a coordinator,  omnisyndenton coordinations are those where all coordinands
have combined with a coordinator, and likewise, 
asyndenton coordinations are those where none of the coordinands have combined with a coordinator.


We turn to  the  analysis of coordinators. 
In other words, what exactly are words like \emph{and}, \emph{or}, 
and others, and how do they combine with coordinands?

\subsection{The status of coordinator expressions}


In HPSG, coordinators are sometimes analyzed as markers \citep{Beavers,Drellishak:Bender:05}. In such a view, the coordinator's lexical entry does not select any arguments, since it has no arguments. In (\ref{le-coord-lexeme-marker}) we show the lexical entry for the conjunction, using current HPSG feature geometry. Note that the \textsc{mrkg} value of the coordinator is the same as the coordinand's, which makes this marker a bit unusual in that it is transparent. Thus, if \emph{and} coordinates S nodes that are \textsc{mrkg} \type{that} (i.e.\ CPs) then the result will be 
an S that is also \textsc{mrkg} \type{that}, and so on, for any given value of
\textsc{mrkg}.\footnote{The semantics and pragmatics of coordination  is a particularly complex topic which we cannot do justice to here, specially when it comes to interactions with other phenomena such as quantifier scope and collective, distributive, and reciprocal readings.
See \crossrefchapterw{semantics} for more discussion and 
in particular \citet{mrs},  \citet{jfast}, 
\citet[Chapters 4--6]{chavesthesis},  \citet{chavesextr}, 
\citet{chavessubjexp}, \citet{Chaves:09}, 
and \citet{sangheepark} for HPSG work that specifically focuses on the semantics of coordination.}


\begin{exe}
 \ex \begin{avm}
 %\type{coordinator} \impl
 \[\type{coord-lexeme}\\
 phon  \, \, $\langle$\type{and}$\rangle$\\
  synsem  \[loc \[cat & \[head & \[\type{coord}\\
             sel \[loc \[cat \[coord \type{none}\\ mrkg \@{1}\]\]\]\]\\
             coord  & \type{and}\\
             mrkg & \@{1}\]\]\]\]
 \end{avm}\label{le-coord-lexeme-marker}
\end{exe}




\noindent
This sign imposes constraints on the head sign it combines with via the feature \textsc{sel}, the same feature that allows other markers and 
adjuncts in general to combine with their
hosts. The syntactic construction that allows such elements with their selected heads is the Head-Functor Construction in (\ref{rulem}).
Since the second daughter is the head, the value of the mother's \textsc{head} feature will have to be the same as the head daughter's, as per the
\isi{Head Feature Principle}.\footnote{The Head Feature Principle \citep{pollardsag} states that the value of
the mother's \textsc{head} feature is identical to that of the head daughter's \textsc{head}
feature. See also \crossrefchaptert{basic-properties}.}\addpages

\inlinetodostefan{The second version was still wrong since the head-dtr pointed to the \synsemv. I fixed it for you. I hope the topmost AVM is
  correct now.}

\eas 
\type{head-func-cxt} \impl

\begin{avm}
 \[synsem & \[loc & \[cat & \[subj  & \@{1}\\ 
                              comps & \@{2}\\
                              coord & \@{3}\\
                              mrgk  & \@{4}\]\]\]\\
 hd-dtr & \@{5}\\
 dtrs & \<\[synsem \[loc & \[cat & \[sel   & \@{6}\\
                                     coord & \@{3}\\ 
                                     mrkg  & \@{4}\]\]\]\],
        \@{5} \[synsem \@{6}\[loc \[ cat \[ subj  & \@{1}\\ 
                                      comps & \@{2}\]\]\]\]\>\]

\end{avm}

\begin{avm}
 \[synsem & \[loc & \[cat & \[subj & \@{3}\\ comps & \@{4}\\
 coord & \@{1}\\
 mrgk & \@{3}\]\]\]\\
 hd-dtr & \@{2}\\
 dtrs & \<\[synsem \[loc & \[cat & \[sel & \@{2}\\
               coord & \@{1}\\ mrkg & \@{3}\]\]\]\],
        \[synsem \@{2}\[loc \[ cat \[ subj & \@{3}\\ comps & \@{4}\]\]\]\]\>\]

\end{avm}

\begin{avm}

 \[synsem & \[loc & \[cat & \[val & \@{0}\\
 coord & \@{1}\\
 mrgk & \@{3}\]\]\]\\
 hd-dtr & \@{2}\\
 dtrs & \<\[synsem \[loc & \[cat & \[sel & \@{2}\\
               coord & \@{1}\\ mrkg & \@{3}\]\]\]\],
        \@{2}\[synsem \[loc \[ cat \[ val \@{0}\]\]\]\]\>\]
 \end{avm}\label{rulem}
\zs
\inlinetodostefan{This is wrong. Structure sharing goes from \textsc{sel} to \synsem, there is no VAL but \subj and \comps.}


\noindent
Thus, the conjunction projects an NP when combined with an NP, an AP when combined with an AP, etc., as Figure~\ref{coordphr} illustrates.

\begin{figure}
\hfill
\Tree[.{NP$[$\textsc{coord} \type{and}$]$}	
[.{C$[$\textsc{coord} \type{and}$]$}  {and} ] [.N {Mary} ] ]
\hfill
\Tree [.{AP$[$\textsc{coord} \emph{or}$]$}  
[.{C$[$\textsc{coord} \emph{or}$]$}   {or} ]
[.AP {tall} ] ]
\hfill\mbox{}
\caption{Coordinate marking constructions}\label{coordphr}
\end{figure}


An alternative HPSG account that yields almost the same representation through different means is adopted by \citet{Abeille:03}, \citet{Abeille:05}, \citet{Mouret:07} and \citet{Bilbiie:17} and others. This approach
instead takes coordinators to be \emph{weak heads}, i.e.\ heads which inherit most of their syntactic properties from their complement,
like argument-marking prepositions. Thus, the coordinator combines with coordinands via the same headed constructions that license non-coordinate structures.
It  preserves the \textsc{marking} feature when conjuncts are themselves marked. The conjunction takes the adjacent conjunct as a complement. This captures its being first in head initial languages like English, and its final position in head final languages like Japanese.

\eal
\settowidth\jamwidth{(Japanese)}
\ex Lee [and Kim]]\jambox{(English)}
\ex 
\gll Lee-to Kim\\
     Lee-and Kim\\\jambox{(Japanese)}
\glt `Lee and Kim'
\zl

\noindent
Since it is a weak head, it inherits most of  its syntactic features (\textsc{head}, \textsc{marking}, \textsc{xarg}) from its complement, and adds its own  \textsc{coord} feature. The relevant constraint over all such 
coordinator lexemes is shown in (\ref{lexcoordentry}).


\begin{exe}
\ex 
\begin{avm}
\type{coord-lexeme} \impl
\[phon & $\langle$\emph{and}$\rangle$\\
synsem & \[loc & \[cat & \[head & \@{0}\\
                   coord & and\\
                   mrkg & none\]\]\]\\
          arg-st & \@{1} $\oplus$
                         \<\[synsem \[loc \[cat \[head & \@{0}\\
                         xarg & \@{1}\\
                         mrkg  & \@{2}\]\]\]\]\>\]
                         \end{avm}\label{lexcoordentry}
\end{exe}

\noindent
The weak head analysis is illustrated in
Figure~\ref{coordphr2}. Here, the category of the coordinator, the conjunct and of the mother node are the same, because the coordinator's head value is lexically required
to be structure-shared with the head value of its valents.


\begin{figure}
\hfill
\Tree[.{NP$[$\textsc{coord} \emph{and}$]$}	
[.{N$[$\textsc{coord} \emph{and}$]$}  {and} ] [.NP {Mary} ] ]
\hfill
\Tree [.{AP$[$\textsc{coord} \emph{or}$]$}  
[.{A$[$\textsc{coord} \emph{or}$]$}   {or} ]
[.AP {tall} ] ]
\hfill\mbox{}
\caption{Coordinate weak-head constructions}\label{coordphr2}
\end{figure}


Before moving on, we note that the weak head analysis of coordinators makes certain problematic predictions that the marker analysis in (\ref{le-coord-lexeme-marker}) does not make. Since coordinands are selected as arguments in the former approach,  additional assumptions need to be made in
 order to prevent the  extraction of conjuncts as in (\ref{cc}).
If coordinands are in \textsc{arg-st} then they are expected to be extractable
(see \crossrefchapterw{udc} and \crossrefchapterw{islands}).

%, and subject to the o-command constraint that governs anaphora (see \crossrefchaptert{binding}).

\ea[*]{
\label{cc}
Which boy did you compare Robin and \trace?\\
 (cf.\ with \emph{which boy did you compare Robin with \trace?})
}
\z

%\begin{exe}
%\ex
%\begin{xlista}
%\ex  We invited [[Betsy's$_i$ mother] and [her$_i$]] to the ceremony.\\
%\citep{gazd1982}

%\ex I wish to inform [[him$_i$] and [all of Dr. Phil$_i$'s viewers] that real counseling
%sessions take place behind closed doors, not in TV.\\
%\citep{chavesthesis}

%\end{xlista}
%\end{exe}\label{bt}

\noindent
For this reason, the members of \textsc{arg-st} of the coordinator are typed as \type{canonical} by \citet{Abeille:03}, to prevent their extraction, analogously to how prepositions in most languages must prevent their complements from being extracted, unlike English and a few other languages.
See \citet{Abeille06} for a weak head analysis of certain French prepositions.




\subsection{Correlative coordination}\label{correlphr}


Having discussed standard coordination structures, we now move on to cases where
multiple inter-dependent coordinators are present, such as \emph{either \ldots{} or \ldots{}},
\emph{neither \ldots{} nor \ldots{}}, 
and \emph{both \ldots{} and \ldots{}}. See \citet{hof} for an account  in HPSG. Given the linearization flexibility of the first coordinator, they  can be analyzed in English as adverbials rather than as true coordinators:

\begin{exe}
 \ex
\begin{xlista}
\ex  Either Fred bought a cooking book or he bought a gardening magazine.
\ex  Fred either bought a cooking book or he bought a gardening magazine.
\ex  Fred can either buy a cooking book or he can buy a gardening magazine.
\end{xlista}
\end{exe}


\begin{exe}
 \ex
\begin{xlista}
\ex John will read both the introduction and the conclusion.
\ex John will both read the introduction and the conclusion.
\end{xlista}
\end{exe}



\noindent
In French, as in other Romance languages, the conjunction itself can be reduplicated, and it is obligatory for some conjunctions (\emph{soit} `or' in French) \citep{Mouret:07, Bilbiie:17}: 

\begin{exe}
 \ex
\begin{xlista}
\ex[] {
\gll Jean lira et l'introduction et la conclusion.\\
     Jean read.\textsc{fut} and the.introduction and the conclusion\\
\glt `Jean read both the introduction and the conclusion.'
}
\ex[*] {
\gll Jean et lira l'introduction et la conclusion.\\
     Jean and read.\textsc{fut} the.introduction and the conclusion\\
}
\ex[] {
\gll Jean  lira soit l'introduction soit la conclusion.\\
    Jean read.\textsc{fut} or the.introduction or the conclusion\\
}
\ex[*] {
\gll Jean  lira l'introduction soit la conclusion.\\
    Jean read.\textsc{fut} the.introduction or the conclusion\\
}
\end{xlista}
\end{exe}
\inlinetodostefan{check (\mex{0}c,d). I took them appart. Check translation}

\noindent
Thus, there are  different structures for different types of correlative, as Figure~\ref{f2} illustrates. The one on the left is for correlatives that exhibit adverbial properties and the one on the right is for correlatives that do not.
See \citet{Bilbiie:08} for arguments that both types are attested in Romanian.



\begin{figure}
    \hfill
    \Tree[.X [.{Adv} both ]  [.X X [.X [.{Coord} and ]  X ] ] ]
\hfill
    \Tree[.X [.X [.{Coord} et ]  X ] [.X [.{Coord} et ]  X ] ]
\hfill\mbox{}
\caption{Two possible structures for correlative coordination}\label{f2}
\end{figure}


The correlative coordinate structure on the right is covered by (\ref{omni}), since it requires the COORD feature to be the same for all conjuncts. 

%See  \S\ref{correlphr} for more discussion of correlative constructions in a broader context, beyond coordinate correlatives.


\subsection{Comparative correlatives}



When there is no overt conjunction, it is not always clear whether a binary clause construction is coordinate or not. Comparative correlatives such as (\ref{cc0}) have been analyzed as coordinate by \citet{culijack} for English (in syntax, though not in semantics) and as universally subordinate  by \citet{dikken}. 

\begin{exe}
\ex The more I read, the more I understand. \label{cc0}
\end{exe}

On the semantic side, the interpretation is something like: `if I read more, I understand
more'. \citet{Abeille:06} and \citet{Abeille:Borsley:08} propose that they are  coordinate in some languages, 
 and subordinate in others. In English, one can add the adverb \emph{then}, whereas in French, one can add the conjunction \emph{et} (`and'). In English, the first clause can also be used as a standard adjunct (\ref{adjcc}).
 \inlinetodostefan{I reinserted the \emph{then}. We had a long discussion about this. I think it is
   correct with \emph{then}.}
 
 
 \begin{exe}
 \ex
\begin{xlista}
\ex The more I read, then the more I understand.
\ex 
\gll Plus je lis (et) plus je comprends.\\
     more I read \hspaceThis{(}and more I understand\\
\glt `If I read more, I understand more.'
\ex I understand more, the more I read.
\end{xlista}\label{adjcc}
\end{exe}

As shown by \citet[549-550]{culijack}, the second clause show matrix clause properties, not the first one:

\begin{exe}
 \ex
\begin{xlista}
\ex[]	{The more we eat, the angrier you get, don't you?}
\ex[*] {The more we eat, the angrier you get, don't we?}
\end{xlista}
\end{exe}

Syntactic parallelism seems to be stricter in French, for example clitic inversion or extraction must take place out of both clauses at the same time \citep{Abeille:Borsley:08}:

\inlinetodostefan{I changed the glosses according to the Leipzig Glossing Rules, please check.}
\begin{exe}
 \ex
\begin{xlista}
\ex 
\gll Paul a     peu  de temps: aussi plus  vite commencera-t-il,  plus   vite  aura-t-il  fini.\\
     Paul has little of time so more fast start-\textsc{fut}-he more fast  \textsc{aux}-\textsc{fut}-he finish.\textsc{ppart} \\
\glt `Paul has little time left: so the faster he starts, the faster he will finish'
\ex \gll C'   est un livre  que      plus   tu    lis, plus  tu    appr\'{e}cies. \\
this is    a  book \textsc{comp} more you read.2\textsc{sg}  more you appreciate.2\textsc{sg} \\
\glt `This is a book that the more you read the more you like.'
\end{xlista}
\end{exe}

In Spanish, comparartive correlatives come in two varieties: one that can be analyzed as subordinate
as in (\ref{spanishab}a), 
 and one that can be analyzed as coordinate, like  (\ref{spanishab}b).

\begin{exe}
 \ex
\begin{xlista}
\ex 
\gll Cuanto   m\'{a}s leo,     (tanto)        m\'{a}s entiendo. \\
     how.much more    read.1\textsc{sg} \hspaceThis{(}that.much more understand.1\textsc{sg} \\
\glt `The more I read, the more I understand.'
\ex 
\gll	M\'{a}s leo        (y) m\'{a}s entiendo. \\
	more read.1\textsc{sg} \hspaceThis{(}and more understand.1\textsc{sg} \\
\glt `The more I read, the more I understand.'\\ 
 \citep{Abeille:Borsley:Espinal:06}
\end{xlista}\label{spanishab}
\end{exe}

Be they coordinate or subordinate, they are special kinds of construction: they are binary, with a fixed order: the meaning changes if the order is reversed
as in (\ref{intell}a).
The internal structure of each clause is also special. In English, it must start with \emph{the} and a comparative phrase like in (\ref{intell}b), which may belong to a long distance dependency 
(\ref{intell}c). Each clause must be finite and allow for copula omission, as shown in (\ref{intell}d).

\begin{exe}
 \ex
\begin{xlista}
\ex[] {The more I understand, the more I read.}
\ex[*] {I understand (the) more, I read (the) more.}
\ex[]  {The more I manage to read, the more I start to understand.}
\ex [] {The more intelligent the students, the better the marks.}
\end{xlista}\label{intell}
\end{exe}

These \emph{the}-clauses  are a special subtype of finite clause, starting with a comparative phrase. \citet{Abeille:Borsley:Espinal:06}, 
\citet{Borsley:11} 
define a \textsc{correl} feature which is a \textsc{left edge} feature (see the \textsc{edge} feature in \citet{Bonami:2004} for French liaison). Assuming a degree word \textit{the}, which can only appear as a specifier of a comparative word, \citet{Borsley:11}  defines \textit{the}-clause as a subtype of head-filler-phrase with [\textsc{correl} \type{the}]; see also \citet{fgsag08}.

Comparative correlatives belong to a more general class of (binary) correlative constructions, including \emph{as \ldots{} so \ldots{}},
and \emph{ if \ldots{} then  \ldots{}} constructions in 
\citew{Borsley:04,Borsley:11}.\footnote{This does not handle Hindi type correlatives, which differ in that  only the first clause is introduced by a correlative word, and the first clause is mobile and optional; see \citet[228]{pollardsag} for an analysis.}
Correlative constructions can be defined as follows, 
where \type{correl-construction} is a sub-type of 
\type{declarative-clause} and the feature \textsc{correl} introduces a \type{correl} type
hierarchy analogous to that of \type{coord} in Figure~\ref{fig:mlabelc} above.
The construction in (\ref{correlphrr}) thus states that all correlative
constructions have in common the fact that both daughters are marked by a special expression. 

\begin{exe}
\ex
            \begin{avm}
            \type{correl-construction}
\impl  
            \[synsem  & \[loc \[cat \[head & finite\\
            correl  & \type{none}\]\]\]\\
            dtrs & \< \[synsem \[loc \[cat \[correl \type{corr-mrk}\]\]\]\],\\ 
            \[synsem \[loc \[cat \[correl \type{corr-mrk}\]\]\]\]\> \] \end{avm}\label{correlphrr}
\end{exe}

Naturally, \type{correl-construction} has various sub-types, each imposing particular patterns of correlative marking, including coordinate correlatives. More specifically,  this family of constructions  comes in two varieties: asymmetric (for the subordinate ones, like English comparative correlatives), and symmetric for coordinate ones, like French comparative correlatives). The symmetric subtype inherits from \type{clausal-coordination-phrase}, while the asymmetric one inherits from the \type{head-adjunct-phrase} as seen in Figure~\ref{figcorr}.

\begin{figure}
\centering
{\small 
\begin{forest}
[\type{construction}
  [\type{causality}
    [{\ldots{}}]
    [\type{declar-clause}
      [{\ldots{}}] 
      [\type{correl-cx}, name=correl
        [{\ldots{}}]
        [\type{symmetric-correl-cx}, name = sym ] ] ] ]
  [\type{headedness}, 
    [\type{non-headed-phr} 
        [{\ldots{}}]
        [\type{coord-phr}, name=coord]  ]
    [\type{headed-phr}
        [{\ldots{}}]
        [\type{head-adj-phr}, name=adj
          [{\ldots{}}]
          [\type{asymmetric-correl-cx}, name = asym ] ] ]    
        ] ] 
\draw  (coord.south) --(sym.north)
       (correl.south) --(asym.north);
\end{forest}}

\caption{Type hierarchy for correlative constructions}\label{figcorr}
\end{figure}


Thus,  asymmetric English comparative correlatives  can be defined as
in (\ref{ecc}), where \type{the} is a sub-type of \type{corr-mrk} (i.e.\ is a coordinate marker).

\begin{exe}
 \ex
	\begin{avm}
	\type{asymmetric-cc-cx} \impl %\type{assymetric-correl-cx} \& 
	\[hd-dtr \@{1}\\
             dtrs \<\[synsem \[loc \[cat \[correl & the\]\]\]\],\\ 
             \@{1}\[synsem \[loc \[cat \[correl & the\]\]\]\]\>\]
             \end{avm}\label{ecc}
\end{exe}

\noindent
Similarly,  symmetric French comparative correlatives can be  defined as
in (\ref{ccf}), where \type{et} and \type{compar} are subtypes of \type{corr-mrk} and  \type{none} is the sub-type of \type{correl} indicating
the absence of  correlative marking.

\begin{exe}
 \ex
	\begin{avm}
	\type{symmetric-cc-cx}
\impl %\type{symmetric-correl-cx} \& 
	\[dtrs \<\[synsem \[loc \[cat \[correl & compar\]\]\]\],\\
	\[synsem \[loc \[cat \[coord \type{ none $\vee$ et}\\
	 correl  \type{compar}\]\]\]\]\>\]\end{avm}\label{ccf}
\end{exe}

A more complete analysis would take into account the semantics as well \citep{fgsag08}. From a syntactic point of view, HPSG seems to be in a good position to handle both the general properties and the idiosyncrasy of the CC construction, as well as its crosslinguistic variation. 
For an analysis of a number of Arabic correlative constructions see \citet{Alqurashi:Borsley:14}.
See also  \citet{Borsley:11} for a comparison with a tentative Minimalist analysis.


\section{Phrasal coordination and feature resolution}

\subsection{Feature sharing between coordinands}

The coordination construction in (\ref{coordparam2}) requires the value of \textsc{cat} to be structure-shared across the coordinands and the mother node. Given the large number of features within \textsc{cat}, such a constraint makes a series of predictions and mispredictions.
For example, this entails that all valence constraints are identical. Thus, in VP coordination all nodes have an empty \textsc{comps} list and share exactly the same singleton \textsc{subj} list, as illustrated in Figure~\ref{valenceif}. Thus, nothing needs to be said from the semantic composition side: the verbs will have to share exactly the same referent for their subject. The same goes for any other combination of categories, of whatever part-of-speech.

\COMMENT{
\begin{figure}
{\small \Tree[.{{S}\\
\begin{avm}\[subj \hspace{5pt} $\langle$ \@{2} $\rangle$\\
               comps  $\langle$ $\rangle$\]
               \end{avm}}
\qroof\emph{Sam}.{
\avmbox{2}NP}
[.{VP\\
\begin{avm}
\[subj \hspace{5pt} $\langle$ \@{2} $\rangle$\\
               comps  $\langle$ $\rangle$\]
\end{avm}} \qroof\emph{ate cheese pizza }.{VP\\
\begin{avm}
\[subj \hspace{5pt} $\langle$ \@{2} $\rangle$\\
               comps  $\langle$ $\rangle$\]
\end{avm}} \qroof\emph{and drank soda}.{VP\\
\begin{avm}
\[subj \hspace{5pt} $\langle$ \@{2} $\rangle$\\
               comps  $\langle$ $\rangle$\]
\end{avm}} ] ]}
\caption{Valence identity in  coordination}\label{valenceif}
\end{figure}
}


\begin{figure}
\begin{forest}
sm edges
[\begin{avm}S\[subj \hspace{5pt} $\langle$ \@{2} $\rangle$\\
               comps  $\langle$ $\rangle$\]
               \end{avm}
 [\avmbox{2}NP [Sam, roof]]
 [ \begin{avm}
   VP\[subj \hspace{5pt} $\langle$ \@{2} $\rangle$\\
               comps  $\langle$ $\rangle$\]
   \end{avm} 
    [\begin{avm}
      VP\[subj \hspace{5pt} $\langle$ \@{2} $\rangle$\\
               comps  \eliste \]
      \end{avm} [ate cheese pizza, roof]]
    [\begin{avm}
      VP\[subj \hspace{5pt} $\langle$ \@{2} $\rangle$\\
                    comps  $\langle$ $\rangle$\]
     \end{avm} [and drank soda, roof] ] ]]
\end{forest}
\caption{Valence identity in  coordination}\label{valenceif}
\end{figure}

All the unsaturated valence arguments become one and the same for all coordinands, and it becomes impossible to have daughters with different subcategorization information. For example, if one daughter requires a complement while the other does not,
\textsc{cat} identity  is impossible. This correctly rules out  a coordination of  VP and V categories
like the one in (\ref{vpbad1}), or S and VP as in (\ref{vpbad2}):

\begin{exe}
\ex
\begin{xlista}
\ex[*]{Fred [read a book]$_{\textup{\scriptsize COMP} \langle \, \rangle}$  and [opened]$_{\textup{\scriptsize COMP}
 \langle \textup{\scriptsize NP} \rangle}$.}\label{vpbad1}
 
 \ex[*] {Fred [she has a hat]$_{\textup{\scriptsize SUBJ} \langle \, \rangle}$ and
 [smiled]$_{\textup{\scriptsize SUBJ} \langle \textup{\scriptsize NP} \rangle}$.}\label{vpbad2}
\end{xlista}
\end{exe}

But there is other information in \textsc{cat} besides valence. For example, the head feature  \textsc{vform}  encodes the verb form, and the coordination of inconsistent \textsc{vform} values is   ruled out as ungrammatical as seen
 in (\ref{vform}), while  consistent values of \textsc{vform} are accepted  as illustrated by (\ref{vform2}).\footnote{That said, some cases are more acceptable such as 
\emph{I expect $[$to be there$]_{\textup{\scriptsize VFORM} \,\, \textit{inf}}$ and 
$[$that you will be there too$]_{\textup{\scriptsize VFORM} \,\, \textit{fin}}$}. See \S\ref{unlikessec} for
more discussion about such cases.}



\begin{exe}
\ex
\begin{xlista}
  \ex[*]{Tom [whistled]$_{\textup{\scriptsize VFORM} \,\, \textit{fin}}$ and
                   [singing]$_{\textup{\scriptsize VFORM} \,\, \textit{prp}}$.}
 \ex[*]{Sue  [buy something]$_{\textup{\scriptsize VFORM} \,\, \textit{prs}}$ and
 [came home]$_{\textup{\scriptsize FORM} \,\, \textit{fin}}$.}
 \end{xlista} \label{vform}
\end{exe}



\begin{exe}
\ex \begin{xlista}
  \ex Tom [is married]$_{\textup{\scriptsize VFORM} \,\, \textit{fin}}$ and
                   [just bought  a house]$_{\textup{\scriptsize VFORM} \,\, \textit{fin}}$.
   \ex Sue [buys groceries here]$_{\textup{\scriptsize VFORM} \,\, \textit{fin}}$ and
                   [could be interested in working with us]$_{\textup{\scriptsize VFORM} \,\, \textit{fin}}$.
   \ex Dan [protested for two years]$_{\textup{\scriptsize VFORM} \,\, \textit{fin}}$ and
                   [will keep protesting]$_{\textup{\scriptsize VFORM} \,\, \textit{fin}}$.
\end{xlista}\label{vform2}
\end{exe}

Yet another feature that resides in the \textsc{cat} value of verbal expressions is the head feature \textsc{inv}, which indicates whether a given verbal expression is invertable or not. Hence, inverted structures cannot be coordinated with non-inverted ones:


\begin{exe}
\ex
\begin{xlista}
\ex[] {[Sue has sung in public]$_{\textup{\scriptsize INV} -}$ and [Kim has tap-danced]$_{\textup{\scriptsize INV} -}$.}
\ex[*] {[Sue has sung in public]$_{\textup{\scriptsize INV} -}$ and [has Kim tap-danced$_{\textup{\scriptsize INV} +}$.}
\end{xlista}
\end{exe}

\begin{exe}
\ex \begin{xlista}
\ex[] {[Elvis is alive]$_{\textup{\scriptsize INV} -}$ and [there was a CIA conspiracy]$_{\textup{\scriptsize INV} -}$.}
\ex[*] {[Elvis is alive]$_{\textup{\scriptsize INV} -}$ and [was there a CIA conspiracy]$_{\textup{\scriptsize INV} +}$.}
\end{xlista}
\end{exe}

\noindent
But if the inverted clause precedes the non-inverted one, then such coordinations become somewhat more acceptable. In fact,  \citet[1332--1333]{rodney2} note attested cases like 
(\ref{tax}).

\begin{exe}
\ex Did you make your own contributions to a complying superannuation fund and
your assessable income is less than \$31,000?\label{tax}
\end{exe}

\noindent
A similar problem arises for the feature \textsc{aux}, which distinguishes auxiliary verbal expressions from those that
are not auxiliary:

\begin{exe}
\ex
\begin{xlista}
\ex {}[I stayed home]$_{\textup{\scriptsize AUX} -}$ but [Fred could have gone fishing]$_{\textup{\scriptsize AUX} +}$.
\ex {}[Tom went to NY yesterday]$_{\textup{\scriptsize AUX} -}$ and [he will return next Tuesday]$_{\textup{\scriptsize AUX} +}$.
\ex Fred [sang well]$_{\textup{\scriptsize AUX} -}$ and [will keep on singing]$_{\textup{\scriptsize AUX} +}$.
\end{xlista}\label{aux}
\end{exe}

\noindent
However, this problem vanishes in the account of the English Auxiliary System detailed in \citet{SagEtAl20}, since in that analysis
the feature \textsc{aux} does not indicate whether the verb is auxiliary or not. Rather, the value of \textsc{aux} for auxiliary verbs is resolved by the construction in which the verb is used. Since all the constructions in (\ref{aux}) are canonical VPs (e.g.\ non-inverted), then 
all the conjuncts in (\ref{aux}) are specified as \textsc{aux--}, in
the \citet{SagEtAl20} analysis.




Similarly, argument-marking PPs cannot be coordinated with modifying-PPs simply because the former are specified with different  \textsc{pform} and \textsc{select} values. This explains the contrast
in (\ref{pp}). The first PP is the complement that \emph{rely}
selects but the second is a modifier. Thus, they have different \textsc{cat} values 
and cannot be coordinated.


\begin{exe}
\ex
\begin{xlista}
\ex[] {Kim relied on Mia on Sunday.}
\ex[*] {Kim relied on Mia and on Sunday.}
\end{xlista}\label{pp}
\end{exe}

\noindent
Consequently, it is in general not possible to coordinate argument marking PPs headed by different prepositions, simply because they bear
different \textsc{pform} values as shown in (\ref{pp2}).

\begin{exe}
\ex
\begin{xlista}
\ex[*] {Kim depends  $[$[on Sandy]$_{\textup{\scriptsize PFORM} \,\, \textit{on}}$
                       or [to Fred]$_{\textup{\scriptsize PFORM} \,\, \textit{to}}$ $]$.}

\ex[*] {Kim is afraid  $[[$of Sandy$]_{\textup{\scriptsize PFORM} \,\, \textit{of}}$
                       and $[$to Fred$]_{\textup{\scriptsize PFORM} \,\, \textit{to}} ]$.}
\end{xlista}\label{pp2}
\end{exe}

Similarly, adjectives that are specified as \textsc{pred}$+$ cannot be
coordinated with  \textsc{pred}$-$ adjectives, without stipulation:

\begin{exe}
\ex
\begin{xlista}
  \ex[*]{I became [former]$_{\textup{\scriptsize PRED} -}$ and [happy]$_{\textup{\scriptsize PRED} +}$.}
 \ex[*] {He is  [happy]$_{\textup{\scriptsize PRED} +}$ and [Fred]$_{\textup{\scriptsize PRED} -}$.}
  \ex[*] {[Mere]$_{\textup{\scriptsize PRED} -}$ and [happy]$_{\textup{\scriptsize PRED} +}$, Fred rode on into the sunset.}
 \end{xlista}
\end{exe}


\noindent
Since case information is also part of \textsc{cat}, it follows that conjuncts must be consistent as in (\ref{case}).\footnote{There are nonetheless collocational cases where the distribution of pronouns defies this pattern, due to presumably prescriptive forces; see \citet{grano} and \citet{binomial}.} Many other examples of \textsc{cat} mismatches exist, but the  list above suffices to
illustrate the breadth of predictions that follow from the feature geometry of \textsc{cat} and the constraints imposed by
the coordination construction.

\begin{exe}
\ex
\begin{xlista}
\ex[*] {I saw [her$_{acc}$ and he$_{nom}$].}
\ex[*] {He likes [she$_{nom}$ and me$_{acc}$].}
\end{xlista}\label{case}
\end{exe}


% (Boolean, non-Boolean, etc.  agreement resolution, 1st conjunct agreement)


Mispredictions also exist. We already discussed one, concerning the feature \textsc{inv}, but there are others. For example, requiring that the \textsc{gap} value of the coordinands be the same readily predicts Coordinate Structure Constraint effects like 
(\ref{cs1}), but it incorrectly rules out asymmetric coordination violation cases like (\ref{assym}). 
See \citet{goldsmith}, \citet{lakoff86}, \citet{levinprince86}, and \citet{kehler} for more examples and discussion.



\begin{exe}
\ex \begin{xlista}
\ex[] {[To him]\sub{\avmbox{1}NP}  [Fred gave a football \trace]$_{\textup{\textsc{slash}} \langle \avmbox{1} \rangle}$ and
[Kim gave a book \trace]$_{\textup{\textsc{slash}} \langle \avmbox{1} \rangle}$}

\ex[*] {[To him]\sub{\avmbox{1}NP} [Fred gave a football \trace]$_{\textup{\textsc{slash}} \langle \avmbox{1} \rangle}$ and
[Kim gave me a book]$_{\textup{\textsc{slash}} \langle \, \rangle}$}

\ex[*] {[To him]\sub{\avmbox{1}NP}  [Fred gave a football to me]$_{\textup{\textsc{slash}} \langle \, \rangle}$ and
[Kim gave a book \trace]$_{\textup{\textsc{slash}} \langle \avmbox{1} \rangle}$}
\end{xlista}\label{cs1}
\end{exe}

%% \inlinetodostefan{The gap is bound off within the relative clause. In PS94 and in
%%   \citet[\page 456]{Sag97a}. Please fix this. (\mex{1}b) is not ruled out because of non-matching
%%   \slashvs but because the relative clause is internally not well-formed: a head-filler phrase needs
%% a gap and \emph{that every kid wants it} does not have a gap and hence the relative pronoun cannot
%% be integrated. I guess you should just remove the example in (\mex{1}).}

%% \begin{exe}
%% \ex \begin{xlista}
%% \ex[] {It offers something$_{i}$  [that every kid wants \trace]$_{\textup{\textsc{slash}} \langle \textup{\textsc{np}}_i \rangle}$ and
%% [that every parent tries to help their child to achieve \trace]$_{\textup{\textsc{slash}} \langle \textup{\textsc{np}}_i \rangle}$.}

%% \ex[*] {It offers something$_i$ [that every kid wants \trace]$_{\textup{\textsc{slash}} \langle 
%% \textup{\textsc{np}}_i \rangle}$ and
%% [that every parent tries to help their child to achieve it]$_{\textup{\textsc{slash}} \langle \, \rangle}$} 

%% \ex[*] {It offers something$_i$  [that every kid wants it]$_{\textup{\textsc{slash}} \langle \, \rangle}$ and
%% [that every parent tries to help their child to achieve \trace]$_{\textup{\textsc{slash}} \langle
%% \textup{\textsc{np}}_i \rangle}$}
%% \end{xlista}\label{cs2}
%% \end{exe}


\begin{exe}
\ex \begin{xlista}
\ex {}[Who]\sub{\avmbox{1}NP} did Sam [[pick up the phone]$_{\textup{\textsc{slash}} \langle \, \rangle}$ and [call \trace]$_{\textup{\textsc{slash}} \langle \avmbox{1} \rangle}$?
\ex What was the maximum amount$_{\avmbox{1}}$ that
I can [contribute \trace]$_{\textup{\textsc{slash}} \langle \avmbox{1} \rangle}$ and [still get a tax deduction]$_{\textup{\textsc{slash}} \langle \, \rangle}$?
\end{xlista}\label{assym}
\end{exe}


\citet{chavesextr} argues that there are no independent grounds to assume that asymmetric coordination is anything other than coordination, and therefore the coordination construction must not impose \textsc{gap} identity across conjuncts. Rather, the Coordinate Structure Constraint, and its asymmetric exceptions are best analyzed as
pragmatic in nature, as \citet{kehler} argues. 
See \crossrefchapterw{udc} for more discussion.\todostefan{Please tell me which area you refer to
  and I add the section/pages.}
In practice, this means that the coordination construction should impose identity of some of the features in \textsc{cat}, though not all, despite the fact that one of the prime motivations for \textsc{cat} was
coordination phenomena.

Like in the case of locally specified valents, the category of the extracted phrase is also structure-shared
in coordination. Hence, case mismatches like (\ref{gapcase}) are
correctly ruled out.


\ea[*]{
\label{gapcase}
{}[Him]$_{acc}$,  [all the critics like to praise \trace]$_{\textup{\textsc{slash}} \langle \textup{\scriptsize NP}\emph{acc} \rangle}$
but [I think \trace would probably not be present at the awards]$_{\textup{\textsc{slash}} \langle \textup{\scriptsize NP}\emph{nom} \rangle}$}
\z

\noindent
There are, however, cases where the case of the ATB-extracted phrase can be syncretic as in (\ref{syn}), 
due to \citet[46]{levineetal}. 

\begin{exe}
\ex
\begin{xlista}
\ex[] {Robin is someone who$_i$ even [good friends of \trace$_i$]
believe \trace$_i$ should be closely watched.}

\ex[] {We went to see [a movie]$_{\avmbox{1}nom\_acc}$  [which the critics praised \trace]$_{\textup{\textsc{slash}} \langle \avmbox{1} \rangle}$
but [that Fred said \trace would probably be too violent for my taste]$_{\textup{\textsc{slash}} \langle \avmbox{1} \rangle}$}
\end{xlista}\label{syn}
\end{exe}

The feature \textsc{case} is responsible for identifying the case of nominal expressions.
Pronouns like \emph{him} are specified as \emph{acc}(\emph{usative}), and pronouns like
\emph{I} are \emph{nom}(\emph{inative}), and expressions like \emph{who} or
\emph{Robin} are left underspecified for case.
According to  \citet[207]{levineetal},  the case system of English involves 
the  hierarchy  in Figure~\ref{qwsa}.



\begin{figure}
\centering

{\small 
\begin{forest}
      [\type{case}, 
        [\type{snom},
        [\type{nom} ]
        [\type{nom\_acc}, name = sinc ]] 
        [\type{sacc}, name=acc,
          [\type{acc}, name = asym ] ]]
\draw  (acc.south) --(sinc.north);
\end{forest}}


\caption{Type hierarchy of (structural) case assignments}\label{qwsa}
\end{figure}


%\pagebreak
Verbs subcategorize for structurally nominative (\emph{snom}) NP subjects and 
 structurally accusative (\emph{sacc}) NP complements. Most nouns and some pronouns like \emph{who} and \emph{what} are underspecified for case, and thus typed as \emph{case}, 
 which makes them consistent with both nominative and accusative positions. Hence, \emph{a movie}
 can be simultaneously be required to be  consistent with \emph{snom} and \emph{sacc}, by resolving
 into the syncretic type \emph{nom\_acc}, which is a subtype of both \emph{snom} and
\emph{sacc}. Pronouns like \emph{him} and \emph{her} are specified as \type{acc} and therefore are not compatible
with the \type{nom\_acc} type. The same goes for 
\type{nom} pronouns like \emph{he} and \emph{she}, etc.
Hence, the problem of case syncretism is easily solved.
See Section~\ref{unlikessec} for more discussion about the related phenomenon of coordination of unlike categories.


\subsection{Coordination and agreement}


Another thorny issue for syntactic theory and coordination structures concerns agreement. According to 
\citet{pollardsag}, agreement information is introduced by the \textsc{index} feature in semantics, not morphosyntax. Hence, different expressions
with inconsistent person, gender and number specifications are free to combine. But \citet{wechsler} have also argued that there should be a distinct feature called \textsc{concord}, which is morphosyntactic in nature (See \crossrefchapterw{agreement}). The motivation for this move, is that there are languages, like Serbo-Croatian,
which display hybrid agreement:

\begin{exe}
\ex \gll Ta dobra deca su do\v{s}-l-a\\
         that.\sg.\fem{} good.\sg.\fem{} children \textsc{aux}.3\pl{} come-\textsc{pprt}-\textsc{pn}\\
  \glt `Those good children came.'   \\
  \citep[51]{wechsler} 
\end{exe}

\noindent
The collective noun \emph{deca} `children' triggers feminine singular (morphosyntactic) agreement on NP-internal items, in this case the determiner \emph{ta} `that' and the adjective \emph{dobra} `good'.   This is a problem for a locality theory of agreement. On the other hand, there are other HPSG analyses that argue that what appears to be closest conjunct agreement is in fact agreement with the whole coordinate NP, which has additional features inherited from the first and last conjuncts. Villavicencio et al. (2005) propose two additional features: \textsc{lagr} (for the left most conjunct) and \textsc{ragr} (for the right most conjunct) for determiner and (attributive) adjective agreement in Romance, which involves the \textsc{concord} feature.
Semantic agreement (i.e.\ Concord), on the other hand, is seen in the verb \emph{su}, which  is inflected for third person plural, in agreement with the semantic properties of the subject \emph{deca}. The two kinds of agreement are also visible in
English:

\begin{exe}
\ex
\begin{xlista}
\ex This/*These committee made a decision.
\ex The committee have/has made a decision.
\end{xlista}
\end{exe}


\noindent
Since (morphosyntacic) agreement is not recorded in \textsc{cat}, it follows that the resolution of agreement information in coordination must be processed elsewhere in the grammar. There are usually strict and non-trivial constraints involved in determining what the agreement of the mother node is given that of the coordinands. We turn to this problem below.







\subsection{Agreement}


In case of conjuncts with conflicting agreement values, resolution strategies are observed 
crosslinguistically. For example, a coordination with a 1st person is 1st, and a coordination with 2nd person (and no 1st person) is 2nd person:

\begin{exe}
 \ex
\begin{xlista}
\ex Paul and I like ourselves / *themselves.
\ex Paul and you like yourselves / *themselves.
\end{xlista}
\end{exe}

In gender marking languages, coordination with conflicting gender values is often resolved to 
masculine, at least for animates \citep{Corbet91}. This is illustrated in (\ref{pt}), for Portuguese.

\begin{exe}
 \ex
\begin{xlista}
\ex \gll o homem e a mulher modernos \\
the.\textsc{m.sg} man.\textsc{m.sg} and the.\textsc{f.sg} woman.\textsc{f.sg} modern.\textsc{m.pl} \\
\glt `the modern man and woman'
\ex \gll morbidez e morte prematuras \\
morbidity.\textsc{f.sg} and death.\textsc{f.sg} premature.\textsc{f.pl}\\
\glt `premature morbidity and death'\\
\citep[433]{Villavicencio:Sadler:ea:05}
\end{xlista}\label{pt}
\end{exe}




 \citet{Sag:03} proposes that 1st person is a supertype of 2nd person which is itself a supertype of 3rd person. This way, person resolution in coordination amounts to type unification. Addressing gender resolution, \citet{Aguila:Crysmann:18} propose a list-based encoding of person and gender values, and list concatenation as a combining operation, as shown in (\ref{aguila}). For gender, they propose a \textsc{m}(\textsc{asculine}) feature that has an empty list value for feminine words, and a non empty list value for masculine words.  The coordination of a masculine noun (\emph{chevaux} `horses') with a feminine noun (\emph{\^{a}nesses} `female donkey')  yields a masculine NP with a non-empty list value for \textsc{m}. Only the coordination of two feminine nouns yields a feminine NP with an empty list value \textsc{m}.
 
\begin{exe}
\ex 
\begin{avm}
\type{nom-coord-phrase} \impl \[synsem & \[cont \[index & \[num & pl\\
                                              gen & \[m & \@{1}$\oplus$\@{2} \]\\
                                              pers & \[me  & \@{4}$\oplus$\@{6}\]
                                             \]
                               \]
                  \]\\      
dtrs & \< \[synsem \[cont \[index & \[gen \[ m \@{1}\]\\
                               pers \[me \@{3}\\
                                         you \@{4}]\] \]\] \] \],\\
               \[synsem \[cont \[index & \[gen \[ m \@{2} \]\\
                               pers \[me \@{5}\\
                                         you \@{6}\]\]\] \] \]\>\]
\end{avm}\label{aguila}
\end{exe}
 
 \noindent
For person agreement , they use two list valued features \textsc{me} and \textsc{you}. A 1st person has a non empty \textsc{me} list, 2d person has an empty \textsc{me} list and a non empty \textsc{you} list, and 3rd person has both empty lists.  Thus, coordinating a 1st with a 3rd person  yields a \textsc{me} feature with a non-empty list, and a \textsc{you} feature with a non-empty list, hence a 1st person phrase. Coordinating a 3rd person with a 2d person yields a non-empty \textsc{you} list  and an empty \textsc{me} list, hence a 2d person phrase. This enables person and gender resolution by list concatenation over
conjuncts. 

\subsubsection{Closest Conjunct agreement}


As observed by \citet{Corbet91}, many languages including Romance, Celtic, Semitic and Bantu languages, also have another strategy, namely partial agreement with only one conjunct, the one closest to the target, called closest conjunct agreement (CCA). 
In the following example, again from Portuguese, the determiner and prenominal adjective agree with the first Noun (\ref{fo}a) and the postnominal adjective with the last Noun (\ref{fo}b).

\begin{exe}
 \ex
\begin{xlista}
\ex \gll suas pr\'{o}prias rea\c{c}\~{o}es ou julgamentos \\
his.\textsc{f.pl} own.\textsc{f.pl} reactions.\textsc{f.pl} or judgements.\textsc{m.pl} \\
\glt `his own reactions or judgements'\\ 
\citep[435]{Villavicencio:Sadler:ea:05}  


\ex \gll Esta canc\~{a}o anima os cora\c{c}\~{o}es e mentes brasileiras. \\
 this.\textsc{f.sg}  song.\textsc{f.sg} animates the.\textsc{m.pl} hearts.\textsc{m.pl} and minds.\textsc{f.pl} Brazilian.\textsc{f.pl} \\
\glt `This song animates Brazilian hearts and minds.'\\
\citep[437]{Villavicencio:Sadler:ea:05} 
\end{xlista} \label{fo}
\end{exe}

For French determiners and attributive adjectives, \citet{An:Abeille:17} and \citet{Abeille:An:Shiraishi:18} show on the basis of corpus data and experiments that number agreement may also obey CCA. As far as gender is concerned, prenominal adjectives always obey CCA while postnominal ones half of the time (in contemporary French). In (\ref{ft}a), the determiner can be singular (CCA) or plural (resolution), while in (\ref{ft}b), CCA (feminine Det) is obligatory. In (\ref{ft}c), the postnominal adjective can be masculine (resolution) or feminine (CCA), with the same meaning.

\begin{exe}
 \ex
\begin{xlista}
\ex  
\gll votre / vos nom et pr\'{e}nom \\
     your.\sg{} {} you.\pl{} name.\textsc{m.sg} and first.name.\textsc{m.sg} \\
\glt `your name and first name'\\ 
\citep{An:Abeille:17}

\ex 
\gll certaines          / *   certains           r\'{e}gions          et  d\'{e}partements \\
     certain.\fem.\pl{} {} {} certain.\mas.\pl{} region.\textsc{f.pl} and department.\textsc{m.pl} \\
\glt `certain regions and departments'\\ \citep{Abeille:An:Shiraishi:18}

\ex 
\gll des d\'{e}partements et r\'{e}gions importants/importantes\\
     some department.\textsc{m.pl} and region.\textsc{f.pl} important.\textsc{m.pl}/\textsc{f.pl}\\
\glt `some important departments and regions'
\end{xlista}\label{ft}
\end{exe}


As proposed by \citet{wechsler}, HPSG distinguishes two agreement features: \textsc{concord} is used for
morphosyntactic agreement and \textsc{index} is used for semantic agreement (see
\crossrefchapterw{agreement}). \citet{Moosally} proposes an account
of single conjunct predicate-argument agreement in Ndebele, which she analyses as  \textsc{index} agreement. She has  a version of the following 
constraint that shares the \textsc{index} value of the (nominal) coordinate mother with that of the last conjunct:

\begin{exe}
\ex \begin{avm}
\type{nom-coord-phrase} \impl
\[synsem  & \[ loc|cont|index & \@{1}\]\\
dtrs & \< \[\, \], \ldots{}, \[synsem|loc|cont|index & \@{1}\]\>\]
\end{avm}
\end{exe}


But in other languages, such as \ili{Welsh}, there is evidence that the \textsc{index} of the coordinate
structure is resolved, even though predicate-argument agreement is controlled by the closest conjunct: 

\begin{exe}
\ex \gll Dw i a Gwenllian heb gael ein talu. \\
be.1\textsc{sg} I and Gwenllian.3\textsc{sg} without get \textsc{cl}.1\textsc{sg} pay \\
\glt  `Gwenllian and I have not been paid.'\\
\citep[\page 12]{Sadler2003a}
\end{exe}

\noindent
This is why \citet{Borsley:2009} proposes that CCA is superficial in Welsh and uses linearization domains to handle partial agreement between the initial verb and the first conjunct, which are not sisters.
The hypothesis  was that verb-subject agreement involves order domains and coordinate structures are not represented in order domains. This allows what looks like agreement with a closest conjunct to be just that. The alternative developed by \citet{Villavicencio:Sadler:ea:05} assumes that coordinate structures have features reflecting the agreement properties of their first and last conjuncts, to which agreement constraints may refer. 
\citet{Villavicencio:Sadler:ea:05} use three features: \textsc{concord}, 
\textsc{lagr} (for the left most conjunct) and 
\textsc{ragr} (for the right most conjunct). 

\begin{exe}
 \ex
\type{nom-coord-ph} \impl\\
\begin{avm}
\[synsem|loc|cat|head \[ lagr & \@{1}\\
                         ragr & \@{2}\]\\
  dtrs \< \[synsem|loc|cat|head|lagr & \@{1}\], \ldots, \[synsem|loc|cat|head|ragr & \@{2}\]\>\]
\end{avm}

\ex
\begin{avm}
\type{noun} \impl    
\[lagr & \@{1}\\
  ragr & \@{1}\\
  concord & \@{1}\]
\end{avm}  
\end{exe}

Nouns have the same value for  \textsc{concoord, lagr} and \textsc{ragr}, and 
determiner and (attributive) adjective agreement in Romance  involves the  \textsc{concord} feature.
Attributive adjectives constrain the agreement features of the noun they modify (via the \textsc{mod} or \textsc{sel} feature). One may distinguish two types for prenominal and postnominal adjectives, by the binary \textsc{lex} $\pm$ feature \citep{Sadler:Arnold:94} or by the \textsc{weight} light/non-light feature \citep{Abeille:Godard:99}. In this perspective, each has its agreement pattern, which we simplify as follows, using `$\vee$' to express a disjunction of feature values:\\

\begin{exe}
 \ex 
          \begin{avm}
          \type{prenominal-adj} \impl 
          \[concord & \@{1}\\
                       sel & \[lagr \@{1}\]\]\end{avm}

 \ex 
  \begin{avm}
  \type{postnominal-adj} \impl  
  \[concord & \@{1} $\vee$ \@{2}\\
            sel & \[concord & \@{1}\\
                    ragr & \@{2}\]\]\end{avm}
\end{exe}

\noindent
In the absence of coordination, these constraints apply vacuously, since CONCORD, LAGR and RAGR all share the same values. 


\section{Lexical coordination}\label{lexcoord}

%(including coordination of word parts, Abeill\'{e} 2006, Chaves 2008)

While conjuncts have often been assumed to be phrasal (see for example \citet{Kayne:94} and \citet{bruening} a.o.), \citet{Abeille:06} gives several arguments in favor of lexical coordination.
In some contexts, words (or phrases with a premodifier) are allowed but not full phrases. In English, it is the case with prenominal adjectives and postverbal particles. See \citet{Abeille:06} for similar examples with various categories in different languages. Most English attributive adjectives are prenominal unless they have a complement. Although complex adjectival expressions with complements are not licit in prenominal position,  it is possible to have complex adjectival expressions if they are coordinate.

\begin{exe}
 \ex
\begin{xlista}
\ex[] {a tall / proud man}
\ex[*] {a [taller than you] man}
\ex[*] {a [proud of his work] man}
\ex[] {a [big and tall] man}
\end{xlista}
\end{exe}

As observed by \citet{hpsg1}, a particle may project a  PP  after the nominal complement, but not before; but coordination is possible, at least for some speakers.

\begin{exe}
 \ex
\begin{xlista}
\ex Paul turned (*completely) off the radio.
\ex Paul turned the radio (completely) off.
\ex Paul was turning [on and off] the radio all the time.
\end{xlista}
\end{exe}

While phrasal coordination can conjoin unlike categories (see below), it is not the case with lexical coordination:

\begin{exe}
 \ex
\begin{xlista}
\ex[] {Paul is  [head of the school] [and proud of it].}
\ex[\#] {Paul is [head and proud] of the school.}
\end{xlista}
\end{exe}

Semantically, lexical coordination is more constrained than phrasal coordination. With \textit{and}, two lexical verbs, sharing a preverbal clitic in French, must share the same verbal root, and in Spanish, they must refer to the same event \citep{Bosque:86}.

\eal
\ex[]{
\gll Je te dis et redis que tu {as tort.} \\
     I you tell and retell that you have wrong\\
\glt `I'm telling you that you are wrong.'
}
\ex[\#]{ 
\gll Je te dis et promets que tu as tort.\\
     I you tell and promise that you have wrong\\
\glt `I'm telling and promising you that you are wrong.'
}
\ex[]{
\gll Lo compro y vendio en una sola operacion.\\
     it buy.1\textsc{sg} and sell.1\textsc{sg} in a single operation\\
\glt `I buy and sell it in one single operation.'
}
\ex[*]{
\gll Lo compro hoy y vendio ma\~{n}ana.\\
     it buy.1\textsc{sg} today and sell.1\textsc{sg} tomorrow \\
\glt `I buy it today and sell it tomorrow.'
}
\zl

Some apparent cases of lexical coordination may be analyzed as Right-Node Raising \citep{Beavers}. They differ semantically and prosodically, however: with typical Right-Node Raising, the two conjuncts must stand in contrast to one another, and do not have to refer to the same event. With Right-Node Raising there must be a prosodic boundary at the ellipsis site (see \citet{chavesrnr} and \crossrefchaptert{ellipsis}). In French, the first conjunct cannot end with a clitic article or with a weak preposition as in (\ref{fr1}b,c),

\eal
\label{fr1}
\ex[]{  
\gll Tout le monde dit et je te promets que tu as tort.\\
     all the world says and I you promise that you are wrong\\
\glt `Everyone says and I promise you that you are wrong.'
}
\ex[*]{ \gll Paul cherche le, et Marie conna{\^{i}t} la responsable.\\
 Paul searches the and Marie knows the responsible\\
\glt `Paul looks for the.\textsc{m.sg} and Marie knows the.\textsc{f.sg}  responsible.'
}
\ex[*]{  \gll Paul parle de, et Marie discute avec Woody Allen.\\
 Paul speaks of and Marie talks with Woody Allen\\
\glt `Paul speaks of and Marie talks with Woody Allen.'}
\zl




No such boundary occurs before the conjunction in lexical coordination. Thus, in French,  clitic
articles or weak prepositions can be conjoined, with a shared argument \citep[\page 14]{Abeille:06}:


\eal
\judgewidth{??}
\ex[]{
\gll Paul cherche   le                ou la                responsable\\
     Paul looks.for the.\textsc{m.sg} or the.\textsc{m.sg} responsible\\
\glt `Paul is looking for the man or woman in charge'
}
\ex[]{
\gll un film de et avec Woody Allen\\
     a film by and with Woody Allen\\
}
\ex[??]{
\gll un film de mais sans Woody Allen\\
     a film by but without Woody Allen\\
}
\zl

\noindent
Not all conjunctions are felicitous with lexical coordination; \textit{but}, for example is less felicitous than \textit{and} or \textit{or}.
Analyzing the conjunction as a weak head (see above), the sub-type for lexical coordination has to allow for the coordination of items waiting for complements: the conjunction (this is done by concatenation of 
\textsc{arg-st} lists as it is for complex predicates, see \crossrefchaptert{complex-predicates}). It thus inherits all the dependents of the word it combines with.


\ea
\begin{avm}
\type{ lex-coord} \impl
   \[synsem|loc|cat  \[head &  \@{0}\\
                       weight &  \@{3}light\]\\
    arg-st  \@{1} $\oplus$ 
    \< \[synsem|loc|cat \[head & \@{0}\\
                          weight & \@{3}\]\\
    arg-st \@{1} $\oplus$ \@{2}\]\> $\oplus$ 
    \@{2}\]\end{avm}
\z
\inlinetodostefan{please give an example}

                                                    
The construct resulting from the coordination of lexical elements has hybrid properties: as a syntactic construct, it must be a phrase, but it also behaves as a word: coordinate verbs behave as lexical heads, coordinate adjectives may occur in positions ruled out for phrases. To overcome this apparent paradox, \citet{Abeille:06} analyses it as an instance of ``light'' phrase, following the \textsc{weight} account of \citet{Abeille:Godard:2000} and \citet{Abeille:Godard:2004}. Light elements can be words or phrases, and can have a restricted mobility (see \crossrefchaptert{order}). For example prenominal modifiers can be constrained to be [\textsc{weight} \emph{light}]. In this theory, light phrases can be coordinate phrases or head-adjunct phrases, provided all their daughters are light as Figure~\ref{light} illustrates.


\begin{figure}
    \hfill
\Tree[.VP [.{VP$[$\emph{light}$]$} [.V {likes} ] ] 
[.{VP$[$\emph{light}$]$}  [.Coord {and} ]  [.V {approves } ]	 ] ]
\hfill
\Tree[.AP [.{AP$[$\emph{light}$]$}  [.A {big} ] ] 
[.{AP$[$\emph{light}$]$} [.Coord {and} ] [.A  {tall} ]  ] ]
\hfill\mbox{}
    \caption{Examples of lexical coordination}
    \label{light}
\end{figure}


\section{Coordination of unlike categories}\label{unlikessec}

The coordination construction in (\ref{coordparam2}) requires that the categories being coordinated are the same.
However, there is some evidence that this requirement is excessive.  Consider the coordinations in (\ref{unlk1}), from
\citet{gpsg}, \citet{bayer}, \citet{rodney2}, among  others.
 Such data pose a classic syntactic problem: 
what is the part of speech and categorial status of the bracketed constituents?


\begin{exe}
\ex \begin{xlista}
\ex Kim is  [alone and without money].\\
 \hfill [AP \& PP]
\ex  Pat is [a Republican and proud of it]. \\
 \hfill [NP \& AP]

\ex  Jack is [a good cook and always improving].\\ \hfill [NP \& VP]

\ex What I would love is [a trip to Fiji and to win \$10,000].\\
\hfill [NP \& VP]

\ex  That was [a rude remark and in very bad taste]. \\
\hfill [NP \& PP]

\ex Chimpanzees hunt [frequently and with an unusual degree of success].\\
\hfill [AdvP \& PP]

\ex I'm  planning [a four-month trip to Africa and  to return to York afterwards].\\
\hfill [NP \& VP]
 \end{xlista}\label{unlk1}
\end{exe}


\noindent
As  \citet{jacobson} pointed out, it is clear that the features of the mother are not simply the intersection of the features of the conjuncts. Verbs like \emph{remain} are compatible with both
AdjP and NP complements whereas \emph{grew}
is only compatible with AdjPs.
This is shown in  (\ref{republican}).
Crucially, however, the information associated with
the phrase \emph{wealthy and a Republican}
somehow allows \emph{grew} to detect the presence of
the nominal, as (\ref{show2}a) illustrates, even
when the verbs
are coordinated, as in (\ref{show2}b--d).


\begin{exe}
\ex
\begin{xlista}
\ex  Kim remained/grew wealthy.
\ex  Kim remained/*grew a Republican.
\end{xlista}\label{republican}
\end{exe}



\begin{exe}
\ex
\begin{xlista}
\ex[] {Kim remained/*grew [wealthy and a Republican].}
\ex[] {Kim grew and remained wealthy.}
\ex[*] {Kim grew and remained a Republican.}
\ex[*] {Kim grew and remained [wealthy and a Republican].}
\end{xlista}\label{show2}
\end{exe}



A number of influential accounts in Type-logical grammar
\citep{morrill90,morrill94,bayer} have used one of the rules of
inference from propositional calculus in order to deal with
coordination of unlikes phenomena, namely, disjunction introduction
(or addition): from $P$ one can infer $P \vee Q$. 
Thus, by assuming that categories like NP, PP
and so on can also be disjunctive, the grammar allows an expression
of type `NP' to lead a double life as an `NP $\vee$ PP' expression,
or the type `AP' to be taken as an `AP $\vee$ PP $\vee$ NP' and so
on. This kind of approach has been adopted in various forms into HPSG, see for example \citet{Daniels02} and  \citet{Yatabe:04}.
Related work aims to achieve the same result using type-underspecification, such as 
 \citet{sag}. Other, more exploratory work, views coordination of unlike categories as the result of   parts-of-speech being gradient and  epiphenomenal rather than hard-coded into the type signature  \citep{bookivan}. 
 Finally,  
 \citet{berthold0}, \citet{yatabe},  \citet{Beavers},
 \citet{chaves06}  argue that
coordination of unlikes can be explained by
a deletion operation that omits the left periphery of
non-initial conjuncts, illustrated in   (\ref{unlk}).


\begin{exe}
\ex
\begin{xlista}
\ex Tom gave a book to Mary, and \sout{gave} a magazine to Sue. 

\ex He drinks coffee with milk at breakfast and \sout{drinks coffee with} cream in the evening.\\ \citep{hudson84}

\ex There was one fatality yesterday, and \sout{there were} two others on the day
before.\\
\citep[339]{chavesthesis}.

\ex I see the music as both going backward and \sout{going} forward.\\
{\small [http://pdxjazz.com/dave-holland; 20 December 2010]}
\end{xlista}\label{unlk}
\end{exe}

\noindent
In such a view, (\ref{unlk1})  are verbal coordinations where the verb (or the verb and the subject) has been deleted  (e.g.\ \emph{Kim is alone and \sout{is} without money}).  The problem is that  left-periphery ellipsis  cannot  fully explain 
coordination of unlikes phenomena. For example, there is no elliptical analysis of  data like (\ref{baaad}). \citet{levine11} offers  arguments against
the coercion account of Chaves (2006),
and against  the existence of left-periphery ellipsis. See \cite{yatabe12} for a reply.


\begin{exe}
\ex
\begin{xlista}
\ex Simultaneously shocked and in awe, Fred couldn't believe his eyes.
\ex  Both tired and in a foul mood, Bob packed his gear and headed North.\\
\citep{chaves06}
\ex Both poor and a Republican, \emph{no one} can possibly be.
\ex  Dead drunk and yet in complete control of the situation, \emph{no one} can be.\\
\citep{levine11}
\end{xlista}\label{baaad}
\end{exe}



\noindent
Further problems for an  ellipsis account of coordination
of unlikes phenomena are posed by the position of  the 
correlative coordinators \emph{both}, \emph{either} and
 \emph{neither} in (\ref{baaad2z}).

\begin{exe}
\ex
\begin{xlista}
\ex Isn't this both illegal and a safety hazard?
\ex It's both odd and in very poor taste to have a fake wedding.
\ex Who's neither tired nor in a hurry?
\ex Isn't she either drunk or on medication?
\end{xlista}\label{baaad2z}
\end{exe}



\noindent
 If (\ref{baaad2z}a) is an elliptical coordination
like \emph{isn't this both illegal and \sout{isn't this} a safety
hazard}, then the location of \emph{both} is unexpected. Instead of
occurring before the first conjunct, it is realized inside the first
conjunct. Crucially, the non-elided counterparts are not
grammatical, e.g.\ *\emph{isn't this both illegal and isn't this a
safety hazard?} The same issue is raised by (\ref{baaad2z}b,c). In
an elliptical account  one would have to stipulate
that \emph{both} can only float in the presence of ellipsis, which
is unmotivated.
Finally, see \citet{Mouret:07} for  an extensive discussion in favor of a non elliptical analysis of unlike coordination, based on correlative coordination.
In sum, left-periphery ellipsis does not 
offer a complete account of coordination of unlikes, and underspecification
accounts are more promising.




\section{Non-constituent Coordination}

The fact that not all coordination of unlike categories can be reduced to deletion  does not entail that
deletion is impossible, or that no phenomena involve deletion.
Consider for example the constructions in (\ref{lpecx}), all of which 
involve non-canonical (i.e.\ non-constituent) coordination, 
some of which were already discussed above. We refer the reader to \crossrefchaptert{ellipsis} for more discussion about other types of ellipsis.

\begin{exe}
\ex
\begin{xlista}
\ex Tom gave a book to Mary, and a magazine to Sue.\\
(Argument Cluster Coordination)


\item Tom loves -- and Mary absolutely hates -- spinach dip.\\
(Right-Node Raising)

\item Tom knows how to cook pizza, and Fred -- spaghetti.\\
(Gapping)

\end{xlista}\label{lpecx}
\end{exe}

Some authors regard Argument Cluster Coordination as elliptical \citep{yatabe01,Crysmann:04,Beavers} others
regard such phenomena as base-generated \citep{Mouret:06}.
In the former,  phonological material in the left periphery of the non-initial conjunct that is identical to
phonological material in the left periphery of the initial conjunct is allowed to be not present in the mother node.
This can be achieved by adding the constraints in (\ref{lpec}) to the coordination construction, here shown in the binary-branching format, for perspicuity.

\begin{exe}
\ex

\begin{avm}
%\textup{ 
\emph{coord-phr} \impl
\[phon \@{1}$\oplus$\@{2}$\oplus$\@{0}$\oplus$\@{3}%\\
    \\
 dtrs \< \[phon \@{1}$\oplus$\@{2}$_{ne-list}$\\
             synsem \| loc \| cat \| coord & none\],
 \[phon  \@{0}$\langle$(crd)$\rangle\oplus$\@{1}$\oplus$\@{3}$_{ne-list}$\\
   synsem \| loc \| cat \| coord  crd\]
 \> \]
%}
\end{avm}

%% \begin{avm}
%% \textup{ \emph{coord-phr} \impl
%% \[phon \@{1}$\oplus$\@{2}$\oplus$\@{3}%\\
%%     \\
%%  dtrs \< \[phon \@{1}$\oplus$\@{2}$_{ne-list}$\],
%%  \[phon  \@{1}$\oplus$\@{3}$_{ne-list}$\\
%%    synsem|loc|cat|coord & crd\]
%%  \> \]}
%%\end{avm}
\label{lpec}
\end{exe}
\inlinetodostefan{What does CRD stand for? Formally? It is small caps so this seems to point to a
  feature name. But this cannot be right. What is it?}

\noindent
If \avmbox{1} is resolved as the empty list then no ellipsis occurs, but if \avmbox{1} is non-empty then ellipsis occurs, as illustrated in Figure~\ref{lpe}. 
Some accounts, like  \citet{yatabe01}, \citet{Crysmann:04}, \citet{Beavers}, and \citet{chaveslp} operate on
linearization domain elements instead of directly on \textsc{phon}.  
See \crossrefchaptert{order} for more discussion about linearization theory.


\begin{figure}
    \centering
    
\Tree[.{VP\begin{avm}\[phon & \@{1}$\oplus$\@{2}$\oplus$\@{0}$\oplus$\@{3}\]\end{avm}} 
          {VP\begin{avm}
             \[phon & \@{1}$\langle$give$\rangle\oplus$\@{2}$\langle$a,book,to,Mary$\rangle$\]\end{avm}}
         [.{VP\begin{avm}
             \[phon & \@{0}$\langle$and$\rangle\oplus$\@{1}$\langle$give$\rangle\oplus$\@{3}$\langle$a,magazine,to,Sue$\rangle$\]\end{avm}}
          {Coord}
             {VP} ] ]
    \caption{Analysis of \emph{give a book to Mary and} \sout{\emph{give}} \emph{a magazine to Sue}}\label{lpe}
\end{figure}

\noindent
This approach is motivated by the existence of ambiguity in 
sentences like (\ref{treesa}), from \citet{Beavers} and \citet{chaves06}. Because (\ref{treesa}a) involves a one-time predicate, the ellipsis must include the subject phrase, otherwise
the intepretation is such that the same two trees are cut down twice. In contrast,  (\ref{treesa}b) does not involve a one-time
predicate, and thus is it possible for the ellipsis to simply
involve the verb.


\begin{exe}
\ex 
\begin{xlista}
\ex Two trees were cut down by Robin in July and by Alex in September.\\
(Two trees were cut down by Robin in July and \sout{two trees were cut down} by Alex in September.

\ex Two trees were photographed by Robin in July and by Alex in September.\\
(Two trees were photographed by Robin in July and \sout{photographed} by Alex in September)
\end{xlista}\label{treesa}
\end{exe}



In the non-elliptical analysis of such data, the missing material is recovered from the preceding conjunct. For example, \citeauthor{Mouret:06} proposes a rule along the lines of (\ref{lines}).  Here, a new head feature \textsc{cluster} is introduced, which  takes as its value the list of \textsc{synsem} 
values of the  daughters.

\begin{exe}
\ex 
\begin{avm} \type{ac-cx} \impl \[head & \[cluster \<\@{1}, \ldots{}, \@{n}\>\]\\
 dtrs & \< \[synsem \@{1}\], \ldots{}, \[synsem  \@{n}\] \> \]
\end{avm}\label{lines}
\end{exe}

\noindent
Mouret defines argument-clusters  as  instances  of  some  underspecified  non"=headed  construction 
\type{ac-cx} with  one  daughter  or  more.  The  construction  is  valence saturated.
He also postulates a lexical rule allowing (for example) a ditransitive verb to take a \textsc{cluster} as complement (this rule will also allow clusters for complements and adjuncts, assuming the latter are included in the \textsc{comps} list):

\begin{exe}
\ex \begin{avm} \[comps \<[loc|cat \@{1}], \ldots{}, [loc|cat \@{n}]\>\] \, $\mapsto$ \,
\[comps  \< \[coord & $+$ \\
cluster & \<[loc|cat \@{1}], \ldots{}, [loc|cat \@{n}]\> \] \>\]
\end{avm}
\end{exe}
\inlinetodostefan{Something is missing here. Are these clusters coordintated? And then the verb
  takes the coordination as argument? Maybe a figure would be good.}


\noindent
This approach is motivated by non clausal conjunctions (\textit{as well as, ainsi que}), which are possible in Argument Cluster Coordination, but cannot conjoin tensed VPs:

\begin{exe}
\ex 
\begin{xlista}
\ex[] {John gave a book to Mary as well as a magazine to Sue.}
\ex[*] {John gave a book to Mary as well as gave a magazine to Sue.}
\ex[] {Paul offrira un disque à Marie ainsi qu'un livre à Jean.\\ 
`Paul will.offer a record to Mary as well as a  book to Jean'\\
\citep{Abeille:Godard:1996}\addpages}
\end{xlista}
\end{exe}
\inlinetodostefan{This example is not in \citet{Abeille:Godard:1996}. Please provide correct reference.}

Another argument is the placement of correlative conjunctions: the first conjunction in (a) must be postverbal; this shows that Argument Cluster Coordination does not include the first verb. The examples below are from \citet[254]{Mouret:06}.

\inlinetodostefan{add glosses and translations to all (!) examples}
\begin{exe}
\ex 
\begin{xlista}
\ex[] {
\gll Jean a donné et un livre à Marie et un magazine à Sue.\\
     Jean has given  and a book to Marie and a magazine to Sue\\
\glt `Jean has given both a book to Marie and a magazine to Sue'
}

\ex[] {
\gll Paul compte         offrir et  un disque à  Marie et  un livre à Jean.\\
     Paul is.planning.to offer  and a  record to Marie and a  book  to Jean\\
}
\ex[*] {
\gll Paul compte et offrir un disque à Marie et un livre à Jean.\\
     Jean is.planning and to.offer a record to Marie and a book to Jean\\}
\end{xlista}
\end{exe}

\noindent
Another argument is negation placement, which is a case of constituent negation
\citep[253]{Mouret:06}: 

\inlinetodostefan{add glosses and translations to all (!) examples}
\begin{exe}
 \ex 
 \begin{xlista}
\ex[] {Paul offrira un disque à Marie et (non) pas un livre à Jean.\\
`Paul will offer a record to Marie and not a book to Jean'}
\ex[] {Paul gave a record to Mary and not a book to Bill. }
\ex[*] {Paul gave a record to Mary and not gave a book to Bill.}
% is it true in English? John gave not a book to Sue but a record to Bob
\end{xlista}
\end{exe}


A syntactic and non-elliptical account of Right-Node Raising is harder to maintain given that this phenomenon does not seem to be sensitive to  syntactic structure as (\ref{rnrex1}) shows. See 
\citet{bresnan74}, \citet[299]{wexlercull},  \citet[45]{grosu81}, \citet{mccawley}, and \citet[382, footnote 30]{sab}
for more data and discussion.\footnote{\citet{steedman85,gapsteed,steedmanbook}
and \citet[183]{dowty88} claim that
Right-Node Raising is bounded, nonetheless.
For example, \citet{
dowty88} argues that  *\emph{an idea that, and a robot which $[$can solve this problem$]$} is  evidence for islands in RNR. But as \citet[95]{phil}
 points out, this oddness is explained by semantic factors: it is impossible to   semantically contrast \emph{that} (which is semantically vacuous) with \emph{which}.
 \citet[17]{steedmanbook}   argues that RNR
exhibits islands effects by claiming that \emph{I hope that I will meet the woman
\textsc{who wrote} and you expect to interview the
consortium \textsc{that published} $[$that
novel about the secret life of legumes$]$} is ungrammatical.
In our experience, informants do not systematically share this judgment.}


\begin{exe}
\ex
\begin{xlista}
\ex  I know a man who \textsc{sells} and you know a person who \textsc{buys}
                     [pictures of Elvis Presley].

\ex John wonders when Bob Dylan
\textsc{wrote} and Mary wants to know when
  he
\textsc{recorded} [his great song about the death of Emmet Till].
 
 \ex Politicians \textsc{win when they defend} and \textsc{lose when they attack}
[the right of a woman to an abortion].

\ex Lucy \textsc{claimed} that -- but \textsc{couldn't say}
exactly when --  $[$the strike would take place$]$.
 
 \ex I found a box \textsc{in} which and Andrea found a blanket \textsc{under}
which [a cat could sleep peacefully for hours without being
noticed].
\end{xlista}\label{rnrex1}
\end{exe}

Another source of evidence against syntactic and non-elliptical accounts of Right-Node Raising is that this phenomenon can involve lexical structure,
as  (\ref{rnrex2}) illustrates:


\begin{exe}
\ex \begin{xlista}
\ex Please list all publications of which you were the \textsc{sole} or
\textsc{co}-[author].\\
 \citep[1325, footnote 44]{rodney2}.
 
\ex  It is neither \textsc{un}- nor \textsc{overly} [patriotic] to tread that path.
 
\ex The \textsc{ex-} or \textsc{current} [smokers] had a higher blood pressure.\\
\citep{chaveslp} 

\ex The \textsc{neuro}- and \textsc{cognitive} [sciences] are
presently in a state of rapid development
(\ldots{})\footnote{\url{http://opinionator.blogs.nytimes.com/2011/12/25/the-future-of-moral-machines/?hp},
  accessed 2020-03-09.}

\ex Are you talking about \textsc{a new}  or about \textsc{an ex}-[boyfriend]?

\end{xlista}\label{rnrex2}
\end{exe}



Elliptical accounts of Right-Node Raising are proposed by \citet{Beavers},
\citet{Yatabe:04}, \citet{chavesrnr} and others. The rule in (\ref{rnrcx}) illustrates the account adopted by 
 \citet{chavesrnr}  and \citet{aoi}
  in simplified format.\footnote{See \citet{chavesrnr} for more details about how `cumulative' Right-Node Raising is modeled by this rule, i.e.\
 cases like \emph{Mia lost -- and Fred spent -- $($a total of$)$ \$10.000}.}
In a nutshell, the \textsc{m(orpho-)p(honology)} feature introduces two list-valued features, namely \textsc{phon}(\textsc{ology}) and \textsc{l(exical-)id(entifier)}. The former encodes phonological content, including phonological phrasing information,  whereas the latter is used to individuate lexical items semantically (i.e.\  the value
of \textsc{lid} is a list of semantic frames that canonically specify the meaning of a lexeme).

 
\begin{exe}
\ex
\begin{avm}
{\small \type{rpe-cx} \impl
\[mp & \@{$L_1$}$\oplus$\@{$R_1$}$\oplus$\@{$R_2$}$\oplus$\@{$R_3$}\\
  synsem & \@{0}\\
 dtrs & \< \[mp   \@{$L_1$}$\oplus$\@{$L_2$}\<\[phon & \@{$p_1$}\\ lid & \@{1}\],\ldots{}, \[phon & \@{$p_n$}\\
 lid & \@{n}\]\>$\oplus$\\
 \hspace{0.7cm}\@{$R_1$}$\oplus$\@{$R_2$}\<\[phon & \@{$p_1$}\\ lid & \@{1}\],\ldots{}, \[phon & \@{$p_n$}\\
 lid & \@{n}\]\>$\oplus$\@{$R_3$}\\
 synsem  \@{0}
             \] \> \]}
\end{avm}\label{rnrcx}
\end{exe}

\noindent
By requiring \textsc{phon} identity, this rule ensures that Right-Node Raising only targets strings that
are phonologically independent, and have the same surface form, ruling out the ungrammatical examples in (\ref{badp1}).
The assumption here is that the value of \textsc{phon} is not simply a list of phonemes, but rather a structured list containing  intonational phrases, 
 phonological phrases, prosodic words, syllables, and segments.

Stressed pronouns, affixes that correspond to independent prosodic words, and compound parts can be RNRaised because  they are  independent prosodic units in their local domains.
See \citet{swingle} for more discussion. 

\begin{exe}
\ex \begin{xlista}
\ex[]  {He tried \textsc{to persuade}  but he couldn't \textsc{convince} [THEM] / *[them].}
\ex[*] {I think that \textsc{I'd} and I know that \textsc{Pat'll} [buy  those portraits of Elvis].}
\ex[*] {They've always \textsc{wanted} a -- and so I've \textsc{given them} a --  [coffee grinder].}
\ex[*] {I bought  \textsc{every red} and Jo liked \textsc{some blue} [t-shirt].}
\end{xlista}\label{badp1}
\end{exe}


\noindent
By requiring  \textsc{lid} identity, the rule prevents homophonous strings that have fundamentally different semantics from being Right-Node Raised, as in (\ref{badp2}). In such cases,  oddness arises because in general the same phrase cannot simultaneously have  two meanings, except in puns  \citep[316]{zaenenkart}. 

\begin{exe}
\ex \begin{xlista} 
\ex[*] {Randy \textsc{saw} and Rene has \textsc{been} [flying planes].}
\ex[*] {Jo \textsc{will} and Sandy \textsc{built the} [drive].\\
\citep{Milward94}\addpages
}
\ex[*] {Mary \textsc{fed} and Tom \textsc{enjoyed} [the lamb].\\
(adapted from \citet[64]{buitelaar1998corelex})}
\ex[*] {Robin \textsc{swung}  and Leslie \textsc{tamed} [an unusual bat].\\
\citep[156]{levhubook}}
    \ex[*] {We need new \textsc{black}- and \textsc{floor}[boards].\\
    \citep{artstein5}}
\ex[*] {We caught \textsc{butter}- and \textsc{fire}[flies].\\
\citep{chaveslp}}
\ex[*] {There stood a \textsc{one-} and \textsc{well}-[armed man].\\
\citep{chavesrnr}}
\end{xlista}\label{badp2}
\end{exe}

\noindent
At the same time, \textsc{lid} identity does not go as far as requiring co-referentiality of the shared material. This is  as intended given ambiguous examples like
\emph{Chris \textsc{likes} and Bill \textsc{loves}  $[$his bike$]$}.
The account of Right-Node Raising is illustrated below. Here, \emph{I} corresponds to an intonational phrase,
and  $\phi$ to a phonological phrase.
Note that this is a unary-branching rule, which means that it can in principle apply to any phrasal node, including non-coordinate cases of RNR:


\begin{figure}
    \centering

    \scalebox{1}{
    {\begin{avm}
S\[\type{phrase}\\
mp \<
\[phon \<\[$I$\\
    \[$\phi$\\
       /\textipa{kIm} \textipa{lAIks} /\]\]\>\\
       lid \ldots{}\],
    \[phon \<\[$I$\\
     \[$\phi$\\
            /\textipa{\ae nd} \textipa{mij@} \textipa{heIts}/\]\]\>\\
            lid \ldots{}\],
         \[phon \<\[$I$\\ \[$\phi$\\  /\textipa{beIg@lz}/\]\]\>\\
         lid \ldots{} \]\>  \]\end{avm}}}
         
      $|$   
      
    \scalebox{1}{
\begin{avm}
S\[\type{phrase}\\
mp
\<  \[phon \< \[$I$\\
    \[$\phi$\\
       /\textipa{kIm} \textipa{lAIks} /\] \]\>\\
      lid ..\],
        \[phon    \<\[$I$\\ \[$\phi$\\  /\textipa{beIg@lz}/\]\]\>\\
              lid \ldots{}\],\\
     \[phon \<\[$I$\\
     \[$\phi$\\
            /\textipa{\ae nd} \textipa{mij@} \textipa{heIts}/\]\]\>\\
          lid \ldots{}\],
         \[phon \<\[$I$\\ \[$\phi$\\  /\textipa{beIg@lz}/ \]\]\>\\
              lid \ldots{}\]   \>  \]\end{avm}}
    
%%     \scalebox{1}{
%%     {\begin{avm}
%% S\[\type{phrase}\\
%% mp \<
%% \[phon \[$I$\\
%%     \[$\phi$\\
%%        /\textipa{kIm} \textipa{lAIks} /\]\]\\
%%        lid \ldots{}\],
%%     \[phon \[$I$\\
%%      \[$\phi$\\
%%             /\textipa{\ae nd} \textipa{mij@} \textipa{heIts}/\]\]\\
%%             lid \ldots{}\],
%%          \[phon \[$I$\\ \[$\phi$\\  /\textipa{beIg@lz}/\]\]\\
%%          lid \ldots{} \]\>  \]\end{avm}}}
         
%%       $|$   
      
%%     \scalebox{1}{
%% \begin{avm}
%% S\[\type{phrase}\\
%% mp
%% \<  \[phon \[$I$\\
%%     \[$\phi$\\
%%        /\textipa{kIm} \textipa{lAIks} /\] \]\\
%%       lid ..\],
%%         \[phon    \[$I$\\ \[$\phi$\\  /\textipa{beIg@lz}/\]\]\\
%%               lid \ldots{}\],\\
%%      \[phon \[$I$\\
%%      \[$\phi$\\
%%             /\textipa{\ae nd} \textipa{mij@} \textipa{heIts}/\]\]\\
%%           lid \ldots{}\],
%%          \[phon \[$I$\\ \[$\phi$\\  /\textipa{beIg@lz}/ \]\]\\
%%               lid \ldots{}\]   \>  \]\end{avm}}
              
               
                         \scalebox{0.9}{
         \Tree[.{    
\qroof{Kim
\textsc{likes} bagels and Mia \textsc{hates} bagels}.{\,}} ] }

               
    \caption{Analysis of \emph{Kim likes, and Mia hates, bagels}}\label{rnrt}
\end{figure}
\inlinetodostefan{I and phi are strange. There is no feature but a bracket for embedding. This needs
  explanation. It is formally incoherent.}




\begin{exe}
\ex \begin{xlista}
\ex  It's interesting to compare the people who \textsc{like} with the people
       who \textsc{dislike} [the power of the big unions].\\
       \citep[550]{hudson}

 \ex Anyone  who \textsc{meets} really comes to \textsc{like} [our sales people].\\
 (adapted from \citet{williams})\label{will}


\ex   Spies who learn \textsc{when} can be more valuable than those
able to learn \textsc{where} [major troop movements are going to occur].

\ex Politicians who fought \textsc{for} may well snub those
 who have fought \textsc{against} [chimpanzee rights]. \\
\citep{postal94}\addpages

\ex Those who voted \textsc{against} far outnumbered those who
voted  \textsc{for} [my father's motion].\\
\citep[1344]{rodney2}


\ex If there are people who \textsc{oppose} then maybe there are also some
  people who actually \textsc{support}  [the hiring of unqualified
  workers].\\
  \citep[\page 840]{chavesrnr}

\end{xlista}


\end{exe}




In the example in Figure~\ref{rnrt}  the sub-list \avmbox{R$_3$} is resolved as the empty list, but this need not be so. When the latter sublist is not resolved as the empty list, we obtain discontinuous Right-Node Raising cases like (\ref{disrnr}), 
where the RNRaised expression is followed by extra material.


\begin{exe}
\ex \begin{xlista}
\ex The blast \textsc{upended} and \underline{\textsc{nearly sliced}} [an armored Chevrolet Suburban] \underline{in half}.

\ex During the War of 1982, American troops
\textsc{occupied}  and \underline{\textsc{burned}} [the town] \underline{to the ground}.

\ex Please move from the exit rows if you are \textsc{unwilling} or \underline{\textsc{unable}}
 [to perform the necessary actions] \underline{without injury}.

\ex The troops that \textsc{occupied} ended up \underline{\textsc{burning}}
[the town] \underline{to the ground}.

\end{xlista}\label{disrnr}
\end{exe}



Finally, let us now turn our attention to Gapping, as in 
\emph{Robin likes Sam and Tim -- Sue}.
There are elliptical accounts of Gapping  \citep{chaves06} as well as direct-interpretation accounts where the missing material is recovered from the preceding linguistic context  \citep{Mouret:06,Abeille:Blbie:Mouret:14,sangheepark}; see \crossrefchaptert{ellipsis}. The latter is illustrated in Figure~\ref{gfig}, in simplified format. Basically, the Question Under Discussion (QUD, \citealp{roberts96}) of the first clause is $\lambda y.\lambda x. \exists e(like(x,y))$ which is information that shared across the clausal daughters as \avmbox{1}.
This allows the second conjunct to combine the two NPs with the verbal semantics, and recover the propositional meaning.

%\fi

\begin{figure}
\oneline{%
\begin{forest}
%sm edges
[S\ms{
      qud     & \ibox{1}\\
      content & \menge{ $\exists e'(like(robin,sam)) \wedge \exists e(like(tim,sue))$ }\\
     } 
 [S\ms{  qud & \ibox{1}\\
         content & \menge{ \ibox{0} $\exists e'(like(robin,sam))$ } % }
      } ]
 [S\ms{ qud & \ibox{1}\\
          content & \menge{ \ibox{0} $\wedge \exists e(like(tim,sue))$ }
        }
 [Conj [and]]
 [S\ms{ qud & \ibox{1} \menge{ $\lambda y. \lambda x. \exists e(like(x,y))$ }\\
        content & \menge{ $\exists e(like(t,s))$ } } 
   [NP\ms{ content & \menge{ $tim$ }}]
   [NP\ms{ content & \menge{ $sue$ }}] ]
]]
\end{forest}
}
\caption{Analysis of \emph{Robin likes Sam and Tim -- Sue} (abbreviated)}\label{gfig}
\end{figure}


%% \begin{figure}
%%     \centering
%% \begin{forest}
%% %sm edges
%% [\begin{avm}S\[qud & \@{1}\\
%%                 content & \{ $\exists e'(like(robin,sam)) \wedge \exists e(like(tim,sue))$ \}\]\end{avm}
%%  [\begin{avm}S\[qud & \@{1}\\
%%                 content & \{ \@{0}$\exists e'(like(robin,sam))$\}\]\end{avm}]
%%  [\begin{avm}S\[qud & \@{1}\\
%%                 content & \{ \@{0} $\wedge \exists e(like(tim,sue))$\}\]\end{avm}
%%    [Conj [and] ]
%%    [\begin{avm}S\[qud & \@{1}\{$\lambda y. \lambda x. \exists e(like(x,y))$\}\\
%%                 content & \{$\exists e(like(t,s))$ \}\]\end{avm}
%%      [\begin{avm}NP\[content & \{$tim$ \}\]\end{avm}]
%%      [\begin{avm}NP\[content & \{$sue$ \}\]\end{avm}] ] ] ]
%% \end{forest} 
%%     \caption{Analysis of \emph{Robin likes Sam and Tim -- Sue} (abbreviated)}\label{gfig}
%% \end{figure}


Like RNR, Gapping is not restricted to coordinate structures as (\ref{gnc}) illustrates, contrary to
widespread assumption.   Thus, the Gapping rule proposed by \citet{sangheepark}  that allows a gapped clause to follow a non-gapped clause, is not specific to coordination. 

\begin{exe}
\ex 
\begin{xlista}
\ex Robin speaks French better than Leslie -- German.
\ex My purpose here is not to resolve the crucial disagreement between two prominent theoreticians in a way that one would be declared true while the other one -- false.
\ex The keynote of their relationship was set when Victoria, already a reigning queen,
had to propose to Albert, rather than he -- to her.
\ex The public remembers all that and usually recognizes us before we -- them.\\
\citep[\page 31]{sangheepark}
\end{xlista}\label{gnc}
\end{exe}




\section{Conclusion}

Coordination is a pervasive phenomenon in all natural languages. Despite intensive research in the last 70 years, its empirical properties continue to challenge most linguistic theories:  the coordination lexemes play a crucial role but do not behave like usual syntactic heads, the conjuncts do not need to be identical but display some parallelism relations and can be unlimited in number, some non constituent sequences can be coordinated, peculiar ellipsis phenomena can optionally occur,  etc. We have shown how HPSG offers precise detailed analyses of various coordinate constructions for a wide variety of languages, factoring out the common properties shared by other constructions and the properties specific to coordination.

Central to the HPSG analyses are two main ideas: (i) coordination structures are non-headed phrases and come with different subtypes; (ii) the parallelism between coordinate daughters is captured by feature sharing; from these ideas, specific properties can be derived, regarding extraction and agreement for instance. Nevertheless, there is no clear consensus about some remaining issues. In some accounts, the coordinator is a weak head, whereas in others it is a marker. Coordinate structures are binary branching in some accounts but not so in others. Agreement is always local (with the whole coordinate phrase) in some account, whereas locality is abandoned by others to account for Closest Conjunct agreement. Finally, in some accounts non-constituent coordination involves some form of deletion, but in others no deletion operation is assumed.

}%avmoptions
 
%\section*{Abbreviations}
\section*{Acknowledgements}

We are thankful to Bob Borsley, Stefan M\"{u}ller, and other reviewers for comments and suggestions on earlier drafts. 
As usual, all errors and omissions are our own.


{\sloppy
\printbibliography[heading=subbibliography,notkeyword=this] 
}
\end{document}


%      <!-- Local IspellDict: en_US-w_accents -->
