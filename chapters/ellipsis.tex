\documentclass[output=paper
                ,modfonts
                ,nonflat
	        ,collection
	        ,collectionchapter
	        ,collectiontoclongg
 	        ,biblatex
                ,babelshorthands
                ,newtxmath
                ,draftmode
                ,colorlinks, citecolor=brown
]{./langsci/langscibook}

\IfFileExists{../localcommands.tex}{%hack to check whether this is being compiled as part of a collection or standalone
  % add all extra packages you need to load to this file 

\usepackage{graphicx}
\usepackage{tabularx}
\usepackage{amsmath} 
\usepackage{tipa}      % Davis Koenig
\usepackage{multicol}
\usepackage{lipsum}


\usepackage{./langsci/styles/langsci-optional} 
\usepackage{./langsci/styles/langsci-lgr}
%\usepackage{./styles/forest/forest}
\usepackage{./langsci/styles/langsci-forest-setup}
\usepackage{morewrites}

\usepackage{tikz-cd}

\usepackage{./styles/tikz-grid}
\usetikzlibrary{shadows}


%\usepackage{pgfplots} % for data/theory figure in minimalism.tex
% fix some issue with Mod https://tex.stackexchange.com/a/330076
\makeatletter
\let\pgfmathModX=\pgfmathMod@
\usepackage{pgfplots}%
\let\pgfmathMod@=\pgfmathModX
\makeatother

\usepackage{subcaption}

% Stefan Müller's styles
\usepackage{./styles/merkmalstruktur,german,./styles/makros.2e,./styles/my-xspace,./styles/article-ex,
./styles/eng-date}

\selectlanguage{USenglish}

\usepackage{./styles/abbrev}

\usepackage{./langsci/styles/jambox}

% Has to be loaded late since otherwise footnotes will not work

%%%%%%%%%%%%%%%%%%%%%%%%%%%%%%%%%%%%%%%%%%%%%%%%%%%%
%%%                                              %%%
%%%           Examples                           %%%
%%%                                              %%%
%%%%%%%%%%%%%%%%%%%%%%%%%%%%%%%%%%%%%%%%%%%%%%%%%%%%
% remove the percentage signs in the following lines
% if your book makes use of linguistic examples
\usepackage{./langsci/styles/langsci-gb4e} 

% Crossing out text
% uncomment when needed
%\usepackage{ulem}

\usepackage{./styles/additional-langsci-index-shortcuts}

%\usepackage{./langsci/styles/langsci-avm}
\usepackage{./styles/avm+}


\renewcommand{\tpv}[1]{{\avmjvalfont\itshape #1}}

% no small caps please
\renewcommand{\phonshape}[0]{\normalfont\itshape}

\regAvmFonts

\usepackage{theorem}

\newtheorem{mydefinition}{Def.}
\newtheorem{principle}{Principle}

{\theoremstyle{break}
%\newtheorem{schema}{Schema}
\newtheorem{mydefinition-break}[mydefinition]{Def.}
\newtheorem{principle-break}[principle]{Principle}
}

% This avoids linebreaks in the Schema
\newcounter{schema}
\newenvironment{schema}[1][]
  {% \begin{Beispiel}[<title>]
  \goodbreak%
  \refstepcounter{schema}%
  \begin{list}{}{\setlength{\labelwidth}{0pt}\setlength{\labelsep}{0pt}\setlength{\rightmargin}{0pt}\setlength{\leftmargin}{0pt}}%
    \item[{\textbf{Schema~\theschema}}]\hspace{.5em}\textbf{(#1)}\nopagebreak[4]\par\nobreak}%
  {\end{list}}% \end{Beispiel}

%% \newcommand{schema}[2]{
%% \begin{minipage}{\textwidth}
%% {\textbf{Schema~\theschema}}]\hspace{.5em}\textbf{(#1)}\\
%% #2
%% \end{minipage}}

%\usepackage{subfig}





% Davis Koenig Lexikon

\usepackage{tikz-qtree,tikz-qtree-compat} % Davis Koenig remove

\usepackage{shadow}




\usepackage[english]{isodate} % Andy Lücking
\usepackage[autostyle]{csquotes} % Andy
%\usepackage[autolanguage]{numprint}

%\defaultfontfeatures{
%    Path = /usr/local/texlive/2017/texmf-dist/fonts/opentype/public/fontawesome/ }

%% https://tex.stackexchange.com/a/316948/18561
%\defaultfontfeatures{Extension = .otf}% adds .otf to end of path when font loaded without ext parameter e.g. \newfontfamily{\FA}{FontAwesome} > \newfontfamily{\FA}{FontAwesome.otf}
%\usepackage{fontawesome} % Andy Lücking
\usepackage{pifont} % Andy Lücking -> hand

\usetikzlibrary{decorations.pathreplacing} % Andy Lücking
\usetikzlibrary{matrix} % Andy 
\usetikzlibrary{positioning} % Andy
\usepackage{tikz-3dplot} % Andy

% pragmatics
\usepackage{eqparbox} % Andy
\usepackage{enumitem} % Andy
\usepackage{longtable} % Andy
\usepackage{tabu} % Andy


% Manfred's packages

%\usepackage{shadow}

\usepackage{tabularx}
\newcolumntype{L}[1]{>{\raggedright\arraybackslash}p{#1}} % linksbündig mit Breitenangabe


% Jong-Bok

%\usepackage{xytree}

\newcommand{\xytree}[2][dummy]{Let's do the tree!}

% seems evil, get rid of it
% defines \ex is incompatible with gb4e
%\usepackage{lingmacros}

% taken from lingmacros:
\makeatletter
% \evnup is used to line up the enumsentence number and an entry along
% the top.  It can take an argument to improve lining up.
\def\evnup{\@ifnextchar[{\@evnup}{\@evnup[0pt]}}

\def\@evnup[#1]#2{\setbox1=\hbox{#2}%
\dimen1=\ht1 \advance\dimen1 by -.5\baselineskip%
\advance\dimen1 by -#1%
\leavevmode\lower\dimen1\box1}
\makeatother


% YK -- CG chapter

%\usepackage{xspace}
\usepackage{bm}
\usepackage{bussproofs}


% Antonio Branco, remove this
\usepackage{epsfig}

% now unicode
%\usepackage{alphabeta}



% Berthold udc
%\usepackage{qtree}
%\usepackage{rtrees}

\usepackage{pst-node}

  %add all your local new commands to this file

\makeatletter
\def\blx@maxline{77}
\makeatother


\newcommand{\page}{}



\newcommand{\todostefan}[1]{\todo[color=orange!80]{\footnotesize #1}\xspace}
\newcommand{\todosatz}[1]{\todo[color=red!40]{\footnotesize #1}\xspace}

\newcommand{\inlinetodostefan}[1]{\todo[color=green!40,inline]{\footnotesize #1}\xspace}


\newcommand{\spacebr}{\hspaceThis{[}}

\newcommand{\danish}{\jambox{(\ili{Danish})}}
\newcommand{\english}{\jambox{(\ili{English})}}
\newcommand{\german}{\jambox{(\ili{German})}}
\newcommand{\yiddish}{\jambox{(\ili{Yiddish})}}
\newcommand{\welsh}{\jambox{(\ili{Welsh})}}

% Cite and cross-reference other chapters
\newcommand{\crossrefchaptert}[2][]{\citet*[#1]{chapters/#2}, Chapter~\ref{chap-#2} of this volume} 
\newcommand{\crossrefchapterp}[2][]{(\citealp*[#1][]{chapters/#2}, Chapter~\ref{chap-#2} of this volume)}
% example of optional argument:
% \crossrefchapterp[for something, see:]{name}
% gives: (for something, see: Author 2018, Chapter~X of this volume)

\let\crossrefchapterw\crossrefchaptert



% Davis Koenig

\let\ig=\textsc
\let\tc=\textcolor

% evolution, Flickinger, Pollard, Wasow

\let\citeNP\citet

% Adam P

%\newcommand{\toappear}{Forthcoming}
\newcommand{\pg}[1]{p.#1}
\renewcommand{\implies}{\rightarrow}

\newcommand*{\rref}[1]{(\ref{#1})}
\newcommand*{\aref}[1]{(\ref{#1}a)}
\newcommand*{\bref}[1]{(\ref{#1}b)}
\newcommand*{\cref}[1]{(\ref{#1}c)}

\newcommand{\msadam}{.}
\newcommand{\morsyn}[1]{\textsc{#1}}

\newcommand{\nom}{\morsyn{nom}}
\newcommand{\acc}{\morsyn{acc}}
\newcommand{\dat}{\morsyn{dat}}
\newcommand{\gen}{\morsyn{gen}}
\newcommand{\ins}{\morsyn{ins}}
\newcommand{\loc}{\morsyn{loc}}
\newcommand{\voc}{\morsyn{voc}}
\newcommand{\ill}{\morsyn{ill}}
\renewcommand{\inf}{\morsyn{inf}}
\newcommand{\passprc}{\morsyn{passp}}

%\newcommand{\Nom}{\msadam\nom}
%\newcommand{\Acc}{\msadam\acc}
%\newcommand{\Dat}{\msadam\dat}
%\newcommand{\Gen}{\msadam\gen}
\newcommand{\Ins}{\msadam\ins}
\newcommand{\Loc}{\msadam\loc}
\newcommand{\Voc}{\msadam\voc}
\newcommand{\Ill}{\msadam\ill}
\newcommand{\INF}{\msadam\inf}
\newcommand{\PassP}{\msadam\passprc}

\newcommand{\Aux}{\textsc{aux}}

\newcommand{\princ}[1]{\textnormal{\textsc{#1}}} % for constraint names
\newcommand{\notion}[1]{\emph{#1}}
\renewcommand{\path}[1]{\textnormal{\textsc{#1}}}
\newcommand{\ftype}[1]{\textit{#1}}
\newcommand{\fftype}[1]{{\scriptsize\textit{#1}}}
\newcommand{\la}{$\langle$}
\newcommand{\ra}{$\rangle$}
%\newcommand{\argst}{\path{arg-st}}
\newcommand{\phtm}[1]{\setbox0=\hbox{#1}\hspace{\wd0}}
\newcommand{\prep}[1]{\setbox0=\hbox{#1}\hspace{-1\wd0}#1}

%%%%%%%%%%%%%%%%%%%%%%%%%%%%%%%%%%%%%%%%%%%%%%%%%%%%%%%%%%%%%%%%%%%%%%%%%%%

% FROM FS.STY:

%%%
%%% Feature structures
%%%

% \fs         To print a feature structure by itself, type for example
%             \fs{case:nom \\ person:P}
%             or (better, for true italics),
%             \fs{\it case:nom \\ \it person:P}
%
% \lfs        To print the same feature structure with the category
%             label N at the top, type:
%             \lfs{N}{\it case:nom \\ \it person:P}

%    Modified 1990 Dec 5 so that features are left aligned.
\newcommand{\fs}[1]%
{\mbox{\small%
$
\!
\left[
  \!\!
  \begin{tabular}{l}
    #1
  \end{tabular}
  \!\!
\right]
\!
$}}

%     Modified 1990 Dec 5 so that features are left aligned.
%\newcommand{\lfs}[2]
%   {
%     \mbox{$
%           \!\!
%           \begin{tabular}{c}
%           \it #1
%           \\
%           \mbox{\small%
%                 $
%                 \left[
%                 \!\!
%                 \it
%                 \begin{tabular}{l}
%                 #2
%                 \end{tabular}
%                 \!\!
%                 \right]
%                 $}
%           \end{tabular}
%           \!\!
%           $}
%   }

\newcommand{\ft}[2]{\path{#1}\hspace{1ex}\ftype{#2}}
\newcommand{\fsl}[2]{\fs{{\fftype{#1}} \\ #2}}

\newcommand{\fslt}[2]
 {\fst{
       {\fftype{#1}} \\
       #2 
     }
 }

\newcommand{\fsltt}[2]
 {\fstt{
       {\fftype{#1}} \\
       #2 
     }
 }

\newcommand{\fslttt}[2]
 {\fsttt{
       {\fftype{#1}} \\
       #2 
     }
 }


% jak \ft, \fs i \fsl tylko nieco ciasniejsze

\newcommand{\ftt}[2]
% {{\sc #1}\/{\rm #2}}
 {\textsc{#1}\/{\rm #2}}

\newcommand{\fst}[1]
  {
    \mbox{\small%
          $
          \left[
          \!\!\!
%          \sc
          \begin{tabular}{l} #1
          \end{tabular}
          \!\!\!\!\!\!\!
          \right]
          $
          }
   }

%\newcommand{\fslt}[2]
% {\fst{#2\\
%       {\scriptsize\it #1}
%      }
% }


% superciasne

\newcommand{\fstt}[1]
  {
    \mbox{\small%
          $
          \left[
          \!\!\!
%          \sc
          \begin{tabular}{l} #1
          \end{tabular}
          \!\!\!\!\!\!\!\!\!\!\!
          \right]
          $
          }
   }

%\newcommand{\fsltt}[2]
% {\fstt{#2\\
%       {\scriptsize\it #1}
%      }
% }

\newcommand{\fsttt}[1]
  {
    \mbox{\small%
          $
          \left[
          \!\!\!
%          \sc
          \begin{tabular}{l} #1
          \end{tabular}
          \!\!\!\!\!\!\!\!\!\!\!\!\!\!\!\!
          \right]
          $
          }
   }



% %add all your local new commands to this file

% \newcommand{\smiley}{:)}

% you are not supposed to mess with hardcore stuff, St.Mü. 22.08.2018
%% \renewbibmacro*{index:name}[5]{%
%%   \usebibmacro{index:entry}{#1}
%%     {\iffieldundef{usera}{}{\thefield{usera}\actualoperator}\mkbibindexname{#2}{#3}{#4}{#5}}}

% % \newcommand{\noop}[1]{}



% Rui

\newcommand{\spc}[0]{\hspace{-1pt}\underline{\hspace{6pt}}\,}
\newcommand{\spcs}[0]{\hspace{-1pt}\underline{\hspace{6pt}}\,\,}
\newcommand{\bad}[1]{\leavevmode\llap{#1}}
\newcommand{\COMMENT}[1]{}


% Andy Lücking gesture.tex
\newcommand{\Pointing}{\ding{43}}
% Giotto: "Meeting of Joachim and Anne at the Golden Gate" - 1305-10 
\definecolor{GoldenGate1}{rgb}{.13,.09,.13} % Dress of woman in black
\definecolor{GoldenGate2}{rgb}{.94,.94,.91} % Bridge
\definecolor{GoldenGate3}{rgb}{.06,.09,.22} % Blue sky
\definecolor{GoldenGate4}{rgb}{.94,.91,.87} % Dress of woman with shawl
\definecolor{GoldenGate5}{rgb}{.52,.26,.26} % Joachim's robe
\definecolor{GoldenGate6}{rgb}{.65,.35,.16} % Anne's robe
\definecolor{GoldenGate7}{rgb}{.91,.84,.42} % Joachim's halo

\makeatletter
\newcommand{\@Depth}{1} % x-dimension, to front
\newcommand{\@Height}{1} % z-dimension, up
\newcommand{\@Width}{1} % y-dimension, rightwards
%\GGS{<x-start>}{<y-start>}{<z-top>}{<z-bottom>}{<Farbe>}{<x-width>}{<y-depth>}{<opacity>}
\newcommand{\GGS}[9][]{%
\coordinate (O) at (#2-1,#3-1,#5);
\coordinate (A) at (#2-1,#3-1+#7,#5);
\coordinate (B) at (#2-1,#3-1+#7,#4);
\coordinate (C) at (#2-1,#3-1,#4);
\coordinate (D) at (#2-1+#8,#3-1,#5);
\coordinate (E) at (#2-1+#8,#3-1+#7,#5);
\coordinate (F) at (#2-1+#8,#3-1+#7,#4);
\coordinate (G) at (#2-1+#8,#3-1,#4);
\draw[draw=black, fill=#6, fill opacity=#9] (D) -- (E) -- (F) -- (G) -- cycle;% Front
\draw[draw=black, fill=#6, fill opacity=#9] (C) -- (B) -- (F) -- (G) -- cycle;% Top
\draw[draw=black, fill=#6, fill opacity=#9] (A) -- (B) -- (F) -- (E) -- cycle;% Right
}
\makeatother


% pragmatics
\newcommand{\speaking}[1]{\eqparbox{name}{\textsc{\lowercase{#1}\space}}}
\newcommand{\name}[1]{\eqparbox{name}{\textsc{\lowercase{#1}}}}
\newcommand{\HPSGTTR}{HPSG$_{\text{TTR}}$\xspace}

\newcommand{\ttrtype}[1]{\textit{#1}}
% \newcommand{\avmel}{\q<\quad\q>} %% shortcut for empty lists in AVM
\newcommand{\ttrmerge}{\ensuremath{\wedge_{\textit{merge}}}}
\newcommand{\Cat}[2][0.1pt]{%
  \begin{scope}[y=#1,x=#1,yscale=-1, inner sep=0pt, outer sep=0pt]
   \path[fill=#2,line join=miter,line cap=butt,even odd rule,line width=0.8pt]
  (151.3490,307.2045) -- (264.3490,307.2045) .. controls (264.3490,291.1410) and (263.2021,287.9545) .. (236.5990,287.9545) .. controls (240.8490,275.2045) and (258.1242,244.3581) .. (267.7240,244.3581) .. controls (276.2171,244.3581) and (286.3490,244.8259) .. (286.3490,264.2045) .. controls (286.3490,286.2045) and (323.3717,321.6755) .. (332.3490,307.2045) .. controls (345.7277,285.6390) and (309.3490,292.2151) .. (309.3490,240.2046) .. controls (309.3490,169.0514) and (350.8742,179.1807) .. (350.8742,139.2046) .. controls (350.8742,119.2045) and (345.3490,116.5037) .. (345.3490,102.2045) .. controls (345.3490,83.3070) and (361.9972,84.4036) .. (358.7581,68.7349) .. controls (356.5206,57.9117) and (354.7696,49.2320) .. (353.4652,36.1439) .. controls (352.5396,26.8573) and (352.2445,16.9594) .. (342.5985,17.3574) .. controls (331.2650,17.8250) and (326.9655,37.7742) .. (309.3490,39.2045) .. controls (291.7685,40.6320) and (276.7783,24.2380) .. (269.9740,26.5795) .. controls (263.2271,28.9013) and (265.3490,47.2045) .. (269.3490,60.2045) .. controls (275.6359,80.6368) and (289.3490,107.2045) .. (264.3490,111.2045) .. controls (239.3490,115.2045) and (196.3490,119.2045) .. (165.3490,160.2046) .. controls (134.3490,201.2046) and (135.4934,249.3212) .. (123.3490,264.2045) .. controls (82.5907,314.1553) and (40.8239,293.6463) .. (40.8239,335.2045) .. controls (40.8239,353.8102) and (72.3490,367.2045) .. (77.3490,361.2045) .. controls (82.3490,355.2045) and (34.8638,337.3259) .. (87.9955,316.2045) .. controls (133.3871,298.1601) and   (137.4391,294.4766) .. (151.3490,307.2045) -- cycle;
\end{scope}%
}


% KdK
\newcommand{\smiley}{:)}

\renewbibmacro*{index:name}[5]{%
  \usebibmacro{index:entry}{#1}
    {\iffieldundef{usera}{}{\thefield{usera}\actualoperator}\mkbibindexname{#2}{#3}{#4}{#5}}}

% \newcommand{\noop}[1]{}

% chngcntr.sty otherwise gives error that these are already defined
%\let\counterwithin\relax
%\let\counterwithout\relax

% the space of a left bracket for glossings
\newcommand{\LB}{\hspaceThis{[}}

\newcommand{\LF}{\mbox{$[\![$}}

\newcommand{\RF}{\mbox{$]\!]_F$}}

\newcommand{\RT}{\mbox{$]\!]_T$}}





% Manfred's

\newcommand{\kommentar}[1]{}

\newcommand{\bsp}[1]{\emph{#1}}
\newcommand{\bspT}[2]{\bsp{#1} `#2'}
\newcommand{\bspTL}[3]{\bsp{#1} (lit.: #2) `#3'}

\newcommand{\noidi}{§}

\newcommand{\refer}[1]{(\ref{#1})}

%\newcommand{\avmtype}[1]{\multicolumn{2}{l}{\type{#1}}}
\newcommand{\attr}[1]{\textsc{#1}}

\newcommand{\srdefault}{\mbox{\begin{tabular}{c}{\large <}\\[-1.5ex]$\sqcap$\end{tabular}}}

%% \newcommand{\myappcolumn}[2]{
%% \begin{minipage}[t]{#1}#2\end{minipage}
%% }

%% \newcommand{\appc}[1]{\myappcolumn{3.7cm}{#1}}


% Jong-Bok


% clean that up and do not use \def (killing other stuff defined before)
%\if 0
\def\DEL{\textsc{del}}
\def\del{\textsc{del}}

\def\conn{\textsc{conn}}
\def\CONN{\textsc{conn}}
\def\CONJ{\textsc{conj}}
\def\LITE{\textsc{lex}}
\def\lite{\textsc{lex}}
\def\HON{\textsc{hon}}

\def\CAUS{\textsc{caus}}
\def\PASS{\textsc{pass}}
\def\NPST{\textsc{npst}}
\def\COND{\textsc{cond}}



\def\hd-lite{\textsc{head-lex construction}}
\def\NFORM{\textsc{nform}}

\def\RELS{\textsc{rels}}
\def\TENSE{\textsc{tense}}


%\def\ARG{\textsc{arg}}
\def\ARGs{\textsc{arg0}}
\def\ARGa{\textsc{arg}}
\def\ARGb{\textsc{arg2}}
\def\TPC{\textsc{top}}
\def\PROG{\textsc{prog}}

\def\pst{\textsc{pst}}
\def\PAST{\textsc{pst}}
\def\DAT{\textsc{dat}}
\def\CONJ{\textsc{conj}}
\def\nominal{\textsc{nominal}}
\def\NOMINAL{\textsc{nominal}}
\def\VAL{\textsc{val}}
\def\val{\textsc{val}}
\def\MODE{\textsc{mode}}
\def\RESTR{\textsc{restr}}
\def\SIT{\textsc{sit}}
\def\ARG{\textsc{arg}}
\def\RELN{\textsc{rel}}
\def\REL{\textsc{rel}}
\def\RELS{\textsc{rels}}
\def\arg-st{\textsc{arg-st}}
\def\xdel{\textsc{xdel}}
\def\zdel{\textsc{zdel}}
\def\sug{\textsc{sug}}
\def\IMP{\textsc{imp}}
\def\conn{\textsc{conn}}
\def\CONJ{\textsc{conj}}
\def\HON{\textsc{hon}}
\def\BN{\textsc{bn}}
\def\bn{\textsc{bn}}
\def\pres{\textsc{pres}}
\def\PRES{\textsc{pres}}
\def\prs{\textsc{pres}}
\def\PRS{\textsc{pres}}
\def\agt{\textsc{agt}}
\def\DEL{\textsc{del}}
\def\PRED{\textsc{pred}}
\def\AGENT{\textsc{agent}}
\def\THEME{\textsc{theme}}
\def\AUX{\textsc{aux}}
\def\THEME{\textsc{theme}}
\def\PL{\textsc{pl}}
\def\SRC{\textsc{src}}
\def\src{\textsc{src}}
\def\FORM{\textsc{form}}
\def\form{\textsc{form}}
\def\GCASE{\textsc{gcase}}
\def\gcase{\textsc{gcase}}
\def\SCASE{\textsc{scase}}
\def\PHON{\textsc{phon}}
\def\SS{\textsc{ss}}
\def\SYN{\textsc{syn}}
\def\LOC{\textsc{loc}}
\def\MOD{\textsc{mod}}
\def\INV{\textsc{inv}}
\def\L{\textsc{l}}
\def\CASE{\textsc{case}}
\def\SPR{\textsc{spr}}
\def\COMPS{\textsc{comps}}
%\def\comps{\textsc{comps}}
\def\SEM{\textsc{sem}}
\def\CONT{\textsc{cont}}
\def\SUBCAT{\textsc{subcat}}
\def\CAT{\textsc{cat}}
\def\C{\textsc{c}}
\def\SUBJ{\textsc{subj}}
\def\subj{\textsc{subj}}
\def\SLASH{\textsc{slash}}
\def\LOCAL{\textsc{local}}
\def\ARG-ST{\textsc{arg-st}}
\def\AGR{\textsc{agr}}
\def\PER{\textsc{per}}
\def\NUM{\textsc{num}}
\def\IND{\textsc{ind}}
\def\VFORM{\textsc{vform}}
\def\PFORM{\textsc{pform}}
\def\decl{\textsc{decl}}
\def\loc{\textsc{loc   }}
% \def\   {\textsc{  }}

\def\NEG{\textsc{neg}}
\def\FRAMES{\textsc{frames}}
\def\REFL{\textsc{refl}}

\def\MKG{\textsc{mkg}}

\def\BN{\textsc{bn}}
\def\HD{\textsc{hd}}
\def\NP{\textsc{np}}
\def\PF{\textsc{pf}}
\def\PL{\textsc{pl}}
\def\PP{\textsc{pp}}
\def\SS{\textsc{ss}}
\def\VF{\textsc{vf}}
\def\VP{\textsc{vp}}
\def\bn{\textsc{bn}}
\def\cl{\textsc{cl}}
\def\pl{\textsc{pl}}
\def\Wh{\ital{Wh}}
\def\ng{\textsc{neg}}
\def\wh{\ital{wh}}
\def\ACC{\textsc{acc}}
\def\AGR{\textsc{agr}}
\def\AGT{\textsc{agt}}
\def\ARC{\textsc{arc}}
\def\ARG{\textsc{arg}}
\def\ARP{\textsc{arc}}
\def\AUX{\textsc{aux}}
\def\CAT{\textsc{cat}}
\def\COP{\textsc{cop}}
\def\DAT{\textsc{dat}}
\def\DEF{\textsc{def}}
\def\DEL{\textsc{del}}
\def\DOM{\textsc{dom}}
\def\DTR{\textsc{dtr}}
\def\FUT{\textsc{fut}}
\def\GAP{\textsc{gap}}
\def\GEN{\textsc{gen}}
\def\HON{\textsc{hon}}
\def\IMP{\textsc{imp}}
\def\IND{\textsc{ind}}
\def\INV{\textsc{inv}}
\def\LEX{\textsc{lex}}
\def\Lex{\textsc{lex}}
\def\LOC{\textsc{loc}}
\def\MOD{\textsc{mod}}
\def\MRK{{\nr MRK}}
\def\NEG{\textsc{neg}}
\def\NEW{\textsc{new}}
\def\NOM{\textsc{nom}}
\def\NUM{\textsc{num}}
\def\PER{\textsc{per}}
\def\PST{\textsc{pst}}
\def\QUE{\textsc{que}}
\def\REL{\textsc{rel}}
\def\SEL{\textsc{sel}}
\def\SEM{\textsc{sem}}
\def\SIT{\textsc{arg0}}
\def\SPR{\textsc{spr}}
\def\SRC{\textsc{src}}
\def\SUG{\textsc{sug}}
\def\SYN{\textsc{syn}}
\def\TPC{\textsc{top}}
\def\VAL{\textsc{val}}
\def\acc{\textsc{acc}}
\def\agt{\textsc{agt}}
\def\cop{\textsc{cop}}
\def\dat{\textsc{dat}}
\def\foc{\textsc{focus}}
\def\FOC{\textsc{focus}}
\def\fut{\textsc{fut}}
\def\hon{\textsc{hon}}
\def\imp{\textsc{imp}}
\def\kes{\textsc{kes}}
\def\lex{\textsc{lex}}
\def\loc{\textsc{loc}}
\def\mrk{{\nr MRK}}
\def\nom{\textsc{nom}}
\def\num{\textsc{num}}
\def\plu{\textsc{plu}}
\def\pne{\textsc{pne}}
\def\pst{\textsc{pst}}
\def\pur{\textsc{pur}}
\def\que{\textsc{que}}
\def\src{\textsc{src}}
\def\sug{\textsc{sug}}
\def\tpc{\textsc{top}}
\def\utt{\textsc{utt}}
\def\val{\textsc{val}}
\def\LITE{\textsc{lex}}
\def\PAST{\textsc{pst}}
\def\POSP{\textsc{pos}}
\def\PRS{\textsc{pres}}
\def\mod{\textsc{mod}}%
\def\newuse{{`kes'}}
\def\posp{\textsc{pos}}
\def\prs{\textsc{pres}}
\def\psp{{\it en\/}}
\def\skes{\textsc{kes}}
\def\CASE{\textsc{case}}
\def\CASE{\textsc{case}}
\def\COMP{\textsc{comp}}
\def\CONJ{\textsc{conj}}
\def\CONN{\textsc{conn}}
\def\CONT{\textsc{cont}}
\def\DECL{\textsc{decl}}
\def\FOCUS{\textsc{focus}}
\def\FORM{\textsc{form}}
\def\FREL{\textsc{frel}}
\def\GOAL{\textsc{goal}}
\def\HEAD{\textsc{head}}
\def\INDEX{\textsc{ind}}
\def\INST{\textsc{inst}}
\def\MODE{\textsc{mode}}
\def\MOOD{\textsc{mood}}
\def\NMLZ{\textsc{nmlz}}
\def\PHON{\textsc{phon}}
\def\PRED{\textsc{pred}}
%\def\PRES{\textsc{pres}}
\def\PROM{\textsc{prom}}
\def\RELN{\textsc{pred}}
\def\RELS{\textsc{rels}}
\def\STEM{\textsc{stem}}
\def\SUBJ{\textsc{subj}}
\def\XARG{\textsc{xarg}}
\def\bse{{\it bse\/}}
\def\case{\textsc{case}}
\def\caus{\textsc{caus}}
\def\comp{\textsc{comp}}
\def\conj{\textsc{conj}}
\def\conn{\textsc{conn}}
\def\decl{\textsc{decl}}
\def\fin{{\it fin\/}}
\def\form{\textsc{form}}
\def\gend{\textsc{gend}}
\def\inf{{\it inf\/}}
\def\mood{\textsc{mood}}
\def\nmlz{\textsc{nmlz}}
\def\pass{\textsc{pass}}
\def\past{\textsc{past}}
\def\perf{\textsc{perf}}
\def\pln{{\it pln\/}}
\def\pred{\textsc{pred}}


%\def\pres{\textsc{pres}}
\def\proc{\textsc{proc}}
\def\nonfin{{\it nonfin\/}}
\def\AGENT{\textsc{agent}}
\def\CFORM{\textsc{cform}}
%\def\COMPS{\textsc{comps}}
\def\COORD{\textsc{coord}}
\def\COUNT{\textsc{count}}
\def\EXTRA{\textsc{extra}}
\def\GCASE{\textsc{gcase}}
\def\GIVEN{\textsc{given}}
\def\LOCAL{\textsc{local}}
\def\NFORM{\textsc{nform}}
\def\PFORM{\textsc{pform}}
\def\SCASE{\textsc{scase}}
\def\SLASH{\textsc{slash}}
\def\SLASH{\textsc{slash}}
\def\THEME{\textsc{theme}}
\def\TOPIC{\textsc{topic}}
\def\VFORM{\textsc{vform}}
\def\cause{\textsc{cause}}
%\def\comps{\textsc{comps}}
\def\gcase{\textsc{gcase}}
\def\itkes{{\it kes\/}}
\def\pass{{\it pass\/}}
\def\vform{\textsc{vform}}
\def\CCONT{\textsc{c-cont}}
\def\GN{\textsc{given-new}}
\def\INFO{\textsc{info-st}}
\def\ARG-ST{\textsc{arg-st}}
\def\SUBCAT{\textsc{subcat}}
\def\SYNSEM{\textsc{synsem}}
\def\VERBAL{\textsc{verbal}}
\def\arg-st{\textsc{arg-st}}
\def\plain{{\it plain}\/}
\def\propos{\textsc{propos}}
\def\ADVERBIAL{\textsc{advl}}
\def\HIGHLIGHT{\textsc{prom}}
\def\NOMINAL{\textsc{nominal}}

\newenvironment{myavm}{\begingroup\avmvskip{.1ex}
  \selectfont\begin{avm}}%
{\end{avm}\endgroup\medskip}
\def\pfix{\vspace{-5pt}}


\def\jbsub#1{\lower4pt\hbox{\small #1}}
\def\jbssub#1{\lower4pt\hbox{\small #1}}
\def\jbtr{\underbar{\ \ \ }\ }


%\fi

  %% hyphenation points for line breaks
%% Normally, automatic hyphenation in LaTeX is very good
%% If a word is mis-hyphenated, add it to this file
%%
%% add information to TeX file before \begin{document} with:
%% %% hyphenation points for line breaks
%% Normally, automatic hyphenation in LaTeX is very good
%% If a word is mis-hyphenated, add it to this file
%%
%% add information to TeX file before \begin{document} with:
%% %% hyphenation points for line breaks
%% Normally, automatic hyphenation in LaTeX is very good
%% If a word is mis-hyphenated, add it to this file
%%
%% add information to TeX file before \begin{document} with:
%% \include{localhyphenation}
\hyphenation{
A-la-hver-dzhie-va
anaph-o-ra
affri-ca-te
affri-ca-tes
Atha-bas-kan
Chi-che-ŵa
com-ple-ments
Da-ge-stan
Dor-drecht
er-klä-ren-de
Ginz-burg
Gro-ning-en
Jon-a-than
Ka-tho-lie-ke
Ko-bon
krie-gen
Le-Sourd
moth-er
Mül-ler
Nie-mey-er
Prze-piór-kow-ski
phe-nom-e-non
re-nowned
Rie-he-mann
un-bound-ed
}

% why has "erklärende" be listed here? I specified langid in bibtex item. Something is still not working with hyphenation.


% to do: check
%  Alahverdzhieva

\hyphenation{
A-la-hver-dzhie-va
anaph-o-ra
affri-ca-te
affri-ca-tes
Atha-bas-kan
Chi-che-ŵa
com-ple-ments
Da-ge-stan
Dor-drecht
er-klä-ren-de
Ginz-burg
Gro-ning-en
Jon-a-than
Ka-tho-lie-ke
Ko-bon
krie-gen
Le-Sourd
moth-er
Mül-ler
Nie-mey-er
Prze-piór-kow-ski
phe-nom-e-non
re-nowned
Rie-he-mann
un-bound-ed
}

% why has "erklärende" be listed here? I specified langid in bibtex item. Something is still not working with hyphenation.


% to do: check
%  Alahverdzhieva

\hyphenation{
A-la-hver-dzhie-va
anaph-o-ra
affri-ca-te
affri-ca-tes
Atha-bas-kan
Chi-che-ŵa
com-ple-ments
Da-ge-stan
Dor-drecht
er-klä-ren-de
Ginz-burg
Gro-ning-en
Jon-a-than
Ka-tho-lie-ke
Ko-bon
krie-gen
Le-Sourd
moth-er
Mül-ler
Nie-mey-er
Prze-piór-kow-ski
phe-nom-e-non
re-nowned
Rie-he-mann
un-bound-ed
}

% why has "erklärende" be listed here? I specified langid in bibtex item. Something is still not working with hyphenation.


% to do: check
%  Alahverdzhieva

  \bibliography{../Bibliographies/stmue,
                ../localbibliography,
../Bibliographies/formal-background,
../Bibliographies/understudied-languages,
../Bibliographies/phonology,
../Bibliographies/case,
../Bibliographies/evolution,
../Bibliographies/agreement,
../Bibliographies/lexicon,
../Bibliographies/np,
../Bibliographies/negation,
../Bibliographies/ellipsis,
../Bibliographies/argst,
../Bibliographies/binding,
../Bibliographies/complex-predicates,
../Bibliographies/coordination,
../Bibliographies/relative-clauses,
../Bibliographies/udc,
../Bibliographies/processing,
../Bibliographies/cl,
../Bibliographies/dg,
../Bibliographies/islands,
../Bibliographies/diachronic,
../Bibliographies/gesture,
../Bibliographies/semantics,
../Bibliographies/pragmatics,
../Bibliographies/information-structure,
../Bibliographies/idioms,
../Bibliographies/cg,
../Bibliographies/udc}

  \togglepaper[20]
}{}


\author{%
	Joanna Nykiel\affiliation{Kyung Hee University, Seoul}%
	\lastand Jong-Bok Kim\affiliation{Kyung Hee University, Seoul}%
}
\title{Ellipsis}

% \chapterDOI{} %will be filled in at production

%\epigram{Change epigram in chapters/03.tex or remove it there }

\abstract{This chapter provides an overview of HPSG analyses of ellipsis. The structure of the chapter follows three types of ellipsis, nonsentential utterances, predicate ellipsis (including VP ellipsis), and nonconstituent coordination, with three types of analyses applied to them. The analyses characteristically don't admit silent syntactic material for any ellipsis phenomena with the exception of certain types of nonconstituent coordination.}


\begin{document}
\maketitle
\label{chap-ellipsis}

{\avmoptions{center}

\section{Introduction}
\label{ellipsis-sec-introduction}

         Ellipsis is a phenomenon that involves a noncanonical mapping between syntax and semantics. What appears to be a syntactically incomplete utterance still receives a semantically complete representation, based on the features of the surrounding context, be the context linguistic or nonlinguistic. The goal of syntactic theory is thus to account for how the complete semantics can be reconciled with the apparently incomplete syntax. One of the key questions here relates to the structure of the ellipsis site, that is, whether or not we should assume the presence of invisible syntactic material. This chapter begins by introducing three types of ellipsis (nonsentential utterances, predicate ellipsis, and nonconstituent coordination) that have attracted considerable attention and received treatment within HPSG. We next overview existing evidence for and against the so-called WYSIWYG (`What You See Is What You Get') approach to ellipsis, where no invisible material is posited at the ellipsis site. Finally, we walk the reader through three types of HPSG analyses applied to the three types of ellipsis presented in section 2.


\section{Three types of ellipsis}
\label{sec-three-types-of-ellipsis}

Depending on the type of analysis by means of which HPSG handles them, elliptical phenomena can be broadly divided into three types:
nonsentential utterances, predicate ellipsis (including VP ellipsis), and nonconstituent coordination.
We overview the key features of these types here before discussing in greater detail how they have been brought to bear on the question of whether there is invisible syntactic structure at the ellipsis site or not. We begin with stranded XPs, which HPSG treats as nonsentential utterances, and then move on to predicate and argument ellipsis, followed by phenomena known as nonconstituent coordination.

\subsection{Nonsentential utterances}
This section introduces utterances smaller than a sentence, which we refer to as nonsentential utterances (NUs). These range from Bare Argument Ellipsis (BAE) (\ref{1}), including fragment answers (\ref{2}), to direct or embedded sluicing (\ref{3})-(\ref{4}). Sluicing hosts stranded wh-phrases and has the function of an interrogative clause, while BAE hosts XPs representing various syntactic categories and typically has the function of a declarative clause.\footnote{Several subtypes of nonsentential utterances can be distinguished, based on their contextual functions, which we don't discuss here (for a recent taxonomy, see \citealt{Ginzburg2011}).}



\ea A: You were angry with them.\\ B: Yeah, angry with them and angry with the situation.\label{1}\z

\ea A: Where are we? \\B: In Central Park.\label{2}\z

\ea A: So what did you think about that?\\ B: About what? \label{3}\z

\ea A: There's someone at the door. \\B: Who?/I wonder who. \label{4}\z
The theoretical question NUs raise is whether they are parts of larger sentential structures or rather nonsentential structures whose semantic and morphosyntactic features are licensed by the surrounding context. To adjudicate between these views, researchers have looked for evidence that NUs in fact behave as if they were fragments of sentences. As we will see in section 3, there is evidence to support both of these views. However, HPSG doesn't assume that NUs are underlyingly sentential structures.

\subsection{Predicate ellipsis and argument ellipsis}
The section looks at three constructions whose syntax includes null, hence noncanonical, elements. They are Verb Phrase Ellipsis (VPE), Null Complement Anaphora (NCA), and argument drop (or pro drop). VPE features stranded auxiliary verbs (\ref{5}) and NCA is characterized by omission of complements to some lexical verbs (\ref{6}). Argument drop refers to omission of a pronominal subject or an object argument, as illustrated in (\ref{7}) for Polish.

\ea A: I didn't ask George to invite you.\\B: Then who did?\label{5}\z

\ea Some mornings you can't get hot water in the shower, but nobody complains. \label{6} \z

\ea
\gll Pia p\'{o}\'{z}no wr\'{o}ci\l a do domu. {Od razu} posz\l a spa\'{c}.\\
\textsc{Pia} \textsc{late} \textsc{got} \textsc{to} \textsc{home} \textsc{right away} \textsc{went} \textsc{sleep}\\
\glt `Anna got home late. She went straight to bed.'
\label{7}
\z
Here the question is whether these null elements should be assumed to be underlyingly present in the syntax of theses constructions, and the answer is no.

\subsection{Nonconstituent coordination}

We focus on two instances of nonconstituent coordination, Right Node Raising (RNR) and gapping \citep{Ross1967}, illustrated in (\ref{8}) and (\ref{9}) respectively. In RNR, a single constituent located in the right-peripheral position is associated with both conjuncts. In gapping, a finite verb is associated with both (or more) conjuncts but only present in the leftmost one. This results in what appears to be coordination of standard constituents and elements not normally defined as constituents (a stranded transitive verb in (\ref{8}) and a cluster of subject NP and object NP in (\ref{9})).

\ea Ethan sold and Rasmus gave away all his CDs. \label{8}\z

\ea Ethan gave away his CDs and Rasmus his old guitar. \label{9}\z

To handle such constructions the grammar must either be permitted to coordinate noncanonical constituents, to generate coordinated constituents parts of which can undergo deletion, or to coordinate nonsentential utterances. As we will see, HPSG makes use of the latter two options.

\section{Evidence for and against invisible material at the ellipsis site}
This section is concerned with NUs and VPE since this is where the contentious issues arise of where ellipsis is licensed (sections 3.3. and 3.4) and whether there is invisible syntactic material in an ellipsis site (sections 3.1 and 3.2). Below we consider evidence for and against invisible structure found in the ellipsis literature; the evidence is based not only on intuitive judgments, but also experimental and corpus data.

\subsection{Connectivity effects}
Connectivity effects refer to parallels between NUs and their counterparts in sentential structures, thus speaking in favor of the existence of silent sentential structure. We focus on two kinds here: case-matching effects and preposition-stranding effects (for other examples of connectivity effects, see \citealt{Ginzburg2018}). It's been known since \citet{Ross1969} that NUs exhibit case-matching effects, that is, they are typically marked for the same case that is marked on their counterparts in sentential structures. (\ref{10}) illustrates this for German, where case matching is seen between a wh-phrase functioning as an NU and its counterpart in the antecedent.

\ea
\gll Er will jemandem schmeicheln, aber sie wissen nicht wem/*wen.\\
\textsc{he} \textsc{will} \textsc{someone.DAT} \textsc{flatter}, \textsc{but} \textsc{they} \textsc{know} \textsc{not} \textsc{who.DAT/*who.ACC}\\
\glt `He wants to flatter someone, but they don't know whom.'\label{10}\z

Case-matching effects are crosslinguistically robust in that they are found in the vast majority of languages with overt case marking systems, and therefore, they have been taken as strong evidence for the reality of silent structure. The argument is that the pattern of case matching follows straightforwardly from the silent structure that embeds an NU and matches the structure of the antecedent clause (\citet{Merchant2001, Merchant2004}). However, a language like Hungarian poses a problem for this reasoning \citep[see][]{Jacobson2016-in-cg}. Hungarian has verbs that assign one of two cases to their object NPs with no meaning difference, but case matching is still required between an NU and its counterpart, whichever case is marked on the counterpart. To see this, consider (\ref{11}) from \citet[356]{Jacobson2016-in-cg}. The verb {\it hasonlit} assigns either sublative (SUBL) or allative (ALL) case to its object, but if SUBL is selected for an NU's counterpart, the NU must match this case.

\ea
A: \gll Ki-re hasonlit P\'{e}ter?\\
 \textsc{who.SUBL} \textsc{resembles} \textsc{Peter}\\
 \glt  `Who does Peter resemble?'\\

B: \gll J\'{a}nos-ra/*J\'{a}nos-hoz.\\
\textsc{J\'{a}nos.SUBL/*J\'{a}nos.ALL}\\
\glt  `J\'{a}nos.'\label{11}\z
\citet{Jacobson2016-in-cg} notes that there is some speaker variation regarding the (un)ac\-cepta\-bi\-li\-ty of case mismatch here at the same time that all speakers agree that either case is fine in a corresponding nonelliptical response to (\ref{11}A). This last point is important, because it shows that the requirement of---or at least a preference for---matching case features applies to NUs to a greater extent than it does to their nonelliptical equivalents, challenging connectivity effects.

Similarly problematic for case-based parallels between NUs and their sentential counterparts are some Korean data. Korean NUs can drop case markers more freely than their counterparts in nonelliptical clauses can, a point made in \citet{Morgan1989}. Observe the following exchanges:

  \ea
A: \gll Nwukwu-ka ku  chaek-ul  sa-ass-ni?\\
\textsc{who.NOM} \textsc{the} \textsc{book.ACC} \textsc{buy.PST.QUE}\\
\glt  `Who bought the book?'\\

B: \gll Yongsu-ka/Yongsu/*Yongsu-lul.\\
\textsc{Yongsu.NOM/Yongsu/*Yongsu.ACC}\\
\glt  `Yongsu.'

B$'$: \gll Yongsu-ka/*Yongsu  ku  chaek-ul  sa-ass-e\\
\textsc{Yongsu.NOM/*Yongsu} \textsc{the} \textsc{book.ACC} \textsc{buy.PST.DECL}\\
\glt  `Yongsu bought the book'\\
\label{12}\z
%
When an NU corresponds to a nominative subject in the antecedent (as in \ref{12}B), it can be either marked for nominative or caseless.
However, replacing the same NU with a full sentential answer, as in (\ref{12}B$'$), rules out case drop from the subject. This strongly suggests that the case-marked and caseless NUs couldn't have identical source sentences if they were to derive via PF-deletion.\footnote{Nominative differs in this respect from three other structural cases, dative, accusative and genitive, in that the latter may also be dropped from nonelliptical clauses \citep[see][]{Morgan1989, Lee2016, Kim2016}.}  Data like these led \citet{Morgan1989} to propose that not all NUs have a sentential derivation, an idea later picked up in \citet{Barton1998}.

The same pattern is associated with semantic case. That is, in (\ref{13}), an NU can optionally be
marked for comitative like its counterpart in the A-sentence or be caseless. But being caseless is
not an option for the NU's counterpart.\todostefan{glossing did not match for
  \emph{ha-ess-e}. Please check}

\ea
A: 
\gll Nwukwu-wa         /  *  nwukwu  hapsek-ul                     ha-yess-e?\\
     who-\textsc{com}  {} {} who     sitting.together-\textsc{acc} do-\textsc{pst}-\textsc{que}\\
\glt  `With whom did you sit together?'\\

B: 
\gll Mimi-wa / * Mimi.\\
     Mimi-\textsc{src} {} {} Mimi\\
\glt `With Mimi/*Mimi.' \label{13}\z

The generalization for Korean is then that NUs may be optionally realized as caseless but may never be marked for a different case than is marked on their counterparts.

Overall, case-marking facts show that there is some morphosyntactic identity between NUs and their antecedents, though not to the extent that NUs have exactly the features that they would have if they were constituents embedded in sentential structures. The Hungarian facts also suggest that aspects of the argument structure of the antecedent relating to case licensing are relevant for an analysis of NUs.\footnote{Hungarian and Korean are in fact not the only problematic languages; for a list, see \citet{Vicente2015}.}

The second kind of connectivity effects goes back to \citet{Merchant2001, Merchant2004} and highlights apparent links between wh- and focus movement and the features of NUs. The idea is that prepositions behave the same under wh- and focus movement as they do under clausal ellipsis, that is, they pied-pipe or strand in the same environments. If a language (e.g., English) permits preposition stranding under wh- and focus movement ({\it With what did Harvey paint the wall?} vs {\it What did Harvey paint the wall with?}), then NUs may surface with or without prepositions, as illustrated in (\ref{14}) for sluicing and BAE.

\ea A: I know what Harvey painted the wall with.\\B: (With) what?/(With) primer.\label{14}\z
If there indeed was a link between between preposition stranding and NUs, then we would not expect prepositionless NUs in languages without preposition stranding. This expectation is disconfirmed by an ever-growing list of nonpreposition-stranding languages that do feature prepositionless NUs: Brazilian Portuguese (\citep{AlmeidaYoshida 2007}, Spanish and French \citep{Rodrigues2009}, Greek \citep{Molimpakis2018}, Bahasa Indonesia \citep{Fortin2007}, Emirati Arabic \citep{Leung2014}, Russian \citep{Philippova2014}, Polish \citep{Szczegielniak2008, Nykiel2013, Sag2011}, Czech \citep{Caha2011}, Bulgarian \citep{Abels2017}, and Serbo-Croatian \citep{Stjepanovic2008, Stjepanovic2012}. A few of these studies have presented experimental evidence that prepositionless NUs are acceptable, though --- for reasons still poorly understood --- they typically don't reach the same level of acceptability as their variants with prepositions do (see \citealt{Nykiel2013} for Polish and \citealt{Molimpakis2018} for Greek).

It is evident from this research that there is no grammatical constraint on NUs that keeps track of what preposition-stranding possibilities exist in any given language. On the other hand, it doesn't seem sufficient to assume that NUs can freely drop prepositions, given examples of sprouting like (\ref{15}), in which prepositions are not omissible (see \citealt{Chung1995} on the inomissibility of prepositions under sprouting). The difference between (\ref{14}) and (\ref{15}) is that there is an explicit phrase the NU corresponds to (in the HPSG literature this phrase is termed a Salient Utterance \citep{Ginzburg:Sag:2000} or a Focus-Establishing Constituent \citep{Ginzburg2012}) in the former but not in the latter.

\ea A: I know Harvey painted the wall.\\B: *(With) what?/Yeah, *(with) primer.\label{15}\z
This issue has not received much attention in the HPSG literature, though see \citet{Kim2015}.


\subsection{Island effects}
One of the predictions of the view that NUs are underlyingly sentential is that they should respect island constraints on long-distance movement. But as illustrated below, NUs (both sluicing and BAE) exhibit island-violating behavior.\footnote{\citet{Merchant2004} argued that BAE, unlike sluicing, does respect island constraints, an argument that was later challenged \citep[see e.g,][]{CJ2005a, Griffiths2014}. However, \citet{Merchant2004} focused specifically on pairs of wh-questions and answers like (\ref{2}) and ran into the difficulty of testing for island-violating behavior, since a well-formed antecedent couldn't be constructed.}

\ea A: Harriet drinks scotch that comes from a very special part of Scotland.\\B: Where? \citep[245]{CJ2005a} \label{16}\z

\ea A: The administration has issued a statement that it is willing to meet with one of the student groups.\\B: Yeah, right---the Gay Rifle Club. \citep[245]{CJ2005a} \label{17}\z

Among \citeauthor{CJ2005a}'s (\citeyear[245]{CJ2005a}) examples of well-formed island-violating NUs are also sprouted NUs (those that correspond to implicit Salient Utterances) like (\ref{18})-(\ref{19}).

\ea A: John met a woman who speaks French.\\B: With an English accent?\label{18}\z

\ea A: For John to flirt with at the party would be scandalous. \\B: Even with his wife?\label{19}\z
Other research assumes that sprouted NUs are one of the two kinds of NUs that respect island constraints, the other kind being contrastive NUs, illustrated in (\ref{20}) \citep{Chung1995, Merchant2001, Griffiths2014}.

\ea A: Does Abby speak the same Balkan language that Ben speaks?\\
B: *No, Charlie. \citep{Merchant2001}  \label{20}\z
%
\citet{Schmeh2015} explore the acceptability of NUs like (\ref{20}) compared to NUs introduced by the particle {\it Yes} depicted in (\ref{21}). (\ref{20}) differs from (\ref{21}) in terms of discourse function in that it corrects rather than supplement the antecedent, which is signaled by a different response particle.

\ea A: John met a guy who speaks a very unusual language. \\B: Yes, Albanian. \citep[245]{CJ2005a} \label{21}\z
%
\citet{Schmeh2015} find that corrections lower acceptability ratings compared to supplementations and propose that this follows from the fact that corrections induce greater processing difficulty than supplementations do, and hence the acceptability difference between (\ref{20}) and (\ref{21}). This finding makes it plausible that the perceived degradation of island-violating NUs could ultimately be attributed to nonsyntactic factors, e.g., the difficulty of successfully computing a meaning for them.

In contrast to NUs, many instances of VPE appear to obey island constraints, as would be expected if there was unpronounced structure from which material was extracted. An example is depicted in (\ref{22}) (note that the corresponding sluicing NU is fine).

\ea *They want to hire someone who speaks a Balkan language, but I don't remember which they do [\sout{want to hire someone who speaks t}]. \citep[6]{Merchant2001}\label{22}\z
(\ref{22}) contrasts with well-formed examples like (\ref{23}) and (\ref{24}), from \citet{Ginzburg2018}.

\ea He managed to find someone who speaks a Romance language, but a Germanic language, he didn't [\sout{manage to find someone who speaks t}].\label{23}\z
\ea He was able to find a bakery where they make good baguette, but croissants, he couldn't [\sout{find a bakery where they make good t}].\label{24}\z
As \citet{Ginzburg2018} rightly point out, we don't yet have a complete understanding of when or why island effects show up in VPE. Its behavior is at best inconsistent, failing to provide convincing evidence for silent structure.


\subsection{Structural mismatches}
Because structural mismatches are not permitted by NUs \citep[see][]{Merchant2005, Merchant2013},\footnote{\citet{Ginzburg2018} cite examples---originally from \citet{Beecher2008}---of sprouting NUs with nominal, hence mismatched, antecedents, e.g., (i).
	\ea We're on to the semi-finals, though I don't know who against.\z
	Somewhat similar examples, where NUs appear to take APs as antecedents, appear in COCA:
	\ea  A: Well, it's a defense mechanism. B: Defense against what?\z
	\ea Our Book of Mormon talks about the day of the Lamanite, when the church would make a special effort to build and reclaim a fallen people. And some people will say, Well, fallen from what? \z
	The NUs in (ii)--(iii) repeat the lexical heads whose complements are being sprouted ({\it defense} and {\it fallen}), that is, they contain more material than is usual for NUs (cf. (i)). It seems that without this additional material it would be difficult to integrate the NUs into the propositions provided by the antecedents and hence to arrive at the intended interpretations.
} this section focuses on VPE and developments surrounding the question of which contexts license it. In a seminal study of anaphora, \citet{Hankamer1976} classified VPE as a surface anaphor with syntactic features closely matching those of an antecedent present in the linguistic context. They argued in particular that VPE is not licensed if it mismatches its antecedent in voice. Compare (\ref{25}) and (\ref{26}) from \citet[327]{Hankamer1976}.
%\todosatz{no reference given}.
\ea[*]{
	The children asked to be squirted with the hose, so we did.  \label{25}
}
\z
\ea[]{
	The children asked to be squirted with the hose, so they were. \label{26}
}
\z
This proposal places tighter structural constraints on VPE than on other verbal anaphors (e.g., \emph{do it/that}) in terms of identity between an ellipsis site and its antecedent and has prompted extensive evaluation in a number of corpus and experimental studies in the decades following \citet{Hankamer1976}. Below are examples of acceptable structural mismatches reported in the literature, ranging from voice mismatch (\ref{27}) to nominal antecedents (\ref{28}) to split antecedents (\ref{29}).

\ea This information could have been released by Gorbachev, but he chose not to [release it]. \citep[37]{Hardt1993} \label{27}\z

\ea Mubarak's survival is impossible to predict and, even if he does [survive], his plan to make his son his heir apparent is now in serious jeopardy. \citep{Miller2014a} \label{28}\z

\ea Wendy is eager to sail around the world and Bruce is eager to climb Mt. Kilimanjaro, but neither of them can [do the things they want], because money is too tight. \citep{Webber79a} \label{29}\z

There are two opposing views that have emerged from the empirical work. The first view takes mismatches to be grammatical and connects degradation in acceptability to violation of certain independent discourse \citep{Kehler2002, Miller2011, Kertz2013, Miller2014a, Miller2014b} or processing constraints \citep{Kim2011}. Two types of VPE have been identified on this view through corpus work---auxiliary choice VPE and subject choice VPE---each with different discourse requirements with respect to the antecedent \citep{Miller2011, Miller2014a, Miller2014b}. The second view assumes that there is a grammatical ban on structural mismatch but violations thereof may be repaired under certain conditions; repairs are associated with differential processing costs compared to matching ellipses and antecedents \citep{Arregui2006, Grant2012}. If we follow the first view, it is perhaps unexpected that voice mismatch should consistently incur a greater penalty under VPE than when no ellipsis is involved, as recently reported in \citet{Kim2017}. \citet{Kim2017} stop short of drawing firm conclusions regarding the grammaticality of structural mismatches, but one possibility is that the observed mismatch effects reflect a construction-specific constraint on VPE. HPSG analyses take structurally mismatched instances of VPE to be unproblematic and fully grammatical, while also recognizing construction-specific constraints: discourse or processing constraints formulated for VPE may or may not extend to other elliptical constructions, such as NUs (see \citealt{Ginzburg2018} for this point).


\subsection{Nonlinguistic antecedents}
Like structural mismatches, the availability of nonlinguistic antecedents for an ellipsis points to the fact that it needn't be interpreted by reference to and licensed by a structurally identical antecedent. Although this option is somewhat limited, VPE does tolerate nonlinguistic antecedents, as shown in (\ref{30})--(\ref{31}) \citep[see also][]{Hankamer1976, Schachter1977}.
\ea Mabel shoved a plate into Tate's hands before heading for the sisters' favorite table in the shop. ``You shouldn't have.'' She meant it. The sisters had to pool their limited resources
just to get by. \citep[ex. 23][]{Miller2014b}\label{30}\z

\ea Once in my room, I took the pills out. ``Should I?'' I asked myself. \citep[ex. 22a][]{Miller2014b}\label{31}\z
\citet{Miller2014b} provide an extensive critique of the earlier work on the ability of VPE to take nonlinguistic antecedents, arguing for a streamlined discourse"=based explanation that neatly captures the attested examples as well as examples of structural mismatch like those discussed in section 3.3. The important point here is again that VPE is subject to construction-specific constraints which limit its use with nonlinguistic antecedents.

NUs appear in various nonlinguistic contexts as well. \citet{Ginzburg2018} distinguish three classes of such NUs: sluices (\ref{32}), exclamative sluices (\ref{33}), and declarative fragments (\ref{34}).

\ea (In an elevator) What floor? \citep[298]{Ginzburg:Sag:2000}\label{32}\z

\ea It makes people ``easy to control and easy to handle,'' he said, ``but, God forbid, at what a cost!''
%(Ginzburg \& Miller To appear, ex. 34a)\todosatz{no reference given}
\label{33}\z

\ea BOBADILLA turns, gestures to one of the other men, who comes forward and gives him a roll of parchment, bearing the royal seal. ``My letters of appointment.'' (COCA)\label{34}\z
In addition to being problematic from the licensing point of view, NUs like these have been put forward as evidence against the idea that they are underlyingly sentential, because it is unclear what the structure that underlies them would be \citep[see][]{Ginzburg:Sag:2000, CJ2005a, Stainton2006}.\footnote{This is not to say that a sentential analysis of fragments without linguistic antecedents hasn't been attempted. For details of a proposal involving a `limited ellipsis' strategy, see \citet{Merchant2004}.}
%\todosatz{Merchant 2010: no reference given}


\section{Analyses of NUs}
It is worth noting at the outset that the analyses of NUs within the framework of HPSG are based on an elaborate theory of dialog \citep{Ginzburg1994, Ginzburg2004, Ginzburg2014a, Larsson2002, Purver2006, Fernandez2006, Fernandez2002, Fernandez2007, Ginzburg2010, Ginzburg2014b, Ginzburg2012, Ginzburg2013} and on a wider range of data than is common practice in the ellipsis literature. Existing analyses of NUs go back to \citet{Ginzburg:Sag:2000}, who recognize declarative fragments (\ref{34a}) and two kinds of sluicing NUs, direct sluices (\ref{35}) and reprise sluices (\ref{36}) (the relevant fragments are bolded). The difference between direct and reprise sluices lies in the fact that the latter are requests for clarification of any part of the antecedent. For instance, in (\ref{36}) the referent of {\it that} is unclear to the interlocutor.

\ea ``I was wrong.'' Her brown eyes twinkled. ``Wrong about what?'' ``\textbf{That night}.'' (COCA) \label{34a}\z

\ea ``You're waiting,'' she said softly. ``\textbf{For what?}'' (COCA) \label{35} \z

\ea ``Can we please not say a lot about that?'' ``\textbf{About what?}'' (COCA) \label{36} \z

\citet[304]{Ginzburg:Sag:2000} make use of the constraint shown in (\ref{hf-cx}) (we have added information about the MAX-QUD) to generate NUs.




                                 \ea{\label{hf-cx}
\item[] Head-Fragment Construction
\item[] \small
\begin{avm}
\[CAT &S\\ %\[HEAD v\]\\
  %CONT &\[NUCL \@1\]\\
  CTXT & \[MAX-QUD $\lambda$\{$\pi$$^{i}$\}\\
  SAL-UTT \{  \[CAT \@2\\
                         CONT\ \[IND & {\it i}\\
                                  \]\]\}\]\]

                    \ \ $\rightarrow$\ \
\[  CAT &  \@2\\
   CONT  &\[IND & {\it i}\]
\]
\end{avm}}\z
Let us see how this constructional constraint allows us to
license NUs and capture their properties, including the connectivity effects we discussed in section 3.1. Note first that any phrasal category
can function as an NU, that is, can be mapped onto a sentential utterance as long as it corresponds to a Salient Utterance (SAL-UTT). This means that
the head daughter's syntactic category must match that of a SAL-UTT, which is an attribute supplied by the surrounding context as a (sub)utterance of another contextual attribute---the Maximal Question under Discussion (MAX-QUD). The two contextual attributes SAL-UTT and MAX-QUD are introduced specifically for the purpose of analyzing NUs. The context gets updated with every new question-under-discussion, and MAX-QUD represents the most recent question-under-discussion, while SAL-UTT is the (sub)utterance with the widest scope within MAX-QUD. To put it informally, SAL-UTT represents a (sub)utterance of a MAX-QUD that has not been resolved yet. Its feature CAT supplies information relevant for establishing morphosyntactic identity with an NU, that is, syntactic category and case information, and (\ref{hf-cx}) requires that an NU match this information. Meanwhile, MAX-QUD provides the propositional semantics for an NU and is, typically, a unary question. The content of MAX-QUD can be supplied by linguistic or nonlinguistic context. In the prototypical case, MAX-QUD arises from the most recent wh-question uttered in a given context (\ref{37}), but can also arise (via accommodation) from other forms found in the context, such as constituents bearing focal accent (\ref{38}) and constituents in need of clarification (\ref{39}), or from a nonlinguistic context (\ref{40}).\footnote{\citet{Ginzburg2012} uses the notion of the Dialog Game Board (DGB) to keep track of all information relating to the common ground between interlocutors. The DGB is also the locus of contextual updates arising from each new question-under-discussion that is introduced.}
\ea A: What did Barry break? \\B: The mike.\label{37}\z

\ea A: Barry broke the MIKE. \\B: Yes, the only one we had.\label{38}\z

\ea A: Barry broke the mike. \\B: Who?\label{39}\z

\ea (Cab driver to passenger on the way to airport) A: Which airline?\label{40}\z

The existing analyses of NUs \citep{Ginzburg2012, Sag2011, Kim2015, Abeille2014, Abeille2017} are based on \citeauthor{Ginzburg:Sag:2000}'s (\citeyear{Ginzburg:Sag:2000}) constraint. Below we illustrate how it is applied to the declarative fragment in (\ref{37}).

\ea\label{41}
\begin{forest}
[\begin{avm}\avml \hfil S\\
\[CAT &\[HEAD v\]\\
  %CONT & \[NUCL \@1\]\\
 CTXT & \[MAX-QUD $\lambda$\{$\pi$$^{i}$\}\[{\it break}({\it b,i})\]\\
  SAL-UTT \{\[CAT\ \@2\\
                          CONT & \[IND & {\it i}\\
                                 \] \]\}\]\]  \avmr \end{avm}
[\begin{avm}\avml \hfil NP\\
             \[CAT  \@2\\
  CONT & \[IND & {\it i}%\\
   %\PARAMS & \{$\pi$$^{i}$\}
   \]\]\avmr\end{avm}
 [The mike]]]
\end{forest}
\z

This construction-based analysis, in which dialogue updating plays
a key role in the licensing of NUs, can also offer a simple account of
sprouting examples like (\ref{35}).\footnote{\citet[330]{Ginzburg:Sag:2000} suggest a way of analyzing sprouted NUs, such as (i). The implied direct object of {\it eat} functioning as SAL-UTT here would appear as a noncanonical synsem on the verb's ARG-ST list, but not on the COMPS list, and thereby be able to provide appropriate morphosyntactic identity information.
	
	\ea  A: And what did you do then? \\B: I ate. \\A: What?\z
	
	
} As discussed in \citet{Kim2015}, we could take an unrealized oblique argument of
the verb {\it wait} as an instance of indefinite null instantiation (INI) \citep[see][]{Ruppenhofer2014}:

\ea Lexical entry for {\it wait}:\\
 \begin{avm}
 \[FORM \q<{\it wait}\q>\\
   ARG-ST \<NP\jbsub{{\it i}}, PP\jbsub{{\it x}}\>\\
   SYN\[SUBJ \<NP[{\it overt}]\>\\
        COMPS \<PP[{\it ini}]\>\]\\
   SEM {\it wait}({\it i, x})\]
   \end{avm}\label{wait}\z
%
The lexical information specifies that the second argument of {\it wait} can be an unrealized indefinite NP while the first argument needs to be an overt one. Now consider the dialogue in (\ref{35}). Uttering
the sentence {\it You're waiting?} would then update the DGB with a SAL-UTT represented by the unrealized NP:
%
\ea
\begin{avm}
\[%{\it dgb}\\
 DGB \[SAT-UTT \[SYN  PP\[{\it ini}\\
                          PFORM {\it for}\\
                        \IND\ {\it  x}\]\\
                 SEM {\it wait.for}({\it i,x})\]\]\] \end{avm}\z
%
The fragment {\it for what?}, matching this SAL-UTT,
projects a well-formed NU in accordance with the Head-Fragment Construction.\footnote{See the detailed analysis of such sprouting examples in \citet{Kim2015}.}

The advantages of the nonsentential analyses sketched here follow from their ability to capture limited morphosyntactic parallelism between NUs and SAL-UTT without having to account for why NUs don't behave like constituents of sentential structures. The island-violating behavior of NUs is unsurprising on this analysis, as are attested cases of structural mismatch and situationally controlled NUs.\footnote{The rarity of NUs with nonlinguistic antecedents can be understood as a function of how easily a situational context can give rise to a MAX-QUD and thus license ellipsis (see \citealt{Miller2014b} for this point with regard to VPE).} However, some loose ends still remain. (\ref{hf-cx}) currently has no means of capturing certain connectivity effects: it can't rule preposition drop out under sprouting and it incorrectly rules out case mismatch in languages like Hungarian for speakers that do accept it (see discussion around example (\ref{11})).\footnote{See, however, \citet{Kim2015} for proposals envoking a case hierarchy specific to Korean to explain case mismatch and introducing an additional constraint to block preposition drop under sprouting.}


\section{Analyses of predicate/argument ellipsis}
The first issue in the analysis of VPE is the status of an elided VP. It is assumed to be a {\it pro} element due to its pronominal properties \citep[see][]{Lobeck1995, Lopez2000, Kim2006, Aelbrecht2015, Ginzburg2018}. For instance, VPE applies only to phrasal categories (\ref{42}--\ref{43}),
can cross utterance boundaries (\ref{44}), can override island constraints (\ref{45}--\ref{46}), is subject to the Backwards Anaphora Constraint (\ref{47}--\ref{48}).

\ea *Mary will meet Bill at Stanford because she didn't  \jbtr John.\label{42}\z
\ea Mary will meet Bill at Stanford because she didn't \jbtr at Harvard.\label{43}\z
\ea A: Tom won't leave Seoul soon.\\
B: I don't think Mary will \jbtr either.\label{44}\z
\ea John didn't hit a home run, but I know a woman who did. (CNPC)\label{45}\z
\ea That Betsy won the batting crown is not surprising, but that
Peter didn't know she did \jbtr is indeed surprising. (SSC)\label{46}\z
\ea *Sue didn't [e] but John ate meat.\label{47}\z
\ea Because Sue didn't [e], John ate meat.\label{48}\z

%Argument ellipsis we find in languages like Polish and Korean can also be taken to be ellipsis of a pronominal %expression, as in (\ref{49}).
%
%\ea
%\gll Mimi-ka {\it pro} po-ass-ta.\\
%Mimi.NOM  pro see.PST.DECL\\
%\glt `Mimi saw (him)'\label{49}\z

In accounting for {\it pro}-drop phenomena, we do not need to posit a phonologically empty pronoun if the level of argument structure is
available \citep[see][]{Bresnan1982a}.  As suggested by \citet{Kim200}
and \citet{Ginzburg2018}, we can simply encode the required pronominal properties in the argument structure. In the framework of HPSG, we represent this as the following Argument Realization Constraint allowing mismatch between argument-structure and syntactic-valence features:\footnote{Expressions have two subtypes: overt and covert ones, the latter of which has two subtypes, {\it pro} and {\it gap}. See \citet{Sag2012a} for details.}

\ea
Argument Realization Constraint (ARC):\\
\begin{avm}
{\it v-word} \;   $\Rightarrow$ \;
\[SYN\|VAL \[SUBJ & \@A\\
                 COMPS & \@B $\ominus$ list({\it pro})\]\\
  ARG-ST \@A $\oplus$ \@B\]
  \end{avm}\label{50}
\z
The argument realization here tells us that a {\it pro} element
in the argument structure need not be realized in the syntax.
 For
example, as represented in (\ref{51}), the auxiliary
verb {\it can} in examples like {\it John cannot dance, but Sandy can}
have its second argument as a {\it pro} VP, which is
thus not instantiated as the syntactic
complement of the verb. \index{transitive}

\ea
Lexical entry for {\it can}\\
\begin{avm}
\[{\it v-word}\\
 FORM \q<can\q>\\
 SYN\[HEAD \|VFORM \ \ {\it fin}\\
      VAL \[SUBJ & \q<\@1\q>\\
           COMPS & \q<  \; \q>\]\]\\
ARG-ST  \q<\@1NP, VP[{\it pro}]\q>\]
 \end{avm} \label{51}\z
%
%
As such, English VPE can be analyzed as a language-particular VP {\it pro} drop phenomenon, trigged
by a constraint like the following:

\ea\label{52}
Aux-Ellipsis Construction:\\
\begin{forest}
 [{\begin{avm}
 \[{\it aux-v-lxm}\\
   %\SYN\|\HEAD\|\AUX\ $+$\\
   ARG-ST \q<\@1XP, \@2YP\q>\]
 \  \ $\mapsto$ \ \
 \[{\it aux-ellipsis-wd}\\
   ARG-ST \q<\@1XP, \@2YP[{\it pro}]\q>\]
 \end{avm}}]
\end{forest}
\z
What this tells us is that an auxiliary verb selecting two arguments
can be projected into an elided auxiliary verb whose second argument
is realized as a small {\it pro}. Note that this argument is not mapped
onto the syntactic grammatical function COMPS. The output auxiliary
in (\ref{51}) will then project a structure like the one
in Figure~\ref{fig-53}.

\begin{figure}
\begin{forest}
[S
  [\begin{avm}\@1NP\end{avm}
      [Sandy]]
  [\begin{avm} \avml \hfil VP\\
      \[%{\it head-only-cxt} \& {\it ellip-cxt}\\
      HEAD \@2\\
      SUBJ \q< \@1\q>\]\avmr \end{avm}
    [{\begin{avm} \avml \hfil  V\\
        \[HEAD \@2\[\AUX\ +\]\\
        SUBJ \q<\@1\q>\\
        COMPS \q<\; \;\q>\\
        ARG-ST \q<\@1NP, VP[{\it pro}]\q>\] \avmr \end{avm}}
      [can]]]]
\end{forest}
\caption{add caption}\label{fig-53}
\end{figure}
The head daughter's COMPS list (VP[bse]) is empty because the second element in the ARG-ST\ is
a {\it pro}.



\section{Analyses of nonconstituent coordination and gapping}
In this section, we focus on RNR and gapping, whose analyses we address in separate subsections below.

\subsection{Right Node Raising}
A characteristic property of RNR is that it's the only phenomenon where seemingly incomplete structure has consistently attracted HPSG analyses involving deletion of silent material. All existing analyses of RNR \citep{Abeille2016, Beavers2004, Chaves2008-in-lexicon, Chavez2014, Crysmann2003, Yatabe2001, Yatabe2012} agree on this point, although some of this work proposes more than one mechanism for accounting for coordination of nonconstituents \citep{Chaves2014, Yatabe2001, Yatabe2012, Yatabe2018}.

The RNR literature engages with the question of what kind of deletion it is that targets shared material, based on the kind of material that may be RNRaised and the range of mismatches permitted between the left and right conjuncts.\footnote{Although we refer to the material on the left and right as conjuncts, it is been known since \citet{Hudson1976, Hudson1984} that RNR extends to other syntactic environments than coordination (see \citealt{Chaves2014} for stressing this point).} For instance, \citet[839--840]{Chaves2014} concludes that RNR can't be syntactic deletion because it exhibits various argument-structure mismatches (\ref{54}--\ref{55}) and can target material below the word level (\ref{56}--\ref{57}).


\ea Sue gave me---but I don't think I will ever read---[a book about relativity]. \label{54}\z

\ea Never let me---or insist that I---[pick the seats].\label{55}\z

\ea We ordered the hard- but they got us the soft-[cover edition].\label{56}\z

\ea Your theory under- and my theory over[generates].\label{57}\z
Furthermore, RNR can target strings that are not subject to any known syntactic operations, such as rightward movement \citep[865]{Chaves2014}.

\ea I thought it was going to be a good but it ended up being a very bad [reception].\label{58}\z

\ea Tonight a group of men, tomorrow night he himself, [would go out there somewhere and wait].\label{59}\z

\ea They were also as liberal or more liberal [than any other age group in the 1986 through 1989 surveys].\label{60}\z
RNRaised material can also be discontinuous, as in (\ref{61})--(\ref{62}) (\citealt[868]{Chaves2014}; citing \citealt[238--240]{Whitman2009}). \todosatz{No reference given, but there is one in cg.bib and one in lfg.bib which can be used by changing W to w.}

\ea Please move from the exit rows if you are unwilling or unable [to perform the necessary actions] without injury.\label{61}\z

\ea The blast upended and nearly sliced [an armored Chevrolet Suburban] in half.\label{62}\z
This evidence leads \citet{Chaves2014} to propose that: (1) only 'true' RNR should be accounted for via the mechanism of surface-based deletion, (2) this deletion is sensitive to morph form identity, and (3) the targets of RNR are any linearized strings, whether constituents or otherwise. \citeauthor{Chaves2014}' (\citeyear[874]{Chaves2014}) constraint licensing RNR is given in \ref{50}. It permits the M(orpho)P(honology) feature of the mother to contain only one instance (represented as $L_{3}$ in (\ref{63})) of the two morphophonologically identical sequences [FORM $F_{1}$],...,[FORM $F_{n}$] present in the daughters; the leftmost of these sequences undergoes deletion. The final list in the mother, $L_{4}$, may be empty or nonempty, depending on whether RNR is discontinuous.


\ea{\label{63}
Backward periphery deletion construction\\
\begin{avm}
\[phrase\\
  MP $L_{1}$:\type{ne-list} $\circ L_{2}$:\type{ne-list} $\circ L_{3} \circ L_{4}$ \] $\rightarrow$
  \end{avm}


   \begin{avm}
\[ phrase\\
   MP $L_{1} \circ \langle$ \[ FORM $F_{1}$\],\ldots,\[ FORM $F_{n}$\]$\rangle \circ L_{2} \circ L_{3}:\langle$\[FORM $F_{1}$\],\ldots,\[FORM $F_{n}$\]$\rangle \circ L_{4}$
\]
\end{avm}}\z



\citeauthor{Chaves2014}' (\citeyear{Chaves2014}) proposal reflects the idea that nonconstituent coordination is a multi-faceted phenomenon, requiring more than one kind of analysis. Indeed, the HPSG literature includes analyses based on NP-ellipsis or A(cross) T(he) B(oard) extraposition \citep{Chaves2014} and on phonological vs syntactic deletion \citep{Yatabe2001, Yatabe2012, Yatabe2018}. \citet{Abeille2016} argue for a finer-grained analysis of French RNR based on phonological deletion. Their empirical evidence reveals a split between functional and lexical categories in French such that the former permit mismatch between the two conjuncts (where determiners or prepositions differ) under RNR, while the latter don't.



\subsection{Gapping}
HPSG analyses of gapping fall into two kinds: one kind draws on \citeauthor{Beavers2004}'s (\citeyear{Beavers2004}) deletion-based analysis of nonconstituent coordination \citep{Chaves2009} and the other on \citeauthor{Ginzburg:Sag:2000}'s (\citeyear{Ginzburg:Sag:2000}) analysis of NUs \citep{Abeille2014, Park2018}. The latter analyses align gapping with analyses of NUs, as discussed in section 4, more than with analyses of nonconstituent coordination, and therefore gapping could be classified together with other NUs. We use the analysis due to \citet{Abeille2014} for illustration below.

\citet{Abeille2014} assume an identity condition on gapping requiring that gapping remnants match major constituents in the antecedent clause, which they term source clause. In other words, gapping remnants are constrained by elements of the argument structure of the verbal head present in the antecedent and absent from the rightmost conjunct. %, as originally proposed by \citet{Hankamer1971}\todosatz{no reference given}.
This identity condition correctly predicts that gapping remnants needn't appear in the same order as their counterparts in the antecedent (\ref{64}) \citep[see][156--158]{Sag1985}, nor are they required to be the same syntactic category as their counterparts (\ref{65}). However, note that despite the syntactic category mismatch, the NP {\it an incredible bore} belongs to the subcategorization frame of the predicate {\it become}.

\ea A policeman walked in at 11, and at 12, a fireman. \label{64}\z

\ea Pat has become [crazy]$_{AP}$ and Chris [an incredible bore]$_{NP}$.  \label{65}\z
\citet{Abeille2014} offer additional evidence from Romance (e.g., case mismatch between gapping remnants and their counterparts and even more possibilities of ordering remnants than is the case in English) to strengthen their point that syntactic identity is relaxed under gapping.


The key assumption in \citeauthor{Abeille2014}'s (\citeyear{Abeille2014}) analysis is that (two or more) gapping remnants form a cluster whose mother has an underspecified syntactic category, that is, is a non-headed phrase. This phrase then serves as the head daughter of a head-fragment phrase, whose syntactic category is also underspecified. This means that there is no unpronounced verbal head in the phrase to which gapping remnants belong. Furthermore, the contextual attribute SAL-UTT introduced for NUs serves to ensure syntactic identity between gapping remnants and their counterparts such that the latter are SAL-UTTs bearing the specification [Major +] as part of their Head feature and being coindexed with the remnants.

With these ingredients of the analysis in place, we reproduce the gapping construction in (\ref{66}). The construction represents asymmetric coordination in the sense that the daughters include both nonelliptical verbal conjuncts and head-fragment phrases with an underspecified syntactic category. The mother only shares its syntactic category with the nonelliptical conjuncts so that its own category is specified to be verbal.


\ea
Gapping construction\\\type{gapping-ph} $\rightarrow$ \type{coord-ph} \&\\
\begin{avm}
%$\Langle$
\< \[HEAD \fbox{H} \type{verbal}\\
CONTXT | BACKGROUND \{ ..., sym-discourse-rel(\fbox{M$_{1}$},..., \fbox{M$_{j}$}, \fbox{M$_{j+1}$},..., \fbox{M$_{n}$} ), ... \}\\

DTRS $\langle$ \[ HEAD \fbox{H} \[ verbal \\ CLUSTER elist \]\\CONTENT \fbox{M$_{1}$} \] ,..., \[
HEAD \fbox{H} \[ verbal \\ CLUSTER elist \]\\CONTENT \fbox{M$_{j}$} \]\] \> %$\Rangle
$\bigoplus$
\end{avm}
\begin{avm}
%$\Langle$
\<\[ HEAD \[CLUSTER $\langle \fbox{1},...,\fbox{n}\rangle$\] \\
             SOURCE \fbox{M$_{j}$}\\
             CONTENT \fbox{M$_{j+1}$} \],..., \[ HEAD \[CLUSTER $\langle$ \fbox{1$'$},...,\fbox{n$'$}$\rangle$\]\\
                                                SOURCE \fbox{M$_{j}$}\\
                                                CONTENT \fbox{M$_{n}$}\]\>%$\rangle$
\end{avm}
\label{66}
\z



\section{Summary}
This chapter has reviewed three types of ellipsis, nonsentential utterances, predicate ellipsis, and nonconstituent coordination, corresponding to three kinds of analysis within HPSG. The pattern that emerges from this overview is that HPSG favors the `what you see is what get' approach to ellipsis and limits a deletion-based approach, common in the minimalist literature on ellipsis, to a subset of nonconstituent coordination phenomena.

%\citep{Chomsky1957}.
%\citep{Comrie1981}




%\begin{table}
%\caption{Frequencies of word classes}
%\label{tab:1:frequencies}
% \begin{tabular}{lllll}
%  \lsptoprule
            %& nouns & verbs & adjectives & adverbs\\
 % \midrule
  %absolute  &   12 &    34  &    23     & 13\\
%  relative  &   3.1 &   8.9 &    5.7    & 3.2\\
 % \lspbottomrule
 %\end{tabular}
%\end{table}




\section*{Abbreviations}

\begin{tabularx}{.99\textwidth}{@{}lX}
NUs & Nonsentential utterances\\
BAE & Bare Argument Ellipsis\\
VPE & Verb Phrase Ellipsis\\
NCA & Null Complement Anaphora\\
SAL-UTT & Salient Utterance\\
MAX-QUD & Maximal-Question-under-Discussion\\
DGB & Dialog Game Board\\
\end{tabularx}


\section*{Acknowledgements}
We thank the editors and Yusuke Kubota for helpful comments.

{\sloppy
\printbibliography[heading=subbibliography,notkeyword=this]
}

}% AVM options

\end{document}

%      <!-- Local IspellDict: en_US-w_accents -->
