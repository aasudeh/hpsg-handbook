\documentclass{scrartcl} 

\usepackage{./langsci/styles/langsci-forest-setup}


\usepackage{./styles/my-xspace,./styles/abbrev,german}

\selectlanguage{USenglish}


\usepackage{./styles/lfg/dalrymple}

\RequirePackage{graphics}
\RequirePackage{./styles/lfg/trees}
%% \RequirePackage{avm}
%% \avmoptions{active}
%% \avmfont{\sc}
%% \avmvalfont{\sc}
\RequirePackage{./styles/lfg/lfgmacrosash}

\usepackage{./styles/lfg/glue}
\usepackage{./styles/lfg/prooftree}

%%%%%%%%%%%%%%%%%%%%%%%%%%%%%%
%% Markup
%%%%%%%%%%%%%%%%%%%%%%%%%%%%%%
\usepackage[normalem]{ulem} % For thinks like strikethrough, using \sout

\newcommand{\high}[1]{\textbf{#1}} % highlighted text
\newcommand{\term}[1]{\textit{#1}\/} % technical term
\newcommand{\qterm}[1]{`{#1}'} % technical term, quotes
%\newcommand{\trns}[1]{\strut `#1'} % translation in glossed example
\newcommand{\trnss}[1]{\strut \phantom{\sqz{}} `#1'} % translation in ungrammatical glossed example
\newcommand{\ttrns}[1]{(`#1')} % an in-text translation of a word
%\newcommand{\feat}[1]{\mbox{\textsc{\MakeLowercase{#1}}}}     % feature name
%\newcommand{\val}[1]{\mbox{\textsc{\MakeLowercase{#1}}}}    % f-structure value
\newcommand{\mg}[1]{\mbox{\textsc{\MakeLowercase{#1}}}}    % morphological gloss
%\newcommand{\word}[1]{\textit{#1}}       % mention of word
\providecommand{\kstar}[1]{{#1}\ensuremath{^*}}
\providecommand{\kplus}[1]{{#1}\ensuremath{^+}}
\newcommand{\template}[1]{@\textsc{\MakeLowercase{#1}}}
\newcommand{\templaten}[1]{\textsc{\MakeLowercase{#1}}}
\newcommand{\templatenn}[1]{\MakeUppercase{#1}}
\newcommand{\tempeq}{\ensuremath{=}}
\newcommand{\predval}[1]{\ensuremath{\langle}\textsc{#1}\ensuremath{\rangle}}
\newcommand{\predvall}[1]{{\rm `#1'}}
\newcommand{\fst}[1]{\textit{#1}\ensuremath{\,}}
\newcommand{\scare}[1]{`#1'} % scare quotes
\newcommand{\bracket}[1]{\ensuremath{\left\langle\mathit{#1}\right\rangle}}
\newcommand{\sectionw}[1][]{Section#1} % section word: for cap/non-cap
\newcommand{\tablew}[1][]{Table#1} % table word: for cap/non-cap
\newcommand{\lfgglue}{LFG+Glue}
\newcommand{\hpsgglue}{HPSG+Glue}
\newcommand{\gs}{GS}
\newcommand{\func}[1]{\ensuremath{\mathbf{#1}}}
\renewcommand{\glue}{Glue}
\newcommand{\exr}[1]{(\ref{ex:#1}}

%%%%%%%%%%%%%%%%%%%%%%%%%%%%%%
% Notation
%\newcommand{\xbar}[1]{$_{\mbox{\textsc{#1}$^{\raisebox{1ex}{}}$}}$}
\newcommand{\xprime}[2][]{\textup{\mbox{{#2}\ensuremath{^\prime_{\hspace*{-.0em}\mbox{\footnotesize\ensuremath{\mathit{#1}}}}}}}}
\providecommand{\xzero}[2][]{#2\ensuremath{^0_{\mbox{\footnotesize\ensuremath{\mathit{#1}}}}}}


\newcommand{\UP}{\Up}



\let\leftangle\langle
\let\rightangle\rangle

\newcommand{\pslabel}[1]{}


%\usepackage{pst-node}

\newcommand{\crossrefchaptert}[2][]{\citet*[#1]{chapters/#2}, Chapter~\ref{chap-#2} of this volume} 
\newcommand{\crossrefchapterp}[2][]{(\citealp*[#1]{chapters/#2}, Chapter~\ref{chap-#2} of this volume)}
\newcommand{\crossrefchapteralt}[2][]{\citealt*[#1]{chapters/#2}, Chapter~\ref{chap-#2} of this volume}
\newcommand{\crossrefchapteralp}[2][]{\citealp*[#1]{chapters/#2}, Chapter~\ref{chap-#2} of this volume}
% example of optional argument:
% \crossrefchapterp[for something, see:]{name}
% gives: (for something, see: Author 2018, Chapter~X of this volume)

\let\crossrefchapterw\crossrefchaptert




	\newcommand{\bstpath}{./langsci/bst/}
	\newcommand{\bbxpath}{\bstpath biblatex-sp-unified/bbx/}
	% \renewcommand{\bbxpath}{\bstpath}
	\newcommand{\cbxpath}{\bstpath biblatex-sp-unified/cbx/}

\usepackage[
	natbib=true,
	% \iflsUndecapitalize
	% style=\bstpath biblatex-langsci-unified-undecap,
	% \else
	% style=\bstpath biblatex-langsci-unified,
	% \fi
	style=\bbxpath biblatex-sp-unified,
	citestyle=\cbxpath sp-authoryear-comp,
	useprefix = true, %sort von, van, de where they should appear
	%refsection=chapter,
	maxbibnames=99,
	uniquename=false,
	mincrossrefs=99,
	maxcitenames=2,
	isbn=false,
	doi=false,
	url=false,
	eprint=false,
        autolang=hyphen,
        useprefix=true,
	backend=biber,
	indexing=cite,
  datamodel=\bstpath langsci   % add authauthor and autheditor as possible fields to bibtex entries
]{biblatex}

  \bibliography{../Bibliographies/stmue,
                ../localbibliography,
../Bibliographies/formal-background,
../Bibliographies/understudied-languages,
../Bibliographies/phonology,
../Bibliographies/case,
../Bibliographies/evolution,
../Bibliographies/agreement,
../Bibliographies/lexicon,
../Bibliographies/np,
../Bibliographies/negation,
../Bibliographies/argst,
../Bibliographies/binding,
../Bibliographies/complex-predicates,
../Bibliographies/coordination,
../Bibliographies/relative-clauses,
../Bibliographies/udc,
../Bibliographies/processing,
../Bibliographies/cl,
../Bibliographies/dg,
../Bibliographies/islands,
../Bibliographies/diachronic,
../Bibliographies/gesture,
../Bibliographies/semantics,
../Bibliographies/pragmatics,
../Bibliographies/information-structure,
../Bibliographies/idioms,
../Bibliographies/cg,
../Bibliographies/udc,
../Bibliographies/lfg}    %SW:  I added this line

%\bibliography{macros,own,general,crossreferences,hpsg-handbook} <-from Ash  ??

% If the user provided a shortauthor in the bibtex entry, we use the authentic author (as with the
% authorindex package) if it is defined, otherwise we use the author.
% This gets F/T as shorthand right and puts the guys in the index.

\renewbibmacro*{citeindex}{%
  \ifciteindex
    {\iffieldequalstr{labelnamesource}{shortauthor} % If biblatex uses shortauthor as the label of a bibitem
      {\ifnameundef{authauthor}                     % we check whether there is something in authauthor
        {\indexnames{author}}                       % if not, we use author
        {\indexnames{authauthor}}}                  % if yes, we use authauthor
      {\iffieldequalstr{labelnamesource}{author}    % if biblatex uses author we similarly test for
                                                    % authauthor and use this field
        {\ifnameundef{authauthor}% if defined use authauthor
          {\indexnames{author}}
          {\indexnames{authauthor}}} % if defined use this field
        {\iffieldequalstr{labelnamesource}{shorteditor} % same for editor
          {\ifnameundef{autheditor}
            {\indexnames{editor}}
            {\indexnames{autheditor}}}
          {\indexnames{labelname}}}}}               % as a fallback we index on whatever biblatex used.
    {}}


\usepackage{./styles/makros.2e}

\usepackage{./styles/langsci-minimal}


%\usepackage{amsmath}

\usepackage{./styles/avm+}

\renewcommand{\feat}[1]{\textsc{#1}}

\usepackage{./styles/additional-langsci-index-shortcuts}
\usepackage{./langsci/styles/langsci-gb4e}




\author{Stephen Wechsler and Ash Asudeh }
\title{HPSG and Lexical Functional Grammar}

% \chapterDOI{} %will be filled in at production


%\bibliography{Bibliographies/wechsler,Bibliographies/jp2,Bibliographies/stmue}

\input lfg-include.tex
