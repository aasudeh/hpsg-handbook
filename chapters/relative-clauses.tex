\documentclass[output=paper,nonflat,draftmode]{./langsci/langscibook} 

%\bibliography{localbibliography-rc}
%%% add all extra packages you need to load to this file 

\usepackage{graphicx}
\usepackage{tabularx}
\usepackage{amsmath} 
\usepackage{tipa}      % Davis Koenig
\usepackage{multicol}
\usepackage{lipsum}


\usepackage{./langsci/styles/langsci-optional} 
\usepackage{./langsci/styles/langsci-lgr}
%\usepackage{./styles/forest/forest}
\usepackage{./langsci/styles/langsci-forest-setup}
\usepackage{morewrites}

\usepackage{tikz-cd}

\usepackage{./styles/tikz-grid}
\usetikzlibrary{shadows}


%\usepackage{pgfplots} % for data/theory figure in minimalism.tex
% fix some issue with Mod https://tex.stackexchange.com/a/330076
\makeatletter
\let\pgfmathModX=\pgfmathMod@
\usepackage{pgfplots}%
\let\pgfmathMod@=\pgfmathModX
\makeatother

\usepackage{subcaption}

% Stefan Müller's styles
\usepackage{./styles/merkmalstruktur,german,./styles/makros.2e,./styles/my-xspace,./styles/article-ex,
./styles/eng-date}

\selectlanguage{USenglish}

\usepackage{./styles/abbrev}

\usepackage{./langsci/styles/jambox}

% Has to be loaded late since otherwise footnotes will not work

%%%%%%%%%%%%%%%%%%%%%%%%%%%%%%%%%%%%%%%%%%%%%%%%%%%%
%%%                                              %%%
%%%           Examples                           %%%
%%%                                              %%%
%%%%%%%%%%%%%%%%%%%%%%%%%%%%%%%%%%%%%%%%%%%%%%%%%%%%
% remove the percentage signs in the following lines
% if your book makes use of linguistic examples
\usepackage{./langsci/styles/langsci-gb4e} 

% Crossing out text
% uncomment when needed
%\usepackage{ulem}

\usepackage{./styles/additional-langsci-index-shortcuts}

%\usepackage{./langsci/styles/langsci-avm}
\usepackage{./styles/avm+}


\renewcommand{\tpv}[1]{{\avmjvalfont\itshape #1}}

% no small caps please
\renewcommand{\phonshape}[0]{\normalfont\itshape}

\regAvmFonts

\usepackage{theorem}

\newtheorem{mydefinition}{Def.}
\newtheorem{principle}{Principle}

{\theoremstyle{break}
%\newtheorem{schema}{Schema}
\newtheorem{mydefinition-break}[mydefinition]{Def.}
\newtheorem{principle-break}[principle]{Principle}
}

% This avoids linebreaks in the Schema
\newcounter{schema}
\newenvironment{schema}[1][]
  {% \begin{Beispiel}[<title>]
  \goodbreak%
  \refstepcounter{schema}%
  \begin{list}{}{\setlength{\labelwidth}{0pt}\setlength{\labelsep}{0pt}\setlength{\rightmargin}{0pt}\setlength{\leftmargin}{0pt}}%
    \item[{\textbf{Schema~\theschema}}]\hspace{.5em}\textbf{(#1)}\nopagebreak[4]\par\nobreak}%
  {\end{list}}% \end{Beispiel}

%% \newcommand{schema}[2]{
%% \begin{minipage}{\textwidth}
%% {\textbf{Schema~\theschema}}]\hspace{.5em}\textbf{(#1)}\\
%% #2
%% \end{minipage}}

%\usepackage{subfig}





% Davis Koenig Lexikon

\usepackage{tikz-qtree,tikz-qtree-compat} % Davis Koenig remove

\usepackage{shadow}




\usepackage[english]{isodate} % Andy Lücking
\usepackage[autostyle]{csquotes} % Andy
%\usepackage[autolanguage]{numprint}

%\defaultfontfeatures{
%    Path = /usr/local/texlive/2017/texmf-dist/fonts/opentype/public/fontawesome/ }

%% https://tex.stackexchange.com/a/316948/18561
%\defaultfontfeatures{Extension = .otf}% adds .otf to end of path when font loaded without ext parameter e.g. \newfontfamily{\FA}{FontAwesome} > \newfontfamily{\FA}{FontAwesome.otf}
%\usepackage{fontawesome} % Andy Lücking
\usepackage{pifont} % Andy Lücking -> hand

\usetikzlibrary{decorations.pathreplacing} % Andy Lücking
\usetikzlibrary{matrix} % Andy 
\usetikzlibrary{positioning} % Andy
\usepackage{tikz-3dplot} % Andy

% pragmatics
\usepackage{eqparbox} % Andy
\usepackage{enumitem} % Andy
\usepackage{longtable} % Andy
\usepackage{tabu} % Andy


% Manfred's packages

%\usepackage{shadow}

\usepackage{tabularx}
\newcolumntype{L}[1]{>{\raggedright\arraybackslash}p{#1}} % linksbündig mit Breitenangabe


% Jong-Bok

%\usepackage{xytree}

\newcommand{\xytree}[2][dummy]{Let's do the tree!}

% seems evil, get rid of it
% defines \ex is incompatible with gb4e
%\usepackage{lingmacros}

% taken from lingmacros:
\makeatletter
% \evnup is used to line up the enumsentence number and an entry along
% the top.  It can take an argument to improve lining up.
\def\evnup{\@ifnextchar[{\@evnup}{\@evnup[0pt]}}

\def\@evnup[#1]#2{\setbox1=\hbox{#2}%
\dimen1=\ht1 \advance\dimen1 by -.5\baselineskip%
\advance\dimen1 by -#1%
\leavevmode\lower\dimen1\box1}
\makeatother


% YK -- CG chapter

%\usepackage{xspace}
\usepackage{bm}
\usepackage{bussproofs}


% Antonio Branco, remove this
\usepackage{epsfig}

% now unicode
%\usepackage{alphabeta}



% Berthold udc
%\usepackage{qtree}
%\usepackage{rtrees}

\usepackage{pst-node}

%%%add all your local new commands to this file

\makeatletter
\def\blx@maxline{77}
\makeatother


\newcommand{\page}{}



\newcommand{\todostefan}[1]{\todo[color=orange!80]{\footnotesize #1}\xspace}
\newcommand{\todosatz}[1]{\todo[color=red!40]{\footnotesize #1}\xspace}

\newcommand{\inlinetodostefan}[1]{\todo[color=green!40,inline]{\footnotesize #1}\xspace}


\newcommand{\spacebr}{\hspaceThis{[}}

\newcommand{\danish}{\jambox{(\ili{Danish})}}
\newcommand{\english}{\jambox{(\ili{English})}}
\newcommand{\german}{\jambox{(\ili{German})}}
\newcommand{\yiddish}{\jambox{(\ili{Yiddish})}}
\newcommand{\welsh}{\jambox{(\ili{Welsh})}}

% Cite and cross-reference other chapters
\newcommand{\crossrefchaptert}[2][]{\citet*[#1]{chapters/#2}, Chapter~\ref{chap-#2} of this volume} 
\newcommand{\crossrefchapterp}[2][]{(\citealp*[#1][]{chapters/#2}, Chapter~\ref{chap-#2} of this volume)}
% example of optional argument:
% \crossrefchapterp[for something, see:]{name}
% gives: (for something, see: Author 2018, Chapter~X of this volume)

\let\crossrefchapterw\crossrefchaptert



% Davis Koenig

\let\ig=\textsc
\let\tc=\textcolor

% evolution, Flickinger, Pollard, Wasow

\let\citeNP\citet

% Adam P

%\newcommand{\toappear}{Forthcoming}
\newcommand{\pg}[1]{p.#1}
\renewcommand{\implies}{\rightarrow}

\newcommand*{\rref}[1]{(\ref{#1})}
\newcommand*{\aref}[1]{(\ref{#1}a)}
\newcommand*{\bref}[1]{(\ref{#1}b)}
\newcommand*{\cref}[1]{(\ref{#1}c)}

\newcommand{\msadam}{.}
\newcommand{\morsyn}[1]{\textsc{#1}}

\newcommand{\nom}{\morsyn{nom}}
\newcommand{\acc}{\morsyn{acc}}
\newcommand{\dat}{\morsyn{dat}}
\newcommand{\gen}{\morsyn{gen}}
\newcommand{\ins}{\morsyn{ins}}
\newcommand{\loc}{\morsyn{loc}}
\newcommand{\voc}{\morsyn{voc}}
\newcommand{\ill}{\morsyn{ill}}
\renewcommand{\inf}{\morsyn{inf}}
\newcommand{\passprc}{\morsyn{passp}}

%\newcommand{\Nom}{\msadam\nom}
%\newcommand{\Acc}{\msadam\acc}
%\newcommand{\Dat}{\msadam\dat}
%\newcommand{\Gen}{\msadam\gen}
\newcommand{\Ins}{\msadam\ins}
\newcommand{\Loc}{\msadam\loc}
\newcommand{\Voc}{\msadam\voc}
\newcommand{\Ill}{\msadam\ill}
\newcommand{\INF}{\msadam\inf}
\newcommand{\PassP}{\msadam\passprc}

\newcommand{\Aux}{\textsc{aux}}

\newcommand{\princ}[1]{\textnormal{\textsc{#1}}} % for constraint names
\newcommand{\notion}[1]{\emph{#1}}
\renewcommand{\path}[1]{\textnormal{\textsc{#1}}}
\newcommand{\ftype}[1]{\textit{#1}}
\newcommand{\fftype}[1]{{\scriptsize\textit{#1}}}
\newcommand{\la}{$\langle$}
\newcommand{\ra}{$\rangle$}
%\newcommand{\argst}{\path{arg-st}}
\newcommand{\phtm}[1]{\setbox0=\hbox{#1}\hspace{\wd0}}
\newcommand{\prep}[1]{\setbox0=\hbox{#1}\hspace{-1\wd0}#1}

%%%%%%%%%%%%%%%%%%%%%%%%%%%%%%%%%%%%%%%%%%%%%%%%%%%%%%%%%%%%%%%%%%%%%%%%%%%

% FROM FS.STY:

%%%
%%% Feature structures
%%%

% \fs         To print a feature structure by itself, type for example
%             \fs{case:nom \\ person:P}
%             or (better, for true italics),
%             \fs{\it case:nom \\ \it person:P}
%
% \lfs        To print the same feature structure with the category
%             label N at the top, type:
%             \lfs{N}{\it case:nom \\ \it person:P}

%    Modified 1990 Dec 5 so that features are left aligned.
\newcommand{\fs}[1]%
{\mbox{\small%
$
\!
\left[
  \!\!
  \begin{tabular}{l}
    #1
  \end{tabular}
  \!\!
\right]
\!
$}}

%     Modified 1990 Dec 5 so that features are left aligned.
%\newcommand{\lfs}[2]
%   {
%     \mbox{$
%           \!\!
%           \begin{tabular}{c}
%           \it #1
%           \\
%           \mbox{\small%
%                 $
%                 \left[
%                 \!\!
%                 \it
%                 \begin{tabular}{l}
%                 #2
%                 \end{tabular}
%                 \!\!
%                 \right]
%                 $}
%           \end{tabular}
%           \!\!
%           $}
%   }

\newcommand{\ft}[2]{\path{#1}\hspace{1ex}\ftype{#2}}
\newcommand{\fsl}[2]{\fs{{\fftype{#1}} \\ #2}}

\newcommand{\fslt}[2]
 {\fst{
       {\fftype{#1}} \\
       #2 
     }
 }

\newcommand{\fsltt}[2]
 {\fstt{
       {\fftype{#1}} \\
       #2 
     }
 }

\newcommand{\fslttt}[2]
 {\fsttt{
       {\fftype{#1}} \\
       #2 
     }
 }


% jak \ft, \fs i \fsl tylko nieco ciasniejsze

\newcommand{\ftt}[2]
% {{\sc #1}\/{\rm #2}}
 {\textsc{#1}\/{\rm #2}}

\newcommand{\fst}[1]
  {
    \mbox{\small%
          $
          \left[
          \!\!\!
%          \sc
          \begin{tabular}{l} #1
          \end{tabular}
          \!\!\!\!\!\!\!
          \right]
          $
          }
   }

%\newcommand{\fslt}[2]
% {\fst{#2\\
%       {\scriptsize\it #1}
%      }
% }


% superciasne

\newcommand{\fstt}[1]
  {
    \mbox{\small%
          $
          \left[
          \!\!\!
%          \sc
          \begin{tabular}{l} #1
          \end{tabular}
          \!\!\!\!\!\!\!\!\!\!\!
          \right]
          $
          }
   }

%\newcommand{\fsltt}[2]
% {\fstt{#2\\
%       {\scriptsize\it #1}
%      }
% }

\newcommand{\fsttt}[1]
  {
    \mbox{\small%
          $
          \left[
          \!\!\!
%          \sc
          \begin{tabular}{l} #1
          \end{tabular}
          \!\!\!\!\!\!\!\!\!\!\!\!\!\!\!\!
          \right]
          $
          }
   }



% %add all your local new commands to this file

% \newcommand{\smiley}{:)}

% you are not supposed to mess with hardcore stuff, St.Mü. 22.08.2018
%% \renewbibmacro*{index:name}[5]{%
%%   \usebibmacro{index:entry}{#1}
%%     {\iffieldundef{usera}{}{\thefield{usera}\actualoperator}\mkbibindexname{#2}{#3}{#4}{#5}}}

% % \newcommand{\noop}[1]{}



% Rui

\newcommand{\spc}[0]{\hspace{-1pt}\underline{\hspace{6pt}}\,}
\newcommand{\spcs}[0]{\hspace{-1pt}\underline{\hspace{6pt}}\,\,}
\newcommand{\bad}[1]{\leavevmode\llap{#1}}
\newcommand{\COMMENT}[1]{}


% Andy Lücking gesture.tex
\newcommand{\Pointing}{\ding{43}}
% Giotto: "Meeting of Joachim and Anne at the Golden Gate" - 1305-10 
\definecolor{GoldenGate1}{rgb}{.13,.09,.13} % Dress of woman in black
\definecolor{GoldenGate2}{rgb}{.94,.94,.91} % Bridge
\definecolor{GoldenGate3}{rgb}{.06,.09,.22} % Blue sky
\definecolor{GoldenGate4}{rgb}{.94,.91,.87} % Dress of woman with shawl
\definecolor{GoldenGate5}{rgb}{.52,.26,.26} % Joachim's robe
\definecolor{GoldenGate6}{rgb}{.65,.35,.16} % Anne's robe
\definecolor{GoldenGate7}{rgb}{.91,.84,.42} % Joachim's halo

\makeatletter
\newcommand{\@Depth}{1} % x-dimension, to front
\newcommand{\@Height}{1} % z-dimension, up
\newcommand{\@Width}{1} % y-dimension, rightwards
%\GGS{<x-start>}{<y-start>}{<z-top>}{<z-bottom>}{<Farbe>}{<x-width>}{<y-depth>}{<opacity>}
\newcommand{\GGS}[9][]{%
\coordinate (O) at (#2-1,#3-1,#5);
\coordinate (A) at (#2-1,#3-1+#7,#5);
\coordinate (B) at (#2-1,#3-1+#7,#4);
\coordinate (C) at (#2-1,#3-1,#4);
\coordinate (D) at (#2-1+#8,#3-1,#5);
\coordinate (E) at (#2-1+#8,#3-1+#7,#5);
\coordinate (F) at (#2-1+#8,#3-1+#7,#4);
\coordinate (G) at (#2-1+#8,#3-1,#4);
\draw[draw=black, fill=#6, fill opacity=#9] (D) -- (E) -- (F) -- (G) -- cycle;% Front
\draw[draw=black, fill=#6, fill opacity=#9] (C) -- (B) -- (F) -- (G) -- cycle;% Top
\draw[draw=black, fill=#6, fill opacity=#9] (A) -- (B) -- (F) -- (E) -- cycle;% Right
}
\makeatother


% pragmatics
\newcommand{\speaking}[1]{\eqparbox{name}{\textsc{\lowercase{#1}\space}}}
\newcommand{\name}[1]{\eqparbox{name}{\textsc{\lowercase{#1}}}}
\newcommand{\HPSGTTR}{HPSG$_{\text{TTR}}$\xspace}

\newcommand{\ttrtype}[1]{\textit{#1}}
% \newcommand{\avmel}{\q<\quad\q>} %% shortcut for empty lists in AVM
\newcommand{\ttrmerge}{\ensuremath{\wedge_{\textit{merge}}}}
\newcommand{\Cat}[2][0.1pt]{%
  \begin{scope}[y=#1,x=#1,yscale=-1, inner sep=0pt, outer sep=0pt]
   \path[fill=#2,line join=miter,line cap=butt,even odd rule,line width=0.8pt]
  (151.3490,307.2045) -- (264.3490,307.2045) .. controls (264.3490,291.1410) and (263.2021,287.9545) .. (236.5990,287.9545) .. controls (240.8490,275.2045) and (258.1242,244.3581) .. (267.7240,244.3581) .. controls (276.2171,244.3581) and (286.3490,244.8259) .. (286.3490,264.2045) .. controls (286.3490,286.2045) and (323.3717,321.6755) .. (332.3490,307.2045) .. controls (345.7277,285.6390) and (309.3490,292.2151) .. (309.3490,240.2046) .. controls (309.3490,169.0514) and (350.8742,179.1807) .. (350.8742,139.2046) .. controls (350.8742,119.2045) and (345.3490,116.5037) .. (345.3490,102.2045) .. controls (345.3490,83.3070) and (361.9972,84.4036) .. (358.7581,68.7349) .. controls (356.5206,57.9117) and (354.7696,49.2320) .. (353.4652,36.1439) .. controls (352.5396,26.8573) and (352.2445,16.9594) .. (342.5985,17.3574) .. controls (331.2650,17.8250) and (326.9655,37.7742) .. (309.3490,39.2045) .. controls (291.7685,40.6320) and (276.7783,24.2380) .. (269.9740,26.5795) .. controls (263.2271,28.9013) and (265.3490,47.2045) .. (269.3490,60.2045) .. controls (275.6359,80.6368) and (289.3490,107.2045) .. (264.3490,111.2045) .. controls (239.3490,115.2045) and (196.3490,119.2045) .. (165.3490,160.2046) .. controls (134.3490,201.2046) and (135.4934,249.3212) .. (123.3490,264.2045) .. controls (82.5907,314.1553) and (40.8239,293.6463) .. (40.8239,335.2045) .. controls (40.8239,353.8102) and (72.3490,367.2045) .. (77.3490,361.2045) .. controls (82.3490,355.2045) and (34.8638,337.3259) .. (87.9955,316.2045) .. controls (133.3871,298.1601) and   (137.4391,294.4766) .. (151.3490,307.2045) -- cycle;
\end{scope}%
}


% KdK
\newcommand{\smiley}{:)}

\renewbibmacro*{index:name}[5]{%
  \usebibmacro{index:entry}{#1}
    {\iffieldundef{usera}{}{\thefield{usera}\actualoperator}\mkbibindexname{#2}{#3}{#4}{#5}}}

% \newcommand{\noop}[1]{}

% chngcntr.sty otherwise gives error that these are already defined
%\let\counterwithin\relax
%\let\counterwithout\relax

% the space of a left bracket for glossings
\newcommand{\LB}{\hspaceThis{[}}

\newcommand{\LF}{\mbox{$[\![$}}

\newcommand{\RF}{\mbox{$]\!]_F$}}

\newcommand{\RT}{\mbox{$]\!]_T$}}





% Manfred's

\newcommand{\kommentar}[1]{}

\newcommand{\bsp}[1]{\emph{#1}}
\newcommand{\bspT}[2]{\bsp{#1} `#2'}
\newcommand{\bspTL}[3]{\bsp{#1} (lit.: #2) `#3'}

\newcommand{\noidi}{§}

\newcommand{\refer}[1]{(\ref{#1})}

%\newcommand{\avmtype}[1]{\multicolumn{2}{l}{\type{#1}}}
\newcommand{\attr}[1]{\textsc{#1}}

\newcommand{\srdefault}{\mbox{\begin{tabular}{c}{\large <}\\[-1.5ex]$\sqcap$\end{tabular}}}

%% \newcommand{\myappcolumn}[2]{
%% \begin{minipage}[t]{#1}#2\end{minipage}
%% }

%% \newcommand{\appc}[1]{\myappcolumn{3.7cm}{#1}}


% Jong-Bok


% clean that up and do not use \def (killing other stuff defined before)
%\if 0
\def\DEL{\textsc{del}}
\def\del{\textsc{del}}

\def\conn{\textsc{conn}}
\def\CONN{\textsc{conn}}
\def\CONJ{\textsc{conj}}
\def\LITE{\textsc{lex}}
\def\lite{\textsc{lex}}
\def\HON{\textsc{hon}}

\def\CAUS{\textsc{caus}}
\def\PASS{\textsc{pass}}
\def\NPST{\textsc{npst}}
\def\COND{\textsc{cond}}



\def\hd-lite{\textsc{head-lex construction}}
\def\NFORM{\textsc{nform}}

\def\RELS{\textsc{rels}}
\def\TENSE{\textsc{tense}}


%\def\ARG{\textsc{arg}}
\def\ARGs{\textsc{arg0}}
\def\ARGa{\textsc{arg}}
\def\ARGb{\textsc{arg2}}
\def\TPC{\textsc{top}}
\def\PROG{\textsc{prog}}

\def\pst{\textsc{pst}}
\def\PAST{\textsc{pst}}
\def\DAT{\textsc{dat}}
\def\CONJ{\textsc{conj}}
\def\nominal{\textsc{nominal}}
\def\NOMINAL{\textsc{nominal}}
\def\VAL{\textsc{val}}
\def\val{\textsc{val}}
\def\MODE{\textsc{mode}}
\def\RESTR{\textsc{restr}}
\def\SIT{\textsc{sit}}
\def\ARG{\textsc{arg}}
\def\RELN{\textsc{rel}}
\def\REL{\textsc{rel}}
\def\RELS{\textsc{rels}}
\def\arg-st{\textsc{arg-st}}
\def\xdel{\textsc{xdel}}
\def\zdel{\textsc{zdel}}
\def\sug{\textsc{sug}}
\def\IMP{\textsc{imp}}
\def\conn{\textsc{conn}}
\def\CONJ{\textsc{conj}}
\def\HON{\textsc{hon}}
\def\BN{\textsc{bn}}
\def\bn{\textsc{bn}}
\def\pres{\textsc{pres}}
\def\PRES{\textsc{pres}}
\def\prs{\textsc{pres}}
\def\PRS{\textsc{pres}}
\def\agt{\textsc{agt}}
\def\DEL{\textsc{del}}
\def\PRED{\textsc{pred}}
\def\AGENT{\textsc{agent}}
\def\THEME{\textsc{theme}}
\def\AUX{\textsc{aux}}
\def\THEME{\textsc{theme}}
\def\PL{\textsc{pl}}
\def\SRC{\textsc{src}}
\def\src{\textsc{src}}
\def\FORM{\textsc{form}}
\def\form{\textsc{form}}
\def\GCASE{\textsc{gcase}}
\def\gcase{\textsc{gcase}}
\def\SCASE{\textsc{scase}}
\def\PHON{\textsc{phon}}
\def\SS{\textsc{ss}}
\def\SYN{\textsc{syn}}
\def\LOC{\textsc{loc}}
\def\MOD{\textsc{mod}}
\def\INV{\textsc{inv}}
\def\L{\textsc{l}}
\def\CASE{\textsc{case}}
\def\SPR{\textsc{spr}}
\def\COMPS{\textsc{comps}}
%\def\comps{\textsc{comps}}
\def\SEM{\textsc{sem}}
\def\CONT{\textsc{cont}}
\def\SUBCAT{\textsc{subcat}}
\def\CAT{\textsc{cat}}
\def\C{\textsc{c}}
\def\SUBJ{\textsc{subj}}
\def\subj{\textsc{subj}}
\def\SLASH{\textsc{slash}}
\def\LOCAL{\textsc{local}}
\def\ARG-ST{\textsc{arg-st}}
\def\AGR{\textsc{agr}}
\def\PER{\textsc{per}}
\def\NUM{\textsc{num}}
\def\IND{\textsc{ind}}
\def\VFORM{\textsc{vform}}
\def\PFORM{\textsc{pform}}
\def\decl{\textsc{decl}}
\def\loc{\textsc{loc   }}
% \def\   {\textsc{  }}

\def\NEG{\textsc{neg}}
\def\FRAMES{\textsc{frames}}
\def\REFL{\textsc{refl}}

\def\MKG{\textsc{mkg}}

\def\BN{\textsc{bn}}
\def\HD{\textsc{hd}}
\def\NP{\textsc{np}}
\def\PF{\textsc{pf}}
\def\PL{\textsc{pl}}
\def\PP{\textsc{pp}}
\def\SS{\textsc{ss}}
\def\VF{\textsc{vf}}
\def\VP{\textsc{vp}}
\def\bn{\textsc{bn}}
\def\cl{\textsc{cl}}
\def\pl{\textsc{pl}}
\def\Wh{\ital{Wh}}
\def\ng{\textsc{neg}}
\def\wh{\ital{wh}}
\def\ACC{\textsc{acc}}
\def\AGR{\textsc{agr}}
\def\AGT{\textsc{agt}}
\def\ARC{\textsc{arc}}
\def\ARG{\textsc{arg}}
\def\ARP{\textsc{arc}}
\def\AUX{\textsc{aux}}
\def\CAT{\textsc{cat}}
\def\COP{\textsc{cop}}
\def\DAT{\textsc{dat}}
\def\DEF{\textsc{def}}
\def\DEL{\textsc{del}}
\def\DOM{\textsc{dom}}
\def\DTR{\textsc{dtr}}
\def\FUT{\textsc{fut}}
\def\GAP{\textsc{gap}}
\def\GEN{\textsc{gen}}
\def\HON{\textsc{hon}}
\def\IMP{\textsc{imp}}
\def\IND{\textsc{ind}}
\def\INV{\textsc{inv}}
\def\LEX{\textsc{lex}}
\def\Lex{\textsc{lex}}
\def\LOC{\textsc{loc}}
\def\MOD{\textsc{mod}}
\def\MRK{{\nr MRK}}
\def\NEG{\textsc{neg}}
\def\NEW{\textsc{new}}
\def\NOM{\textsc{nom}}
\def\NUM{\textsc{num}}
\def\PER{\textsc{per}}
\def\PST{\textsc{pst}}
\def\QUE{\textsc{que}}
\def\REL{\textsc{rel}}
\def\SEL{\textsc{sel}}
\def\SEM{\textsc{sem}}
\def\SIT{\textsc{arg0}}
\def\SPR{\textsc{spr}}
\def\SRC{\textsc{src}}
\def\SUG{\textsc{sug}}
\def\SYN{\textsc{syn}}
\def\TPC{\textsc{top}}
\def\VAL{\textsc{val}}
\def\acc{\textsc{acc}}
\def\agt{\textsc{agt}}
\def\cop{\textsc{cop}}
\def\dat{\textsc{dat}}
\def\foc{\textsc{focus}}
\def\FOC{\textsc{focus}}
\def\fut{\textsc{fut}}
\def\hon{\textsc{hon}}
\def\imp{\textsc{imp}}
\def\kes{\textsc{kes}}
\def\lex{\textsc{lex}}
\def\loc{\textsc{loc}}
\def\mrk{{\nr MRK}}
\def\nom{\textsc{nom}}
\def\num{\textsc{num}}
\def\plu{\textsc{plu}}
\def\pne{\textsc{pne}}
\def\pst{\textsc{pst}}
\def\pur{\textsc{pur}}
\def\que{\textsc{que}}
\def\src{\textsc{src}}
\def\sug{\textsc{sug}}
\def\tpc{\textsc{top}}
\def\utt{\textsc{utt}}
\def\val{\textsc{val}}
\def\LITE{\textsc{lex}}
\def\PAST{\textsc{pst}}
\def\POSP{\textsc{pos}}
\def\PRS{\textsc{pres}}
\def\mod{\textsc{mod}}%
\def\newuse{{`kes'}}
\def\posp{\textsc{pos}}
\def\prs{\textsc{pres}}
\def\psp{{\it en\/}}
\def\skes{\textsc{kes}}
\def\CASE{\textsc{case}}
\def\CASE{\textsc{case}}
\def\COMP{\textsc{comp}}
\def\CONJ{\textsc{conj}}
\def\CONN{\textsc{conn}}
\def\CONT{\textsc{cont}}
\def\DECL{\textsc{decl}}
\def\FOCUS{\textsc{focus}}
\def\FORM{\textsc{form}}
\def\FREL{\textsc{frel}}
\def\GOAL{\textsc{goal}}
\def\HEAD{\textsc{head}}
\def\INDEX{\textsc{ind}}
\def\INST{\textsc{inst}}
\def\MODE{\textsc{mode}}
\def\MOOD{\textsc{mood}}
\def\NMLZ{\textsc{nmlz}}
\def\PHON{\textsc{phon}}
\def\PRED{\textsc{pred}}
%\def\PRES{\textsc{pres}}
\def\PROM{\textsc{prom}}
\def\RELN{\textsc{pred}}
\def\RELS{\textsc{rels}}
\def\STEM{\textsc{stem}}
\def\SUBJ{\textsc{subj}}
\def\XARG{\textsc{xarg}}
\def\bse{{\it bse\/}}
\def\case{\textsc{case}}
\def\caus{\textsc{caus}}
\def\comp{\textsc{comp}}
\def\conj{\textsc{conj}}
\def\conn{\textsc{conn}}
\def\decl{\textsc{decl}}
\def\fin{{\it fin\/}}
\def\form{\textsc{form}}
\def\gend{\textsc{gend}}
\def\inf{{\it inf\/}}
\def\mood{\textsc{mood}}
\def\nmlz{\textsc{nmlz}}
\def\pass{\textsc{pass}}
\def\past{\textsc{past}}
\def\perf{\textsc{perf}}
\def\pln{{\it pln\/}}
\def\pred{\textsc{pred}}


%\def\pres{\textsc{pres}}
\def\proc{\textsc{proc}}
\def\nonfin{{\it nonfin\/}}
\def\AGENT{\textsc{agent}}
\def\CFORM{\textsc{cform}}
%\def\COMPS{\textsc{comps}}
\def\COORD{\textsc{coord}}
\def\COUNT{\textsc{count}}
\def\EXTRA{\textsc{extra}}
\def\GCASE{\textsc{gcase}}
\def\GIVEN{\textsc{given}}
\def\LOCAL{\textsc{local}}
\def\NFORM{\textsc{nform}}
\def\PFORM{\textsc{pform}}
\def\SCASE{\textsc{scase}}
\def\SLASH{\textsc{slash}}
\def\SLASH{\textsc{slash}}
\def\THEME{\textsc{theme}}
\def\TOPIC{\textsc{topic}}
\def\VFORM{\textsc{vform}}
\def\cause{\textsc{cause}}
%\def\comps{\textsc{comps}}
\def\gcase{\textsc{gcase}}
\def\itkes{{\it kes\/}}
\def\pass{{\it pass\/}}
\def\vform{\textsc{vform}}
\def\CCONT{\textsc{c-cont}}
\def\GN{\textsc{given-new}}
\def\INFO{\textsc{info-st}}
\def\ARG-ST{\textsc{arg-st}}
\def\SUBCAT{\textsc{subcat}}
\def\SYNSEM{\textsc{synsem}}
\def\VERBAL{\textsc{verbal}}
\def\arg-st{\textsc{arg-st}}
\def\plain{{\it plain}\/}
\def\propos{\textsc{propos}}
\def\ADVERBIAL{\textsc{advl}}
\def\HIGHLIGHT{\textsc{prom}}
\def\NOMINAL{\textsc{nominal}}

\newenvironment{myavm}{\begingroup\avmvskip{.1ex}
  \selectfont\begin{avm}}%
{\end{avm}\endgroup\medskip}
\def\pfix{\vspace{-5pt}}


\def\jbsub#1{\lower4pt\hbox{\small #1}}
\def\jbssub#1{\lower4pt\hbox{\small #1}}
\def\jbtr{\underbar{\ \ \ }\ }


%\fi

%\input{localcommands-rc.tex} 

%% the notactive option is necessary for this document:
\avmoptions{notactive,center}

%%  \crossrefchaptert{ChXXXX}) -- should be a cross ref to the chapter that introduces
%%  multiple inheritance from different dimensions
%%  \crossrefchapterp{ChYYYY} -- should be a cross ref to the chapter on resumptive pronouns
%% In Figure 3, and Figure 9 I took your advice and used \trace for the empty leaf -- is this what you wanted?

\title{Relative clauses in HPSG} 
\author{%
 Doug Arnold\affiliation{University of Essex}\lastand 
 Danièle Godard\affiliation{University de Paris Diderot}
}
% \chapterDOI{} %will be filled in at production

% \epigram{}



\abstract{ We provide an extended discussion of analyses of relative clauses
  (prototypically clauses with a noun modifying function) and related constructions that
  have appeared in the HPSG literature.  The basic theoretical approaches are presented
  (specifically, the lexical ``head-driven'' approach associated with earlier
  work in HPSG and the more recent constructional approach), followed by descriptions of
  analyses of different kinds of relative clause across a range of typologically diverse
  languages (notably Arabic, English, French, German, Japanese, and Korean). Phenomena
  discussed include \lic{wh}-relatives, relatives headed by complementisers, ``bare'' relatives,
  non-restrictive relatives, extraposition of relative clauses, relative clause-like
  constructions that function as complements, various kinds of ``dependent noun'' and
  ``pseudo'' relative clause, and free (headless) relatives.
}

\begin{document}
\maketitle
\label{chap-relative-clauses}
%% index cross references
\is{pied-piping|see{relative inheritance}}
\is{wh-percolation@\lic{wh}-percolation|see{relative inheritance}}
\is{REL@\textsc{rel} inheritance|see{relative inheritance}}
\is{relative clause!headless relative|see{relative clause, free}}
\is{relative clause!fused relative|see{relative clause, free}}
\is{relative clause!infinitival|see{relative clause, non-finite}}
\is{relative clause!supplemental|see{relative clause, non-restrictive}}
\is{relative clause!supplementary|see{relative clause, non-restrictive}}
\is{relative clause!appositive|see{relative clause, non-restrictive}}
\is{relative clause!that-less@\emph{that}-less|see{relative clause, bare}}
\is{relative clause!R, RP|see{relative clause, empty relativiser}}


\is{relative clause|(}
\section{Introduction}
\label{sec:rc-introduction}

The goal of this paper is to give an overview of HPSG analyses of
relative clauses. Relative clauses are, typically, sentential constructions that function as nominal
modifiers, like the italicised part of \pref{x:rc-1}, for example.
\begin{exe}\ex
    \label{x:rc-1} The person \emph{to whom Kim spoke yesterday} claimed to know nothing.
\end{exe}
Relative clauses have been an important topic in HPSG: not only as the focus on a
considerable amount of descriptive and theoretical work across a range of languages, but
also in terms of the theoretical development of the framework. Notably, \citegen{Sag:97}
analysis of English relative clauses was the first fully developed realisation of the
constructional approach involving cross-classifying phrase types that has dominated work
in HPSG in the last two decades, and was thus the first step towards the development of
\isi{Sign-based Construction Grammar} (cf.\ \crossrefchapteralt{cxg}).

The basic organisation of the discussion is as follows. Section~\ref{sec:rc-approaches}
introduces basic ideas and overviews the main analytic techniques that have been used, focusing
on one kind of relative clause. Section~\ref{sec:rc-varieties} looks at other kinds of
relative clause in a variety of languages. Section~\ref{sec:rc-other-functions-other-issues}
looks at a variety of constructions which have some similarity with relative clauses, but which are in
some way untypical (e.g.\ clauses that resemble relative clauses, but which are not
nominal modifiers, or which are not adjoined to the nominals they
modify). Section~\ref{sec:rc-conclusion} provides a conclusion.

\section{Basic ideas and approaches}
\label{sec:rc-approaches}
This section introduces basic ideas and intuitions about relative clauses, viewed from an
HPSG perspective (Section~\ref{sec:rc-basic-}), then introduces the two main approaches that
have been taken in HPSG: the lexical approach of \cite{Pollard:Sag:94} which makes use of
phonologically empty elements (Section~\ref{sec:rc-pollard--sag}), and the constructional
approach of \cite{Sag:97}, which makes phonologically empty elements unnecessary (Section~\ref{sec:rc-sag-1997}).
Section~\ref{sec:rc-interim-conclusions} presents some interim conclusions, and 
provides some discussion of some brief discussion alternative approaches.

\subsection{Basic ideas and intuitions}
\label{sec:rc-basic-}
\is{relative clause!wh-relative@\lic{wh}-relative|(}
\is{unbounded dependency|(}

Relative clauses  are, prototypically, sentential constructions which modify a
nominal.  \pref{x:rc-3} is an example of one kind of \ili{English} relative clause, which we will call a
``\lic{wh}-relative''. In \pref{x:rc-4} it is used  as a modifier of the nominal \lic{person}
(the \emph{antecedent}\is{relative clause!antecedent of} of the relative clause).
\begin{exe}\label{x:rc-2}
  \ex\label{x:rc-3} to whom Kim spoke yesterday
  \ex\label{x:rc-4} The person to whom Kim spoke yesterday claimed to know nothing.
\end{exe}
\is{relative clause!wh-phrase@\lic{wh}-phrase in|(}
Syntactically, this kind of relative clause consists of a preposed \lic{wh}-phrase
(\lic{to whom}), i.e.\ a phrase containing a \isi{relative pronoun} (\lic{whom}), and a
clause with a missing constituent --- a gap (the complement of \lic{speak}: \lic{Kim spoke
  \trace yesterday}). This is often called the \is{relative clause!relativised constituent}\emph{relativised constituent}.
Semantically, in \pref{x:rc-3} the interpretation of the relative clause is
\emph{intersective}: \pref{x:rc-3} denotes the intersection of the set of people and the
set of entities that Kim spoke to. Getting this interpretation involves combining the
descriptive content of the antecedent nominal and the propositional content of the
relative clause, and equating the referential indices of the nominal and the \isi{relative
  pronoun}, to produce something along the lines of ``the set of $x$ where $x$ is a person
and Kim spoke to $x$''.

Not all relative clauses have these properties, but they provide a good starting point.
In the remainder of this section, we will show, in broad terms, how these properties can
be accounted for.

As regards their function and distribution, relative clauses are subordinate clauses,
which can be captured by assuming they have a \feat{head} feature like [\feat{mc}~{--}],
``\textsc{main-clause}~\type{minus}''. They are naturally assumed to be adjuncts: their
distribution as nominal adjuncts can be dealt with by assuming that (like other adjuncts)
they indicate the sort of head they can modify via a feature like \feat{mod} or
\feat{select}. %(cf.\ \crossrefchaptert{ChXXX} on the treatment of adjuncts).
That is,
relative clauses such as \pref{x:rc-3} will be specified as in \pref{x:rc-5}, whereas
adjunct clauses headed by a subordinator like \lic{because} (as in \lic{We're late because
  it's raining}) will be specified as \pref{x:rc-6}, and normal, non-adjunct, clauses will
typically be specified as \pref{x:rc-7}:
\begin{exe}\ex\begin{xlist}
  \ex\label{x:rc-5}
  \begin{avm}
   \[ synsem\|loc\|cat\|head\|mod &  
      \[ loc\|cat\|head & noun \\
      \]\\
   \]
   \end{avm}
  \ex\label{x:rc-6}
  \begin{avm}
   \[ synsem\|loc\|cat\|head\|mod &  
      \[ loc\|cat\|head & verb \\
      \]\\
   \]
   \end{avm}
 \ex\label{x:rc-7} \begin{avm}\[synsem\|loc\|cat\|head\|mod & none\]\end{avm}
\end{xlist}
\isfeat{synsem}\isfeat{local}\isfeat{cat}\isfeat{head}\isfeat{mod}
\istype{verb}\istype{noun}
%?? \istype{none}
\end{exe}

With this in hand, we will look in more detail at the internal structure of this kind of
relative clause (Section~\ref{sec:rc-intern-struct-relat}), and at the relation between
the relative clause and its antecedent (Section~\ref{sec:rc-relative-clause-ante}).

\subsubsection{The internal structure of the relative clause}
\label{sec:rc-intern-struct-relat}
As regards internal structure, it is characteristic of \lic{wh}-relatives that
they consist of a preposed \lic{wh}-phrase and a clause containing a gap.
The dependency between the \lic{wh}-phrase and the associated gap is potentially
unbounded, as can be seen from examples like \pref{x:rc-11}.
\begin{exe}\ex\label{x:rc-11}
  the person to whom [ Sam said [ Kim intended [ to speak \trace yesterday]]]
\end{exe}
As regards the \lic{wh}-phrase, it is notable that it must be preposed --- \ili{English}
does not allow examples like \pref{x:rc-9} without a relative phrase, or \pref{x:rc-10} where
the relative phrase is \emph{in situ}.
\begin{exe}\ex\begin{xlist}\label{x:rc-8}
  \ex[*]{\label{x:rc-9} a person Kim spoke to her yesterday}
  \ex[*]{\label{x:rc-10} a person Kim spoke to whom yesterday}
\end{xlist}\end{exe}
Despite being forbidden \lic{in situ}, the preposed \lic{wh}-phrase behaves in some
respects as though it occupied the gap. For example, in the  examples above \lic{to whom}
satisfies the subcategorisation requirements of \lic{speak}, and makes a semantic
contribution in the gapped clause.  Assuming some kind of co-indexation relation
between the \is{relative clause!antecedent of}antecedent and the \lic{wh}-phrase, the same behaviour can be seen with
subject-verb agreement, as in \pref{x:rc-166}, and binding, as in \pref{x:rc-167}:
\begin{exe}\ex\begin{xlist}
  \ex\label{x:rc-166} a person who [ everyone thinks [ \trace is/*are weird]]
  \ex\label{x:rc-167} a person who [ everyone thinks [ \trace hates herself/*her]]
\end{xlist}\end{exe}
In fact, this dependency between the \lic{wh}-phrase and the gap appears to be a typical
filler-gap dependency, with the \lic{wh}-phrase as the filler,  which can be handled by
standard \feat{slash} inheritance techniques (see \crossrefchapteralt{udc}), so that these
properties are accounted for.

In examples like \pref{x:rc-3} the \lic{wh}-phrase must contain a \isi{relative pronoun}. Here we
have another apparently unbounded dependency, because the \isi{relative pronoun} can be embedded
arbitrarily deeply inside the \lic{wh}-phrase (example
  \pref{x:rc-16} is due to \citealt{Ross67a-Eng}):
  \begin{exe}\ex\begin{xlist}\label{x:rc-12}
  \ex\label{x:rc-13} the person  [to [whose friends]] Kim spoke \trace   
  \ex\label{x:rc-14} the person [to [[whose children's] friends]] Kim spoke \trace   
  \ex\label{x:rc-15} the person [to [the children [of [whose friends]]] Kim spoke \trace   
  \ex\label{x:rc-16} books [the height [of [the letters [on [the covers [of which]]]]]
  the government regulates \trace
\end{xlist}\end{exe}
This dependency between a \isi{relative pronoun}
and the phrase that contains it is often called ``\lic{wh}-percolation''\is{wh-percolation@\emph{wh}-percolation}, ``\isi{relative
percolation}'', or, following \cite{Ross67a-Eng}, ``\isi{pied-piping}''. We will talk about
\emph{\isi{relative inheritance}}.

Notice that as well as being unbounded, \isi{relative inheritance} resembles \feat{slash}
inheritance in that the ``bottom'' of the inheritance path (i.e.\ the actual \isi{relative pronoun}, or the gap in a filler-gap dependency) is typically not a head (e.g.\
\lic{whom} is not the head of \lic{to whom}).  Moreover, though examples involving
multiple independent \isi{relative pronoun}s are rather rare in \ili{English} (i.e.\ there are few, if
any, relative clauses parallel to interrogatives like \lic{Who gave what to whom?}) they
exist in other languages, so it is reasonable to assume that \isi{relative inheritance} involves a set of some kind.\footnote{Examples of languages which allow
  multiple \isi{relative pronoun}s include \ili{Hindi} \citep[e.g.][]{Srivastav1991} and \ili{Marathi}
  \citep[e.g.][Chapter 7]{dhongde2009marathi}. See \cite[227--232]{Pollard:Sag:94} for HPSG
  analyses. In \ili{English}, multiple \isi{relative pronoun}s occur in cases of co-ordination (e.g.\
  \lic{the person with whom or for whom you work}), but they are not independent (they
  relate to the same entity). \cite{Kayne17SomeEvenMoreUnusual} gives some \ili{English}
  examples that appear to involve multiple \isi{relative pronoun}s, but they are rather
  marginal.}  This motivates the introduction of a \feat{rel} feature which is subject
to the same kind of formal mechanisms as \feat{slash}.\footnote{The assumption that
  \isi{relative inheritance} should be treated as involving an unbounded dependency (i.e.\
  handled with a \feat{non-local} feature, like \feat{slash}), has been challenged in
  \cite{Van-Eynde:04} (\citeauthor{Van-Eynde:04} argues it should be treated as local
  dependency).}

\is{relative inheritance|(}
The idea is that a \isi{relative pronoun} will register its presence by introducing a non-empty
\feat{rel} value, which will be inherited upwards until it reaches the preposed
\lic{wh}-phrase at the top of the relative clause (equivalently: a relative clause
introduces a non-empty \feat{rel} value on its \lic{wh}-phrase daughter that is inherited downwards till it is
realised as a \isi{relative pronoun}). Within the \lic{wh}-phrase, \feat{rel} inheritance can
be handled by the same sort of formal apparatus as is used for handling \feat{slash}
inheritance. Blocking \feat{rel} inheritance from carrying a \feat{rel} element
upwards beyond the top of relative clause can be achieved with the same formal apparatus
as is used to block \feat{slash} inheritance from carrying information about a gap
higher than the level at which the associated filler appears.\footnote{In case it is not
  obvious why further upward inheritance of a \feat{rel} value would be problematic,
  notice that while a relative clause can \emph{contain} a \lic{wh}-phrase, it cannot
  \emph{be} a \lic{wh}-phrase, e.g.\ it cannot function as the filler in a relative
  clause. Suppose, counter-factually, the \feat{rel} value of \lic{who} could be inherited
  beyond the relative clause \lic{to whom Kim spoke}, so that e.g.\ \lic{a person to whom
    Kim spoke} was marked as [\feat{rel}~\{\idx{1}\}]. This phrase would be able to
  function as the \lic{wh}-phrase in a relative clause like *[\lic{a person to whom Kim
    spoke}] \lic{Sam recognised \trace}, which would be able to combine with a noun specified as
  [\feat{index}~\idx{1}] to produce something like *\lic{a person} [[\lic{a person to whom
    Kim spoke}] \lic{Sam recognised \trace}].}
\is{relative inheritance|)}

Co-indexation of the \is{relative clause!antecedent of}antecedent nominal and the \isi{relative pronoun} can be achieved simply if
the \feat{rel} value contains an index which is shared by both the antecedent and the
\isi{relative pronoun}.  As regards the \isi{relative pronoun}, at the ``bottom'' of the \feat{rel}
dependency, this can be a matter of lexical stipulation: \isi{relative pronoun}s can be
lexically specified as having a \feat{rel} value that contains their \feat{index}
value, roughly as in \pref{x:rc-17}, which we abbreviate to \pref{x:rc-18}.\footnote{Here, and
  below, we will abbreviate attribute paths where no confusion arises, and use a number of
  other standard abbreviations, in particular, we write \feat{index} values as
  subscripts on nouns and NPs. We use \ibar{N} to indicate a noun with an empty \comps
  list, i.e.\ one which has combined with its complements, if any, and NP for a \ibar{N} with
  an empty \feat{spec} (\feat{specifier}) list (e.g.\ a combination of determiner and a \ibar{N}).
  % In X-bar terms this is an \ibar{N}.
  Similarly, we use PP to abbreviate a phrase consisting of a preposition and its complement,
  VP for a phrase consisting of verb with its complements, and S for a phrase consisting of
  a subject and a VP.}
\begin{exe}\ex\begin{xlist}
  \ex\label{x:rc-17}  
  \begin{avm}
   \[synsem & 
      \[ loc & 
         \[ cat  & \[ head & noun \]\\
            cont & \[ index & \@1 \]
         \]\\
         non-loc & \[ inher\|rel & \q\{\@1\q\} \]
      \]
   \]  
   \end{avm}
  \ex\label{x:rc-18} \ibar{N}\subtag{1}~\begin{avm}\[rel& \q\{\@1\q\}\]\end{avm}
\end{xlist}
\isfeat{synsem}\isfeat{local}\isfeat{cat}\isfeat{content}\isfeat{head}\isfeat{index}\isfeat{non-local}
\istype{noun}
\end{exe}
This index can then be \is{relative inheritance}inherited upwards via the \feat{rel} value to the level of the
\lic{wh}-phrase. At the top, the index of the \is{relative clause!antecedent of}antecedent can be accessed via the
\feat{mod} value of the relative clause: this is simply a matter of replacing the
specification of the \feat{mod} value in \pref{x:rc-5} with that in \pref{x:rc-19},
abbreviated as in \pref{x:rc-20}, where
\idx{1} is the index that appears in the \feat{rel} value of the associated
\lic{wh}-phrase.\footnote{We assume, for simplicity,
  that the value of \feat{rel} is a set of indices. This is consistent with e.g.\
  \cite{Pollard:Sag:94} and \cite{Sag:97}, but not with \cite[188]{Ginzburg:Sag:00}, who
  assume it is a set of \istype{parameter}\type{parameters}, that is, indices with restrictions (a kind of
  \istype{scope-object}\type{scope-object}), like the \feat{que} attribute which is used for
  \lic{wh}-inheritance in interrogatives. It is not clear that anything important hangs on
  this.}
\begin{exe}\ex\begin{xlist}
  \ex\label{x:rc-19}
  \begin{avm}
   \[ synsem\|loc\|cat\|head\|mod &  
      \[ loc & 
         \[ cat  & \[ head & noun \]\\
            cont & \[ index & \@1 \]
         \]\\
      \]
   \]
   \end{avm}
   \ex\label{x:rc-20} S~\begin{avm}\[mod&\upshape{\ibar{N}\subtag{1}}\]\end{avm}
 \end{xlist}
 \isfeat{synsem}\isfeat{local}\isfeat{cat}\isfeat{head}\isfeat{mod}\isfeat{index}\isfeat{content}
 \istype{noun}
\end{exe}

\begin{figure}
    \begin{forest}  %qtree edges
   [{\TnodeDA{S}{\[mod& \upshape{\ibar{N}\subtag{1}}\]}}, baseline 
      [{\TnodeDA{\idx{3} PP}{\[ rel &\setof{\@1} \]}}   [{P}  [ {to} ] ]
         [{\TnodeDA{NP\subtag{1}}{\[ rel&\setof{\@1} \]}}  [ {whom} ] ]
      ]
      [{\TnodeDA{S}{\[slash&\setof{\@3}\]}} 
         [{\TnodeDA{NP}{}}  [ {Kim} ] ]
         [{\TnodeDA{VP}{\[slash&\setof{\@3}\]}}    
            [{\TnodeDA{V}{\[slash&\setof{\@3}\]}}  [ {spoke} ] ]
         ]
      ]
   ]
   \end{forest}
  \caption{Representation of \lic{to whom Kim spoke}}
  \label{fig:rc-1}
\end{figure}
Schematically, then, \lic{wh}-relatives should have
structures along the lines of Figure~\ref{fig:rc-1}.
The top structure here is a head-filler
structure. Notice how \feat{slash} inheritance ensures the relevant properties of the
PP are shared by lower nodes so that the subcategorization requirements of the verb can
be satisfied, with the PP being interpreted as a complement of the verb (equivalently:
\feat{slash} inheritance ensures that the gap caused by the missing complement of
\lic{speak} is registered on higher nodes until it is filled by the PP). Similarly,
\feat{rel} \is{relative inheritance}inheritance means that the \feat{index} of the \isi{relative pronoun} appears on
higher nodes so that it can be identified with the \feat{index} of the \is{relative clause!antecedent of}antecedent noun,
via the \feat{mod} value of the highest S (equivalently: the index of the antecedent
nominal appears on lower nodes down to the \isi{relative pronoun}, so that the nominal and the
\isi{relative pronoun} are co-indexed).

As regards \feat{content}, the effect of this will be to give the relative clause \lic{to whom\subscr{i} Kim
  spoke} an interpretation along the lines of \lic{Kim spoke to whom\subscr{i}}, where $i$ is
the index of its antecedent. In terms of standard HPSG semantics, this
``internal'' content (i.e.\ the content associated with a verbal head with its complements and
modifiers) is a \istype{state-of-affairs}\type{state-of-affairs} (\istype{soa}\type{soa}), and can be represented as in
\pref{x:rc-21}, abbreviated to  \pref{x:rc-22}:\footnote{In fact \pref{x:rc-35} is already somewhat abbreviated:
   [\feat{speaker}~\type{Kim}] is an abbreviation for a structure including an index, and a
   \feat{background} restriction on that index indicating that it stands in the
   \istype{naming}\type{naming} relation to the name \type{Kim}.}
\begin{exe}\ex\begin{xlist}
  \ex\label{x:rc-21}
  \begin{avm}
   \[\asort{soa} nuc & \[ \asort{speak\_to} speaker & Kim\\addressee & \@1 \] \]
   \end{avm}
   \ex\label{x:rc-22} $speak\_to(Kim,\idx{1})$
 \end{xlist}
 \isfeat{speaker}\isfeat{nucleus}\isfeat{addressee}
 \istype{state-of-affairs}
\end{exe}

\is{relative inheritance|(}
There are restrictions on what can occur as the preposed \lic{wh}-phrase in a relative
clause, as can be seen in \pref{x:rc-23}. To a first approximation, NPs and PPs are fine in
\ili{English}, but VPs and Ss are not.\footnote{This is a considerable simplification: e.g.\
  \ili{English} allows VPs like that in \pref{x:rc-27} so long as they function as subjects, so
  \lic{person to speak to whom is a privilege} is allowed. \ili{German} allows \feat{rel}
  \is{relative inheritance}inheritance to VP more freely than \ili{English}, and an analogue of \pref{x:rc-27} is
  grammatical in \ili{German}. See \cite{deKuthy99a}, \cite{HN99a} and
  \cite{Mueller99a} for discussion and HPSG analyses.}
\begin{exe}\ex\begin{xlist}\label{x:rc-23}
\ex[] {the person  [\sub{NP} who] we want Kim to speak to \trace}  \label{x:rc-24}
\ex[] {the person  [\sub{PP} to whom] we want Kim to speak \trace} \label{x:rc-25}
\ex[*] {the person [\sub{VP} speak to whom] we want Kim to \trace}\label{x:rc-26}
\ex[*] {the person [\sub{VP} to speak to whom] we want Kim \trace}\label{x:rc-27}
\ex[*] {the person [\sub{S} Kim to speak to whom] we want \trace} \label{x:rc-28}
\end{xlist}\end{exe}
It is rather natural to interpret this as indicating restrictions on \isi{relative inheritance}
(i.e.\ pied-piping in relative clauses) --- e.g.\ as indicating that while \isi{relative inheritance} from NP to PP (and through an upward chain of NPs and PPs, as in \pref{x:rc-16}),
is permitted, it is blocked by VP and S nodes. This can be achieved by requiring that 
the \feat{rel} value on VP and S nodes be empty (cf.\ the Clausal \feat{rel}
Prohibition of \citealt[220]{Pollard:Sag:94}). While important, these restrictions are
poorly understood and we will have nothing to say about them here, except to emphasise
that they are different from the restrictions on \feat{que} inheritance
(i.e.\ pied-piping in interrogatives). For example, \feat{que}-inheritance is not
possible from the complement of a noun, but \feat{rel} inheritance is fine, so \lic{some
  pictures of whom} is not possible as the focus of a question, as in \pref{x:rc-29}, but is fine as the initial
phrase of a relative clause, as in \pref{x:rc-30}:
\eal
\ex[*]{\label{x:rc-29}
I wonder [some pictures of whom] they were admiring.  
}
\ex[]{\label{x:rc-30} 
the children [some pictures of whom] they were admiring
}
\zl
Notice that \feat{rel} and \feat{que} also differ in other ways: e.g.\ as
\cite[490--493]{Sag:10b} emphasises, there are \lic{wh}-words that can function as
interrogative pronouns, but not as \isi{relative pronoun}s (i.e.\ which have non-empty
\feat{que} values, but empty \feat{rel} values), and \emph{vice versa}. For example,
\lic{how} and (in standard \ili{English}) \lic{what} function as interrogative pronouns,
but not \isi{relative pronoun}s, as the following examples show (as \citealt[493]{Sag:10b} puts it, there
is ``no morphological or syntactic unity underlying the concept of an \ili{English}
\lic{wh}-expression''):\footnote{See also \cite[81--85]{Mueller99b} on differences
  between interrogative and \isi{relative pronoun}s in \ili{German}. Several non-standard \ili{English}
  dialects allow NP \lic{what} as a \isi{relative pronoun} like \lic{which} (cf.\ non-standard
  \lic{\%the book what she bought}, vs.\ standard \lic{the book which she bought}). No
  dialect allows determiner \lic{what} as a \isi{relative pronoun} (though it is fine as an
  interrogative, as can be seen in \pref{x:rc-33}). \cite[491, note 10]{Sag:10b} suggests
  that NP \lic{which} is only ever a \isi{relative pronoun} (an apparent counter-example like
  \lic{Which did you buy?}  involves determiner \lic{which} with an elliptical noun).}
\begin{exe}\ex\begin{xlist}
  \ex[]{\label{x:rc-31} I wonder how she did it. \hfill(interrogative)}
  \ex[*]{\label{x:rc-32} the way how she did it \hfill(relative)}
\end{xlist}\end{exe}
\begin{exe}\ex\begin{xlist}
  \ex[]{\label{x:rc-33} I wonder what (things) she bought. \hfill(interrogative)}
  \ex[*]{\label{x:rc-34} the book what (things) she bought \hfill(relative)}
\end{xlist}\end{exe}
\is{relative inheritance|)}

With this overview of the internal structure of a relative clause in place, we now turn to
relation between the relative clause and the nominal it modifies (its antecedent).

\subsubsection{The relative clause and its antecedent}
\label{sec:rc-relative-clause-ante}
\is{relative clause!antecedent of|(}

The combination of a relative clause and the nominal it modifies is traditionally
regarded as a head-adjunct structure, where the nominal is the head and the
relative clause is the adjunct, as in Figure~\ref{fig:rc-2}.
\begin{figure}
  \begin{forest}  %qtree edges
   [{\ibar{N}\subtag{1}} ,baseline
      [{\idx{2} \ibar{N}\subtag{1}}  [ {person} ] ]
      [{\TnodeDA{S}{\[mod&\@2\]}}
         [{to whom Kim spoke}, roof]
      ]
   ]
   \end{forest}
   \caption{A relative clause and its antecedent}
   \label{fig:rc-2}
 \end{figure}
 
 The content we want for a modified nominal such as \lic{person to whom Kim spoke}, as for
 an unmodified nominal such as \lic{person}, is a \emph{restricted index}, i.e.\ in HPSG
 terms a \istype{scope-object}\type{scope-object} --- an \feat{index} and a \feat{restr}
 (\textsc{restriction}) set (a set of objects of type \istype{fact}\type{fact}).\footnote{%
   In \cite{Pollard:Sag:94}, \istype{scope-object}\type{scope-object}s were called
   \istype{nom-object}\type{nom-object}s, and restrictions were sets of
   \emph{parameterized states of affairs} \istype{psoa}(\type{psoa}s), rather than
   \istype{fact}\type{fact}s. The difference reflects the more comprehensive semantics of
   \cite{Ginzburg:Sag:00}, which involves different kinds of
   \istype{message}\type{message} (e.g.\ \istype{proposition}\type{proposition},
   \istype{outcome}\type{outcome}, and \istype{question}\type{question}, as well as
   \istype{fact}\type{fact}). For our purposes, this is just a minor change in feature
   geometry: \istype{fact}\type{fact}s contain \citeauthor{Pollard:Sag:94}-style
   \istype{state-of-affairs}\type{state-of-affairs} content as the value of the
   \feat{prop}$\,\vert\,$\feat{soa} path, as can be seen in \pref{x:rc-35}.}  For
 \lic{person}, this is as in \pref{x:rc-35}, abbreviated as in \pref{x:rc-36}, for
 \lic{person to whom Kim spoke} it is as in \pref{x:rc-37}, abbreviated as in
 \pref{x:rc-38}.
\begin{exe}
  \ex\label{x:rc-35}
  \begin{avm}
      \[\asort{scope-obj}
         index & \@1\\
         restr & 
         \{ \[\asort{fact} prop\|soa & \[\asort{soa} nuc & \[\asort{person} instance & \@1 \] \]
            \]
         \} 
      \] 
   \end{avm}
   \istype{scope-object}
   \istype{soa}\istype{fact}
  \ex\label{x:rc-36}\ibox{1} : \ensuremath{\{person(\idx{1})\}}
\end{exe}
\begin{exe}
  \ex\label{x:rc-37}
  \begin{avm}
      \[\asort{scope-obj}
         index & \@1\\
         restr & 
         \{ \begin{tabular}{l}
            \[\asort{fact} prop\|soa  & \[\asort{soa} nuc & \[\asort{person} instance & \@1\]\]\] ,\\
            \[\asort{fact} prop\|soa  & \[\asort{soa} nuc & \[\asort{speak\_to} speaker & Kim\\addressee & \@1\]\]\]
            \end{tabular}
      \} 
      \] 
   \end{avm}
   \isfeat{index}\isfeat{restr}\isfeat{prop}\isfeat{soa}\isfeat{nucleus}\isfeat{instance}\isfeat{speaker}\isfeat{addressee}
   \istype{scope-object}
 \end{exe}
\begin{exe}
\ex\label{x:rc-38}\ibox{1} : \ensuremath{\{person(\idx{1}),
        speak\_to(Kim,\idx{1})\}}
\end{exe}
To get the content of \lic{person to whom Kim spoke} from the content of \lic{person} is a
matter of producing a \istype{scope-object}\type{scope-object} whose index is the index of \lic{person} (and
the \isi{relative pronoun}), and whose restrictions are the union of the restrictions of
\lic{person} with a set containing a \istype{fact}\type{fact} corresponding to the
\istype{state-of-affairs}\type{state-of-affairs} that is the content of the relative clause. Unioning the restrictions gives
the intersective interpretation.

Conceptually, this is straightforward, but there is technical difficulty: the structure in
Figure~\ref{fig:rc-2} is a head-adjunct structure, and in such structures the content should
come from the adjunct daughter, the relative clause. That is, for ``external'' semantic
purposes (purposes of semantic composition) relative clauses should have
\istype{scope-object}\type{scope-object} content, but as we have seen, their ``internal'' content is a
\istype{state-of-affairs}\type{soa}. So some special apparatus will be required, as will appear in the following
discussion.\footnote{Though the details are HPSG-specific, this is a general problem,
  regardless of semantic theory. For example, in a setting using standard logical types,
  relative clauses \emph{qua} clauses (saturated predications) might be assigned type $t$,
  but in order to act as nominal modifiers this predicative semantics must be converted
  into ``attributive'' (noun-modifying) semantics, i.e.\  logical type
  \tuple{et,et}. See e.g.\ \cite[521--524]{Sag:10b} where an HPSG syntax is combined with a conventional
  predicate-logic-based semantics for relative clauses. }

This should give the reader an idea of the general shape of an approach to relative
clauses like \pref{x:rc-3} using the HPSG apparatus. In the following sections we will make this
more precise by outlining the two main approaches that have been taken to the analysis of
relative clauses in HPSG: the lexical approach of \cite[Chapter~5]{Pollard:Sag:94}, which
makes use of phonologically empty elements, and the constructional approach of
\cite{Sag:97}, which does not.\footnote{\cite{Mueller99b} presents what might be
  considered a third approach, which resembles \cite{Sag:97} in avoiding empty elements,
  but uses a rule schema for \ili{German} relative clauses rather than the constructional apparatus
  of phrasal types (see \citealt[95]{Mueller99b} for details). The overview of HPSG in
  \cite{MuellerCurrentApproaches} also presents a rule schema for relative clauses
  (\emph{loc cit} Section 6.1). Rule schemas were a crucial piece of apparatus in the framework of
  \cite{Pollard:Sag:94}, but they have fallen out of favour with the rise of construction-based 
  analyses since \cite{Sag:97}. A rule schema is essentially just a phrasal type ---
  that is, a type describing constraints on a mother and daughters --- with the difference
  that unlike phrasal types, rule schemas do not stand in inheritance relations, so it is
  not possible to factor out generalisations in the way of construction-based
  analyses. This is not an issue for \citeauthor{Mueller99b}, who claims that a
  description of restrictive relatives in \ili{German} requires only a single schema
  \citep[74]{Mueller99b}.}
\is{relative clause!antecedent of|)}

\subsection{The lexical approach of \citew{ps2}}
\label{sec:rc-pollard--sag}
\is{relative clause!empty relativiser|(}

The idea that relative clauses have a lexical head is appealing for some kinds of relative
clause in many languages (see below, e.g.\ Section~\ref{sec:rc-comp-relatives},
Section~\ref{sec:rc-bare-relatives}), but it is problematic for relative clauses like
\pref{x:rc-3} --- there is no obvious candidate to serve as the head.  This is clearly
problematic for a lexical, ``head-driven'' approach such as HPSG. Building on an approach originally
proposed by \cite{borsley1989phrase}, the analysis proposed in
\cite[Chapter~5]{Pollard:Sag:94} overcomes this problem by assuming that relative clauses
involve a phonologically empty head, which \citeauthor{Pollard:Sag:94} call R
(``relativiser''), and which projects an RP (that is, a relative clause).

R is lexically specified to be a nominal modifier (i.e.\ [\feat{mod}~\istype{noun}\type{noun}]) which
takes two arguments. The first is an XP, the \lic{wh}-phrase, with a \feat{rel} value
which contains the index of the \is{relative clause!antecedent of}antecedent nominal. The second is sentential, and
constrained to have a \feat{slash} value that includes the XP. With some simplifications
and some minor modifications to fit the framework we assume here, this is along the lines
of \pref{x:rc-39} \citep[cf.][216]{Pollard:Sag:94}. Here XP~\idx{4} is intended to
mean an XP whose \feat{local} value is \idx{4}, and S:\idx{3} means a clause (a
saturated \istype{verb}\type{verb} -- i.e.\ one with empty \feat{subj} and \comps
specifications) whose \feat{content} is \idx{3}. The \idx{2} that appears in the
value of \feat{restr} is intended to be the
\feat{restr} set of the \is{relative clause!antecedent of}antecedent nominal (this should be specified as part of the
\feat{mod} value, but we have not done this, in the interests of
readability).
\ea
\label{x:rc-39}
Lexical item for the empty relativizer:\\*
  \begin{avm}
     \[ synsem & \[ loc & 
         \[cat & 
            \[ head & \[ mod & \upshape{\ibar{N}\subtag{1}} \]\]\\
               cont & \[\asort{scope-obj} index & \@1\\ restr &  \@2 $\cup$  \{ \[\asort{fact}prop\|soa & \@3\] \} \] \] 
      \]\\ 
      arg-st & \< \cPile{{\upshape XP} \@4\\ \[rel&\setof{\@1}\]}, 
                  \cPile{ {\upshape S}:\idx{3}\\\[slash&\setof{\@4}\]}   \>
     \]
   \end{avm}
   \isfeat{synsem}\isfeat{local}\isfeat{cat}\isfeat{head}\isfeat{content}\isfeat{mod}\isfeat{index}\isfeat{restr}\isfeat{prop}\isfeat{soa}\isfeat{arg-st}\isfeat{rel}\isfeat{slash}
\end{exe}

Standard schemas for combining heads with arguments will produce structures like the RP in
Figure~\ref{fig:rc-3}, which (since \feat{mod} is a head feature) will inherit the
\feat{mod} feature from R, and hence combine with a nominal like \lic{person} in a
head-adjunct phrase to produce the structure in Figure~\ref{fig:rc-3}.\footnote{Here again
  we have used PP \idx{4} to indicate a PP whose \feat{local} value is \idx{4}.}
\begin{figure}
  \begin{forest} % qtree edges
   [{\ibar{N}\subtag{1}}%, s sep=5em, baseline
      [{\idx{7} \ibar{N}\subtag{1}}  [ {person} ]  
      ]  
      [{\TnodeDA{RP}{\[rel&\setof{\@1}\]}} 
         [{\TnodeDA{PP~\idx{4}}{\[rel&\setof{\@1}\]}}  ~
            [ {to whom}, roof] 
         ]
         [{\TnodeDA{\ibar{R}}{}}
            [{\TnodeDA{R}{\[mod&\@7\\arg-st&\tuple{\@4,\@6}\]}}     [ {\trace} ] ]
            [{\TnodeDA{\idx{6}~S}{\[slash&\setof{\@4}\]}}  [ {Kim spoke} ,roof] ]
         ]
      ]
   ]
   \end{forest}
   \caption{A \cite{Pollard:Sag:94}-style structure involving a finite \lic{wh}-relative clause}
   \label{fig:rc-3}
\end{figure}

This captures the properties described above, and resolves the issues mentioned in the
following way.

The first argument of R is specified as [\feat{rel}~\setof{\idx{1}}]. Thus, it must contain
a \isi{relative pronoun}. Moreover, \pref{x:rc-39} specifies that the first argument must
correspond to a gap in the second argument. Hence cases like \pref{x:rc-8} where there is no
\lic{wh}-phrase, or where the \lic{wh}-phrase is \emph{in situ}, are excluded.

Since R, not the slashed S, is the head of RP, there is no problem of mismatch between the
content of the S and the relative clause: R is lexically specified as having \istype{fact}\type{fact}
(i.e.\ \istype{scope-object}\type{scope-object}) content incorporating the ``internal'' content of its complement clause
(tagged \idx{3}) in the appropriate way. This \istype{fact}\type{fact} content will be
projected to RP by normal principles of semantic composition relating to heads,
complements, and subjects, and RP will produce the right content by unioning the
restrictions that come from the head nominal with this \istype{fact}\type{fact} content.

\is{relative inheritance|(}
This leaves the question of how upward inheritance of the \feat{rel} and \feat{slash}
values can be prevented. The same method is used for both. The idea is that for features
like \feat{rel} and \feat{slash} (non-local features) the value on the mother is the
union of the values on the daughters, less any indicated as being discharged
(``bound off'') on the head daughter (the values that are bound off in this way are
specified as elements of the value of a \feat{to-bind} attribute). Thus, R can be specified so as to discharge the
\feat{slash} value on its S sister (so that \ibar{R} is [\feat{slash}~\setof{}]), and we can
ensure that the topmost \ibar{N} is [\feat{rel}~\setof{}], so long as its head \ibar{N} daughter is specified as
binding-off the \feat{rel} value on RP. This specification can be imposed by stipulation
in the \feat{mod} value of R.  See \cite[164]{Pollard:Sag:94} for details.
\is{relative inheritance|)}

The approach can be extended to deal with other kinds of relative clause by positing
alternative forms of empty relativiser (see below and \citealt[Chapter~5]{Pollard:Sag:94}).

The great attraction of the approach is that, apart from R, it requires no special
apparatus of any kind. On the other hand, it requires the introduction of a novel part
of speech (R), and the need to posit phonologically empty elements for which there is no
independent evidence. Reservations about this lead Sag to develop the constructional
approach presented in \cite{Sag:97}.\footnote{\label{fn:rc-1}%
One detail we ignore here concerns the analysis of \is{relative clause!subject relative}``subject'' relatives: relative clauses where
the relative phrase is a grammatical subject inside the relative clause, as in (i):
\begin{exe}
  \ex\label{x:rc-40} person who spoke to Kim 
%   \ex\label{x:rc-41} person who everyone thinks spoke to Kim
\end{exe}
\cite{Pollard:Sag:94} treat such examples specially (cf.\
\citealt[218--219]{Pollard:Sag:94}), using the ``\isi{Subject Extraction Lexical Rule}'' (SELR)
which in essence permits a VP to replace an S in an \feat{arg-st} in the presence of a
gap \citep[174]{Pollard:Sag:94}, so that R combines with a VP rather than an S. But this
is not an essential part of the analysis of relative clauses: it is motivated by quite
independent theoretical considerations (specifically, the assumption that gaps are
associated only with non-initial members of \feat{arg-st} lists --- cf.\ the
``\isi{Trace-Principle}''; \citealt[172]{Pollard:Sag:94}). Hence we ignore it here.}

\is{relative clause!empty relativiser|)}

\subsection{The constructional approach of Sag (1997)}
\label{sec:rc-sag-1997}
\is{relative clause!construction|(}

The analysis of \ili{English} relative clauses in \cite{Sag:97} is constructional and completely
dispenses with phonologically empty elements.\footnote{See \crossrefchaptert{cxg}, for
    broader discussion of the constructional approach to HPSG.} It involves three main constructions: one
for combining relative clauses and nominals, and two for relative clauses themselves. One
of these is the standard construction for head-filler phrases.  The other involves a
number of sub-constructions specific to relative clauses, which are treated as a subtype
of \istype{clause}\type{clause} (alongside e.g.\ \type{declaratives} and \type{imperatives}). These
are outlined (with some simplifications and minor adjustments) in
Figure~\ref{fig:rc-42}.\footnote{See \cite[Chapter~11]{KimSellsIntroduction} for an introductory
  overview of \ili{English} relative clauses on similar lines to
  \cite{Sag:97}. \cite[521--524]{Sag:10b} outlines an approach which is stated using the
  \isi{Sign-Based Construction Grammar} style notation \citep{BS2012a-ed}. Apart from the
  semantics (which is formulated using the conventional $\lambda$-calculus apparatus), it is
  generally compatible with the earlier analysis described here.  One simplification we
  make here is that we follow the more recent work \citep[e.g.][523]{Sag:10b} and do not
  distinguish \is{relative clause!subject relative}subject and non-subject finite relative clauses: \cite{Sag:97} follows
  \cite{Pollard:Sag:94} in treating them differently (cf.\ footnote~\ref{fn:rc-1}; and see
  Sag \citeyear[452--454]{Sag:97}), but it is not clear how important this is in the
  framework of \cite{Sag:97}.}

%% The inheritance hierarchy for relative clauses from Sag 1997.
\newsavebox{\RelcHierarchyDA}
\savebox{\RelcHierarchyDA}{
\newcommand{\finwhrelcegsDA}{
  \begin{tabular}[t]{@{}l@{}}
    whose bagels I like/who won the prize\\
    to whose bagels I owe everything\\
    whose playing the guitar amazed me\\
    that I admire/that admires me                           
  \end{tabular}
}
\newcommand{\redrelcegsDA}{
  \begin{tabular}[t]{@{}l@{}}
     standing by the door\\
     given the pay rise\\
     by the window\\
     happy with the idea\\
  \end{tabular}
}
\hspace{-3em}%
\begin{forest}
   where n children=0{edge=dotted,font=\upshape\small}{font=\itshape}% dashed/dotted lines above leaf nodes
   [clause, baseline, for tree={parent anchor=south, child anchor=north, s sep=0pt} % lines join, boundaries between nodes are 0pt
      [rel-cl, l=2\baselineskip, %calign with current edge 
         [wh-rel-cl ,l=2\baselineskip
            [fin-wh-rel-cl
               [{\finwhrelcegsDA}, l=7\baselineskip, text width=5em ]
            ]
            [ {inf-wh-rel-cl}
               [ {\makebox[4em]{\cPile{in whom to place our trust}}},l=5\baselineskip,
                  text width=4em, inner sep=0pt ]
            ]
         ]
         [non-wh-rel-cl ,l=2\baselineskip
            [bare-rel-cl
               [ {\makebox[3em]{\cPile{everyone likes}}},  l=3.5\baselineskip, text width=3em ]
            ]
            [simp-inf-rel-cl
               [ \small\cPile{(for us) to talk to}, l=2\baselineskip ]
            ]
         ]
         [ red-rel-cl 
            [{\makebox[5em]{\cPile{\redrelcegsDA}}}, l=8\baselineskip, text width=5em ]
         ]
      ]
      [ core-cl, 
         [ {\makebox[3em]{decl-cl}}, text width=3em [{}] ]
         [ imp-cl [{}]       ]
      ]
   ]
   \end{forest}
}

\begin{figure}
  \usebox{\RelcHierarchyDA}
  \caption{Type hierarchy for {\protect\istype{clause}\type{clause}}, based on \cite{Sag:97}}
  \label{fig:rc-42}
  \istype{rel-cl}\istype{wh-rel-cl}\istype{fin-wh-rel-cl}\istype{inf-wh-rel-cl}\istype{non-wh-rel-cl}\istype{simp-inf-rel-cl}\istype{red-rel-cl}\istype{core-cl}\istype{decl-cl}\istype{imp-cl}
\end{figure}
The \istype{rel-cl}\type{rel-cl} clause type is associated with the constraints in \pref{x:rc-44}, which
simply state that relative clauses are subordinate clauses ([\feat{mc}~{--}]) that
modify nouns and have \istype{proposition}\type{propositional}
content, and that they do not permit
subject-aux inversion ([\feat{inv}~{--}]).\footnote{Giving relative clauses
  \istype{proposition}\type{propositional} content puts them on a par with other kinds of clause, and is not
  very different from \citeauthor{Pollard:Sag:94}'s assumption that clauses have
  \istype{state-of-affairs}\type{state-of-affairs} content (since \istype{proposition}\type{proposition}s are simply semantic objects which contain a \feat{soa}).}
\begin{exe}\ex\label{x:rc-44}
  \type{rel-cl} \(\Rightarrow\)
  \begin{avm}
   \[ %\asort{rel-cl} 
      head & 
      \[ mc & --\\
         inv & --\\
         mod & \[ head & noun \]
      \]\\
      cont & proposition\\ 
   \]
   \end{avm}
   \isfeat{head}\isfeat{mc}\isfeat{inv}\isfeat{mod}\isfeat{head}\isfeat{content}
   \istype{noun}\istype{proposition}
\end{exe}
Relative clauses such as that in \pref{x:rc-3} are what Sag calls
\istype{fin-wh-rel-cl}\type{fin-wh-rel-cl}, a sub-type of \istype{wh-rel-cl}\type{wh-rel-cl}. This is associated with the
constraints in \pref{x:rc-46}. In words: \lic{wh}-relatives are a subtype of relative clause
(as stated in the type hierarchy in Figure~\ref{fig:rc-42}), where the non-head daughter is
required to have a \feat{rel} value which contains the \feat{index} of the
\is{relative clause!antecedent of}antecedent.\footnote{\label{fn:rc-2}For simplicity and to avoid distractions, we have
  presented \lic{wh}-relatives as \ibar{N} modifiers in \pref{x:rc-46}. This is a conventional
  assumption, because standard methods of semantic composition ensure that the content of
  the relative clause is included in the restrictions of a quantificational determiner (as
  in \lic{every person to whom Kim spoke}), but it is not Sag's analysis. Instead he takes
  \lic{wh}-relatives to be NP modifiers, which allows him to account for facts about the
  ordering of \lic{wh}-relatives and bare relatives (see
  \citealt[465--469]{Sag:97}). \cite[293--294]{Kiss2005a} gives a number of arguments in
  favour of this view, for example, the existence of what \cite{Link84a-u} called
  ``hydras'',\is{relative clause!hydra} like \pref{x:rc-45}, where the relative clause must be interpreted as modifying
  the coordinate structure consisting of the conjoined NPs.
  \begin{exe}
    \ex\label{x:rc-45} The boy\subscr{i} and the girl\subscr{j} who\subscr{i+j} dated each other are
    Kim's friends.
  \end{exe}
  Sag's analysis requires a different approach to semantic composition to that assumed
  here, e.g.\ one using \isi{Minimal Recursion Semantics} (MRS, \citealt{CFPS2005a}) or \isi{Lexical Resource Semantics} (LRS, \citealt{richtersailer-lrs04}) --- see, in  particular
  \cite{chaves07}, which provides, \emph{inter alia} an analysis of coordinate structures and
  relative clauses using MRS, and \cite{Walker2017}, where an approach to the semantics of relative clauses using LRS is worked out in detail.}
\begin{exe}\ex\label{x:rc-46}
  \istype{wh-rel-cl}\type{wh-rel-cl} \(\Rightarrow\)
  \begin{avm}
   \[ head & \[mod & \nbar\subtag{1}\]\\
     non-hd-dtrs & \< \[rel &  \setof{\@1}\] \>
   \]
   \end{avm}
   \isfeat{head}\isfeat{non-head-daughters}\isfeat{mod}\isfeat{rel}
 \end{exe}
The framework assumed in \cite{Sag:97} allows multiple inheritance of
constraints from different dimensions (cf.\ \crossrefchapteralt{properties}). As well as inheriting
properties in the clausal dimension, expressions of type \type{fin-wh-rel-cl} are
also classified in the phrasal dimension as belonging to a sub-type of head-filler phrase
(\istype{hd-fill-phrase}\type{hd-fill-ph}), thus inheriting constraints as in \pref{x:rc-47}.\footnote{The
  $\uplus$ symbol here signifies \emph{disjoint union}. This is like normal set union,
  except that it is undefined for pairs of sets that share common elements. Here, the
  intention is that restrictions are distributed between the noun and
  the clause, so the restrictions associated with the noun do not include the restrictions
  associated with the clause, and \emph{vice versa}.}
\begin{exe}\ex\label{x:rc-47}
  \istype{hd-fill-phrase}\type{hd-fill-ph}    \(\Rightarrow\)
  \begin{avm}
   \[ slash & \@2\\
      hd-dtr & \[ head & verbal\\
                  slash & \setof{\@1}\,  $\uplus$ \@2
               \]\\
     non-hd-dtrs & \< \[local &  \@1 \] \>
   \]
   \end{avm}
   \isfeat{head-daughter}\isfeat{slash}\isfeat{non-head-daughters}\isfeat{head}\isfeat{local}
   \istype{verbal}
\end{exe}
In words: they are \istype{verbal}\type{verbal} ---
e.g.\ clausal --- phrases where the \feat{slash} value of the head daughter is the
\feat{slash} value of the mother plus the \feat{local} value of the non-head daughter
(equivalently, the \feat{slash} value of the mother is the \feat{slash} value of the
head daughter less the \feat{local} value of the non-head daughter). Head-filler phrases are a
sub-type of another phrase type (\istype{head-nexus-phrase}\type{head-nexus-phrase}) which specifies identity of
content between mother and head daughter.

Putting these together with a constraint that requires clauses to have empty \feat{rel}
values will license local trees like that in Figure~\ref{x:rc-48} for a finite relative
clause (\istype{fin-wh-rel-cl}\type{fin-wh-rel-cl}) like \pref{x:rc-3} (simplifying, and
disregarding most empty and irrelevant attributes).\footnote{\label{fn:rc-3}This
  assumption about \feat{rel} values is one of many minor technical differences between
  \cite{Sag:97} and \cite{Pollard:Sag:94}, where the non-empty \feat{rel} value is
  \is{relative inheritance}inherited upwards to RP, and is discharged there. This means
  that for \citeauthor{Pollard:Sag:94}, but not for \cite{Sag:97}, a \lic{wh}-relative
  clause is a \feat{rel} marked clause.}
\begin{figure}
  \newcommand{\MotherDA}{S
      \begin{avm}
   \[ head & \@4 \[ mod&\nbar\subtag{1} \]\\
      cont & \@3\\ % \[ \asort{fact} prop\|soa & \@4 \]\\
      slash & \setof{ }\\
      rel & \setof{ }\\
   \]
   \end{avm}
 }
\newcommand{\FillerDDA}{PP~\idx{2}~\begin{avm}\[rel&\setof{\@1}\]\end{avm}}
  \newcommand{\HeadDDA}{S~\begin{avm}\[head&\@4\\cont&\@3\\slash&\setof{\@2}\\rel&\setof{ }\]\end{avm}}
    \begin{forest}
   [{\MotherDA}, baseline 
      [{\FillerDDA} [ {to whom} ,roof] ]
      [{\HeadDDA}   [ {Kim spoke} ,roof] ]
   ]
   \end{forest}
   \isfeat{slash}\isfeat{head}\isfeat{content}\isfeat{mod}\isfeat{rel}
   \caption{A \cite{Sag:97}-style structure for a finite \lic{wh}-relative clause}
   \label{x:rc-48}
\end{figure}
The \feat{rel} specification on the non-head daughter
(the PP) in \pref{x:rc-46} ensures the presence of a \lic{wh}-phrase, and the fact that this
is a head-filler phrase ensures that the \lic{wh}-phrase cannot be \emph{in situ} (cf.\
\pref{x:rc-8}, above); the [\feat{rel}~\setof{}] on the daughter S excludes the
possibility of additional \isi{relative pronoun}s inside the S (i.e.\ the possibility of multiple
\isi{relative pronoun}s, cf.\ \lic{*(the person) to whom Kim spoke about whom}). \feat{rel}
inheritance will carry the index of the \is{relative clause!antecedent of}antecedent down into the PP,
guaranteeing the presence of a \isi{relative pronoun} co-indexed with any nominal that this
relative clause is used to modify. Further upward \is{relative inheritance}inheritance of this \feat{rel} value
is prevented by a requirement that all clauses (including relative clauses) have empty
\feat{rel} values.\footnote{%
  Sag's account of the \is{relative inheritance}propagation of \feat{rel} values is a special case of the
  apparatus that is now standardly assumed for propagation of all non-local features,
  \feat{slash}, \feat{wh} (i.e.\ \feat{que}), and \feat{background}
  \citep[Chapter 5]{Ginzburg:Sag:00}. Upward inheritance is handled by a constraint on
  \istype{word}\type{word}s that says that (by default) the \feat{rel} value of a word is the union
  of the \feat{rel} values of its arguments.  In the absence of a lexical head with
  arguments (e.g.\ in \lic{of whom} and \lic{of whose friends} if \lic{of} is treated
  simply as a marker) the \feat{rel} value on a phrase is that of its head daughter (the
  ``\isi{Wh-Inheritance Principle}'', WHIP); see \citealt[449]{Sag:97}.  Since these are only
  default principles, they can be overridden, e.g.\ by the requirement that clauses have
  empty \feat{rel} values.}  The \feat{slash} specification on the head S daughter
will ensure that the \feat{local} value of the PP is inherited lower down inside the S, so that the
subcategorisation requirements of \lic{speak} can be satisfied, and the right content is
produced for this S (and passed to the mother S, because this is a head-filler
phrase).

The task of combining a nominal and a relative clause (in particular, identifying indices
and unioning restrictions) involves a further phrase type \istype{head-relative-phrase}\type{head-relative-phrase},
as in \pref{x:rc-49}.
\begin{exe}\ex\label{x:rc-49}
  \istype{head-relative-phrase}\type{head-relative-phrase}  \(\Rightarrow\)
  \begin{avm}
   \[
      head & noun\\
      cont & 
      \[ index & \@2\\
         restr & \@3  $\uplus$ \{\[\asort{fact} prop & \@4 \] \}
      \]\\
      hd-dtr & 
      \[ index & \@2\\
         restr & \@3
      \]\\
      non-hd-dtr & \[ cont & \@4 \]
   \]
   \end{avm}
   \isfeat{head}\isfeat{content}\isfeat{index}\isfeat{restr}\isfeat{prop}
   \istype{noun}
 \end{exe}

In words, this specifies a nominal construction (i.e.\ one whose head is
a noun), whose \feat{content} is the same as that of its head daughter, except that the
content of the non-head-daughter (the relative clause) has been added to its restriction
set. (Thus, it is this construction that takes care of the mismatch between the
``internal'', propositional, \feat{content} of the relative clause itself, and its
``external'' contribution of restrictions on the nominal it modifies).  Since
\istype{head-relative-phrase}\type{head-relative-phrase}s are a subtype of \istype{head-adjunct-phrase}\type{head-adjunct-phrase}, which
requires the \feat{mod} value of the non-head to be identical to the \feat{synsem}
value of the head \citep[475]{Sag:97}, this will give rise to structures like that in
Figure~\ref{fig:rc-4}.\footnote{This is not the normal semantics associated with head-adjunct
  phrases (where the content is simply the content of the adjunct daughter). This could be
  dealt with by introducing a separate sub-type of \istype{head-adjunct-phrase}\type{head-adjunct-ph} which deals
  with content in this way: \istype{head-adjunct-phrase}\type{head-adjunct-ph} itself would impose no constraints on
  content. Notice that we again follow \cite{Ginzburg:Sag:00} in taking restrictions to be
  sets of \istype{fact}\type{facts} (\citealt{Sag:97} assumes they are sets of
  \istype{proposition}\type{propositions}). Nothing hangs on this.}

\begin{figure}
\newcommand{\NContentDA}{
   \begin{avm} 
   \[ cont &
      \[  index & \@1 \\  restr & \@2 \]
   \]
   \end{avm}
 }
 \newcommand{\TopNContentDA}{
   \begin{avm}
   \[ cont & 
      \[ index & \@1 \\
         restr & \@2 \, $\uplus$ \{ \[\asort{fact} prop & \@3 \] \} 
      \]
   \]
   \end{avm}
 }
 \newcommand{\RelContentDA}{
   \begin{avm}
   \[ mod & \@4\\ cont & \@3  \]
   \end{avm}
 }
    \begin{forest}  baseline %qtree edges, 
   [{\ibar{N}~\TopNContentDA}
      [{\idx{4}~\ibar{N}~\NContentDA} [ {person} ,roof] ]
      [{S~\RelContentDA} 
         [ {to whom Kim spoke}, roof ]
      ]
   ]
   \end{forest}
   \caption{\citegen{Sag:97} analysis of a relative clause plus its antecedent}
  \label{fig:rc-4}
  \isfeat{content}\isfeat{index}\isfeat{restr}\isfeat{prop}\isfeat{mod}
  \istype{fact}
\end{figure}\is{relative clause!antecedent of}

From a purely formal point of view, the \istype{head-relative-phrase}\type{head-relative-phrase} construction is not
strictly necessary. It would be possible to build its semantic effects into the
\istype{rel-cl}\type{rel-cl} construction, so that the structure in Figure~\ref{fig:rc-4} would be an entirely normal head-adjunct
phrase where the content comes from the adjunct daughter. There are two arguments against
this. One is that it would require the relative clause to have nominal
(i.e.\ \istype{scope-object}\type{scope-object}) content, which is somewhat at odds with its status as a
clause. The other is that it would push the semantic mismatch into the relative clause
itself. That is, semantically, relative clauses like \lic{to whom Kim spoke} would no
longer be normal head-filler phrases where \feat{content} is shared between head and
mother. Perhaps neither argument is compelling --- and in fact, the discussion of relative clauses in \citet[522]{Sag:10b} employs essentially this approach, making the \lic{wh}-relative clause construction responsible for converting the propositional semantics of its head daughter into the noun-modifying semantics appropriate for a relative clause \citep[522]{Sag:10b}.
\is{relative clause!wh-relative@\lic{wh}-relative|)}
\is{relative clause!construction|)}

\subsection{Interim Conclusions}
\label{sec:rc-interim-conclusions}
The discussion so far has focused on one kind of relative clause, sketched the basic ideas
and intuitions behind the HPSG approach, and outlined the two main approaches:
that of \cite{Pollard:Sag:94} and that of \cite{Sag:97}. At some levels they seem very
different (e.g.\ in the use of phonologically empty lexical heads vs.\ the use of constructions), and there
are differences in terms of low level technical details (e.g.\ precisely which
phrases are specified as having empty \feat{rel} values, and in the precise way
\is{relative inheritance}inheritance of \feat{slash} and \feat{rel} values is terminated). But in other respects they are
very similar: for the most part the same features are used in ways that are not radically
different.

More significantly, the approaches involve a common view of the relation between relative
clause and \is{relative clause!antecedent of}antecedent: the view that the relative clause is adjoined to the \is{relative clause!antecedent of}antecedent,
with the relation between the \is{relative clause!antecedent of}antecedent and the \is{relative clause!relativised constituent}relativised constituent within the
relative clause being one of co-indexation (a more or less anaphoric relation): a view
that can be traced back to \cite{Chomsky77a-u}.

Outside HPSG this style of analysis stands in contrast to two others: the \emph{raising}
analysis \citep[see \emph{inter alia}][]{Schachter73a-u,Vergnaud74a-u,Kayne94a-u}, and the
\emph{matching} analysis \citep[see \emph{inter
  alia}][]{Chomsky65a,Lees61Constituent,Sauerland98MeaninChain}. Under the \is{relative
  clause!raising analysis of}raising analysis, the relative clause contains a DP of the
form \lic{which}+noun, which is preposed to the beginning of the clause; then the noun is
moved out of the relative clause (``raised'') to combine with a determiner, which selects
both the noun and the relative clause. According to the \is{relative clause!matching
  analysis of}matching analysis, the relative clause is adjoined to the \is{relative
  clause!antecedent of}antecedent, as in the adjunction analysis, but, as in the
\is{relative clause!raising analysis of}raising analysis, the relative clause contains a
DP \lic{which}+noun, which is preposed to the beginning of the clause; the noun is not
raised, but the noun is deleted under identity with the \is{relative clause!antecedent
  of}antecedent nominal.

Neither analysis has any appeal from an HPSG perspective: as normally understood, both are
fundamentally derivational in nature, presupposing at least two levels of syntactic
structure. Moreover, many of the motivations usually cited are absent given standard HPSG
assumptions (e.g.\ arguments from binding theory which can be taken as indicating the
presence of a \lic{wh}-phrase inside the relative clause fall out naturally without this
assumption given the argument-structure-based account of binding theory which is standard
in HPSG, see \crossrefchapteralt{arg-st}). More important, as discussed in
\cite{Webelhuth18Idioms}, both face numerous empirical difficulties and miss important
generalisations which are unproblematic for the style of analysis described
here.\footnote{For example, both analyses treat \lic{wh}-words like
  \lic{who}, \lic{what}, \lic{which}, and their equivalents as determiners, whereas in
  fact they behave like pronouns. Case assignment appears to pose a fundamental problem
  for the \is{relative clause!raising analysis of}raising analysis, since it seems to predict that the case properties of the
  \is{relative clause!antecedent of}antecedent NP should be assigned ``downstairs'' inside the relative clause. But they
  never are \citep[see][]{Webelhuth18Idioms}.}
\is{relative clause!wh-phrase@\lic{wh}-phrase in|)}

\section{Varieties of relative clause}
\label{sec:rc-varieties}
In this section we will look at how the approaches introduced above have been adapted and
extended to deal with other kinds of relative clause in a variety of
languages.\footnote{In addition to the phenomena and languages we discuss, the HPSG
  literature includes more or less detailed treatments of relative clauses in
  \ili{Bulgarian} \citep{Avgustinova96-Eng}, \ili{German}
  \citep{Mueller99b,Mueller99a,MuellerCurrentApproaches}, \ili{Hausa}
  \citep{Crysmann:16}, \ili{Polish} \citep{MMPK2003a-u,Bolc:05}, and \ili{Turkish}
  \citep{Guengoerdue:96}.  } Section~\ref{sec:rc-wh-relatives} looks at other kinds of
relative clauses which involve a \isi{relative pronoun}, notably ones which do not involve
a finite verb. Section~\ref{sec:rc-comp-relatives} and Section~\ref{sec:rc-bare-relatives}
look at relative clauses which do not involve \isi{relative pronoun}s:
Section~\ref{sec:rc-comp-relatives} looks at relative clauses which can be analysed as
involving a complementiser; Section~\ref{sec:rc-bare-relatives} looks at ``bare''
relatives, which involve neither \isi{relative pronoun}s nor
complementisers. Section~\ref{sec:rc-non-restr-suppl} looks at non-restrictive relative
clauses, which lack the intersective semantics associated with prototypical relative
clauses.

One dimension of variation among \is{relative clause!construction}relative clause
constructions which we will discuss only in passing relates to whether, in the case of
relative clauses that involve a filler-gap construction, the gap is genuinely absent
phonologically (as in the examples we have looked at so far), or whether it is realised as
a full pronoun (a so-called \emph{\isi{resumptive pronoun}}) as in \pref{x:rc-50} from
\cite[28]{Alqurashi:Borsley:12}, or the \ili{English} example in \pref{x:rc-51} --- the
resumptive pronouns are indicated in bold.
\begin{exe}\ex\label{x:rc-50}
\gll wajadtu    l-kitab-a    [{llaði} tuħib-\textbf{hu}     Hind-un] \\
     found.{\sc 1.sg} {\DEF-book-\ACC} \hspaceThis{[}{that.{\sc m.sg}}   like.{\sc 3.f.sg-3.m.sg} Hind-\NOM\\\jambox*{(\ili{Arabic})}
\glt `I found the book that Hind likes.' 
\end{exe}
\begin{exe}\ex\label{x:rc-51}
  This is the road which I don't know where \textbf{it} goes.
\end{exe}
The analysis of resumptive pronouns is discussed elsewhere in this volume \crossrefchapterp{udc}, and while they
are an important feature of relative clause constructions in many languages
(see e.g.\ \citealt{Vaillette2001a-u}; \citealt{Vaillette2001b}; \citealt{Taghvaipour:05,AbeilleGodard07,AB2013a-u}), the issues seem to be similar in all
constructions involving unbounded dependencies, and not specific to relative clauses.

\subsection{\lic{Wh}-relatives}
\label{sec:rc-wh-relatives}
\is{relative clause!wh-relative@\lic{wh}-relative|(}
\is{relative clause!construction|(}
\is{relative clause!wh-phrase@\lic{wh}-phrase in|(}

Finite \lic{wh}-relatives in \ili{English} have been discussed above
(Section~\ref{sec:rc-approaches}). \ili{English} also allows \lic{wh}-relatives which are headed by
non-finite verbs, such as \pref{x:rc-52}; \pref{x:rc-53} is a similar example from \ili{French}.
\begin{exe}
\ex\label{x:rc-52}  a person [on whom to place the blame]
\ex\label{x:rc-53}
\gll un paon                 [dans les plumes duquel]  mettre le courrier\\
     a  peacock \hspaceThis{[}in the feathers of.which to.place the mail\\\jambox*{(\ili{French})}
     \glt `a  peacock in whose feathers to place the mail'
\end{exe}

\is{relative clause!non-finite|(}
Non-finite relatives were not discussed by \cite{Pollard:Sag:94}, but \citegen{Sag:97}
constructional approach provides a straightforward account. It involves
distinguishing two sub-types of \istype{hd-fill-phrase}\type{hd-fill-ph}: a finite subtype which has an empty
\feat{subj} list, and a non-finite subtype whose \feat{subj} list is required to contain
just a \isi{PRO} (that is, a pronominal that is not syntactically expressed as a
syntactic daughter). This requirement reflects the fact that non-finite
\lic{wh}-relatives do not allow overt subjects:
\begin{exe}\ex[*]{\label{x:rc-54}
  a person [on whom (for) Sam to place the blame]}
\end{exe}
The relative clause in \pref{x:rc-52} receives a structure like that in
Figure~\ref{fig:rc-5}. Apart from the finite specification, this differs from the finite
\lic{wh}-relative in \pref{x:rc-48} above only in the presence of the \isi{PRO} on the
\feat{subj} list.\footnote{The use of S\subscr{inf} in Figure~\ref{fig:rc-5} is an
  approximation. First, S is standardly an abbreviation for something of type
  \istype{verb}\type{verb} with empty \feat{subj} and \comps values,
  and here there is a non-emtpy \feat{subj}. Second, Sag would have CP instead of S here,
  reflecting his analysis of \lic{to} as a complementiser rather than an auxiliary verb,
  as is often assumed in HPSG analyses \citep[e.g.][51--52]{Ginzburg:Sag:00}. S and CP are
  not very different (both \istype{verb}\type{verb} and \istype{comp}\type{comp} are
  subtypes of \istype{verbal}\type{verbal}), but Sag is careful to treat \lic{to} as a
  \istype{comp}\type{comp} and non-finite \lic{wh}-relatives as CPs because this gives a
  principled basis for excluding overt subjects.}
\begin{figure}
  \newcommand{\MotherDA}{S\subscr{inf}
      \begin{avm}
   \[ head & \@4 \[ mod& \nbar\subtag{1} \]\\
      cont & \@3\\ % \[ \asort{fact} prop\|soa & \@4 \]\\
      slash & \{\}\\
      rel & \{\}\\
      subj & \tuple{\@5}
   \]
   \end{avm}
 }
  \newcommand{\FillerDDA}{\idx{2}~PP~\begin{avm}\[rel&\{\@1\}\]\end{avm}}
  \newcommand{\HeadDDA}{S\subscr{inf}~\begin{avm}\[head&\@4\\cont&\@3\\slash&\{\@2\}\\rel&\{\}\\subj&\sliste{\@5\thinspace\
           \mathrm{PRO}}\]\end{avm}}
%  \newcommand{\HeadDDA}{S\subscr{inf}~\begin{avm}\[head&\@4\\cont&\@3\\slash&\{\@2\}\\rel&\{\}\\subj&\tuple{\@5\thinspace\
%           \mathrm{PRO}}\]\end{avm}}

    \begin{forest}  %qtree edges
   [{\MotherDA}, baseline 
      [{\FillerDDA} [ {on whom} ,roof] ]
      [{\HeadDDA}   [ {to put the blame} ,roof] ]
   ]
   \end{forest}
   \caption{\citegen{Sag:97} analysis of a non-finite \lic{wh}-relative clause (\type{inf-wh-rel-cl})}
   \label{fig:rc-5}
   \istype{inf-wh-rel-cl}
   \isfeat{head}\isfeat{content}\isfeat{slash}\isfeat{rel}\isfeat{subj}\is{PRO}
 \end{figure}
 
The exclusion of overt subjects is not peculiar to non-finite relatives (it is shared by
non-finite interrogatives, cf.\ \lic{I wonder on whom (*for Sam) to put the
  blame}), but non-finite \lic{wh}-relatives are subject to the apparently
idiosyncratic restriction that the \lic{wh}-phrase must be a PP:
\begin{exe}\ex\begin{xlist}
  \ex[*]{a person who(m) to place the blame on\hfill (relative)}
  \ex[] {I wonder  who(m) to place the blame on\hfill (interrogative)}
\end{xlist}\end{exe}
The relevant constraints can be stated directly --- roughly as in \pref{x:rc-55}
(disregarding constraints that are inherited from elsewhere). In words, these constraints say that a non-finite
head-filler phrase must have an unexpressed subject, and a non-finite \lic{wh}-relative
clause is a non-finite head-filler phrase whose non-head daughter is a PP.
\begin{exe}\ex\begin{xlist}\label{x:rc-55}
  \ex \istype{inf-hd-fill-phrase}\type{inf-hd-fill-ph}    \(\Rightarrow\)
  \begin{avm}
   \[ hd-dtr &
      \[ head & \[ vform & non-finite \]\\
         subj & \tuple{\mathrm{PRO}}
      \]\\
   \]
   \end{avm} 
   \isfeat{head-daughter}\isfeat{head}\isfeat{subj}\isfeat{vform}
   \istype{non-finite}\is{PRO}
 \ex\label{x:rc-56}
   \istype{inf-hd-fill-rel-cl}\type{inf-hd-fill-rel-cl}    \(\Rightarrow\)
   \istype{inf-hd-fill-phrase}\type{inf-hd-fill-ph} \&
   \begin{avm}
   \[ non-hd-dtrs & \tuple{\mathrm{PP}}  \]
   \end{avm} 
 \end{xlist}
   \isfeat{non-head-daughters}
\end{exe}
\is{relative clause!non-finite|)}
\is{relative clause!wh-relative@\lic{wh}-relative|)} 
\is{relative clause!construction|)}
\is{relative clause!wh-phrase@\lic{wh}-phrase in|)}

\subsection{Complementizer relatives}
\label{sec:rc-comp-relatives}
\is{relative clause!headed by complementizer|(} 

As well as \lic{wh}-relatives, which involve \isi{relative pronoun}s, there are cases of relative clauses
which appear to be headed by what is plausibly analysed as a complementiser. In this
section we look first at \ili{Arabic}, where a complementiser analysis has been proposed, 
then at \ili{English}, where such an analysis seems possible for some cases, but where it is
controversial, and an interesting construction in French.\footnote{There are also cases which involve a \isi{relative pronoun} \emph{and} a
  complementiser, as in the following from \citegen{Hinrichs:Nakazawa:02} discussion of Bavarian \ili{German}:
  \begin{exe}
    \ex \gll der Mantl (den) wo i kaffd hob\\
    the coat \hspaceThis{(}which that I bought have\\\jambox*{(Bavarian \ili{German})}
    \glt `the coat which I bought’
  \end{exe}
  \cite{Hinrichs:Nakazawa:02} analyse these as \lic{wh}-relatives,\is{relative clause!wh-relative@\lic{wh}-relative} even when
  the \isi{relative pronoun} is omitted, as it can be under certain circumstances. 
}
\is{relative clause!construction|)}

\subsubsection{Arabic}
\label{sec:rc-arabic}
\is{relative clause!headed by complementizer!in Arabic|(}

\cite{Alqurashi:Borsley:12} argue that in \ili{Arabic} finite relatives the word 
\emph{ʔallaði} `that' (transliterated as \emph{llaði} in \pref{x:rc-57}, from
\citealt[27]{Alqurashi:Borsley:12}) and its inflectional variants should be analysed as a
complementiser, with a \feat{synsem} value roughly as in \pref{x:rc-58}.\footnote{Here S\subscr{fin}  means a
  finite clause (a \istype{verb}\type{verb} which is \comps and \feat{subj} saturated).
  NP\subscr{def} in the \feat{mod} means a fully saturated definite nominal whose
  \feat{content} is given after the colon. According to \pref{x:rc-58} the content of the
  S\subscr{fin} is merged
  with the restrictions of this modified NP. This is imprecise: as discussed above, what
  should be merged is a \istype{fact}\type{fact} constructed from the content of the S\subscr{fin}.}
\begin{exe}\ex\label{x:rc-57}
\gll jaaʔa l-walad-u   llaði qaabala l-malik-a.\\
     {came.3.{\sc m.sg}} {\DEF-boy-\NOM} {that.{\sc m.sg}} {met.3.{\sc m.sg}} {\DEF-king-\ACC}\\\jambox*{(\ili{Arabic})}
\glt `The boy who met the king came.'
\end{exe}
\begin{exe}\ex\label{x:rc-58}
  \begin{avm}
   \[ \asort{synsem}
      loc & 
         \[ cat & 
            \[
               head & 
               \[\asort{c} 
                  mod & {\upshape NP\subscr{def}:} \[ index & \@1\\restr&\@2\]
               \]\\
               comps &
               \<
                  \[ loc &
                     \[ cat & {\upshape S\subscr{fin}}\\
                        cont & \@3\\
                     \]\\
                     non-loc &  \[ slash  & \{ NP\subscr{\idx{1}}\} \]
                  \]
               \>\\
            \]\\
            cont &
            \[ index & \@1\\
               restr & \@2 $\uplus$ \{ \@3 \}
            \]
         \]\\
         non-loc & \[ slash  & \{ \} \]
      \]
   \end{avm}
   \isfeat{local}\isfeat{cat}\isfeat{head}\isfeat{mod}\isfeat{index}\isfeat{restr}\isfeat{comps}\isfeat{non-local}\isfeat{slash}
\end{exe}
According to this, \emph{ʔallaði} will combine with a slashed finite sentential
complement, to produce a phrase which will modify a definite NP.  When it combines with
that NP, its content will have the same \feat{index} as the NP, and the restrictions of
the NP combined with the propositional content of the sentential complement. The
\feat{slash} value on the sentential complement means that it will contain a gap (or a
\isi{resumptive pronoun}) which also bears the same index.

Notice that there is no role for a \feat{rel} feature here (obviously, since there is no
\isi{relative pronoun}).  The presence of the \feat{slash} value indicates that
\citeauthor{Alqurashi:Borsley:12} assume that \ili{Arabic} relatives involve an unbounded
dependency (i.e.\ that the gap or \isi{resumptive pronoun} may be embedded arbitrarily
deeply within the relative clause). In \is{relative
  clause!wh-relative@\lic{wh}-relative}\lic{wh}-relatives, as described above, the
unbounded dependency is what \cite[155]{Pollard:Sag:94} call a ``strong'' unbounded
dependency, \is{unbounded dependency!strong} i.e.\ one that is terminated by at the top by
a filler (the \is{relative clause!wh-phrase@\lic{wh}-phrase in}\lic{wh}-phrase), in a head-filler phrase. This is not the case here ---
here there is no filler, and upward inheritance of the gap is halted by the head
\emph{ʔallaði} itself (cf.\ its own empty \feat{slash} specification). That is,
\ili{Arabic} relatives (and complementiser relatives generally) are normal head-complement
structures, involving what \citeauthor{Pollard:Sag:94} (\emph{loc cit}) call a ``weak''
unbounded dependency \is{unbounded dependency!weak} construction (like \ili{English}
purpose clauses and \lic{tough}-constructions).\footnote{\cite[42]{Alqurashi:Borsley:12}
  assume that the \feat{slash} inheritance is governed by a default principle, so the
  empty \feat{slash} specification on \emph{ʔallaði} prevents upward
  inheritance. The same effect could be achieved with an appropriate \feat{to-bind}
  specification. }


Since \emph{ʔallaði} shows inflections agreeing with the \is{relative clause!antecedent of}antecedent NP for
\feat{number}, \feat{gender}, and \feat{case}, different forms will impose
additional restrictions on the modified NP (e.g.\ the form transliterated as
\emph{llaði} in \pref{x:rc-57} will add to \pref{x:rc-58} the additional requirement that
the NP which is modified must be masculine singular).

Notice that \citeauthor{Alqurashi:Borsley:12}'s account is entirely lexical: no
constructional apparatus is used at all. \cite{Hahn:12} argues for a constructional
alternative.\footnote{\ili{Arabic} also has finite relatives that do not have an overt
  relativiser (and which occur with indefinite
  antecedents).\is{relative clause!antecedent of} \citeauthor{Alqurashi:Borsley:12} analyse these as involving a
  phonetically null complementiser. In addition, \ili{Arabic} also has non-finite and free
  relatives, which have received some attention. See \citet{Melnik:06},
  \citet{Haddar:Boukedi:09,Zalila:Haddar:11}, \citet{Hahn:12}, and \citet{Crysmann:Reintges:14}
  for further discussion.}
\is{relative clause!headed by complementizer!in Arabic|)}

\subsubsection{English}
\label{sec:rc-english}
\is{relative clause!headed by complementizer!in English|(}
\is{relative clause!construction|(}

A similar analysis could be proposed for \ili{English} \lic{that}-relatives as in
\pref{x:rc-59}. However, this is controversial: \cite{Pollard:Sag:94} treat some uses of
\lic{that} as simply a marker (i.e.\ the realisation of a \feat{marking} feature whose value
is \istype{that}\type{that}, as opposed to \istype{unmarked}\type{unmarked}), and
others as  a \isi{relative pronoun}, see \cite[221--222]{Pollard:Sag:94}. 
\cite[462--464]{Sag:97} prefers to treat \lic{that} as
a \isi{relative pronoun}.\footnote{\label{fn:rc-4}\cite{Pollard:Sag:94} treat instances of 
  \lic{that} in relative clauses involving relativisation of a top level subject, like \pref{x:rc-60},
  as a \isi{relative pronoun}.  In other relative clauses, in particular those involving
  relativisation of embedded subjects, like 
  \pref{x:rc-61}, or non-subjects, \lic{that} is treated as a marker, meaning that such
  clauses are treated as instances of bare relatives. It is hard to find clear empirical
  evidence against this, but an analysis which provides a uniform treatment of \ili{English}
  \lic{that}-relatives is clearly more appealing.}
\begin{exe}\ex\begin{xlist}\label{x:rc-59}
  \ex\label{x:rc-60}  person that \trace admires Kim
  \ex\label{x:rc-61} person that everyone thinks \trace admires Kim
\end{xlist}\end{exe}
On \citegen{Pollard:Sag:94} analysis, some support for a \isi{relative pronoun} analysis comes
from coordination. It is possible to coordinate \lic{that} relatives with normal
\lic{wh-}relatives quite freely, as in \pref{x:rc-62}. This is a natural consequence if the
\feat{rel} value of the coordinate structure is shared by both conjuncts (implying that
both conjuncts contain \isi{relative pronoun}s, of course).\footnote{The same argument can be
  made given \citegen{Sag:97} assumptions, but it is less direct. Recall that, on
  \citegen{Sag:97} analysis, relative clauses have empty \feat{rel} values, so a
  coordination of relative clauses will have an empty \feat{rel} value too  (cf.\ above Section~\ref{sec:rc-sag-1997},
  especially  Footnote~\ref{fn:rc-3}). For \cite{Sag:97} the argument relies on the assumption that all
  and only \lic{wh}-relatives are NP modifiers, rather than \ibar{N} modifiers as we have
  presented them here (cf.\ Footnote~\ref{fn:rc-2}). Since coordination involves identity of
  \feat{mod} values, data like \pref{x:rc-62} show that \lic{that}-relatives must be NP
  modifiers, and consequently must be \lic{wh}-relatives, i.e.\ must contain a
  \isi{relative pronoun} (namely, \lic{that}).}
\begin{exe}\ex\label{x:rc-62}
 a book [that/which you own or that/which you can borrow]
\end{exe}
Potential evidence against this, and in favour of a complementiser-style (or perhaps marker-style) analysis, would be that \lic{that} differs from normal \isi{relative pronoun}s in not
allowing pied-piping, cf.\ \pref{x:rc-65}.
\begin{exe}\ex\begin{xlist}\label{x:rc-63}
  \ex[] {the person that I spoke to \trace}\label{x:rc-64}
  \ex[*]{the person to that I spoke \trace}\label{x:rc-65}
\end{xlist}\end{exe}
\cite[464]{Sag:97} and \cite[220]{Pollard:Sag:94} argue that this restriction is
compatible with a \isi{relative pronoun} analysis on the assumption that \lic{that} has
nominative case, so that it cannot occur as e.g.\ the complement of a preposition.  Notice
also that \lic{who} (which is generally regarded as a \isi{relative pronoun}) follows the same
pattern:
\begin{exe}\ex\begin{xlist}\label{x:rc-66}
  \ex [] {the person who I spoke to \trace}\label{x:rc-67}
  \ex [*]{the person to who I spoke \trace}\label{x:rc-68}
\end{xlist}\end{exe}

However, this response is not very convincing.  What \pref{x:rc-63} and \pref{x:rc-66} show is
that \lic{that} and \lic{who} cannot appear as complement of a preposition, but can be
associated with a gap that is complement of a preposition. But this is inconsistent with
them being fillers in a head-filler phrase, where \feat{slash} inheritance ensures
identity between the \feat{local} values of filler and gap (including, of course
\feat{case}): if \lic{that} and \lic{who} are nominative, then they should not be
compatible with non-nominative gaps, such as we see in \pref{x:rc-64} and \pref{x:rc-67}. But
if they are not fillers, then they must be heads (or markers). Developing an analysis
along these lines is beyond the scope of this paper, but it is worth pointing out that it
would not involve a radical change to the analyses described above (for example, modifying
\citeauthor{Sag:97}'s \citeyear{Sag:97} analysis might involve creating a new subtype of \istype{rel-cl}\type{rel-cl} for
\lic{that} and \lic{who} relatives, separate from \istype{wh-rel-cl}\type{wh-rel-cl}, and new lexical
entries for \lic{that} and \lic{who}, but could otherwise use the same apparatus, and
produce the same distribution of properties).
\is{relative clause!headed by complementizer!in English|)}
\is{relative clause!construction|)}

\subsubsection{French}
\label{sec:rc-french}
\is{relative clause!dont-relative@\lic{dont}-relative|(}
\is{relative clause!headed by complementizer!in French|(}

Besides \lic{wh}-relatives, \ili{French} has relatives introduced by complementisers:
\lic{que} `that' and \lic{dont} `of which'. \lic{Dont}-relatives present
something of a challenge, which is addressed in \cite{AbeilleGodard07}. \lic{Dont} is
generally analysed as a complementiser introducing finite relatives
\citep{Godard92a-u}. It can introduce a relative with a PP\sub{\lic{de}} gap (i.e.\ a gap
that could be occupied by a PP marked with the preposition
\lic{de} `of'). The contrast between the grammatical \pref{x:rc-70} and the
ungrammatical \pref{x:rc-71} arises because whereas \lic{parler} `talk' in \pref{x:rc-70} takes
a PP\sub{\lic{de}} complement, \lic{comprendre} `understand' in
\pref{x:rc-71} takes an NP complement, and so cannot cannot contain a gap licensed by
\lic{dont}, as can be seen in \pref{x:rc-169} and \pref{x:rc-170}.
\begin{exe}\ex\begin{xlist}\label{x:rc-69}
  \ex[]{\gll
    un problème 	dont 	on  a 	parlé\\
    a  problem of-which	one has talked\\\jambox*{(\ili{French})}
    \glt `a problem that we have talked about'}\label{x:rc-70}
  \ex[*]
  {\gll un problème  dont         on résoudra\\
         a   problem  of-which	one will.resolve\\\jambox*{(\ili{French})}
  \glt Intended: `a problem that we will resolve'}\label{x:rc-71}
\end{xlist}\end{exe}
\begin{exe}\ex\begin{xlist}\label{x:rc-168}
  \ex[]{\gll On  a 	{parlé} d' un problème.\\
           One  has {talked} of a problem\\\jambox*{(\ili{French})}
       \glt `We have talked about a problem.'}\label{x:rc-169}
  \ex[*]{ \gll On  	{résoudra}  d' un problème.\\
           One  will.resolve of a problem\\\jambox*{(\ili{French})}
       \glt Intended: `We will resolve a problem.'}\label{x:rc-170}
\end{xlist}\end{exe}

\citeauthor{AbeilleGodard07} suggest a lexical entry for \lic{dont} with a \feat{synsem}
value along the lines of \pref{x:rc-72}.
\begin{exe}\ex\label{x:rc-72}
  \begin{avm}
   \[ \asort{synsem}
      loc &
      \[ cat &
         \[ head & \[\asort{c} mod & \nbar\subscr{\@1} \]\\
            comps &
            \<
               \[ loc & \[ cat & \upshape{S\subscr{fin}}\\cont &\@2\]\\
                  non-loc&
                  \[slash&
                     \{\@3
                        \[\asort{nprl} cat & \upshape{PP\subscr{de}}\thinspace\subscr{\@1}\]
                        %   \upshape{NP\subscr{de}}\thinspace\subscr{\@1}
                     \}
                  \]
               \]
            \>\\
         \]\\
         cont & \@2\\
      \]\\
      to-bind & \{\@3\}
   \]
   \end{avm}
   \isfeat{local}\isfeat{head}\isfeat{mod}\isfeat{comps}\isfeat{non-local}\isfeat{cat}\isfeat{to-bind}\isfeat{content}
   \istype{nplr}
\end{exe}
In words: \lic{dont} is a complementiser that takes a finite S complement, and heads a
phrase that can act as an \ibar{N} modifier. \lic{Dont} itself has no inherent semantic
content (its \feat{content} is just that of its complement S). The complement S is
associated with a \feat{slash} value that contains a PP\sub{\lic{de}} which is
co-indexed with the \is{relative clause!antecedent of}antecedent nominal, as specified in the \feat{mod} value. The
\feat{to-bind} value simply prevents this \feat{slash} element being inherited upwards
beyond the phrase headed by \lic{dont}. This \feat{slash} element is non-pronominal
(\istype{nprl}\type{nprl}) --- that is, a genuine gap, rather than a \isi{resumptive pronoun}.\footnote{\cite{AbeilleGodard07} assume that gaps and resumptive pronouns are
  associated with distinct subtypes of \istype{local}\type{local} value: \istype{prl}\type{prl} (pronominal) for
  pronouns and \istype{nprl}\type{nprl} (non-pronominal) for genuine gaps. The relevance of this will
  appear directly.}

\is{resumptive pronoun|(} Given this, one might expect that it is generally impossible for
a \lic{dont}-relative to have an NP as the \is{relative clause!relativised
  constituent}relativised constituent, but this is not the case. It is in fact possible,
provided that the \is{relative clause!relativised constituent}relativised constituent is
realised by an overt pronoun (i.e.\ a resumptive pronoun) and is somewhere inside the
complement of (some) propositional attitude and communication predicates. For example, in
\pref{x:rc-76} the pronoun \lic{le} represents the \is{relative clause!relativised
  constituent}relativised constituent, which appears in the complement of \lic{être
  certain} `be sure'.\footnote{One might consider an alternative analysis where \lic{dont}
  is associated with a PP\sub{\lic{de}} gap dependent of \lic{certain}, and the resumptive
  pronoun is a normal anaphoric pronoun --- this would correspond to a main clause along
  the lines of \lic{Paul is sure, of this problem, that we will resolve it}. One problem
  with this alternative is that this sort of PP\sub{\lic{de}} dependent is not very good
  with \lic{certain}, see \pref{x:rc-73}. Another is that it would not explain the fact
  that the personal pronoun is obligatory --- \pref{x:rc-74}, with no personal pronoun, is
  ungrammatical, though semantically coherent:
  \begin{exe}
    \ex[??]{\gll
	Paul est certain de ce problème   qu’  on le résoudra.\\
        Paul is   sure   of this  problem that one it
        will.solve.\\\jambox*{(\ili{French})}}\label{x:rc-73}
    \ex[*]{\gll
      un problème dont	[Paul est certain que tout va se résoudre]\\
      a problem      of-which \hspaceThis{[}Paul  is   sure that everything goes itself
      to.solve\\\jambox*{(French)}}\label{x:rc-74}
  \end{exe}%
  }
\begin{exe}\ex\label{x:rc-76}
  \gll
  un problème dont	[Paul est certain [qu' on \textbf{le} résoudra]]\\
  a problem   of-which \spacebr Paul  is   sure \spacebr that one it will.solve\\\jambox*{(French)}
  \glt `a problem that Paul is sure that we will solve'
\end{exe}

Unsurprisingly, the presence of a resumptive pronoun is associated with immunity to island
constraints. So, for example, in \pref{x:rc-78} we have a relative where the \is{relative clause!relativised constituent}relativised constituent is within a relative clause inside an embedded NP, which is 
impossible for a genuine gap.
\begin{exe}
  \ex{\gll
    un problème dont [Paul est certain [qu' {il y a}  [quelqu’un  qui \textbf{le} résoudra]]]\\
    a problem of-which \spacebr Paul is sure \spacebr that {there is}  \spacebr someone
    that it will.solve\\\jambox*{(French)}\label{x:rc-78}
  \glt `a problem such that Paul is sure that there is someone  who will solve it'}
\end{exe}
What is surprising, however, is that the path between \lic{dont} and the predicate that
licenses the resumptive \emph{is} sensitive to island constraints. To see this, compare
the grammatical \pref{x:rc-76} and \pref{x:rc-78} with the ungrammatical
\pref{x:rc-171}. All involve a \lic{dont} relative containing a resumptive pronoun
licensed by \lic{être certain}, but in \pref{x:rc-171},
\lic{être certain} is separated from \lic{dont} by an island boundary (\lic{être
  certain} is inside a relative clause).
\begin{exe}
  \ex[*]{\gll
    un problème dont {il y a} [quelqu’un qui est certain qu’ on le résoudra]\\
    a problem of-which {there is} \spacebr someone who is sure that one it will.solve\\\jambox*{(French)}
  }\label{x:rc-171}
\end{exe}

In short, though the dependency between the licensing predicate and the resumptive pronoun
can cross island boundaries, the dependency between the licensing predicate and \lic{dont}
cannot. \citegen{AbeilleGodard07} account of this is that while the dependency between the
licensing predicate and the \is{relative clause!relativised constituent}relativised constituent involves inheritance of a resumptive element, the dependency
between the licensing predicate and \lic{dont} involves inheritance of a gap. They suggest
that this should be dealt with by a lexical rule along the lines of \pref{x:rc-79}, 
where $\oplus$ signifies the `append' relation -- in combination with the ellipsis it allows the possibility that the \comps list may contain additional elements.
\begin{exe}\ex\label{x:rc-79}  Lexical Rule for Propositional attitude predicates in \ili{French}\\
  \begin{avm}
   \[ % cat & \[ head & v \]\\
      comps & \< \cPile{CP\\\[slash & \{ \@1 \[\asort{prl} cont\|index&\@2\] \}\]} \>\; $\oplus$ \ldots\\
    %  cont & \@3 proposition
   \]
   \end{avm}
   \isfeat{comps}\isfeat{slash}\isfeat{content}\isfeat{index}\istype{plr}
   $\mapsto$
  \begin{avm}
      \[ slash & \{ \[\asort{nprl} cat & \upshape{PP}\subscr{de}\subscr{\@2} \]\}\\
         to-bind & \{\@1\} \]
   \end{avm}
   \isfeat{slash}\isfeat{to-bind}\isfeat{cat}
 \end{exe}
In words, the left-hand side of this describes a lexeme that takes a CP complement with a
\feat{slash} value containing pronominal (\istype{plr}\type{plr}) elements (that is, a CP that can
contain resumptive pronouns). The effect of the rule is to provide a lexical entry that
binds off the resumptive pronoun, and introduces an PP\sub{\lic{de}} gap co-indexed to
the resumptive pronoun, that is, the sort of gap that can legitimately be associated with
\lic{dont}. Thinking from the top down, this rule produces a predicate that can appear in
a context with an inherited requirement for a PP\sub{\lic{de}} gap (e.g.\ a relative clause
headed by \lic{dont}), and convert this into a requirement for a resumptive pronoun
further down. Thinking from the bottom up, the predicate can bind off a resumptive
pronoun, and replace it with a gap dependency.\footnote{As \cite{AbeilleGodard07} point
  out, the facts are not quite as simple as this. In particular there is an interesting
  complication involving coordination. It is possible for a \lic{dont}-clause containing a
  predicate like \lic{être certain} to involve a coordinate structure, where one conjunct
  contains a PP\sub{\lic{de}} gap and the other contains a pronoun, as in \pref{x:rc-80} (the
  second conjunct here contains the pronominal \lic{y} `to it'; the English translation is
  intended to make it clear that the second conjunct is in the scope of \lic{être
    certain}).
  \begin{exe}\ex\label{x:rc-80}\gll
    un problème dont         Paul est certain [que nous avons parlé  \trace] [et que nous \textbf{y} reviendrons plus tard]\\
    a     problem   of-which Paul is   sure      \hspaceThis{[}that we have 
    spoken \hspaceThis{\trace} \hspaceThis{[}and that we to-it will.come.back more late\\\jambox*{(\ili{French})}
    \glt Lit: `a problem of which Paul is sure that we have  spoken and that
    he is sure that we will come back to it later'
  \end{exe}
  Dealing with this
  involves a formal complication that we leave aside here. See \cite{AbeilleGodard07}.}
\is{resumptive pronoun|)}
\is{relative clause!dont-relative@\lic{dont}-relative|)}
\is{relative clause!headed by complementizer!in French|)}
\is{relative clause!headed by complementizer|)}


\subsection{Bare relatives}
\label{sec:rc-bare-relatives}
\is{relative clause!bare|(}
\is{relative clause!construction|(}

Not all languages realise relative clauses using \isi{relative pronoun}s or complementisers. In this section we
will discuss HPSG analyses of what we will call \emph{bare relatives} in \ili{Japanese} and \ili{Korean}
(Section~\ref{sec:rc-bare-relat-japan}) and in \ili{English}, where they are often called
``\lic{that}-less'' relatives (Section~\ref{sec:rc-bare-relat-engl}).  The absence of \isi{relative pronoun}s means there is no question of pied-piping, hence no role for a \feat{rel}
feature in these constructions.

\subsubsection{Bare relatives in Japanese and Korean}
\label{sec:rc-bare-relat-japan}
\is{relative clause!bare!in Korean|(}
\is{relative clause!bare!in Japanese|(}
\is{relative clause!headed by verb|(}

\ili{Japanese}
relative clauses corresponding to \pref{x:rc-3} contain a gap, but are otherwise similar to
normal clauses, cf.\ \pref{x:rc-81} \citep[from][18]{SiraiGunjiRelative}; in \ili{Korean} they are
distinguished by special marking on the topmost verb --- cf.\ the \lic{-nun} affix on
\lic{sayngkakha} `think' in \pref{x:rc-82}
\citep[from][285]{Kim16SyntacticStrKorean}.
\begin{exe}\ex\label{x:rc-81}
\gll Naomi-ga \trace\subscr{i} yon-da hon\subscr{i}\\
     Naomi-{\sc nom} {} read-\textsc{past} book\\\jambox*{(\ili{Japanese})}
\glt `the book (that) Naomi read'
\end{exe}
\begin{exe}\ex\label{x:rc-82}
 \gll [motwu-ka           [Kim-i       \trace\subscr{i} ilk-ess-ta-ko]  sayngkakha-nun] chayk\subscr{i} \\
      \hspaceThis{[}everyone-{\sc nom} \hspaceThis{[}Kim-{\sc nom}  {}     read-{\sc pst-decl-comp} think-{\sc pres.mod}   book\\\jambox*{(\ili{Korean})}
 \glt `the book (that) everyone thinks Kim read'  
\end{exe}
Evidence for a gap in these examples is that it is not possible to put an overt NP in
place of the gap (e.g.\ putting \lic{sore-wo} `it-{\sc acc}' in \pref{x:rc-81}, or
\lic{sosel-u} `novel-{\sc acc}' in \pref{x:rc-82} renders them ungrammatical).\footnote{As
  well as these ``standard'' relatives, \ili{Korean} and \ili{Japanese} both have other kinds of
  relative construction, notably what are sometimes called \emph{\is{relative clause!internally headed}internally headed} relatives, and so-called
  \emph{pseudo-relatives}, which are briefly discussed below. See
  Section~\ref{sec:rc-depend-noun-pseudo}.}

\cite{SiraiGunjiRelative} provide a non-constructional account of \ili{Japanese} bare relatives
like \pref{x:rc-81}. They show how an account that uses \feat{slash} inheritance could
work, but their actual proposal is \feat{slash}-less. They assume that the tense affixes
are heads of verbal predications, and operate via ``predicate composition'' --- by
inheriting the subcategorisation requirements of the associated verb. The adnominal tense
affixes are special in that a) they are specified as nominal modifiers, and b) they
inherit the subcategorisation requirements of the associated verb, less an NP that is
co-indexed with the modified nominal. (A lexical equivalent of this could be implemented with a lexical rule which removes an
element from a verb's \feat{arg-st} and introduces a \feat{mod} value containing a
nominal with the corresponding index). 
Of course, a \feat{slash}-less account like this
will only deal with cases of local relativization --- where the relativised NP is an
argument of the highest verb. \citeauthor{SiraiGunjiRelative} argue that cases of
non-local relativization, like \pref{x:rc-85}, should be treated as involving
null-pronominals (which are a common feature of \ili{Japanese}). They suggest that the requirement
that the modified noun and the pronoun be co-indexed should be captured via a pragmatic
condition that requires the relative clause be ``about'' the modified noun.
 \begin{exe}\ex\label{x:rc-85}
   \gll [Ken-ga                                  [Eiko-ga     \trace\subscr{i} yon-da] to sinzitei-ru] hon\subscr{i}\\
        \hspaceThis{[}Ken-{\sc nom} \hspaceThis{[}Eiko-{\sc nom} {}        read.\textsc{past} \textsc{comp}
          believe-\textsc{pres} book\\\jambox*{(\ili{Japanese})}
 \glt `the book that Ken believes Eiko read'
 \end{exe}

%% There is extensive discussion of this in \cite[Ch13,283--317]{Kim16SyntacticStrKorean}

\cite{Kim16SyntacticStrKorean} provides a constructional analysis for \ili{Korean} which
resembles \citegen{Sag:97} analysis of \ili{English} --- see also \cite{kim1998head-driven-KoreanRC} and
\cite{kim-yang2003koreanlkb}. He suggests that \ili{Korean} allows verb lexemes
to be realised as ``modifier verbs'' (\istype{v-mod}\type{v-mod}) subject to a constraint along the
lines of \pref{x:rc-86} --- these are verbs that can head a subordinate clause ([\feat{mc}~{--}])
which modifies a nominal (N).\footnote{Different sub-types of \istype{v-mod}\type{v-mod} are associated
  with different tense affixes. \pref{x:rc-86} differs from
  \citeauthor{Kim16SyntacticStrKorean}'s formulation, e.g.\ Kim's formulation involves a
  \feat{pos} (part-of-speech) feature and he assumes that \feat{mod} is list valued
  \citep[see][285]{Kim16SyntacticStrKorean}. This is not important here.}
\begin{exe}\ex\label{x:rc-86}
  \begin{avm}
   \[head & \[\asort{verb} mc & --\\ mod & noun\]\]
   \end{avm}
   \isfeat{head}\isfeat{mc}\isfeat{mod}\istype{noun}
\end{exe}
He   also   proposes   a  construction   (the   \istype{head-relative-mod}\type{head-relative-mod}  construction,
\citealt[see][290]{Kim16SyntacticStrKorean})  to  combine a  structure  headed  by such  a
modifier verb  with a head  nominal, along  the lines of  \pref{x:rc-87}.\footnote{Again, our
  formulation is slightly different from Kim's for  the sake of consistency with the rest of
  our presentation. }
\begin{exe}\ex\label{x:rc-87}
\istype{hd-relative-mod-phrase}\type{hd-relative-mod-phrase}  \(\Rightarrow\)
  \begin{avm}
   \[ head & noun\\
      slash &  \setof{}\\
      hd-dtr & \@2 \upshape{N}\subscr{\idx{1}}\\
      non-hd-dtrs &
      \<
         \upshape{S} \[
            head\|mod \; \@2\\
            slash \;  \setof{\mathrm{NP\subscr{\idx{1}}}}
         \]
      \>
   \]
   \end{avm}
   \isfeat{head}\isfeat{slash}\isfeat{head-daughter}\isfeat{non-head-daughters}\isfeat{mod}
\end{exe}
In words: a nominal structure can consist of a head noun, and a clause headed by a
modifier verb containing an NP gap which is co-indexed with the head noun. The empty
\feat{slash} value on the mother is necessary to prevent the gap being inherited
upwards. The \feat{slash} value on the S daughter ensures the presence of an appropriate gap, and the
\feat{mod} value on the S daughter ensures that it is headed by a verb with the right
morphology. It will license structures like that in Figure~\ref{fig:rc-6}. Kim does not discuss the
semantics, but it would be straightforward to add constraints to this construction along
the lines of those presented above.
\begin{figure}
\begin{forest}  %qtree edges
sm edges
[{N\subtag{1}}  ,baseline
      [{\TnodeDA{S}{\[mod&\@2\\slash&\setof{\@3 \mathrm{NP\subscr{\@1}}}\]}} 
         [{NP}  [ motwu-ka;everyone-\textsc{nom} ] ]
         [{\TnodeDA{VP}{\[mod&\@2\\slash&\setof{\@3}\]}}   
            [{\TnodeDA{S}{\[slash&\setof{\@3}\]}} 
               [{\TnodeDA{NP}{}}  [ Kim-i;Kim-\textsc{nom} ] ]
               [{\TnodeDA{VP}{\[slash&\setof{\@3}\]}}    
                  [{\TnodeDA{V}{\[slash&\setof{\@3}\]}}  [ ilk-ess-ta-ko;read-\textsc{pst-decl-comp} ] ]
               ]
            ]
            [{\TnodeDA{V}{\[mod&\@2\]}}  [ sayngkakha-nun;think-\textsc{mod} ]   ]
         ]
      ]
      [{\idx{2}N\subtag{1}}  [ chayk;book ] ]
      ]
\end{forest}
\caption{A Korean relative clause, based on \cite[295]{Kim16SyntacticStrKorean}}
\label{fig:rc-6}
\end{figure}
\is{relative clause!headed by verb|)}
\is{relative clause!bare!in Korean|)}
\is{relative clause!bare!in Japanese|)}

\subsubsection{Bare relatives in English}
\label{sec:rc-bare-relat-engl}
\is{relative clause!bare!in English|(}


\ili{English} also has bare relative clauses, both finite, as in \pref{x:rc-89}, and non-finite
as in \pref{x:rc-90}:
\begin{exe}\ex\begin{xlist}\label{x:rc-88}
  \ex\label{x:rc-89} the cakes Kim bought \trace
  \ex\label{x:rc-90} some cakes (for Sam) to eat \trace
\end{xlist}\end{exe}
In \ili{English}, there is no obvious motivation for suggesting a special sub-type of ``relative
clause heading'' verb, so an alternative way of licensing noun-modifying clauses with
appropriate \feat{slash} values is required. In \cite{Pollard:Sag:94} this was the role
of an empty relativiser similar to that described above, differing only in taking a single
argument --- a slashed clause (see \citealt[222]{Pollard:Sag:94}; recall that the
relativiser discussed above takes two arguments: a \is{relative
  clause!wh-phrase@\lic{wh}-phrase in}\lic{wh}-phrase, and a slashed
clause). This gives structures like that in Figure~\ref{fig:rc-7}.\footnote{According to
  \cite[222]{Pollard:Sag:94}, the clausal argument of this single argument version of R
  can either be bare, as here, or marked by \lic{that}. Thus, terminological accuracy
  demands the observation that for \citeauthor{Pollard:Sag:94} some instances of
  \lic{that}-relatives are actually ``bare'' in the sense of containing neither a \isi{relative pronoun} nor a complementiser (though others, in particular those involving
  relativisation of a top level subject, are analysed as containing a version of
  \lic{that} which is actually a \isi{relative pronoun}). See above Footnote~\ref{fn:rc-4}.}
\begin{figure}
    \begin{forest}  %qtree edges
   [{\ibar{N}\subtag{1}} , baseline
      [{\idx{7}\ibar{N}\subtag{1}}  [ {cakes} ]  
      ]
      [{RP} 
            [{\TnodeDA{R}{\[mod&\@7\\arg-st&\tuple{\@6}\]}}     [ {\trace} ] ]
            [{\TnodeDA{\idx{6}~S:\idx{3}}{\[slash&\setof{\mathrm{NP\subscr{\idx{1}}}}\]}}  [ {Kim bought} ,roof] ]
      ]
   ]
   \end{forest}
   \caption{A \cite{Pollard:Sag:94}-style structure for an English bare relative}
   \label{fig:rc-7}
\end{figure}

In \cite{Sag:97} the task of licensing such bare relatives is carried out by a construction (an
immediate subtype of \istype{rel-cl}\type{rel-cl}) as in \pref{x:rc-91}. In words: a relative clause can be
a noun-modifying clause whose head daughter contains an NP gap that is co-indexed with the
modified nominal.
\begin{exe}\ex\label{x:rc-91}
  \istype{non-wh-rel-cl}\type{non-wh-rel-cl} \(\Rightarrow\)
  \begin{avm}
   \[ %\asort{rel-cl} 
      head & \[ mod & \[ head & \nbar\subscr{\idx{1}} \]\]\\
      slash & \setof{}\\
      hd-dtr &  \[ slash & \setof{\mathrm{NP\subscr{\idx{1}}}}\]
   \]
   \end{avm}
   \isfeat{head}\isfeat{mod}\isfeat{slash}\isfeat{head-daughter}
\end{exe}
This licenses structures like that in Figure~\ref{fig:rc-8}.\footnote{Sag also proposes a
  subtype of \pref{x:rc-91} to deal with non-finite bare relatives, like \pref{x:rc-92}, which he calls
  \emph{simple infinitival relatives}, cf.\ \istype{simp-inf-rel-cl}\type{simp-inf-rel-cl} in
  Figure~\ref{fig:rc-42}. See \cite[469]{Sag:97}. \cite{AGMS98a} includes discussion of a
  similar construction in French --- `infinitival \lic{\`{a}}-relatives', like \pref{x:rc-93}:
  \begin{exe}
    \ex\label{x:rc-92}book (for Sam) to read
    \ex\label{x:rc-93}\gll un livre \`{a} lire\\
            a book    to   read\\ \jambox*{(\ili{French})}
  \end{exe}
  Neither discussion addresses the
  special modal semantics associated with non-finites, e.g.\ \pref{x:rc-92}
  means something like ``books that Sam can (or should) read''.} 
\begin{figure}
  \newcommand{\SlashDA}[1]{\begin{avm}\[slash&\setof{#1}\]\end{avm}}
  \newcommand{\SlashModDA}{\begin{avm}\[mod&\@2 \nbar\subscr{\idx{1}}\\slash&\setof{}\]\end{avm}}
    \begin{forest}  %qtree edges
   [\ibar{N}    , baseline
      [ {\idx{2}} [ cakes ] ]
      [ {S~\SlashModDA}
         [{NP} [ Kim ] ]
         [ {VP~\SlashDA{\mathrm{NP\subscr{\idx{1}}} }} [ {bought} ,roof]   ]
      ]
   ]
   \end{forest}
   \caption{A \cite{Sag:97}-style structure for an English bare relative}
   \label{fig:rc-8}
   \isfeat{slash}\isfeat{mod}
 \end{figure}
 
This differs from Kim's proposal for \ili{Korean} in which the \feat{slash} value is
bound off: in particular, where Kim's analysis involves a nominal and a slashed S, Sag's
involves a nominal and an \emph{un}slashed S --- the clause is [\feat{slash}~\setof{}], it
is the VP which is [\feat{slash}~\setof{\mathrm{NP}}]. This reflects the fact that in \ili{English} the gap in the
relative clause cannot be the subject, accounting for the contrast
in \pref{x:rc-94}.\footnote{Examples like \pref{x:rc-95} are acceptable in some non-standard dialects of
  \ili{English}. Sag suggests this is not problematic, since they could be analysed as
  \is{relative clause!reduced}reduced relatives
  (see \citealt[471]{Sag:97}), but see immediately below where we cast doubt on this. If we
  are right, then the non-standard dialects would have something like \pref{x:rc-87}
  instead of \pref{x:rc-91}.}
\begin{exe}\ex\begin{xlist}\label{x:rc-94}
  \ex[*]{person spoke to Sam}\label{x:rc-95}
  \ex[]{person who spoke to Sam}\label{x:rc-96}
\end{xlist}\end{exe}

The issue of where upwards termination of \feat{slash} inheritance should occur
highlights the impossibility of having an entirely lexical and non-constructional account
of bare relatives that does not employ empty elements. At first glance, a purely lexical
approach might seem
simple: since all we need is to create clauses specified as [\feat{mod}~\ibar{N}] which contain a
co-indexed gap, all we seem to need is verbs specified as in \pref{x:rc-97}.
\begin{exe}\ex\label{x:rc-97}
    \begin{avm}
   \[ head & 
      \[ \asort{verb}
         mod & \upshape{\ibar{N}\subscr{\idx{1}}}
      \]\\
      comps & \< \upshape{PP} \>\\
      slash & \{ \upshape{ NP\subscr{\idx{1}} } \}
   \]
   \end{avm}
   \isfeat{head}\isfeat{mod}\isfeat{comps}\isfeat{slash}
\end{exe}
In the absence of special constructions or empty elements, this would license structures
like that in Figure~\ref{fig:rc-8}, except that the upward inheritance of the \feat{slash} value will not be
terminated, allowing an additional spurious filler for the gap, as in
\pref{x:rc-98}:\footnote{The \feat{slash} based analysis of \ili{Japanese} relatives outlined in
  \cite{SiraiGunjiRelative} manages to avoid this problem, without either special
  constructions or empty elements, but it is not fully lexical, because it assumes 
  tense affixes combine with the associated lexical verb in the syntax (hence the affix is
  able to block higher inheritance of the gap introduced by the lexical verb).}
\begin{exe}\ex[*]{\label{x:rc-98}
  That book\subscr{i}, I enjoyed [ the book\subscr{i}  Kim read \trace\subscr{i} ]}
\end{exe} 

\is{relative clause!reduced|(}
There is one class of exceptions to this --- that is, phrases which might be analysed as
relative clauses for
which a purely lexical account \emph{is} possible. Examples involving participal phrases
and a variety of other post-nominal modifiers, notably APs and PPs, are often called
\emph{reduced relatives}, and analysed as a type of relative clause.  \cite[471]{Sag:97}
follows this tradition (\istype{red-rel-cl}\type{red-rel-cl} in Figure~\ref{fig:rc-42}). What this comes down to is the assumption that such examples
involve clauses containing predicative phrases with \isi{PRO} subjects, co-indexed with
the nominals they modify. 
\begin{exe}\ex\begin{xlist}\label{x:rc-99}
  \settowidth\jamwidth{(VP-\lic{passive-part})}
  \ex\label{x:rc-100} a person standing by the door \jambox {(VP-\lic{pres-part})}
  \ex\label{x:rc-101} a train recently arrived at platform four \jambox {(VP-\lic{past-part})}
  \ex\label{x:rc-102} a person given a pay rise \jambox {(VP-\lic{passive-part})}
  \ex\label{x:rc-103} a person in the doorway\jambox {(PP)}
  \ex\label{x:rc-104} a person fond of children\jambox {(AP)}
\end{xlist}\end{exe}
It is not obvious to us what is gained by treating these as relative clauses introduced by
a special construction. A lexical account seems at least as appealing, where the relevant
properties of the phrases (e.g.\ noun modifying semantics) are projected directly from
lexical entries for the head words. The reason such a non-constructional approach is
possible is that such examples involve neither \isi{relative pronoun}s nor genuine gaps, so
there are neither \is{relative inheritance}\feat{rel} nor \feat{slash} dependencies to terminate.\footnote{This
  argument does not necessarily carry over to languages which allow relativisation of
  non-subjects in reduced relatives, such as \ili{Arabic}. See \cite[241]{Melnik:06}.
}
This
approach seems particularly appealing in the cases like \pref{x:rc-104}, which would be
analysed as just involving an attributive adjective (\lic{fond}) which happens to take a
complement, along the lines of \pref{x:rc-105}, where \setof{\ldots} stands for the restrictions
the adjective itself imposes. But we think a similar account of verbal
participles and prepositions is equally plausible.\footnote{For example, \cite[159--164]{Mueller2002b} deals with
adjectival passive participles in this way.}
\begin{exe}\ex\label{x:rc-105}
  \begin{avm}
   \[ head & 
      \[ mod & 
         \[\asort{noun}
            index & \@1\\
            restr & \@2
         \]\\
      \]\\
      cont & 
      \[ index & \@1\\
         restr & \@2 $\uplus\ \lbrace \ldots \rbrace$
      \]
   \]
   \end{avm}
   \isfeat{head}\isfeat{mod}\isfeat{index}\isfeat{restr}\isfeat{content}\istype{noun}
\end{exe}
Notice that in \pref{x:rc-105} we omit mention of the \feat{subj}. If we assume the
noun-modifying entry is derived from a predicative entry, there are two obvious
alternatives: a) that the predicative subject is suppressed; or b) that it is constrained
to be unexpressed (i.e.\ \isi{PRO}). In the latter case, the two approaches are very
similar, the only difference being whether examples like those in \pref{x:rc-99} are
classified as clausal. It is not clear whether this has empirical consequences.
\is{relative clause!reduced|)}
\is{relative clause!bare!in English|)}
\is{relative clause!bare|)}
\is{relative clause!construction|)}

\subsection{Non-restrictive (supplemental) relatives}
\label{sec:rc-non-restr-suppl}
\is{relative clause!non-restrictive|(}

The examples of relative clauses considered so far have been \emph{restrictive relatives} (RRCs); they are
interpreted as restricting the denotation of their antecedent to a subset of what
it would be without the relative clause. So-called \emph{supplemental}, \emph{supplementary}, \emph{appositive}, or \emph{non-restrictive}
relatives (NRCs) are different. They do not affect the interpretation of any associated
nominal, and are generally interpreted with wide scope, much like independent
utterances. For example, if \lic{who understand logic} is read as an NRC as in \pref{x:rc-107}
it will be interpreted outside the scope of \lic{Kim thinks}.
\begin{exe}\ex\begin{xlist}\label{x:rc-106}
  \ex\label{x:rc-107} Kim thinks linguists, who understand logic, are clever. \hfill (NRC)
  \ex\label{x:rc-108} Kim thinks linguists who understand logic are clever. \hfill (RRC)
\end{xlist}\end{exe}
NRCs are often set off intonationally, and are subject to a number of surface
morphosyntactic restrictions in \ili{English}.  In particular, they must be finite and contain a
\lic{wh}-pronoun, witness the ungrammaticality of \pref{x:rc-109} and
\pref{x:rc-110}.\footnote{More extensive discussion of differences between NRCs and RRCs can
  be found in \cite{Arnold07}.}
\begin{exe}\ex\begin{xlist}
  \ex[*]{\label{x:rc-109} Kim, for Sandy to speak to, will arrive later.}
  \ex[*]{ \label{x:rc-110} Kim, (that) Sandy spoke to, will arrive later.}
\end{xlist}\end{exe}

The analysis of non-restrictives has attracted some attention in the HPSG
literature.\footnote{\cite{BlbieLaurens09} discuss what they call \emph{verbless relative
  adjuncts}, such as \pref{x:rc-111}, in \ili{French} and \ili{Romanian}:
\begin{exe}\ex\label{x:rc-111}
  \gll Trois personnes, [parmi lesquelles Jean], sont venues.\\
  three people({\sc fem}) \hspaceThis{[}among which.{\sc fem} John {\sc aux} come\\\jambox*{(\ili{French})}
  \glt `Three people, among which John, have come.’ 
\end{exe}
These have non-restrictive semantics, and some similarities with relative clauses, but
\citeauthor{BlbieLaurens09} point out significant differences, and argue for an analysis that
treats them rather differently, as a distinct construction.}

Where RRCs are typically nominal modifiers, NRCs are compatible with a wide range of
\is{relative clause!antecedent of}antecedents. \cite{Holler:03} provides an analysis of \ili{German} non-restrictives which are adjoined to S, as
in \pref{x:rc-112}. Her account uses a version of the empty relativiser from \cite{Pollard:Sag:94} whose
\feat{mod} value specifies a clausal (rather than nominal) target for modification, and
looks for an appropriate \is{relative clause!antecedent of}antecedent for its first
argument (the \is{relative clause!wh-phrase@\lic{wh}-phrase in}\lic{wh}-phrase) among
the discourse referents contributed by the modification target (for example, the discourse
referent corresponding to the proposition expressed by the main clause in \pref{x:rc-112}).
The \isi{relative pronoun} is thus treated rather like a normal pronoun.
\is{relative clause!empty relativiser}
\begin{exe}\ex\label{x:rc-112}
 \gll Anna gewann die Schachpartie, was Peter \"{a}rgerte.\\
      Anna won the {game of chess} which Peter annoyed \\\jambox*{(\ili{German})}
\glt `Anna won the game of chess, which annoyed Peter.' 
\end{exe}

\cite{Arnold:04a} provides an analysis of \ili{English} non-restrictives of all kinds. This
analysis also takes the \isi{relative pronoun}s involved in NRCs to be much like normal
pronouns, but accounts for the syntactic restrictions by making minor modifications to
constructions given in \citegen{Sag:97} analysis of restrictives. It assumes a uniform
syntax for restrictives and NRCs, but provides a way for relative clauses to combine with
their heads in two semantically distinct ways, either restrictively (in the normal way) or
non-restrictively (making their semantic contribution at the same level as the root
clause, accounting for the wide-scope interpretation). The fact that supplementary
relatives are required to be finite and contain a \lic{wh}-pronoun can then be simply
stated (e.g.\ non-restrictive semantics entails a non-head daughter which is a
\istype{fin-wh-rel-cl}\type{fin-wh-rel-cl}).\footnote{As stated, given \citegen{Sag:97}  assumption that
  \lic{that}-relatives are a variety of \lic{wh}-relative, this wrongly predicts that
  supplemental \lic{that}-relatives should normally be allowed. One way around this is to
  adopt a different analysis of \lic{that}, but \cite{Arnold:04a} also considers an analysis whereby
  \lic{that} has a different kind of \feat{rel} value from ``real'' \isi{relative pronoun}s.}
Likewise, the wider range of \is{relative clause!antecedent of}antecedents available to NRCs can be captured by relaxing the
[\feat{mod}~\istype{noun}\type{noun}] constraint associated with \istype{rel-cl}\type{rel-cl} (so in principle all
kinds of relative clause are compatible with any \is{relative clause!antecedent of}antecedent), and adding it as requirement
associated with restrictive semantics.
 
The approach to NRCs developed in \cite{Arnold:04a} is \emph{syntactically integrated} --- NRCs
are treated as normal parts of the syntactic structure on a par with restrictive
relatives. 
On the face of it, examples like \pref{x:rc-114} are problematic for such an
approach:
\begin{exe}\ex\begin{xlist}
  \ex\label{x:rc-113} What did Jo think?
  \ex\label{x:rc-114} You should say nothing, which is regrettable.
\end{xlist}\end{exe}
When uttered in the context provided by \pref{x:rc-113}, the interpretation of \pref{x:rc-114}
is that it is regrettable that \emph{Jo thinks} you should say nothing. This has been
taken as an indication that the interpretation of NRCs requires \is{relative clause!antecedent of}antecedents that are not
syntactically realised and only available at a level of conceptual structure
\citep[see][]{Blakemore06Divisions}.  However, \cite{Arnold:Borsley:08} show that this is
incorrect, and in fact a syntactically integrated account combined with the approach to
\isi{ellipsis} and \is{fragmentary utterance}fragmentary utterances of \cite{Ginzburg:Sag:00} makes precisely the right
predictions in this case and in a range of others.

\cite{Arnold:Borsley:10} look at NRCs where the \is{relative clause!antecedent of}antecedent is a VP, and where the gap is the
complement of an auxiliary, as in \pref{x:rc-115}.
\begin{exe}\ex
  \label{x:rc-115}Kim has ridden a camel, which Sam never would \trace.   
\end{exe}
This is unexpected, because such examples seem to involve an NP filler (\lic{which}) being
associated with a gap in a position where an NP is generally impossible, cf.\ \lic{*Sam
  never would that activity}. \citeauthor{Arnold:Borsley:10} consider a number of
analyses, including an analysis which treats \lic{which} as a potential VP, and an
analysis which introduces a special \is{relative clause!construction}relative clause construction. However, they argue that
the best analysis is one which relates examples like \pref{x:rc-115} to cases of VP \isi{ellipsis}
(as in \lic{Kim has ridden a camel but Sam never would}), which involve the VP argument of
an auxiliary verb being omitted from its \comps list. The idea is that auxiliary
verbs allow such an elided VP argument to have (optionally) a \feat{slash} value that
contains an appropriately co-indexed NP. If such a \feat{slash} value is present, normal
\feat{slash} amalgamation and inheritance will yield \pref{x:rc-115} as a normal relative
clause, without further stipulation.
  
NRCs normally follow their \is{relative clause!antecedent of}antecedents. However, as \cite{Goldman12Supplemental} observes,
there are some special cases where the NRC precedes the \is{relative clause!antecedent of}antecedent. Such cases involve the
\isi{relative pronoun}s \lic{which} and \lic{what} with \is{relative clause!antecedent of}antecedents that have clausal interpretations,
i.e.\ either actual clauses, as in \pref{x:rc-117} and \pref{x:rc-118}, or other expressions interpreted
elliptically as with \lic{later} in \pref{x:rc-119}.
\begin{exe}\ex\begin{xlist}\label{x:rc-116}
  \ex\label{x:rc-117} It may happen now, or --- \emph{which would be worse} --- it may
  happen later.
  \ex\label{x:rc-118}  It may happen now, or --- \emph{which would be worse} 
  --- later.
  \ex\label{x:rc-119} It may happen now.  \emph{What is worse}, it may happen later.
\end{xlist}\end{exe}
\citeauthor{Goldman12Supplemental} provides a constructional account. It makes use of a
feature \feat{relzr}, introduced by \cite{Sag:10b}, which is shared between a relative
clause and its filler daughter, and whose value reflects the identity of the \isi{relative pronoun} (so possible values include \lic{which}, \lic{what}, etc.). Cases like
\pref{x:rc-119} are dealt with simply by means of a special \is{relative clause!construction}construction which combines a
\lic{what}-relative clause with its \is{relative clause!antecedent of}antecedent in the desired order. The account of cases
like \pref{x:rc-117} and \pref{x:rc-119} makes use of the idea of constituent order domains for
\isi{linearisation} originally proposed by \citeauthor{Reape94a} (e.g.\
\citealt{Reape94a}, and \crossrefchapteralt{order}). The relevant construction
combines a phrase whose \feat{relzr} value is \istype{which}\type{which} (e.g.\ \lic{which would be
  worse}) with a clause whose constituent order \feat{domain} has a coordinator as its
first element (e.g.\ the \feat{domain} associated with \lic{or it may happen later}) and
produces a phrase where the \feat{domain} value of the \lic{which} phrase appears after
the coordinator and before the remainder of the clause, giving the desired
result.\footnote{\citeauthor{Goldman12Supplemental} handles the wide scope interpretation
  of NRCs by implementing a multidimensional notion of \feat{content} inspired by
  \cite{Potts05Logic}. He also extends the analysis described here to deal with cases of
  \lic{as}-parentheticals (e.g.\ \lic{As most of you are aware, we have been under severe
    stress lately}), arguing that \lic{as} should be analysed as a relativiser, and that
  such clauses should be analysed as relative clauses.}
\is{relative clause!non-restrictive|)}
 
\section{Other functions, other issues}
\label{sec:rc-other-functions-other-issues}
For reasons of space, we have so far restricted the notion \emph{relative clause} to the
typical case: clauses which are nominal modifiers, adjoined to nominals. This ignores a
number of relevant phenomena, notably the fact that relative clauses
are not necessarily nominal modifiers, and the possibility that even when they function as
nominal modifiers they need not be adjoined to nominals. In this section we will provide
some discussion of these issues. Section~\ref{sec:rc-extraposition} will briefly review
HPSG analyses of cases where relative clauses are not adjoined to nominals. Section
\ref{sec:rc-other-functions} will overview HPSG approaches to cases where clauses resembling relative
clauses are not nominal modifiers.\footnote{Among the other phenomena we have neglected,
  one should mention \emph{amount} relatives (e.g.\ \citealt{grosu2016amount}), that is, relative clauses where what is
  modified semantically is not a nominal, but an \emph{amount} related to the nominal, as
  for example in \pref{x:rc-120} where the relative clause clause gives information about the \emph{amount}
  of wine, rather than the wine itself.
  \begin{exe}\ex\label{x:rc-120}
    It would take me a year to drink the wine [that Kim drinks on a normal night].  
  \end{exe}
}

\subsection{Extraposition}
\label{sec:rc-extraposition}
\is{relative clause!extraposition of|(}

As noted above, relative clauses are typically nominal modifiers, and typically adjoined
to the nominals they modify. However, this is not invariably the case: under certain
circumstances relative clauses can be \lic{extraposed}, as in \pref{x:rc-121}, where the
relative clauses (emphasised) have been extraposed from the subject NP to the end of the
clause.
\begin{exe}\ex\begin{xlist}\label{x:rc-121}
  \ex\label{x:rc-122} Someone might win \emph{who does not deserve it}.
  \ex\label{x:rc-123} Something happened then \emph{(that) I can't really talk about here}.
  \ex\label{x:rc-124} Something may arise \emph{for us to talk about}.
\end{xlist}\end{exe}
Several different approaches to extraposition have been proposed in the HPSG literature.

One approach uses the idea of constituent order domains, mentioned briefly in
Section~\ref{sec:rc-non-restr-suppl} above (and see \crossrefchapteralt{order}). The idea is
that an extraposed relative clause is composed with its \is{relative clause!antecedent
  of}antecedent nominal in the normal way as regards syntax and semantics, but that rather
than being \emph{compacted} into a single \feat{domain} element, the nominal and the
relative clause remain as separate \feat{domain} elements, with the effect that that
relative clause can be \emph{liberated} away from the nominal, so that its phonology is
contributed discontinuously from the phonology of the nominal, as in the examples in
\pref{x:rc-121}.  See e.g.\ \cite{Nerbonne94a} and \cite{Kathol:Pollard:95}
for details.

A second approach treats extraposition as involving a non-local dependency, introducing a
non-local feature, typically called something like \feat{extra}, which functions much
like other non-local features (e.g.\ \feat{slash}).  The idea is that a relative
clause can make its semantic contribution as a nominal modifier ``downstairs'', but rather
than being realised as a syntactic \textsc{daughter} (sister to the nominal), the relevant
properties (e.g.\ the \feat{local} features) are added to the \feat{extra} list of the
head, and inherited up the tree until they are discharged from the \feat{extra} list by
the appearance of an appropriate phrase-final daughter constituent, which contributes its
phonology in the normal way, but makes no semantic contribution. Thinking from the top
downwards, this is equivalent to having a construction which allows a relative clause to
appear e.g.\ as sister to a VP (as in \pref{x:rc-122}) without affecting the VP's syntax or
semantics, so long as it is pushed onto the \feat{extra} list of the VP, from where it
will be inherited downwards until a nominal occurs which it can be interpreted as modifying (the
apparatus needed to deal with the ``bottom'' of the dependency might be a family of lexical
items derived by lexical rule, or a non-branching construction). See e.g.\
\cite{Keller:95c}, \cite{Bouma96},  \cite{Mueller99b}, \cite{Mueller2004b}, \cite{Crysmann2005a-u},
and \cite{Crysmann2013a}. 

\is{Minimal Recursion Semantics|(}
\is{Lexical Resource Semantics|(}
A third approach is suggested in \cite{Kiss2005a}, and adopted in \cite{Crysmann2004a} and
\cite{Walker2017}. This
approach exploits the more flexible approach to semantic composition provided by \isi{Minimal Recursion Semantics} (MRS, \citealt{CFPS2005a}), in the case of \cite{Kiss2005a}, and \isi{Lexical Resource Semantics} (LRS, \citealt{richtersailer-lrs04}) in \cite{Walker2017}. The idea is
that an extraposed relative clause appears as a normal syntactic daughter in its surface
position, but the notion of semantic modification is generalised so that rather than the
index of a modifying phrase being identified with that of a sister constituent (as
standardly assumed), it may be identified with that of any suitable constituent
\emph{within} the sister. That is, adjuncts can be interpreted as modifying not just their sisters, but
anything \emph{contained in} their sisters --- words and phrase to which they have no
direct syntactic connection.  This is implemented by means of a set valued
\feat{anchors} feature, which is inherited upwards in the manner of a non-local
feature, and which allows access to the indices of constituents from lower down. The
flexibility of semantic composition afforded by MRS and LRS means that the right
interpretations can be obtained.
\is{Minimal Recursion Semantics|)}
\is{Lexical Resource Semantics|)}

A number of authors have argued for the superiority of an approach using
\feat{extra}-style apparatus (e.g.\ \citealt{Mueller2004b} and \citealt{Crysmann2013a}), but in
terms of theoretical costs and benefits there seems to be little to choose between these
alternatives --- the first and third approaches rely on particular approaches to constituent order
and semantic composition, while \feat{extra}-style analyses involve only the more
commonplace apparatus of non-local features (though with the added cost of special
constructions or lexical operations to introduce and remove elements from \feat{extra}
lists). Empirically, there are several issues that all approaches deal with more or less
successfully (for example, the Right Roof Constraint from \citealt{Ross67a-Eng} that prevents
extraposition beyond the clause, cf.\ \pref{x:rc-126}). However, a more significant factor may
be how well different accounts integrate with analyses of extraposition involving other kinds of
adjunct and complement (e.g.\ complement clauses, as in \pref{x:rc-127}), capturing similarities
and differences (see e.g.\ \citealt{Crysmann2013a}).
\begin{exe}\ex\begin{xlist}
  \ex[]{[That someone might win \emph{who does not deserve it}] is irrelevant.}\label{x:rc-125}
  \ex[*]{[That someone might win] is irrelevant \emph{who does not deserve it}.}\label{x:rc-126}
\end{xlist}\end{exe}
\begin{exe}\ex{\label{x:rc-127}
  The question then arises \emph{whether we should continue in this way}.}
\end{exe}
\is{relative clause!extraposition of|)} 
  
\subsection{Other functions}
\label{sec:rc-other-functions}

In this section we will briefly discuss phenomena involving clauses whose internal
structures resemble relative clauses but which do not function as nominal
modifers.\footnote{One omission here is discussion of \emph{\is{relative clause!relative-corelative}relative-corelative} constructions,
  which can be found in \ili{Hindi} and \ili{Marathi}, \emph{inter alia}, and which were given an
  analysis in \cite[227--232]{Pollard:Sag:94}. These involve the paratactic combination of
  a clause that contains one or more \isi{relative pronoun}s, and what looks like a main clause
  containing coreferential pronouns, something like `which boy\subscr{i} saw which girl\subscr{j}, he\subscr{i} proposed
  to her\subscr{j}' (meaning \lic{the boy who saw the girl proposed to
    her}). \citeauthor{Pollard:Sag:94}'s analysis involves associating a set of
  indices in the \feat{rel} value of the first clause, which are realised by \isi{relative pronoun}s in the
  normal way, and an identical set of indices as encoded as the value of a
  \feat{correlative} feature in the main clause, which
  are realised by normal pronouns.}

\subsubsection{Complement clauses}
\label{sec:rc-complement-clauses}

Perhaps the most obvious cases of this kind involve clauses with the internal structure of
a relative clause which occur as complements, rather than adjuncts. The following are some
examples.\footnote{Another case where a relative clause should be analysed as a complement
  is discussed in \cite{Arnold:Lucas:16}.}
\begin{exe}\ex\begin{xlist}
\ex\label{x:rc-128} This story is the *(most) interesting \emph{that we have heard}.
\ex\label{x:rc-129}
\gll diejenige Frau *(\emph{die dort steht})\\ 
    the.that woman \hspaceThis{*(}{who there stands}\\\jambox*{(\ili{German})}
    \glt `the very woman who is standing there'
\ex\label{x:rc-130} It was Kim \emph{that solved the problem}.
\ex\label{x:rc-131} It was from Kim \emph{that we got the news}. 
\ex\label{x:rc-132} \gll On l' a  vu \emph{qui} \emph{s'enfuyait}\\
                      We him have seen who run.away.{\sc imperf}\\\jambox*{(\ili{French})}
                      \glt`We saw him running away'
% \ex\label{x:rc-133} \gll Le prof a \'{e}t\'{e} vu \emph{qui fumait}\\
%                       The professor has been seen who smoke-{\sc imperf}\\\jambox*{ (French)}
%                       \glt`The professor was seen smoking'
\end{xlist}\end{exe}

In \pref{x:rc-128} we have what looks like a \lic{that} relative which is plausibly analysed as
the complement of the superlative (notice that omitting the superlative makes \pref{x:rc-128}
ungrammatical).\is{relative clause!as complement!of superlative adjective}

The \ili{German} example in \pref{x:rc-129} exemplifies the \lic{diejenigen} class of determiners,
which require a complement that looks like a relative clause (and is analysed as such in
\citealt{Walker2017}).\is{relative clause!as complement!of \lic{diejenigen}}%

\is{relative clause!as complement!in cleft construction|(}
In \pref{x:rc-130} we have a so-called \lic{it}-cleft, a construction which features a clause
resembling a relative clause, but rather than adding information about an associated
nominal (as it would if it were a normal relative clause), the clause is interpreted as
providing a presupposition (``someone/something solved the problem''), for an associated
focus phrase (here the nominal \lic{Kim}, so the interpretation is roughly ``\ldots{} and
that person/thing was Kim''). Notice that the focus phrase need not be nominal (e.g.\ in
\pref{x:rc-131} it is a PP \lic{from Kim}), again this is unlike normal (restrictive)
relatives clauses (which are nominal modifiers).\footnote{Notice also that
  \lic{that}-relatives are usually incompatible with proper name \is{relative clause!antecedent of}antecedents, but proper
  names are perfectly acceptable as the focus of an \lic{it}-cleft with a
  \lic{that}-clause, as in \pref{x:rc-130} \citep[1416-1417]{HP2002a-ed}.}  In HPSG,
following \cite[260--262]{Pollard:Sag:94}, \lic{it}-clefts have typically been analysed as
involving a lexical entry for \lic{be} that takes an \lic{it} subject, and two
complements: an XP and an S which is marked as containing an XP gap. This makes
\lic{it}-clefts look rather different from relative clauses (the only real similarity
being the existence of an unbounded dependency). One problem is that it is not clear how this
approach can be extended to examples like \pref{x:rc-134}, where we seem to have an NP focus
(\lic{Sam}) which is not directly associated with an XP gap --- we have instead a PP gap
that seems to be associated with a normal relative phrase filler (\lic{on whom}),
i.e.\ where the similarity of the clefted clause to a relative clause is quite
strong. It is not obvious how this problem should be dealt with.
\begin{exe}\ex\label{x:rc-134}
 It was Sam [on whom she particularly focused her attention \trace].
\end{exe}
\is{relative clause!as complement!in cleft construction|)}
%% This might suggest that a better analysis could be produced if the relationship of
%% relative clauses and the clausal elements of \lic{it}-clefts were taken more
%% seriously. Unfortunately, it is not immediately obvious how to do this

\is{relative clause!predicative|(}
The \ili{French} example in \pref{x:rc-132} contains a so-called \emph{predicative
  relative clause} (PRC).\footnote{The \ili{French} term is \emph{proposition relative dépendante attribut} \citep{Sandfeld65Syntaxe}.} Such clauses have the superficial form of a finite relative
clause, but differ from them syntactically, semantically, and pragmatically.
\cite{KL99a-u} analyse them as a form of \is{secondary predicate}\emph{secondary predicate} (cf.\
\lic{running away} in \ili{English} \lic{We saw them running away}). Syntactically, they are
restricted to post-verbal positions, and are only permitted with certain kinds of verb
(notably verbs of perception, like \lic{voir} `see', and discovery, like \lic{trouver}
`find'), and the \isi{relative pronoun} must be a top level subject. Semantically, they are
subject to constraints on tense, modality, and negation (there must be temporal overlap
between the perception/discovery event and the event reported in the relative clause, and
the relative clause content cannot be either modal or negative). Pragmatically, their
content must be asserted (rather than presupposed). \citeauthor{KL99a-u} provide an
analysis which treats PRCs as \feat{rel} marked clauses with both an internal and an
external subject (instances of \istype{head-subject-phrase}\type{head-subject-ph} which have a non-empty
\feat{subj} value), and which can consequently function as secondary
predicates.
\is{relative clause!predicative|)}

\subsubsection{Dependent noun and pseudo-relative constructions}
\label{sec:rc-depend-noun-pseudo}
\is{relative clause!as complement!in dependent noun construction|(}

The following exemplifies a \ili{Korean} structure that contains what looks superficially like a relative clause:
\begin{exe}\ex
  \gll Kim-u       [[sakwa-ka          cayngpan-wi-ey                 iss-nun] kes]-ul mek-ess-ta.\\
  Kim-\textsc{top} \hspaceThis{[[}apple-\textsc{nom} tray-\textsc{top}-\textsc{loc} exist-\textsc{mod}
  \textsc{kes-acc}  eat-\textsc{pst}-\textsc{decl}\\\jambox*{(\ili{Korean})}
  \glt `Kim ate an apple which was on the tray.' 
\end{exe}
Here what is traditionally called a \emph{dependent noun} (\lic{kes}) is preceded by a clause
whose verb bears the morphological marking that is characteristic of relative clauses (the
-\lic{nun} affix).\footnote{\ili{Japanese} has a similar construction, 
 involving the nominalising particle \lic{no}, which has received some attention in the
 HPSG literature (e.g.\ \citealt{kikuta1998multiple};
 \citeyear{kikuta2001japanese}; \citeyear{Kikuta2002a-u}). A difference is that there is no special
 morphology on the clause in \ili{Japanese}, as noted above, in Section~\ref{sec:rc-bare-relat-japan}.
}

However, unlike a normal relative clause, this ``dependent'' clause does not contain a gap,
instead it contains what might be regarded as the semantic head of the construction (in
this case, \lic{sakwa-ka} `apple'), notice that the clause+\lic{kes} constituent satisfies
the selection restriction of the verb \lic{mek-ess-ta} `ate'; this is what motivates the
translation and explains why such clauses are often regarded as ``\is{relative clause!internally headed}internally headed''
relatives.  \cite[303--317]{Kim16SyntacticStrKorean} notes a number of differences
between \lic{kes}-clauses and normal relatives (e.g.\ \lic{kes}-clauses do not allow the
full range of relative affixes to appear), and suggests these clauses are better analysed
as complements of \lic{kes}. See also \cite{kim1996internally}, \cite{Chan:Kim:03}, \cite{kim2016copular}, and
references there.\footnote{\cite[232--236]{Pollard:Sag:94} discuss a number of cases of what
  appear to be more plausible instances of \is{relative clause!internally headed}\emph{internally headed} relatives from a number of
  languages (\ili{Lakhota}, \ili{Dogon}, and \ili{Quechua}); the following is from \ili{Dogon}:% \pref{x:rc-135}
\begin{exe}\ex\label{x:rc-135}
  \gll [ya indɛ mi wɛ gɔ]  yimaa   boli.\\
       \hspaceThis{[}yesterday person 1sg see.PN.$\emptyset$ DEF die.PSP go.PN.3sg\\\jambox*{(\ili{Dogon})}
  \glt `The person I saw yesterday is dead.' 
\end{exe}
Here we have a determiner \lic{gɔ} preceded by a clause containing what would be
the external head of a standard relative clause (in this case \lic{indɛ} `person'). The key
difference between this and the \ili{Korean} case is the absence here of any obvious clause-external
nominal like \lic{kes} which can be treated as the head which takes the relative
clause as a complement.  \cite[234]{Pollard:Sag:94} suggest (following \citealt{Culy:90}) that NPs
like that in \pref{x:rc-135} involve an exocentric construction, but no empty elements
(neither an empty nominal, nor an empty relativiser). The NP consists of a determiner and
a nominal, where the nominal consists of just a clause whose \feat{rel} value contains
the index of the nominal. This \feat{rel} value is \is{relative inheritance}inherited downwards into the clause where it
is identified with the index of one of the NPs, here the index of \lic{indɛ}
`person': the effect of this is that the index of \lic{indɛ} `person' becomes
the index of the whole NP. (This ignores a number of technical and empirical issues to do
with the inheritance and binding-off of \feat{rel} values.)}
\is{relative clause!as complement!in dependent noun construction|)}

\is{relative clause!as complement!in pseudo-relative construction|(}
Another \ili{Korean} structure that has some similarity with relative clauses is the so-called
\emph{pseudo-relative} construction, exemplified in \pref{x:rc-136}.\footnote{A similar
  construction can be found in \ili{Japanese}, \citep[cf.][]{kikuta1998multiple,kikuta2001japanese,Kikuta2002a-u,Chan:Kim:03}.}
\begin{exe}\ex\label{x:rc-136}
 \gll [komwu-ka tha-nun] naymsay\\
      \hspaceThis{[}rubber-\textsc{nom} burn-\textsc{mod} smell\\\jambox*{(\ili{Korean})}
      \glt `the smell that characterises the burning of rubber' 
\end{exe}
There is again no gap in the relative clause; again, only one kind of marking is allowed
on the verb (only past tense \lic{-un}); and only a limited range of nouns allow this kind
of relative clause; this makes them rather like complement clauses.  However, it is less
plausible to think of a noun like \lic{naymsay} `smell' taking a complement (unlike
\lic{kes}), and these clauses are like prototypical relative clauses in not allowing topic
marking. Kim suggests this is
a special construction where the relation of head noun and relative clause is that the
noun describes the perceptive result of the situation described by the clause (e.g.\ the
smell is the perceptive result of the rubber burning). See 
\cite{kim1998pseudoRC}, \cite{Yoon:93}, \cite{Chan:Kim:03}, \cite{Cha:05}, and \cite{Kim16SyntacticStrKorean}.
\is{relative clause!as complement!in pseudo-relative construction|)}

\subsubsection{Free relatives}
\label{sec:rc-free-relatives}
\is{relative clause!free|(}

Perhaps the most significant case of a clause type that resembles a relative clause but
which does not function as a nominal modifier consists of the so-called \emph{free}
(\emph{headless}, or \emph{fused}) \emph{relatives}, exemplified in \pref{x:rc-137}. These
have received considerable attention in the HPSG literature.
\begin{exe}\ex\begin{xlist}\label{x:rc-137}
    \ex\label{x:rc-138} She ate \emph{what I suggested}.
    \ex\label{x:rc-139} She ate \emph{whatever I suggested}.
    \ex\label{x:rc-140} She put it \emph{where I suggested}.
\end{xlist}\end{exe}
As these examples suggest, free relatives can be interpreted as involving either definite
descriptions, as in \pref{x:rc-138} ``the thing that I suggested'', or universal
quantification, as in \pref{x:rc-139} ``everything that I suggested''. They can also have
adverbial or prepositional interpretations, as in \pref{x:rc-140} ``in the place that I
suggested''. The interpretation is related to the choice of \is{relative
  clause!wh-phrase@\lic{wh}-phrase in}\lic{wh}-phrase.   There are some special
restrictions.  For example, in \ili{English} free relatives must be finite, as can be seen
from \pref{x:rc-142}, and there are restrictions on what \lic{wh}-words are allowed (e.g.\
\lic{what} is permitted, as in \pref{x:rc-138}, but \lic{which} is not, witness \pref{x:rc-143}).
\begin{exe}\ex\begin{xlist}\label{x:rc-141}
  \ex[*]{\label{x:rc-142} She ate \emph{what to cook}.}
  \ex[*]{\label{x:rc-143} She ate \emph{which I suggested}.}
\end{xlist}\end{exe}

Free relatives resemble prototypical \lic{wh}-relatives (and interrogative clauses) in containing a
gap, and an initial \is{relative clause!wh-phrase@\lic{wh}-phrase in}\lic{wh}-phrase which is interpreted as filling the gap.  They
differ from interrogatives in having  the external
distribution of NPs or PPs rather than clauses (for example in \pref{x:rc-138} \lic{what I
  suggested} is the complement of \lic{eat}, and in \pref{x:rc-140} \lic{where I suggested}
is a complement of \lic{put}, neither of which allow clausal complements). They differ
from prototypical relative clauses in not being associated with a nominal \is{relative clause!antecedent of}antecedent. They can
contain \isi{relative pronoun}s which are not permitted in normal \lic{wh}-relatives, notably the
\lic{-ever} pronouns, \lic{whatever}, \lic{whoever}, etc., and \lic{what}, witness the
ungrammaticality of the following:\footnote{\lic{What} is not a \isi{relative pronoun} in
  standard \ili{English}, but it is in some other varieties, and \pref{x:rc-146} is grammatical in
  those.}
\begin{exe}\ex\begin{xlist}\label{x:rc-144}
  \ex[*]{\label{x:rc-145} She ate the thing(s) \emph{whatever I suggested}.}
  \ex[*]{\label{x:rc-146} She ate the things(s) \emph{what I suggested}.}
\end{xlist}\end{exe}
In general the possibilities of \isi{relative inheritance} (pied-piping) in free relatives are
dramatically reduced compared to prototypical relatives and interrogatives. For example in \ili{English}, 
\isi{relative inheritance} is not possible from the complement of a preposition, as can be seen
from \pref{x:rc-148}:
\begin{exe}\ex\begin{xlist}
  \ex[]{\label{x:rc-147} Try to describe \emph{what you talked about}.}
  \ex[*]{\label{x:rc-148} Try to describe \emph{about what you talked}.}
\end{xlist}\end{exe}
In fact, in \ili{English} \isi{relative inheritance}  only seems to be possible from \is{relative clause!wh-phrase@\lic{wh}-phrase in}\lic{wh}-phrases in in pre-nominal position
(determiners and genitive NPs), as in \pref{x:rc-149}, and \pref{x:rc-155} below.\footnote{Other
  languages are less restrictive, e.g.\ \cite[57]{Mueller99b} gives \ili{German} examples
  analogous to \pref{x:rc-148}. See Footnote~\ref{fn:rc-5}.}
\begin{exe}\ex
  \label{x:rc-149} They will steal \emph{what(ever) things they can carry}.
\end{exe}

As with prototypical relatives, the initial \is{relative clause!wh-phrase@\lic{wh}-phrase in}\lic{wh}-phrase in a free relative has to
satisfy restrictions imposed ``downstairs'' in the relative clause (i.e.\ restrictions that
follow from the location of the gap). In addition, however, it seems that with free
relatives the \is{relative clause!wh-phrase@\lic{wh}-phrase in}\lic{wh}-phrase is also sensitive to restrictions imposed from the
outside the relative clause --- the \is{relative clause!wh-phrase@\lic{wh}-phrase in}\lic{wh}-phrase of a free relative has to be of the
appropriate category for the position where the free relative appears. For example, as a
first approximation, a free relative with \lic{what} is only possible where an NP is
possible, and a free relative with \lic{where} is only possible where a locative PP is
possible. This is the so-called \emph{matching effect} in free relatives.\footnote{In
  fact, things are more complicated. For example, in \lic{He walked to [where his horse
    was waiting].} we have a free relative with \lic{where} in an NP position (object of a
  preposition) rather than a PP position. See e.g.\ \cite{KimMixed2017} for discussion.}

One interesting instance of this involves case marking. Consider, for example, the \ili{German} data in
\pref{x:rc-150}. These show a free relative in a position which requires nominative case
marking, containing a \isi{relative pronoun} whose role within the relative clause requires
nominative marking. Since \lic{wer} `who' is nominative, all is well. By contrast, in
\pref{x:rc-152} while the nominative \lic{wer} satisfies the requirements within the relative
clause, there is a case conflict because the free relative as a whole is the complement of
a verb \lic{vertrauen} `trust' that requires a dative complement. The result is
ungrammatical. Examples like \pref{x:rc-153} show a complication. Here again there is a case
conflict: within the relative clause, the \isi{relative pronoun} is required to be accusative
(complement of \lic{empfehlen} `recommend'), and the free relative as a whole is in a
nominative position. However, the result is grammatical, presumably because the
morphological form of the neuter \isi{relative pronoun} \lic{was} `what' can realise either
nominative or accusative case (unlike the masculine \lic{wer}).
\begin{exe}\ex\begin{xlist}\label{x:rc-150}
  \ex[]{
       \gll Wer schwach ist, muss klug sein. \\
            who.{\sc nom} weak is must clever be\\\jambox*{(\ili{German})}
        \glt `Whoever is weak must be clever.'}\label{x:rc-151}
        \ex[*]{
        \gll Wer klug ist, vertraue ich immer.\\
             who.{\sc nom} clever is trust I ever\\\jambox*{(\ili{German})}
        \glt intended: `I trust whoever is clever.'}\label{x:rc-152} 
        \ex[]{
        \gll Was du mir empfiehlst, macht einen guten Eindruck.\\
        what.{\sc nom/acc} you me recommend makes a good impression\\\jambox*{(\ili{German})}
        \glt `What you recommend me makes a good impression.'} \label{x:rc-153}
\end{xlist}\end{exe}  

The agreement properties of free relatives are somewhat surprising, and reveal a potential
complication in the matching effect. Notice that in \pref{x:rc-155} the \is{relative clause!wh-phrase@\lic{wh}-phrase in}\lic{wh}-phrase,
\lic{whoever's dogs}, is plural, and triggers plural agreement on the verb in relative
clause.
\begin{exe}\ex\label{x:rc-154}
  \begin{xlist}
    \ex\label{x:rc-155} {}[[Whoever's\subscr{sg} dogs]\subscr{pl} are running around]\subscr{sg} is in trouble.
    \ex\label{x:rc-156} Whoever is/*are running around (is in trouble).
  \end{xlist}
\end{exe}
This is not surprising since \lic{whoever's dogs} is headed by a plural noun
(\lic{dogs}). However, the free relative as a whole triggers singular agreement,
consistent with the agreement properties coming from the \isi{relative pronoun} ---
\lic{whoever} is singular, as can be seen from \pref{x:rc-156}. This is also consistent with
the semantics: the free relative in \pref{x:rc-155} denotes the person whose dogs are running
around, not the dogs (in this it resembles an NP like \lic{anyone whose dogs are running
  around}, which involves a normal relative clause construction).\footnote{This is not a
  universal property: \cite{Borsley:08} notes that examples in \ili{Welsh} resembling
  \pref{x:rc-155} are interpreted as meaning that the dogs are in big trouble, not the owner.}
This shows a complication of the matching effect: it seems that within-clause requirements
are reflected on the initial \is{relative clause!wh-phrase@\lic{wh}-phrase in}\lic{wh}-phrase (\lic{whoever's dogs} is the subject of the
relative), but the external distribution reflects the properties of the relative
\emph{word} (\lic{whoever}). Of course, the fact that \isi{relative inheritance} is so limited
in free relatives means that usually the \is{relative clause!wh-phrase@\lic{wh}-phrase in}\lic{wh}-phrase consists of just the
\lic{wh}-word, so that is very difficult to tease these things
apart.\footnote{\label{fn:rc-5}\cite[90]{Mueller99b} discusses the following \ili{German} example
  of a free relative with an initial PP containing the nominal relative word \lic{wem}
  `whom' (i.e.\ showing \isi{relative inheritance} to PP):
  \begin{exe}\ex\label{x:rc-157}
    \gll Ihr könnt beginnen, [mit wem ihr (beginnen) wollt].\\
    you can start with whom you start want\\\jambox*{(\ili{German})}
    \glt `You can start with whoever you like.’
  \end{exe}
  He observes that the free relative functions as a PP, just like \lic{mit wem}, and in
  the variant where the parenthesised instance of \lic{beginnen} is present, the within-clause
  role is also that of a PP.  However, \lic{mit} here is a non-predicative
  preposition, so the index associated with the PP is just that of the \lic{wh}-word that
  it contains, so it is still not possible to fully distinguish the properties of the
  \lic{wh}-phrase and the properties of the \lic{wh}-word it contains.
}



Following \cite{Mueller99b} on \ili{German}, free relatives have received considerable
attention in the HPSG literature, with analyses dealing with a variety of languages,
including: \ili{Arabic} \citep{Alqurashi:12,Hahn:12}, \ili{Danish} \citep{Bjerre:12,Bjerre:14},
\ili{English}
\citep{kim-park1996english,Kim01Constructional,WK03a,Francis07,Yoo:08,KimMixed2017},
\ili{German} \citep{Hinrichs:Nakazawa:02,Kubota:03}, \ili{Persian} \citep{Taghvaipour:05}, and \ili{Welsh}
\citep{Borsley:08}.

The central analytic problem is this: leaving aside the complication arising from case
syncretism and \isi{relative inheritance} just mentioned, the existence of matching effects has
suggested to some (e.g.\ \citealt{Kubota:03}) that the \is{relative clause!wh-phrase@\lic{wh}-phrase in}\lic{wh}-phrase should be the head
of the free relative, because the distribution of free relatives depends on the properties
of the \is{relative clause!wh-phrase@\lic{wh}-phrase in}\lic{wh}-phrase. So, for
example, the NP \lic{what} would be the head of \lic{what I suggested}. But this is
inconsistent with \lic{what} being the filler of the gap in \lic{what I suggested}
(i.e.\ the missing object of \lic{suggested}), because in a normal filler-gap construction
the filler is \emph{not} the head. If, instead, we assume that \lic{what} is primarily the
filler of the gap in the free relative, then we should assume that the clause \lic{I
  suggested \trace} is the head of the free relative --- and the distributional properties
of the free relative are unexplained.
\is{relative clause!free|)}

\subsubsection{Pseudo-clefts and transparent free relatives}
\label{sec:rc-pseudo-clefts-transp}
\is{specificational pseudo-cleft|(}
\is{relative clause!transparent free|(}

Two constructions that show some similarity with free relatives, and have received
some attention in the HPSG literature, are \emph{specificational pseudo-clefts}, exemplified in \pref{x:rc-158}, and so-called
\emph{transparent free relatives} (TFRs), exemplified in \pref{x:rc-163}.
\begin{exe}\ex\begin{xlist}\label{x:rc-158}
  \ex\label{x:rc-159} A new coat is [what Kim will be wearing].
  \ex\label{x:rc-160} {}[What Kim will be wearing] is a new coat.
  \ex\label{x:rc-161} {}[What she did] was cut her hair. 
  \ex\label{x:rc-162} {}[What she did not bring] was any wine.
\end{xlist}\end{exe}
\begin{exe}\ex\begin{xlist}\label{x:rc-163}
 \ex\label{x:rc-164} She replied in [what anyone would consider \trace a belligerent tone].
 \ex\label{x:rc-165} Her reply was [what anyone would consider \trace belligerent].  
\end{xlist}\end{exe}

Specificational pseudo-clefts typically consist of a \lic{wh}-clause, \lic{be}, and a
\emph{focal phrase} (e.g.\ \lic{any wine} in \pref{x:rc-162}). The focal phrase corresponds to a
gap in the \lic{wh}-clause (e.g.\ in \pref{x:rc-162} \lic{any wine} is interpreted as the
missing object of \lic{bring}). They raise a number of issues that are not typical of
relative clauses, notably the existence of \emph{connectivity effects} whereby the focal
phrase behaves as though it was part of the \lic{wh}-clause (e.g.\ in \pref{x:rc-162} the
negative polarity item \lic{any} is licensed by the negation in the
\lic{wh}-clause). Beyond this, it is not obvious whether the \lic{wh}-clauses should be
analysed as related to interrogatives, as in \cite{Yoo2003a-u}, or as related to free
relatives\is{relative clause!free}, as in \cite{Gerbl2007a}.\footnote{It can be
  difficult to distinguish this kind 
  of pseudo-cleft from cases involving a normal free relative. An example like \lic{What
    she is wearing is a mess} is superficially similar to \pref{x:rc-160}, but it involves a
  free relative. Notice, for example, it can be paraphrased with a normal NP plus relative
  clause (as ``The thing that she is wearing is a mess'') and \lic{what} can be replaced
  with \lic{whatever}. It does not have a paraphrase with an \lic{it}-cleft or a simple
  proposition --- it cannot be paraphrased as ``It is a mess that she is wearing'' or ``She is
  wearing a mess''.\is{relative clause!free}}

In TFRs the relative appears to function somewhat like a parenthetical modifier of a
\emph{nucleus} (e.g.\ \lic{a belligerent tone} in \pref{x:rc-164}), which seems to provide the head
properties of the phrase as a whole --- so for example the TFR in \pref{x:rc-164} has the
characteristics of an NP, that in \pref{x:rc-165} has those of an AP (it is a natural starting
point to assume the nucleus is internal to the relative clause, since otherwise one has
the puzzle of a relative clause which is both incomplete and occurs before the head it
modifies). TFRs are in some ways even more restricted than other kinds of relative (only
\lic{what} is allowed as the relative expression), but in others less restricted (e.g.\
free relatives\is{relative clause!free}
have the external distribution of NPs, but the TFR in \pref{x:rc-165} has the distribution of
an AP, like its nucleus \lic{belligerent}). Some approaches to TFRs employ novel kinds of
structure (e.g.\ \emph{grafts}, cf.\ \citealt{Riemskijk06Grafts}), but \cite{Yoo:08} and 
\cite{kim2011engtranspfrs} provide HPSG analyses which capture the relevant properties
using the existing apparatus with only minor adjustments.
\is{specificational pseudo-cleft|)}
\is{relative clause!transparent free|)}

\section{Conclusion}
\label{sec:rc-conclusion}

The analysis of relative clauses has been important in the theoretical evolution of HPSG,
notably in the development of a constructional approach involving inheritance from cross-classifying
dimensions of description. Empirically, relative clauses have been the focus
of a significant amount of descriptive work in a variety of typologically diverse
languages. Our goal in this paper has been exposition and survey rather than argumentation
towards particular conclusions, but, perhaps paradoxically given what we have just said, we
think one conclusion that clearly emerges is that, from an HPSG perspective at least,
\emph{relative clauses are not a natural kind}.
There is \emph{nothing} one can
say that will be true of everything that has been  described as a ``relative clause'' in
the literature. As
regards internal structure, some are \lic{head-filler} structures (\lic{wh}-relatives),
while others are \lic{head-complement} structures (complementiser relatives, some kinds of
bare relative); correspondingly, some involve \isi{relative pronoun}s (hence a \feat{rel}
feature), some do not. It is true that most involve some kind of \feat{slash}
dependency, but this is hardly unique to relative clauses, and even this does not hold of
the dependent noun and pseudo-relatives mentioned in
Section~\ref{sec:rc-depend-noun-pseudo}. There is no semantic unity --- while restrictive
relatives are noun-modifiers, non-restrictives function more like independent clauses, and
free relatives have nominal or adverbial semantics. Similarly, as regards external distribution: prototypical
relatives are noun modifiers, and appear in \istype{head-adjunct-phrase}\type{head-adjunct-ph} structures, but
expressions with similar internal structure occur as complements (e.g.\ free
relatives, clefts, and complements of superlative adjectives).

We do not think it is a bad thing that this conclusion should emerge from a discussion of HPSG
approaches. Rather, it suggests to us that an approach that tries to impose unity will end
up being procrustean. In fact, discussion of relative clauses seems to us to show some of
the best features of HPSG --- the analyses we have summarised are generally well
formalised, carefully constructed (detailed, precise, and coherent), and both empirically
satisfying and insightful, with relatively few \emph{ad hoc} assumptions or special
stipulations. The discussion shows how the expressivity and flexibility of the descriptive
machinery of the framework are compatible with a wide range of phenomena across a range of
languages.

\is{unbounded dependency|)}
\is{relative clause|)}

\section*{Abbreviations}

\begin{tabularx}{.99\textwidth}{@{}lX}
RP & a phrase headed by the empty relativiser R\\
SELR & Subject Extraction Lexical Rule         \\            
MRS & Minimal Recursion Semantics              \\            
LRS & Lexical Resource Semantics               \\            
WHIP &  Wh-Inheritance Principle                \\             
NRC & non-restrictive relative clause           \\             
RRC & restrictive relative clause               \\             
PRC & predicative relative clause               \\             
TFR & transparent free relative                 \\             
\end{tabularx}




%% \begin{itemize}
%% \item AP, XP, PP, NP, VP, CP, S, DP, PRO (standard linguistic abbreviations)
%% \item RP -- a phrase headed by the empty relativiser R% -- put in the index?
%% \item SELR -- Subject Extraction Lexical Rule
%% \item MRS -- Minimal Recursion Semantics
%% \item LRS -- Lexical Resource Semantics
%% \item WHIP --  Wh-Inheritance Principle
%% \item NRC -- non-restrictive relative clause
%% \item RRC -- restrictive relative clause
%% \item PRC -- predicative relative clause 
%% \item TFR -- transparent free relative                              
%% \end{itemize}

\section*{Acknowledgements}
We are grateful to the editors of this volume and two anonymous referees for their careful
and insightful comments on earlier versions of this contribution. Remaining flaws are our
sole responsibility, of course.


{\sloppy 
\printbibliography[heading=subbibliography,notkeyword=this] 
}

\end{document}

 
