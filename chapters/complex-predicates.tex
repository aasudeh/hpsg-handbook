\documentclass[output=paper
                ,modfonts
                ,nonflat
	        ,collection
	        ,collectionchapter
	        ,collectiontoclongg
 	        ,biblatex
                ,babelshorthands
                ,newtxmath
                ,draftmode
                ,colorlinks, citecolor=brown
]{./langsci/langscibook}

\IfFileExists{../localcommands.tex}{%hack to check whether this is being compiled as part of a collection or standalone
  % add all extra packages you need to load to this file 

\usepackage{graphicx}
\usepackage{tabularx}
\usepackage{amsmath} 
\usepackage{tipa}      % Davis Koenig
\usepackage{multicol}
\usepackage{lipsum}


\usepackage{./langsci/styles/langsci-optional} 
\usepackage{./langsci/styles/langsci-lgr}
%\usepackage{./styles/forest/forest}
\usepackage{./langsci/styles/langsci-forest-setup}
\usepackage{morewrites}

\usepackage{tikz-cd}

\usepackage{./styles/tikz-grid}
\usetikzlibrary{shadows}


%\usepackage{pgfplots} % for data/theory figure in minimalism.tex
% fix some issue with Mod https://tex.stackexchange.com/a/330076
\makeatletter
\let\pgfmathModX=\pgfmathMod@
\usepackage{pgfplots}%
\let\pgfmathMod@=\pgfmathModX
\makeatother

\usepackage{subcaption}

% Stefan Müller's styles
\usepackage{./styles/merkmalstruktur,german,./styles/makros.2e,./styles/my-xspace,./styles/article-ex,
./styles/eng-date}

\selectlanguage{USenglish}

\usepackage{./styles/abbrev}

\usepackage{./langsci/styles/jambox}

% Has to be loaded late since otherwise footnotes will not work

%%%%%%%%%%%%%%%%%%%%%%%%%%%%%%%%%%%%%%%%%%%%%%%%%%%%
%%%                                              %%%
%%%           Examples                           %%%
%%%                                              %%%
%%%%%%%%%%%%%%%%%%%%%%%%%%%%%%%%%%%%%%%%%%%%%%%%%%%%
% remove the percentage signs in the following lines
% if your book makes use of linguistic examples
\usepackage{./langsci/styles/langsci-gb4e} 

% Crossing out text
% uncomment when needed
%\usepackage{ulem}

\usepackage{./styles/additional-langsci-index-shortcuts}

%\usepackage{./langsci/styles/langsci-avm}
\usepackage{./styles/avm+}


\renewcommand{\tpv}[1]{{\avmjvalfont\itshape #1}}

% no small caps please
\renewcommand{\phonshape}[0]{\normalfont\itshape}

\regAvmFonts

\usepackage{theorem}

\newtheorem{mydefinition}{Def.}
\newtheorem{principle}{Principle}

{\theoremstyle{break}
%\newtheorem{schema}{Schema}
\newtheorem{mydefinition-break}[mydefinition]{Def.}
\newtheorem{principle-break}[principle]{Principle}
}

% This avoids linebreaks in the Schema
\newcounter{schema}
\newenvironment{schema}[1][]
  {% \begin{Beispiel}[<title>]
  \goodbreak%
  \refstepcounter{schema}%
  \begin{list}{}{\setlength{\labelwidth}{0pt}\setlength{\labelsep}{0pt}\setlength{\rightmargin}{0pt}\setlength{\leftmargin}{0pt}}%
    \item[{\textbf{Schema~\theschema}}]\hspace{.5em}\textbf{(#1)}\nopagebreak[4]\par\nobreak}%
  {\end{list}}% \end{Beispiel}

%% \newcommand{schema}[2]{
%% \begin{minipage}{\textwidth}
%% {\textbf{Schema~\theschema}}]\hspace{.5em}\textbf{(#1)}\\
%% #2
%% \end{minipage}}

%\usepackage{subfig}





% Davis Koenig Lexikon

\usepackage{tikz-qtree,tikz-qtree-compat} % Davis Koenig remove

\usepackage{shadow}




\usepackage[english]{isodate} % Andy Lücking
\usepackage[autostyle]{csquotes} % Andy
%\usepackage[autolanguage]{numprint}

%\defaultfontfeatures{
%    Path = /usr/local/texlive/2017/texmf-dist/fonts/opentype/public/fontawesome/ }

%% https://tex.stackexchange.com/a/316948/18561
%\defaultfontfeatures{Extension = .otf}% adds .otf to end of path when font loaded without ext parameter e.g. \newfontfamily{\FA}{FontAwesome} > \newfontfamily{\FA}{FontAwesome.otf}
%\usepackage{fontawesome} % Andy Lücking
\usepackage{pifont} % Andy Lücking -> hand

\usetikzlibrary{decorations.pathreplacing} % Andy Lücking
\usetikzlibrary{matrix} % Andy 
\usetikzlibrary{positioning} % Andy
\usepackage{tikz-3dplot} % Andy

% pragmatics
\usepackage{eqparbox} % Andy
\usepackage{enumitem} % Andy
\usepackage{longtable} % Andy
\usepackage{tabu} % Andy


% Manfred's packages

%\usepackage{shadow}

\usepackage{tabularx}
\newcolumntype{L}[1]{>{\raggedright\arraybackslash}p{#1}} % linksbündig mit Breitenangabe


% Jong-Bok

%\usepackage{xytree}

\newcommand{\xytree}[2][dummy]{Let's do the tree!}

% seems evil, get rid of it
% defines \ex is incompatible with gb4e
%\usepackage{lingmacros}

% taken from lingmacros:
\makeatletter
% \evnup is used to line up the enumsentence number and an entry along
% the top.  It can take an argument to improve lining up.
\def\evnup{\@ifnextchar[{\@evnup}{\@evnup[0pt]}}

\def\@evnup[#1]#2{\setbox1=\hbox{#2}%
\dimen1=\ht1 \advance\dimen1 by -.5\baselineskip%
\advance\dimen1 by -#1%
\leavevmode\lower\dimen1\box1}
\makeatother


% YK -- CG chapter

%\usepackage{xspace}
\usepackage{bm}
\usepackage{bussproofs}


% Antonio Branco, remove this
\usepackage{epsfig}

% now unicode
%\usepackage{alphabeta}



% Berthold udc
%\usepackage{qtree}
%\usepackage{rtrees}

\usepackage{pst-node}

  %add all your local new commands to this file

\makeatletter
\def\blx@maxline{77}
\makeatother


\newcommand{\page}{}



\newcommand{\todostefan}[1]{\todo[color=orange!80]{\footnotesize #1}\xspace}
\newcommand{\todosatz}[1]{\todo[color=red!40]{\footnotesize #1}\xspace}

\newcommand{\inlinetodostefan}[1]{\todo[color=green!40,inline]{\footnotesize #1}\xspace}


\newcommand{\spacebr}{\hspaceThis{[}}

\newcommand{\danish}{\jambox{(\ili{Danish})}}
\newcommand{\english}{\jambox{(\ili{English})}}
\newcommand{\german}{\jambox{(\ili{German})}}
\newcommand{\yiddish}{\jambox{(\ili{Yiddish})}}
\newcommand{\welsh}{\jambox{(\ili{Welsh})}}

% Cite and cross-reference other chapters
\newcommand{\crossrefchaptert}[2][]{\citet*[#1]{chapters/#2}, Chapter~\ref{chap-#2} of this volume} 
\newcommand{\crossrefchapterp}[2][]{(\citealp*[#1][]{chapters/#2}, Chapter~\ref{chap-#2} of this volume)}
% example of optional argument:
% \crossrefchapterp[for something, see:]{name}
% gives: (for something, see: Author 2018, Chapter~X of this volume)

\let\crossrefchapterw\crossrefchaptert



% Davis Koenig

\let\ig=\textsc
\let\tc=\textcolor

% evolution, Flickinger, Pollard, Wasow

\let\citeNP\citet

% Adam P

%\newcommand{\toappear}{Forthcoming}
\newcommand{\pg}[1]{p.#1}
\renewcommand{\implies}{\rightarrow}

\newcommand*{\rref}[1]{(\ref{#1})}
\newcommand*{\aref}[1]{(\ref{#1}a)}
\newcommand*{\bref}[1]{(\ref{#1}b)}
\newcommand*{\cref}[1]{(\ref{#1}c)}

\newcommand{\msadam}{.}
\newcommand{\morsyn}[1]{\textsc{#1}}

\newcommand{\nom}{\morsyn{nom}}
\newcommand{\acc}{\morsyn{acc}}
\newcommand{\dat}{\morsyn{dat}}
\newcommand{\gen}{\morsyn{gen}}
\newcommand{\ins}{\morsyn{ins}}
\newcommand{\loc}{\morsyn{loc}}
\newcommand{\voc}{\morsyn{voc}}
\newcommand{\ill}{\morsyn{ill}}
\renewcommand{\inf}{\morsyn{inf}}
\newcommand{\passprc}{\morsyn{passp}}

%\newcommand{\Nom}{\msadam\nom}
%\newcommand{\Acc}{\msadam\acc}
%\newcommand{\Dat}{\msadam\dat}
%\newcommand{\Gen}{\msadam\gen}
\newcommand{\Ins}{\msadam\ins}
\newcommand{\Loc}{\msadam\loc}
\newcommand{\Voc}{\msadam\voc}
\newcommand{\Ill}{\msadam\ill}
\newcommand{\INF}{\msadam\inf}
\newcommand{\PassP}{\msadam\passprc}

\newcommand{\Aux}{\textsc{aux}}

\newcommand{\princ}[1]{\textnormal{\textsc{#1}}} % for constraint names
\newcommand{\notion}[1]{\emph{#1}}
\renewcommand{\path}[1]{\textnormal{\textsc{#1}}}
\newcommand{\ftype}[1]{\textit{#1}}
\newcommand{\fftype}[1]{{\scriptsize\textit{#1}}}
\newcommand{\la}{$\langle$}
\newcommand{\ra}{$\rangle$}
%\newcommand{\argst}{\path{arg-st}}
\newcommand{\phtm}[1]{\setbox0=\hbox{#1}\hspace{\wd0}}
\newcommand{\prep}[1]{\setbox0=\hbox{#1}\hspace{-1\wd0}#1}

%%%%%%%%%%%%%%%%%%%%%%%%%%%%%%%%%%%%%%%%%%%%%%%%%%%%%%%%%%%%%%%%%%%%%%%%%%%

% FROM FS.STY:

%%%
%%% Feature structures
%%%

% \fs         To print a feature structure by itself, type for example
%             \fs{case:nom \\ person:P}
%             or (better, for true italics),
%             \fs{\it case:nom \\ \it person:P}
%
% \lfs        To print the same feature structure with the category
%             label N at the top, type:
%             \lfs{N}{\it case:nom \\ \it person:P}

%    Modified 1990 Dec 5 so that features are left aligned.
\newcommand{\fs}[1]%
{\mbox{\small%
$
\!
\left[
  \!\!
  \begin{tabular}{l}
    #1
  \end{tabular}
  \!\!
\right]
\!
$}}

%     Modified 1990 Dec 5 so that features are left aligned.
%\newcommand{\lfs}[2]
%   {
%     \mbox{$
%           \!\!
%           \begin{tabular}{c}
%           \it #1
%           \\
%           \mbox{\small%
%                 $
%                 \left[
%                 \!\!
%                 \it
%                 \begin{tabular}{l}
%                 #2
%                 \end{tabular}
%                 \!\!
%                 \right]
%                 $}
%           \end{tabular}
%           \!\!
%           $}
%   }

\newcommand{\ft}[2]{\path{#1}\hspace{1ex}\ftype{#2}}
\newcommand{\fsl}[2]{\fs{{\fftype{#1}} \\ #2}}

\newcommand{\fslt}[2]
 {\fst{
       {\fftype{#1}} \\
       #2 
     }
 }

\newcommand{\fsltt}[2]
 {\fstt{
       {\fftype{#1}} \\
       #2 
     }
 }

\newcommand{\fslttt}[2]
 {\fsttt{
       {\fftype{#1}} \\
       #2 
     }
 }


% jak \ft, \fs i \fsl tylko nieco ciasniejsze

\newcommand{\ftt}[2]
% {{\sc #1}\/{\rm #2}}
 {\textsc{#1}\/{\rm #2}}

\newcommand{\fst}[1]
  {
    \mbox{\small%
          $
          \left[
          \!\!\!
%          \sc
          \begin{tabular}{l} #1
          \end{tabular}
          \!\!\!\!\!\!\!
          \right]
          $
          }
   }

%\newcommand{\fslt}[2]
% {\fst{#2\\
%       {\scriptsize\it #1}
%      }
% }


% superciasne

\newcommand{\fstt}[1]
  {
    \mbox{\small%
          $
          \left[
          \!\!\!
%          \sc
          \begin{tabular}{l} #1
          \end{tabular}
          \!\!\!\!\!\!\!\!\!\!\!
          \right]
          $
          }
   }

%\newcommand{\fsltt}[2]
% {\fstt{#2\\
%       {\scriptsize\it #1}
%      }
% }

\newcommand{\fsttt}[1]
  {
    \mbox{\small%
          $
          \left[
          \!\!\!
%          \sc
          \begin{tabular}{l} #1
          \end{tabular}
          \!\!\!\!\!\!\!\!\!\!\!\!\!\!\!\!
          \right]
          $
          }
   }



% %add all your local new commands to this file

% \newcommand{\smiley}{:)}

% you are not supposed to mess with hardcore stuff, St.Mü. 22.08.2018
%% \renewbibmacro*{index:name}[5]{%
%%   \usebibmacro{index:entry}{#1}
%%     {\iffieldundef{usera}{}{\thefield{usera}\actualoperator}\mkbibindexname{#2}{#3}{#4}{#5}}}

% % \newcommand{\noop}[1]{}



% Rui

\newcommand{\spc}[0]{\hspace{-1pt}\underline{\hspace{6pt}}\,}
\newcommand{\spcs}[0]{\hspace{-1pt}\underline{\hspace{6pt}}\,\,}
\newcommand{\bad}[1]{\leavevmode\llap{#1}}
\newcommand{\COMMENT}[1]{}


% Andy Lücking gesture.tex
\newcommand{\Pointing}{\ding{43}}
% Giotto: "Meeting of Joachim and Anne at the Golden Gate" - 1305-10 
\definecolor{GoldenGate1}{rgb}{.13,.09,.13} % Dress of woman in black
\definecolor{GoldenGate2}{rgb}{.94,.94,.91} % Bridge
\definecolor{GoldenGate3}{rgb}{.06,.09,.22} % Blue sky
\definecolor{GoldenGate4}{rgb}{.94,.91,.87} % Dress of woman with shawl
\definecolor{GoldenGate5}{rgb}{.52,.26,.26} % Joachim's robe
\definecolor{GoldenGate6}{rgb}{.65,.35,.16} % Anne's robe
\definecolor{GoldenGate7}{rgb}{.91,.84,.42} % Joachim's halo

\makeatletter
\newcommand{\@Depth}{1} % x-dimension, to front
\newcommand{\@Height}{1} % z-dimension, up
\newcommand{\@Width}{1} % y-dimension, rightwards
%\GGS{<x-start>}{<y-start>}{<z-top>}{<z-bottom>}{<Farbe>}{<x-width>}{<y-depth>}{<opacity>}
\newcommand{\GGS}[9][]{%
\coordinate (O) at (#2-1,#3-1,#5);
\coordinate (A) at (#2-1,#3-1+#7,#5);
\coordinate (B) at (#2-1,#3-1+#7,#4);
\coordinate (C) at (#2-1,#3-1,#4);
\coordinate (D) at (#2-1+#8,#3-1,#5);
\coordinate (E) at (#2-1+#8,#3-1+#7,#5);
\coordinate (F) at (#2-1+#8,#3-1+#7,#4);
\coordinate (G) at (#2-1+#8,#3-1,#4);
\draw[draw=black, fill=#6, fill opacity=#9] (D) -- (E) -- (F) -- (G) -- cycle;% Front
\draw[draw=black, fill=#6, fill opacity=#9] (C) -- (B) -- (F) -- (G) -- cycle;% Top
\draw[draw=black, fill=#6, fill opacity=#9] (A) -- (B) -- (F) -- (E) -- cycle;% Right
}
\makeatother


% pragmatics
\newcommand{\speaking}[1]{\eqparbox{name}{\textsc{\lowercase{#1}\space}}}
\newcommand{\name}[1]{\eqparbox{name}{\textsc{\lowercase{#1}}}}
\newcommand{\HPSGTTR}{HPSG$_{\text{TTR}}$\xspace}

\newcommand{\ttrtype}[1]{\textit{#1}}
% \newcommand{\avmel}{\q<\quad\q>} %% shortcut for empty lists in AVM
\newcommand{\ttrmerge}{\ensuremath{\wedge_{\textit{merge}}}}
\newcommand{\Cat}[2][0.1pt]{%
  \begin{scope}[y=#1,x=#1,yscale=-1, inner sep=0pt, outer sep=0pt]
   \path[fill=#2,line join=miter,line cap=butt,even odd rule,line width=0.8pt]
  (151.3490,307.2045) -- (264.3490,307.2045) .. controls (264.3490,291.1410) and (263.2021,287.9545) .. (236.5990,287.9545) .. controls (240.8490,275.2045) and (258.1242,244.3581) .. (267.7240,244.3581) .. controls (276.2171,244.3581) and (286.3490,244.8259) .. (286.3490,264.2045) .. controls (286.3490,286.2045) and (323.3717,321.6755) .. (332.3490,307.2045) .. controls (345.7277,285.6390) and (309.3490,292.2151) .. (309.3490,240.2046) .. controls (309.3490,169.0514) and (350.8742,179.1807) .. (350.8742,139.2046) .. controls (350.8742,119.2045) and (345.3490,116.5037) .. (345.3490,102.2045) .. controls (345.3490,83.3070) and (361.9972,84.4036) .. (358.7581,68.7349) .. controls (356.5206,57.9117) and (354.7696,49.2320) .. (353.4652,36.1439) .. controls (352.5396,26.8573) and (352.2445,16.9594) .. (342.5985,17.3574) .. controls (331.2650,17.8250) and (326.9655,37.7742) .. (309.3490,39.2045) .. controls (291.7685,40.6320) and (276.7783,24.2380) .. (269.9740,26.5795) .. controls (263.2271,28.9013) and (265.3490,47.2045) .. (269.3490,60.2045) .. controls (275.6359,80.6368) and (289.3490,107.2045) .. (264.3490,111.2045) .. controls (239.3490,115.2045) and (196.3490,119.2045) .. (165.3490,160.2046) .. controls (134.3490,201.2046) and (135.4934,249.3212) .. (123.3490,264.2045) .. controls (82.5907,314.1553) and (40.8239,293.6463) .. (40.8239,335.2045) .. controls (40.8239,353.8102) and (72.3490,367.2045) .. (77.3490,361.2045) .. controls (82.3490,355.2045) and (34.8638,337.3259) .. (87.9955,316.2045) .. controls (133.3871,298.1601) and   (137.4391,294.4766) .. (151.3490,307.2045) -- cycle;
\end{scope}%
}


% KdK
\newcommand{\smiley}{:)}

\renewbibmacro*{index:name}[5]{%
  \usebibmacro{index:entry}{#1}
    {\iffieldundef{usera}{}{\thefield{usera}\actualoperator}\mkbibindexname{#2}{#3}{#4}{#5}}}

% \newcommand{\noop}[1]{}

% chngcntr.sty otherwise gives error that these are already defined
%\let\counterwithin\relax
%\let\counterwithout\relax

% the space of a left bracket for glossings
\newcommand{\LB}{\hspaceThis{[}}

\newcommand{\LF}{\mbox{$[\![$}}

\newcommand{\RF}{\mbox{$]\!]_F$}}

\newcommand{\RT}{\mbox{$]\!]_T$}}





% Manfred's

\newcommand{\kommentar}[1]{}

\newcommand{\bsp}[1]{\emph{#1}}
\newcommand{\bspT}[2]{\bsp{#1} `#2'}
\newcommand{\bspTL}[3]{\bsp{#1} (lit.: #2) `#3'}

\newcommand{\noidi}{§}

\newcommand{\refer}[1]{(\ref{#1})}

%\newcommand{\avmtype}[1]{\multicolumn{2}{l}{\type{#1}}}
\newcommand{\attr}[1]{\textsc{#1}}

\newcommand{\srdefault}{\mbox{\begin{tabular}{c}{\large <}\\[-1.5ex]$\sqcap$\end{tabular}}}

%% \newcommand{\myappcolumn}[2]{
%% \begin{minipage}[t]{#1}#2\end{minipage}
%% }

%% \newcommand{\appc}[1]{\myappcolumn{3.7cm}{#1}}


% Jong-Bok


% clean that up and do not use \def (killing other stuff defined before)
%\if 0
\def\DEL{\textsc{del}}
\def\del{\textsc{del}}

\def\conn{\textsc{conn}}
\def\CONN{\textsc{conn}}
\def\CONJ{\textsc{conj}}
\def\LITE{\textsc{lex}}
\def\lite{\textsc{lex}}
\def\HON{\textsc{hon}}

\def\CAUS{\textsc{caus}}
\def\PASS{\textsc{pass}}
\def\NPST{\textsc{npst}}
\def\COND{\textsc{cond}}



\def\hd-lite{\textsc{head-lex construction}}
\def\NFORM{\textsc{nform}}

\def\RELS{\textsc{rels}}
\def\TENSE{\textsc{tense}}


%\def\ARG{\textsc{arg}}
\def\ARGs{\textsc{arg0}}
\def\ARGa{\textsc{arg}}
\def\ARGb{\textsc{arg2}}
\def\TPC{\textsc{top}}
\def\PROG{\textsc{prog}}

\def\pst{\textsc{pst}}
\def\PAST{\textsc{pst}}
\def\DAT{\textsc{dat}}
\def\CONJ{\textsc{conj}}
\def\nominal{\textsc{nominal}}
\def\NOMINAL{\textsc{nominal}}
\def\VAL{\textsc{val}}
\def\val{\textsc{val}}
\def\MODE{\textsc{mode}}
\def\RESTR{\textsc{restr}}
\def\SIT{\textsc{sit}}
\def\ARG{\textsc{arg}}
\def\RELN{\textsc{rel}}
\def\REL{\textsc{rel}}
\def\RELS{\textsc{rels}}
\def\arg-st{\textsc{arg-st}}
\def\xdel{\textsc{xdel}}
\def\zdel{\textsc{zdel}}
\def\sug{\textsc{sug}}
\def\IMP{\textsc{imp}}
\def\conn{\textsc{conn}}
\def\CONJ{\textsc{conj}}
\def\HON{\textsc{hon}}
\def\BN{\textsc{bn}}
\def\bn{\textsc{bn}}
\def\pres{\textsc{pres}}
\def\PRES{\textsc{pres}}
\def\prs{\textsc{pres}}
\def\PRS{\textsc{pres}}
\def\agt{\textsc{agt}}
\def\DEL{\textsc{del}}
\def\PRED{\textsc{pred}}
\def\AGENT{\textsc{agent}}
\def\THEME{\textsc{theme}}
\def\AUX{\textsc{aux}}
\def\THEME{\textsc{theme}}
\def\PL{\textsc{pl}}
\def\SRC{\textsc{src}}
\def\src{\textsc{src}}
\def\FORM{\textsc{form}}
\def\form{\textsc{form}}
\def\GCASE{\textsc{gcase}}
\def\gcase{\textsc{gcase}}
\def\SCASE{\textsc{scase}}
\def\PHON{\textsc{phon}}
\def\SS{\textsc{ss}}
\def\SYN{\textsc{syn}}
\def\LOC{\textsc{loc}}
\def\MOD{\textsc{mod}}
\def\INV{\textsc{inv}}
\def\L{\textsc{l}}
\def\CASE{\textsc{case}}
\def\SPR{\textsc{spr}}
\def\COMPS{\textsc{comps}}
%\def\comps{\textsc{comps}}
\def\SEM{\textsc{sem}}
\def\CONT{\textsc{cont}}
\def\SUBCAT{\textsc{subcat}}
\def\CAT{\textsc{cat}}
\def\C{\textsc{c}}
\def\SUBJ{\textsc{subj}}
\def\subj{\textsc{subj}}
\def\SLASH{\textsc{slash}}
\def\LOCAL{\textsc{local}}
\def\ARG-ST{\textsc{arg-st}}
\def\AGR{\textsc{agr}}
\def\PER{\textsc{per}}
\def\NUM{\textsc{num}}
\def\IND{\textsc{ind}}
\def\VFORM{\textsc{vform}}
\def\PFORM{\textsc{pform}}
\def\decl{\textsc{decl}}
\def\loc{\textsc{loc   }}
% \def\   {\textsc{  }}

\def\NEG{\textsc{neg}}
\def\FRAMES{\textsc{frames}}
\def\REFL{\textsc{refl}}

\def\MKG{\textsc{mkg}}

\def\BN{\textsc{bn}}
\def\HD{\textsc{hd}}
\def\NP{\textsc{np}}
\def\PF{\textsc{pf}}
\def\PL{\textsc{pl}}
\def\PP{\textsc{pp}}
\def\SS{\textsc{ss}}
\def\VF{\textsc{vf}}
\def\VP{\textsc{vp}}
\def\bn{\textsc{bn}}
\def\cl{\textsc{cl}}
\def\pl{\textsc{pl}}
\def\Wh{\ital{Wh}}
\def\ng{\textsc{neg}}
\def\wh{\ital{wh}}
\def\ACC{\textsc{acc}}
\def\AGR{\textsc{agr}}
\def\AGT{\textsc{agt}}
\def\ARC{\textsc{arc}}
\def\ARG{\textsc{arg}}
\def\ARP{\textsc{arc}}
\def\AUX{\textsc{aux}}
\def\CAT{\textsc{cat}}
\def\COP{\textsc{cop}}
\def\DAT{\textsc{dat}}
\def\DEF{\textsc{def}}
\def\DEL{\textsc{del}}
\def\DOM{\textsc{dom}}
\def\DTR{\textsc{dtr}}
\def\FUT{\textsc{fut}}
\def\GAP{\textsc{gap}}
\def\GEN{\textsc{gen}}
\def\HON{\textsc{hon}}
\def\IMP{\textsc{imp}}
\def\IND{\textsc{ind}}
\def\INV{\textsc{inv}}
\def\LEX{\textsc{lex}}
\def\Lex{\textsc{lex}}
\def\LOC{\textsc{loc}}
\def\MOD{\textsc{mod}}
\def\MRK{{\nr MRK}}
\def\NEG{\textsc{neg}}
\def\NEW{\textsc{new}}
\def\NOM{\textsc{nom}}
\def\NUM{\textsc{num}}
\def\PER{\textsc{per}}
\def\PST{\textsc{pst}}
\def\QUE{\textsc{que}}
\def\REL{\textsc{rel}}
\def\SEL{\textsc{sel}}
\def\SEM{\textsc{sem}}
\def\SIT{\textsc{arg0}}
\def\SPR{\textsc{spr}}
\def\SRC{\textsc{src}}
\def\SUG{\textsc{sug}}
\def\SYN{\textsc{syn}}
\def\TPC{\textsc{top}}
\def\VAL{\textsc{val}}
\def\acc{\textsc{acc}}
\def\agt{\textsc{agt}}
\def\cop{\textsc{cop}}
\def\dat{\textsc{dat}}
\def\foc{\textsc{focus}}
\def\FOC{\textsc{focus}}
\def\fut{\textsc{fut}}
\def\hon{\textsc{hon}}
\def\imp{\textsc{imp}}
\def\kes{\textsc{kes}}
\def\lex{\textsc{lex}}
\def\loc{\textsc{loc}}
\def\mrk{{\nr MRK}}
\def\nom{\textsc{nom}}
\def\num{\textsc{num}}
\def\plu{\textsc{plu}}
\def\pne{\textsc{pne}}
\def\pst{\textsc{pst}}
\def\pur{\textsc{pur}}
\def\que{\textsc{que}}
\def\src{\textsc{src}}
\def\sug{\textsc{sug}}
\def\tpc{\textsc{top}}
\def\utt{\textsc{utt}}
\def\val{\textsc{val}}
\def\LITE{\textsc{lex}}
\def\PAST{\textsc{pst}}
\def\POSP{\textsc{pos}}
\def\PRS{\textsc{pres}}
\def\mod{\textsc{mod}}%
\def\newuse{{`kes'}}
\def\posp{\textsc{pos}}
\def\prs{\textsc{pres}}
\def\psp{{\it en\/}}
\def\skes{\textsc{kes}}
\def\CASE{\textsc{case}}
\def\CASE{\textsc{case}}
\def\COMP{\textsc{comp}}
\def\CONJ{\textsc{conj}}
\def\CONN{\textsc{conn}}
\def\CONT{\textsc{cont}}
\def\DECL{\textsc{decl}}
\def\FOCUS{\textsc{focus}}
\def\FORM{\textsc{form}}
\def\FREL{\textsc{frel}}
\def\GOAL{\textsc{goal}}
\def\HEAD{\textsc{head}}
\def\INDEX{\textsc{ind}}
\def\INST{\textsc{inst}}
\def\MODE{\textsc{mode}}
\def\MOOD{\textsc{mood}}
\def\NMLZ{\textsc{nmlz}}
\def\PHON{\textsc{phon}}
\def\PRED{\textsc{pred}}
%\def\PRES{\textsc{pres}}
\def\PROM{\textsc{prom}}
\def\RELN{\textsc{pred}}
\def\RELS{\textsc{rels}}
\def\STEM{\textsc{stem}}
\def\SUBJ{\textsc{subj}}
\def\XARG{\textsc{xarg}}
\def\bse{{\it bse\/}}
\def\case{\textsc{case}}
\def\caus{\textsc{caus}}
\def\comp{\textsc{comp}}
\def\conj{\textsc{conj}}
\def\conn{\textsc{conn}}
\def\decl{\textsc{decl}}
\def\fin{{\it fin\/}}
\def\form{\textsc{form}}
\def\gend{\textsc{gend}}
\def\inf{{\it inf\/}}
\def\mood{\textsc{mood}}
\def\nmlz{\textsc{nmlz}}
\def\pass{\textsc{pass}}
\def\past{\textsc{past}}
\def\perf{\textsc{perf}}
\def\pln{{\it pln\/}}
\def\pred{\textsc{pred}}


%\def\pres{\textsc{pres}}
\def\proc{\textsc{proc}}
\def\nonfin{{\it nonfin\/}}
\def\AGENT{\textsc{agent}}
\def\CFORM{\textsc{cform}}
%\def\COMPS{\textsc{comps}}
\def\COORD{\textsc{coord}}
\def\COUNT{\textsc{count}}
\def\EXTRA{\textsc{extra}}
\def\GCASE{\textsc{gcase}}
\def\GIVEN{\textsc{given}}
\def\LOCAL{\textsc{local}}
\def\NFORM{\textsc{nform}}
\def\PFORM{\textsc{pform}}
\def\SCASE{\textsc{scase}}
\def\SLASH{\textsc{slash}}
\def\SLASH{\textsc{slash}}
\def\THEME{\textsc{theme}}
\def\TOPIC{\textsc{topic}}
\def\VFORM{\textsc{vform}}
\def\cause{\textsc{cause}}
%\def\comps{\textsc{comps}}
\def\gcase{\textsc{gcase}}
\def\itkes{{\it kes\/}}
\def\pass{{\it pass\/}}
\def\vform{\textsc{vform}}
\def\CCONT{\textsc{c-cont}}
\def\GN{\textsc{given-new}}
\def\INFO{\textsc{info-st}}
\def\ARG-ST{\textsc{arg-st}}
\def\SUBCAT{\textsc{subcat}}
\def\SYNSEM{\textsc{synsem}}
\def\VERBAL{\textsc{verbal}}
\def\arg-st{\textsc{arg-st}}
\def\plain{{\it plain}\/}
\def\propos{\textsc{propos}}
\def\ADVERBIAL{\textsc{advl}}
\def\HIGHLIGHT{\textsc{prom}}
\def\NOMINAL{\textsc{nominal}}

\newenvironment{myavm}{\begingroup\avmvskip{.1ex}
  \selectfont\begin{avm}}%
{\end{avm}\endgroup\medskip}
\def\pfix{\vspace{-5pt}}


\def\jbsub#1{\lower4pt\hbox{\small #1}}
\def\jbssub#1{\lower4pt\hbox{\small #1}}
\def\jbtr{\underbar{\ \ \ }\ }


%\fi

  %% hyphenation points for line breaks
%% Normally, automatic hyphenation in LaTeX is very good
%% If a word is mis-hyphenated, add it to this file
%%
%% add information to TeX file before \begin{document} with:
%% %% hyphenation points for line breaks
%% Normally, automatic hyphenation in LaTeX is very good
%% If a word is mis-hyphenated, add it to this file
%%
%% add information to TeX file before \begin{document} with:
%% %% hyphenation points for line breaks
%% Normally, automatic hyphenation in LaTeX is very good
%% If a word is mis-hyphenated, add it to this file
%%
%% add information to TeX file before \begin{document} with:
%% \include{localhyphenation}
\hyphenation{
A-la-hver-dzhie-va
anaph-o-ra
affri-ca-te
affri-ca-tes
Atha-bas-kan
Chi-che-ŵa
com-ple-ments
Da-ge-stan
Dor-drecht
er-klä-ren-de
Ginz-burg
Gro-ning-en
Jon-a-than
Ka-tho-lie-ke
Ko-bon
krie-gen
Le-Sourd
moth-er
Mül-ler
Nie-mey-er
Prze-piór-kow-ski
phe-nom-e-non
re-nowned
Rie-he-mann
un-bound-ed
}

% why has "erklärende" be listed here? I specified langid in bibtex item. Something is still not working with hyphenation.


% to do: check
%  Alahverdzhieva

\hyphenation{
A-la-hver-dzhie-va
anaph-o-ra
affri-ca-te
affri-ca-tes
Atha-bas-kan
Chi-che-ŵa
com-ple-ments
Da-ge-stan
Dor-drecht
er-klä-ren-de
Ginz-burg
Gro-ning-en
Jon-a-than
Ka-tho-lie-ke
Ko-bon
krie-gen
Le-Sourd
moth-er
Mül-ler
Nie-mey-er
Prze-piór-kow-ski
phe-nom-e-non
re-nowned
Rie-he-mann
un-bound-ed
}

% why has "erklärende" be listed here? I specified langid in bibtex item. Something is still not working with hyphenation.


% to do: check
%  Alahverdzhieva

\hyphenation{
A-la-hver-dzhie-va
anaph-o-ra
affri-ca-te
affri-ca-tes
Atha-bas-kan
Chi-che-ŵa
com-ple-ments
Da-ge-stan
Dor-drecht
er-klä-ren-de
Ginz-burg
Gro-ning-en
Jon-a-than
Ka-tho-lie-ke
Ko-bon
krie-gen
Le-Sourd
moth-er
Mül-ler
Nie-mey-er
Prze-piór-kow-ski
phe-nom-e-non
re-nowned
Rie-he-mann
un-bound-ed
}

% why has "erklärende" be listed here? I specified langid in bibtex item. Something is still not working with hyphenation.


% to do: check
%  Alahverdzhieva

  \bibliography{../Bibliographies/stmue,
                ../localbibliography,
../Bibliographies/formal-background,
../Bibliographies/understudied-languages,
../Bibliographies/phonology,
../Bibliographies/case,
../Bibliographies/evolution,
../Bibliographies/agreement,
../Bibliographies/lexicon,
../Bibliographies/np,
../Bibliographies/negation,
../Bibliographies/argst,
../Bibliographies/binding,
../Bibliographies/complex-predicates,
../Bibliographies/coordination,
../Bibliographies/relative-clauses,
../Bibliographies/udc,
../Bibliographies/processing,
../Bibliographies/cl,
../Bibliographies/dg,
../Bibliographies/islands,
../Bibliographies/diachronic,
../Bibliographies/gesture,
../Bibliographies/semantics,
../Bibliographies/pragmatics,
../Bibliographies/information-structure,
../Bibliographies/idioms,
../Bibliographies/cg,
../Bibliographies/udc}

  \togglepaper[12]
}{}

\author{%
	Danièle Godard\affiliation{Université Paris Diderot}%
	\lastand Pollet Samvelian\affiliation{Université Sorbonne Nouvelle}
}
\title{Complex predicates}

% \chapterDOI{} %will be filled in at production

%\epigram{Change epigram in chapters/03.tex or remove it there }
\abstract{Complex predicates are constructions in which the head attracts arguments from its predicate complement. It characterizes auxiliaries, copulas and predicative verbs, certain control or raising and causative verbs, and light verbs. It has been studied in HPSG in different languages, Romance and Germanic languages, Korean and Persian, which illustrate different aspects of the phenomenon. Romance languages, as well Korean, show that argument inheritance is distinct from structure: it is compatible with two different structures. German and Dutch and Korean show that argument inheritance induces different word order properties, and Persian that it can be preserved by a derivation rule (nominalization from a verb), and, most importantly in that language which has relatively few simplex verbs, that light verb constructions are used to turn a noun into a verb. }

\begin{document}
\maketitle
\label{chap-complex-predicates}


{\avmoptions{center}

\section{What are Complex Predicates?}

\subsection{Definition}\label{GSsection1.1}


The term \textit{complex predicate} does not have a universally accepted definition. In the HPSG tradition, a complex predicate is composed of two or more words, which are predicates. By predicate, we mean either a verb or a word of a different category (noun, adjective, preposition) which is associated with an argument structure. A complex predicate is a construction in which the head attracts the arguments of the other predicate, which is its complement: the arguments selected by the complement predicate ``become'' the arguments of the head \citep{hinrichs1989flipped, HN98a}. The phenomenon is called \textit{argument attraction}, \textit{composition}, \textit{inheritance} or \textit{sharing}.

To take an example, tense auxiliaries and the participle in Romance languages are two different words, since they can be separated by adverbs, as in (\ref{GSexemple1}), but the two verbs belong to the same clause, and, more precisely, the syntactic arguments belong to one argument structure. We admit that the property of monoclausality can manifest itself differently in different languages \citep{Butt2010a}. In the case of Romance auxiliary constructions, the first verb (the auxiliary) hosts the clitics which pronominalize the arguments of the participle: corresponding to the NP complement in (\ref{GSexemple1a}), the pronominal clitic \textit{l(e)} is hosted by the auxiliary \textit{a} in (\ref{GSexemple1b}), (\ref{GSexemple1c}). This contrasts with the construction of a control verb such as \textit{vouloir} (`to want') in French, where the clitic corresponding to the argument of the infinitive is hosted by the infinitive: 

\begin{exe}
	\ex \label{GSexemple1} 
	\begin{xlist}
		\ex[]{
		Paul a rapidement lu son livre.\jambox*{(\ili{French})}
		\glt `Paul has quickly read his book'} \label{GSexemple1a} 
		
		\ex[]{
		\gll Paul l'a rapidement lu.\\
			Paul it-has quickly read\\} \label{GSexemple1b}  
		
		\ex[*]{
		\gll Paul a rapidement le-lu.\\
		Paul has quickly it-read\\} \label{GSexemple1c} 
	\end{xlist}
\end{exe}


\begin{exe}
	\ex \label{GSexemple2} 
	\begin{xlist}
		\ex[]{
		Paul veut lire son livre.\jambox*{(\ili{French})}
		\glt `Paul wants to read his book'}\label{GSexemple2a}
		
		\ex[]{
		\gll Paul veut le lire. | * Paul le veut lire.{\footnotemark}\\
		Paul wants it read\\ \footnotetext{Possible in an earlier stage of French.}}\label{GSexemple2b}
		
		
	\end{xlist}
\end{exe}


This definition of complex predicate goes back to Relational Grammar \citep{aissen1983clause}: although formalized in a different way, their analysis of causative constructions in Romance languages relies on such argument attraction, under the name of \textit{clause union}. Similarly, in Lexical Functional Grammar, \cite{andrews1999complex} speak of complex predicates as building a domain of grammatical relations sharing. It is also present in Categorial Grammar \citep{geach1970program}, with complex categories whose definition takes into account the nature of the argument they combine with, and the operation of function attraction. In particular, \cite{kraak1998deductive} accommodates complex predicates by introducing a specific mode of combination called \textit{clause union mode}, where two verbs (two lexical heads) are combined. But, in this account, there is no argument attraction in general, the mechanism being specifically defined in order to account for clitic climbing.


There are other definitions of complex predicates. The term has been used to describe the complex content of a word, when it can be decomposed. For instance, the verb \textit{dance} has been analyzed as incorporating the noun \textit{dance} and considered a ``complex predicate'' \citep{hale1997complex}. In the sense adopted here, complex predicates involve at least two words, and are syntactic constructions. Closer to what we consider here complex predicates is the case of Japanese passive or causative verbs, illustrated in (\ref{GSexemple3}).

\begin{exe}
	\ex[]{
		\gll tabe -rare -sasete -i -ta.\\
		eat \textsc{-pass} \textsc{-caus} \textsc{-progr} \textsc{-past}\\\jambox*{(\ili{Japanese})}
		\glt `(someone1) was causing (something) to be eaten'}\label{GSexemple3}
\end{exe}

The causative morpheme adds a causer argument, and behaves as if it took the verb stem as its complement (more precisely the verb stem with the passive morpheme in this case), whose expected subject appears as the object of the causative verb. This operation is like argument attraction. However, it happens in the lexicon rather than in syntax: the elements in (\ref{GSexemple3}) are bound morphemes, and they form a word \citep{manning1999lexical, gunji2012topics}. Thus, we do not consider causative verbs in Japanese to constitute complex predicates.

Complex predicates are sometimes given a semantic definition: the two elements together describe one situation \citep{butt1995structure}. Such a semantic definition does not coincide with the syntactic one. It is true that the head verb of a complex predicate tends to add tense, aspectual or modal information while the other element describes a situation type. Thus, in (\ref{GSexemple1}), the two verbs jointly describe one situation, the auxiliary adding tense and aspect information. But the semantics of a complex predicate is not always different from that of ordinary verbal complements. Thus, there is no evident semantic distinction depending on whether the Italian restructuring verb \textit{volere} (`want') is the head of a complex predicate (\ref{GSexemple4a}) or not (\ref{GSexemple4b}), and the two verbs do not seem to describe just one situation \citep{Monachesi98a}.  

\begin{exe}
	\ex \label{GSexemple4} 
	\begin{xlist}
		
		\ex[]{ 
		\gll  Anna lo vuole comprare.\\ 
		Anna it wants buy \\\jambox*{(\ili{Italian})}
		\glt `Anna wants to buy it'}\label{GSexemple4a}
		
		\ex[]{ 
		\gll Anna vuole comprarlo.\\
		Anna wants buy-it\\}\label{GSexemple4b} 
		
	\end{xlist}
\end{exe}

The same point is made for Hindi in \cite{poornima2009hindi}. They show that there exist two structures combining an aspectual verb and a main verb; in one of them the aspectual verb is the head of a complex predicate while, in the other one, it is a modifier of the main verb. In more general terms, complex predicates show that syntax and semantics are not always isomorphic in a language. Thus, although the semantic definition of complex predicates may be useful for some purposes, we will ignore it here.

The distinction between complex predicates and \textit{Serial verb constructions} (SVC), illustrated in (\ref{GSexemple5}) (from \citealt{MH2016}), is not evident (e.g. \citealt{andrews1999complex, MH2016}). The main reason is that the constructions which have been dubbed SVC are different in different languages: following \cite{andrews1999complex}, they do not share a grammatical mechanism, but more superficial tendencies, such as their resemblance to paratactic constructions, due to the absence of marking of complementation or coordination, and involve more semantic relations than are usually associated with complementation and coordination.

\begin{exe}
	\ex[]{
	\gll \`Oz\'o s\`a\'an  rr\'a  \'ogb\`a.\\
	Ozo jump cross fence\\\jambox*{(\ili{Edo})}
	\glt `Ozo jumped over the fence'}\label{GSexemple5} 
\end{exe}

\subsection{Constructions involving Complex Predicates}\label{GSsection1.2}

Complex predicates enter into a number of constructions across languages. They differ from ordinary constructions by different properties, depending on the construction, such as the position of pronominal clitics in Romance languages (`clitic climbing'), word order, or special semantic combinations. 

The following have been particularly studied in HPSG:

\begin{itemize}
	
	\item Tense auxiliaries, the copulas and other verbs taking predicative complements, restructuring verbs, headed by certain subject raising or control verbs, and certain causative and perception verbs in Romance languages \citep{abeille1994complementation, abeille2000french, abeille2001deux, abeille2001varieties, AG2002b-u, AG2010, abeille1995doublestructure, abeille1998romance, AGS1998, Monachesi98a};
	
	\item Certain constructions in German and Dutch, called coherent constructions, headed by tense auxiliaries, certain raising and control verbs, certain verbs with predicative complements, as well as the copula and particle verbs \citep{hinrichs1989flipped, HN94a, Rentier94, Kiss94, kiss1995infinite, BvN98a, HN98a, Kathol98b, kathol2000linear, Meurers2000, DM2002, dKM2001a,  Mueller2002b, muller2003solving, muller2018clause};
	
	\item Causatives in different languages (among which German, Italian, Turkish), including both analytical causatives (complex predicates in the sense adopted her) and synthetic causatives \citep{webelhuth1998causatives}. 
	
	\item Korean auxiliaries, control verbs and \textit{ha} causative verb \citep{CC1998, Sells1991, Yoo2003, Kim2016a-u};
	
	\item Hindi aspectual predicates \citep{poornima2009hindi}. 
	
	\item Light verb constructions (combination of a semantically light verb with a predicate belonging to diverse categories) in Persian \citep{bonami2010persian, muller2010persian, pollet2012grammaire, bonami2015diversity}, and Korean \citep{Ryu:93, lee2001argument, choi2001mixed, Kim2016a-u}.  
	
\end{itemize}


In this chapter, we examine some of these constructions which illustrate the different ways in which complex predicates differ from ordinary verbs with their complements.


\section{The basic mechanism in HPSG: argument attraction}\label{GSsection2}

In HPSG, complex predicates are analyzed in the following way: one of the predicates is the head of the construction, and it attracts the syntactic arguments of the other predicate, that is, its complements and, possibly, its subject. The phenomenon is called \textit{argument attraction}, \textit{composition}, \textit{inheritance}, \textit{raising}, or \textit{sharing}. We illustrate it with tense auxiliaries in French \citep{abeille1994complementation, AG2002b-u}.

In French, auxiliary constructions consist of a tense auxiliary (\textit{avoir} or \textit{\^etre}) followed by a past participle and its complements as illustrated in (\ref{GSexemple1}). The auxiliary is the head: it bears inflectional affixes (for tense and person), like any other verb; in (\ref{GSexemple1}), it has the form of a present indicative, 3\textsuperscript{rd} person; it is in the indicative, as expected in a declarative sentence. It hosts pronominal clitics, like verbal heads in general (\ref{GSexemple1b}), (\ref{GSexemple1c}). Moreover, it can be gapped alone (\ref{GSexemple6a}), while the participle can only be gapped with the auxiliary (\ref{GSexemple6b}), (\ref{GSexemple6c})\footnote{Note that (\ref{GSexemple6c}) is acceptable with the possession verb \textit{avoir}.}; this is expected if the auxiliary is the head, since it behaves like \textit{pense} (`think') (\ref{GSexemple6d}) while the participle behaves like the infinitive in (\ref{GSexemple6e}), (\ref{GSexemple6f}). 

\begin{exe}
	\ex \label{GSexemple6}
	\begin{xlist}
		
		\ex[]{
		Lola a achet\'e des pommes, et Alice (a) cueilli des p\^eches.
		\glt `Lola has bought apples, and Alice (has) picked peaches'}\label{GSexemple6a}
		
		\ex[]{
		Lola a achet\'e des pommes, et Alice (a achet\'e) des p\^eches.
		\glt `Lola has bought apples, and Alice (has bought) peaches'} \label{GSexemple6b}
		
		\ex[\#]{
		Lola a achet\'e des pommes, et Alice a des p\^eches.
		\glt `Lola has bought apples, and Alice has peaches'} \label{GSexemple6c}
				
		\ex[]{
		Lola pense acheter des pommes, et Alice (pense) cueillir des p\^eches.
		\glt `Lola is thinking of buying apples, and Alice (is thinking of) picking peaches'} \label{GSexemple6d}
		
		\ex[]{
		Lola pense acheter des pommes, et Alice (pense acheter) des  p\^eches.
		\glt `Lola is thinking of buying apples, and Alice (is thinking of picking) peaches'}\label{GSexemple6e}
		
		\ex[*]{
		Lola pense cueillir des pommes et Alice pense des p\^eches.
		\glt `Lola is thinking of picking apples and Alice is thinking of (picking) peaches'} \label{GSexemple6f}
	\end{xlist}
\end{exe}

The auxiliary construction in French is a complex predicate: The clitic corresponding to a complement of the participle is hosted by the auxiliary (it is said to``climb'') as in (\ref{GSexemple1b}); moreover, it occurs in bounded dependencies such as the infinitival complement of adjectives like facile `easy', whose nominal complement is unexpressed (\ref{GSexemple7a}); this unexpressed complement can be that of a participle (\ref{GSexemple7c}) but not that of an infinitive complement (\ref{GSexemple7b}). This follows if this unexpressed complement is in fact treated as the complement of the auxiliary.


\begin{exe}
	\ex \label{GSexemple7}
	\begin{xlist}
		
		\ex[]{
		Cette technique est impossible \`a ma\^itriser en un jour. \jambox*{(\ili{French})}
		\glt `This technique is impossible to master in one day'} \label{GSexemple7a}
		
		\ex[*]{
		Cette technique est impossible \`a r\'eussir \`a ma\^itriser en un jour.
		\glt (intended) This technique is impossible to manage to master in one day} \label{GSexemple7b}
		
		\ex[]{
		Cette technique est impossible \`a avoir ma\^itris\'e en un jour.
		\glt `This technique is impossible to have mastered in one day'} \label{GSexemple7c}
		

	\end{xlist}
\end{exe}

These two properties follow if the complements of the participle become those of \textit{avoir}: the auxiliary ``attracts'' the complements of the participle. In addition, the tense auxiliary \textit{avoir} is a subject raising verb (see chap ?): the subject is selected by the participle and shared by the auxiliary. For instance, \textit{Paul} is an agent in (\ref{GSexemple1a}) (\textit{Paul a lu son livre}, `Paul has read his book') because \textit{lire} (`to read') requires an agent subject, and it is the impersonal subject \textit{il} in \textit{Il a fait froid} (lit. It has made cold, `It [the weather] was cold'), because the subject of \textit{faire froid} is impersonal. Thus, the auxiliary \textit{avoir} (like tense auxiliary \textit{\^etre}) is, in fact, a generalized raising verb: its whole argument structure is identified with that of the participle. A simplified description of subject raising verbs and tense auxiliaries is given in (\ref{GSexemple8}) (for the feature [LIGHT$\pm$], see Section~\ref{GSsection3}).

\begin{exe}
	\ex 	\label{GSexemple8}
	\begin{xlist}
        \ex[]{Subject raising verb:	\begin{avm}
		\[{} arg-st \ibox{1} $\oplus$ \< \,\[{}subj \ibox{1} \]\, \> \,$\oplus$ \ibox{2} \] 
	\end{avm}\label{GSexemple8a}
	}
	
	    \ex[]{Tense auxiliary:	\begin{avm}
		\[arg-st \ibox{1} $\oplus$ \< \,\[{}light +\\ arg-st \ibox{1} $\oplus$ \ibox{2}\\ \]\, \> \,$\oplus$ \ibox{2} \] 
	\end{avm}\label{GSexemple8b}
	}
	\end{xlist}
\end{exe}

The subject raising verb takes a complement saturated complement, which is described as the second element of the argument structure, expecting a subject \ibox{1} identified with the subject of the raising verb. The notation \ibox{1} without $\left< \, \right>$ indicates that this element may be absent: it is meant to accommodate subjectless verbs. In addition, the raising verb may have its own complements, noted here \ibox{2}. On the other hand, the auxiliary is not only a subject raising verb, but takes as a complement a participle which has not combined with any complements.

The arguments of a word are made up of subject and complements. The relation between (expected) arguments and realized subject and complements, is as in (\ref{GSexemple9}) (see \citealt{ginzburg2000interrogative, bouma2001satisfying}). The arguments include the subject, the complements (and the specifier), but also a list of non-canonical elements (possibly empty) (see below).

\begin{exe}
        \ex[]{Argument Realization Principle\\	
        \begin{avm}
		{{\rm \textit{word}} $\Rightarrow$ \[ synsem \| loc
		            \[cat \[subj & \ibox{1}\\ comps & \ibox{2}\\ spr & \ibox{3}\\ \]\\
		            arg-st \ibox{1} $\oplus$ \ibox{2} $\oplus$ \ibox{3} $\bigcirc$ {\rm list (\textit{non-canon})}\\ \]
		        \]}
          	\end{avm}\label{GSexemple9}
	}
\end{exe}

In (\ref{GSexemple10a}), the participle \textit{lu} selects the argument son \textit{livre}, which is attracted by the auxiliary \textit{a}.
Accordingly, it is realized as the complement of the auxiliary \textit{a}. The structure of the VP in (\ref{GSexemple10a}) is given in Figure~\ref{GSfigure1}.

\begin{exe}
	\ex \label{GSexemple10}
	\begin{xlist}
		\ex[]{
		Marie a lu son livre.
		\glt `Marie has read her book'}\label{GSexemple10a}
		
		\ex[]{
		\gll Marie l'a lu.\\
		Marie it-has read\\
		\glt `Mary has read it'}\label{GSexemple10b}
	\end{xlist}
\end{exe}
%%%%%%%%%%%%%%%%%%%%%%%%%%%%%%%%%%%%%%%%%%%%%%%%%%%%%%%%%%%%%%%%%

\begin{figure}
    {\centering
\begin{forest}
 [VP
 [V [\ms{
            head & \ms{\normalfont{\textit{basic verb}}\\
                        vform \normalfont{\textit{indic.}}}\\
            subj & \liste{ \ibox{1} } \\
            comps & \liste{ \ibox{3}, \ibox{2} }\\
            arg-st & \liste{ \ibox{1}, \ibox{3}, \ibox{2} }
            }[a\\has, align=center, base=bottom]]] 
 [\ibox{3} V [\ms{
            head & \ms{\normalfont{\textit{basic verb}}\\
                        vform \normalfont{\textit{pst-ptcp}}}\\
            subj & \liste{ \ibox{1} }\\
            comps & \liste{ \ibox{2} }\\
            arg-st & \liste{ \ibox{1}, \ibox{2} }
            }[lu\\read, align=center, base=bottom, tier=word]]]
 [\ibox{2} NP, before computing xy={s'+=30pt} 
            [son livre\\her book, base=bottom, tier=word, roof]]]
\end{forest}} \caption{}
    \label{GSfigure1}
\end{figure}

%%%%%%%%%%%%%%%%%%%%%%%%%%%%%%%%%%%%%%%%%%%%%%%%%%%%%%%%%%%%%%%%%
\begin{figure}
\begin{forest}
	for tree={font=\itshape}
 [synsem
 [non-canon
    [aff]
    [gap]
    [null-pro]]
 [canon]]
\end{forest}
\caption{Provide caption}\label{GSexemple11}
\end{figure}

Let us turn to pronominal clitics. The arguments are \textit{synsems}, which can have different subtypes (\ref{GSexemple11}). Usually, they are not specified on the lexeme description, but they are on
words.

Romance clitics, illustrated by \textit{l(e)} in (\ref{GSexemple10b}), are analyzed as affixes (\textit{aff}) on verbs, which correspond to arguments of the verb \citep{miller1997french}. They belong to the argument structure of the participle, and are attracted by the auxiliary, although they are not realized as complements. In (\ref{GSexemple10b}) and Figure~\ref{GSfigure2}, the arguments of the auxiliary are the subject \ibox{1}, the participle \ibox{3}, and \ibox{2}; \ibox{2} is typed as an affix, 3\textsuperscript{rd} person, masculine singular. It belongs to the argument structure, but not to the complement list of the auxiliary (see (\ref{GSexemple9})).


We distinguish between \textit{basic verbs} and \textit{reduced verbs}, following \cite{AGS1998}. With basic verbs, the argument list is simply the concatenation of the subject and complements, while reduced verbs have at least one affix argument which belongs to the argument list, but not to the complement list. Such verbs are subject to a morphological rule which realizes this affixal argument as an affix, the so-called clitic pronoun \textit{l(e)}.

In French, past participles never host clitics (\ref{GSexemple1c}), which we assume to be a morphological property. But, in Italian, past participles may host clitics, although never when they combine
with the auxiliary. The specification that the participle complement of the auxiliary is a basic verb accounts for this property, because basic verbs are not the target of the morphological rule realizing the affixal argument as an affix. Although both verbs in Figure~\ref{GSfigure2} have an affixal argument, one is a basic verb (the participle), the affixal argument being also an expected complement, and the other is a reduced verb (the auxiliary), this affixal argument not being an expected complement.

%%%%%%%%%%%%%%%%%%%%%%%%%%%%%%%%%%%%%%%%%%%%%%%%%%%%%%%%%%%%%%%%%

\begin{figure}
    {\centering
\begin{forest}
 [VP
 [V [\ms{
            head & \textit{reduced verb}\\
            subj & \liste{ \ibox{1} }\\
            comps & \liste{ \ibox{3} }\\
            arg-st & \liste{ \ibox{1}, \ibox{3}, \ibox{2} }
            }[l'a\\it-has, align=center, base=bottom]]] 
 [\ibox{3} V [\ms{
            head & \textit{basic verb}\\
            subj & \liste{ \ibox{1} }\\
            comps & \liste{ \ibox{2} }\\
            arg-st & \liste{ \ibox{1}, \ibox{2}, \normalfont{[\textit{aff, 3\textsuperscript{rd}, msg}]} } }
            [lu\\read, align=center, base=bottom, tier=word]]] ]
\end{forest}}\caption{}
    \label{GSfigure2}
\end{figure}

%%%%%%%%%%%%%%%%%%%%%%%%%%%%%%%%%%%%%%%%%%%%%%%%%%%%%%%%%%%%%%%%%

\section{Different structures for Complex Predicates: Restructuring verbs and the copula in Romance languages}\label{GSsection3}

In addition to tense auxiliaries, Romance languages have other cases of complex predicates: they are headed by restructuring verbs, by the copula and other verbs taking predicative complements, and by certain causative and perception verbs. We focus here on restructuring
verbs and the copula. An analysis of causative and perception verbs is proposed in \citep{abeille1995doublestructure, abeille1998romance, AG2010}. 

A comparison of the properties of constructions headed by restructuring verbs in different Romance Languages illustrates an important aspect of the phenomenon: argument attraction is compatible with different syntactic structures. Restructuring verbs enter either a flat structure, or a verbal complex \citep{Monachesi98a, abeille2001deux, AG2010}. As for the copula, it differs from tense auxiliaries and restructuring verbs in two respects: its complement always behaves like a phrase, although it can be fully saturated for its complements, partially saturated, or not saturated at all \citep{abeille2001varieties, AG2002b-u}; and it has a uniform behavior and analysis across the Romance Languages.

\subsection{Romance Restructuring verbs as head of complex predicates} \label{GSsection3.1}

Certain verbs in Romance Languages, called Restructuring verbs, exhibit two behaviors: either as ordinary verbs taking a VP complement or as heads of complex predicates \citep{rizzi1982issues, aissen1983clause}. Restructuring verbs are modal, aspectual, or movement verbs (such as \textit{venire} `to come’, \textit{andare} `to go’, \textit{correre} `to run’, \textit{tornare} `to come back’ in Italian). However, it must be kept in mind that this behavior is lexical: verbs which are close semantically may not be heads of complex predicates. 

Several properties show that they may head complex predicates \citep{Monachesi98a}. First, clitic climbing, which is optional (while it is obligatory with tense auxiliaries). The examples in (\ref{GSexemple12}) all mean `John wants to eat them’ (examples from \citealt{AG2010}). For each language, the first example illustrates the complex predicate, and the second one the VP complement construction, with the clitic downstairs.

\begin{exe}
	\ex \label{GSexemple12} 
	\begin{xlist}
		\ex[]{
		\gll Giovanni {\textit{le} vuole} mangiare.\\
			Giovanni them-wants eat	\\\jambox*{(\ili{Italian})}} \label{GSexemple 12a}  
		
		\ex[]{
		\gll Giovanni vuole mangiar\textit{le}. \\
		Giovanni wants eat-them\\ } \label{GSexemple12b} 
	
		\ex[]{
		\gll Juan {\textit{las} quiere} comer.\\
		Juan them-wants eat 	\\\jambox*{(\ili{Spanish})}} \label{GSexemple12c} 
		
		\ex[]{
		\gll Juan quiere comer\textit{las}.	\\
		Juan wants  eat-them  \\ } \label{GSexemple12d} 
		
		\ex[]{
		\gll O Jo\~ao quere-\textit{as} comer.	\\
		O Jo\~ao wants-them eat \\\jambox*{(\ili{Portuguese})}} \label{GSexemple12e} 
		
		\ex[]{
		\gll O Jo\~ao quer    com\^e-\textit{las}.\\
		O Jo\~ao wants eat-them\\} \label{GSexemple12f} 
		
		\ex[]{
		\gll En Joan {\textit{les} vol} menjar.	\\
		En Joan	them-wants eat \\\jambox*{(\ili{Catalan})}} \label{GSexemple12g} 
		
		\ex[]{
		\gll En Joan vol menjar-\textit{les}.\\
		En Joan wants eat-them   \\} \label{GSexemple12h} 
	\end{xlist}
\end{exe}

Second, the medio-passive or middle \textit{si} construction, where the verb hosts the reflexive clitic \textit{si} or \textit{se} (\ref{GSexemple13b}) (depending on the language), and the subject corresponds to the object of the active construction (\ref{GSexemple13a}), with an interpretation close to that of passive. The construction is possible with restructuring verbs such as \textit{potere} (\ref{GSexemple13c}), (\ref{GSexemple13d}), but not with verbs only taking an infinitival complement such as \textit{parere} (\ref{GSexemple13e}).


\begin{exe}
	\ex \label{GSexemple13} 
	\begin{xlist}
        \ex[]{
		Giovanni stira queste camicie facilmente.\jambox*{(\ili{Italian})}
		\glt `Giovanni irons these shirts easily´} \label{GSexemple13a} 
		
		\ex[]{ 
		\gll Queste camicie {si stirano} facilmente.\\ 
		These shirts \textsc{si}-iron easily\\
		\glt `These shirts iron easily'}\label{GSexemple13b}
		
		 \ex[]{
		Giovanni pu\`o stirare queste camicie facilmente.
		\glt `Giovanni can iron these shirts easily'} \label{GSexemple13c} 
		
		\ex[]{ 
		\gll Queste camicie {si possono} stirare facilmente.\\ 
		These shirts \textsc{si}-can iron easily\\
		\glt `These shirts can be ironed easily'}\label{GSexemple13d}		
		
		\ex[*]{ 
		\gll Queste camicie {si paiono} stirare facilmente.\\ 
		These shirts \textsc{si}-appear iron easily\\
		\glt (intended) These shirts appear to be ironed easily}\label{GSexemple13e}	
		
	\end{xlist}
\end{exe}


The medio-passive verb alternates with a transitive verb: it is the result of a Lexical Rule (\ref{GSexemple14}), which takes a transitive verb like \textit{stirare} (\ref{GSexemple13a}) to give a verb whose subject corresponds to the expected object of the transitive verb, and acquires a reflexive clitic noted as \textit{a-aff} (realized \textit{si} or \textit{se}) as in (\ref{GSexemple13b}) \citep{AGS1998, Monachesi98a}. 

\begin{exe} 
        \ex[]{Medio-Passive Lexical Rule \label{GSexemple14}\\	
        \begin{avm}
		\[{}arg-st \<np, np [\normalfont{\textit{acc}}]\normalfont{\textsubscript{\textit{j}}} \> \, $\oplus$ \ibox{1}\] \, $\Rightarrow$ \[{}arg-st \<np\normalfont{\textsubscript{\textit{j}}}, [\normalfont{\textit{a-aff, acc}}]\normalfont{\textsubscript{\textit{j}}}  \>\, $\oplus$ \ibox{1}\]
	\end{avm}}
\end{exe}

What is crucial here is that the input is a verb taking an accusative NP complement. Hence, a verb taking a VP complement like Italian \textit{potere} or \textit{parere} cannot be the input, since it lacks an NP complement. On the other hand, the corresponding restructuring verb \textit{potere} can since it inherits such a complement from the infinitive: the verb \textit{potere} in (\ref{GSexemple13c}) inherits \textit{queste camicie} from \textit{stirare}, and is the input to Rule (\ref{GSexemple14}), giving the verb which occurs in (\ref{GSexemple13d}). On the other hand, the verb \textit{parere} which is not a restructuring verb does not have an NP object and cannot be the input to this Rule (\ref{GSexemple14}).

The third relevant property is their acceptability in bounded dependencies, as illustrated in (\ref{GSexemple7}) for tense auxiliaries, and (\ref{GSexemple15}) for restructuring verbs. (\ref{GSexemple15b}) relies on \textit{cominciare} being a restructuring verb, while \textit{promettere} is not (\ref{GSexemple15c}).  

\begin{exe}
	\ex \label{GSexemple15} 
	\begin{xlist}
        \ex[]{
		Questa canzone \`e facile da apprendere.\jambox*{(\ili{Italian})}
		\glt `This song is easy to learn'} \label{GSexemple15a} 
		
		\ex[]{ 
		Questa canzone \`e facile da cominciare a apprendere.
		\glt `This song is easy to begin to learn'}\label{GSexemple15b}
		
		 \ex[*]{
		Questa canzone \`e facile da promettere a apprendere.
		\glt (intended) This song is easy to promise to learn} \label{GSexemple15c} 	
	\end{xlist}
\end{exe}

The complement of adjectives such as `easy' in Romance Languages is a bounded dependency: they take an infinitival complement whose own expected complement (we analyze it as a null pronoun, see (\ref{GSexemple11})) (see \citealt{AGS1998, Monachesi98a}). The value of MARKING is that of Italian.

\begin{exe}
        \ex[]{
        \begin{avm}
		{\[head \normalfont{\textit{adjective}}\\
        arg-st \<{}xp\normalfont{\textsubscript{\textit{j}}, \textsc{vp}} 
                    \[vform \normalfont{\textit{infinitive}}\\ 
                    marking \normalfont{\textit{da}}\\
                    comps \<\normalfont{\textit{null-pro}} \[\normalfont{\textit{acc}}\] \,\normalfont{\textit{\textsubscript{j}}}\> \, $\oplus$ \ibox{2}\\ \]
                \, \>\\\]}
          	\end{avm}}
\end{exe}

Complex predicates can occur in this construction because their head attracts the complement of their complement. Thus, in (\ref{GSexemple15b}), \textit{cominciare} is expecting the same object as \textit{apprendere}, which is coindexed with the subject of the copular construction, in the same way as \textit{apprendere} is expecting an object in (\ref{GSexemple15a}). 

Finally, the possibility of preposing the verbal complement of a verb which can take a VP complement or be the head of a complex predicate disappears when there is evidence of a complex predicate. For the sake of simplification, we now concentrate on Italian and Spanish. The data in (\ref{GSexemple17}), with a preposed VP, contrast with those in (\ref{GSexemple18}), where the head verb bears a clitic corresponding to the expected complement of the infinitive. Preposing of the verbal complement is associated with pronominalization (\textit{lo}) in Italian (\ref{GSexemple17a}) not in Spanish (\ref{GSexemple17b}), where it is more natural in contrastive contexts.

\begin{exe}
	\ex{
[Context] Does he want to talk to Mary?}\label{GSexemple17} 
	\begin{xlist}
        \ex[]{
		--Parlare a Maria, certamente lo vuole.\jambox*{(Italian)} \label{GSexemple17a} }
		\ex[]{ 
		--Hablarle a Mar\'ia, seguramente quiere (pero no a su madre).\jambox*{(\ili{Spanish})}
		\glt `Talk to Maria, certainly he wants to (but not to her mother)'}\label{GSexemple17b}
	\end{xlist}
\end{exe}
\begin{exe}
	\ex \label{GSexemple18} 
	\begin{xlist}
        \ex[*]{
		\gll --Parlare, certamente glielo vuole.\\
		Talk certainly to.her-it wants\\\jambox*{(\ili{Italian})}} \label{GSexemple18a} 	
		\ex[*]{ 
		\gll --Hablar, le-quiere (pero no mucho tiempo).\\
		Talk to.her-wants but not for.a.long time\\\jambox*{(\ili{Spanish})}}\label{GSexemple18b}
	\end{xlist}
\end{exe}
We assume that restructuring verbs have two possible descriptions: as ordinary verbs taking an infinitival VP complement, or as heads of complex predicates. They are related by the Argument Attraction Lexical Rules (\ref{GSexemple19}) (adapted from \citealt{Monachesi98a}).\footnote{We leave aside the object control and object raising verbs (verbs of influence or perception verbs) which can also be the head of a complex predicate, hence be the target of a similar Lexical Rule \citep{abeille1998romance, AG2010}, concentrating on the case of so-called Restructuring verbs.}  

\begin{exe}
	\ex {Argument Attraction Lexical Rules for Romance restructuring verbs}\label{GSexemple19} 
	\begin{xlist}
        \ex[]{Subject control verbs \\
        \begin{avm}
		{\[head \normalfont{\textit{verb}}\\
        arg-st \<xp\normalfont{\textsubscript{\textit{i}}}, \ibox{2} 
                    \[head \[\normalfont{\textit{verb}}\\
                    vform \normalfont{\textit{inf.}}\\\]\\ 
                    subj \<xp\normalfont{\textsubscript{\textit{i}}}\>\\
                    comps \< \>\]\,\>\, $\oplus$ \ibox{3}\\\]}
          	\end{avm}\label{GSexemple19a}\vspace{.2cm} \\ 
          	$\Rightarrow$ \\
        \begin{avm}
		{\[arg-st \<xp\normalfont{\textsubscript{\textit{i}}}, v
                    \[\normalfont{\textit{basic-verb}}\\
                    light + \\
                    comps \ibox{4}\\\]\,\> \,$\oplus$ \ibox{4} $\oplus$ \ibox{3}\]}
          	\end{avm}}
	
	    \ex[]{Subject raising verbs\\
        \begin{avm}
		{\[head \normalfont{\textit{verb}}\\
        arg-st \ibox{1} $\oplus$ \<\ibox{2}
                    \[head \[\normalfont{\textit{verb}}\\
                    vform \normalfont{\textit{inf.}}\\\]\\ 
                    subj \ibox{1}\\
                    comps \< \>\]\,\>\, $\oplus$ \ibox{3}\\\]}
          	\end{avm}\label{GSexemple19b}\vspace{.2cm} \\ 
          	$\Rightarrow$ \\
        \begin{avm}
		{\[arg-st \ibox{1} $\oplus$ \<v
                    \[\normalfont{\textit{basic-verb}}\\
                    light + \\
                    comps \ibox{4}\\\]\,\> \,$\oplus$ \ibox{4} $\oplus$ \ibox{3}\]}
          	\end{avm}}
	\end{xlist}
\end{exe}

In the input description, the verbal complement is saturated for its complements. The verb may have other complements in addition to the saturated infinitival VP, noted as list \ibox{3}. We distinguish between subject control verbs and subject raising verbs to accommodate the case where the complement verb is subjectless, but with complements that can be attracted. In (\ref{GSexemple20a}), the verb \textit{sembra} is a raising verb, and the infinitive \textit{piacere} is an impersonal with no subject, but with a complement, realized by \textit{gli} on the head verb \textit{sembra} (there is another interpretation where \textit{gli} is the complement of \textit{sembra}, which is irrelevant).\footnote{Alternatively, in a grammar with null pronouns, impersonal and unaccusative verbs in Romance Languages could be analyzed as having a null pronoun subject, a representation which allows a common input for subject control and raising verbs in the Argument Composition Lexical Rule (as in \citealt{Monachesi98a}).} Note that there is speakers' variation: \textit{sembrare} is not a restructuring verb for all Italian speakers (hence \% on the examples).   

The category of the subject of control verbs is not specified: it can be an infinitival VP as well as an NP (or even a sentence); in the first case, the index is that of the situation (\ref{GSexemple20c}), in the second, it is the index of the nominal entity (\ref{GSexemple20b}). Again, the upstairs clitic \textit{gli} corresponds to the argument of \textit{piacere}:

\begin{exe}
	\ex \label{GSexemple20} 
	\begin{xlist}
        \ex[\%]{
		\gll Gli sembra piacere molto.\\ 
		To.him seems please a.lot\\\jambox*{(\ili{Italian})}
		\glt `It seems that he likes it a lot'}\label{GSexemple20a}

		\ex[\%]{ 
		\gll [Questo regalo] gli sembra piacere.\\ 
		[This gift] to.him seems please\\
		\glt `This gift seems to please him/her'}\label{GSexemple20b}		
		
		\ex[\%]{ 
		\gll [Andare in vacanza] gli sembra piacere\\
		[Go.away on vacation] to.him seems please\\
		\glt `To go away on vacation seems to please him/her'}\label{GSexemple20c}	
	\end{xlist}
\end{exe}

\subsection{The different structures of complex predicates with Restructuring verbs} \label{GSsection3.2}

The point of this section is to show that argument attraction is compatible with different structures: these are two different phenomena. In Romance Languages, restructuring verbs enter either a flat structure or a verbal complex. We contrast Italian and Spanish.\footnote{In Portuguese restructuring verb constructions are also a flat structure, but with different ordering constraints from Italian; the other variety of Spanish is similar to Portuguese. Except for the copula (see Section~\ref{GSsection3.4}), complex predicate constructions with head verbs entering only one structure also distribute between these two structures among Romance Languages: tense auxiliaries in French, Italian, Portuguese as well as Romanian modal \textit{a putea} (`can') are the head of a flat structure, while tense auxiliaries in the variety of Spanish described here and in Romanian enter a verbal complex \citep{AG2010}.} Note that in Spanish, there is variation among speakers: we describe here one usage of Spanish complex predicates. 

The impossibility of Preposing illustrated in (\ref{GSexemple18}) shows that the sequence of the complement verb and its complements does not form a constituent (a VP) when there is a complex predicate, a point made by \cite{rizzi1982issues} for Italian, on the basis of a series of constructions (pied-piping, clefting, Right Node Raising, Complex NP shift). However, the two languages differ with respect to other properties. The presence of a clitic on the head verb indicates that there is a complex predicate.
 
First, adverbs occur between the restructuring verb and the infinitive in Italian (\ref{GSexemple21a}), but not in Spanish in a general way (\ref{GSexemple21b}) (a few adverbs, such as \textit{casi} `nearly', \textit{ya} `already', \textit{apenas} `barely' are possible). In Spanish, an adverb may occur after the verb and before the infinitive if the complement is a VP (\ref{GSexemple21c}).

\begin{exe}
	\ex \label{GSexemple21} 
	\begin{xlist}
        \ex[]{
		\gll Giovanni \textit{lo} vuole spesso leggere.\\ 
		Giovanni it wants often read\\\jambox*{(\ili{Italian})}
		\glt `Giovanni wants to read it often'}\label{GSexemple21a}

		\ex[*]{ 
		\gll Juan \textit{lo} quiere {a menudo} leer.\\ 
		Juan it wants often read\\\jambox*{(\ili{Spanish})}
		\glt (intended) Juan wants to read it often}\label{GSexemple21b}		
		
		\ex[]{ 
		\gll Juan quiere {a menudo} leer\textit{lo}.\\
		Juan wants often {read-it}\\
		\glt `Juan wants to read it often'
		}\label{GSexemple21c}	
	\end{xlist}
\end{exe}

Second, an inverted subject NP can occur between the two verbs of a complex predicate in Italian (\ref{GSexemple22a}), but not in Spanish (\ref{GSexemple22b}). The subject can occur postverbally in interrogative sentences. In Italian, it can occur between the two verbs with a special prosody, indicated by the small capitals in (\ref{GSexemple22a}), and with speaker’s variation \citep{salvi1980ausiliari}. In Spanish, this is not possible (except for the pronominal subject) \citep{suner1982syntax}.

\begin{exe}
	\ex \label{GSexemple22} 
	\begin{xlist}
        \ex[\%]{
		\gll Lo comincia \textsc{Maria} a capire, il problema, oppure no?\\ 
		It begins Maria to understand, the problem, or no\\\jambox*{(\ili{Italian})}
		\glt `Maria, she's beginning to understand it, the problem, yes or no'}\label{GSexemple22a}

		\ex[*]{ 
		\gll ?`Lo comienza Juan a comprender?\\
		It begins Juan to understand\\\jambox*{(\ili{Spanish})}}\label{GSexemple22b}		
		
		\ex[]{ 
		\gll ?`Comienza Juan a comprenderlo?\\
		Begins Juan to understand-it\\
		\glt `Is Juan beginning to understand it?'}\label{GSexemple22c}	
	\end{xlist}
\end{exe}

Finally, Italian heads of complex predicates can have scope over the coordination of infinitives with their complements (\ref{GSexemple23a}), while this is not the case in Spanish (\ref{GSexemple23b}) (from \citealt{AG2010}). Again, the presence of a clitic on the head verb (\textit{lo vuole}, \textit{le volvi\'o}) shows that this is a complex predicate construction.

\eal
\label{GSexemple23} 
\ex[\%]{
\gll Giovanni lo vuole comprare subito      e   dare a  Maria.\\ 
     Giovanni it wants buy      immediately and give to Maria\\\jambox*{(\ili{Italian})}
\glt `Giovanni wants to buy it immediately and give it to Maria.'
}\label{GSexemple23a}

\ex[*]{ 
\gll Le     volvi\'o      a  pedir un aut\'ografo y   a  hacer proposiciones.\\
     to.him started.again to ask   an autograph   and to make  propositions\\\jambox*{(\ili{Spanish})}
\glt Intended: `He started again to ask him for an autograph and to make propositions to him.'
}\label{GSexemple23b}	
\zl

Constituency tests such as preposing (\ref{GSexemple18}) show that the verbal complement is not a VP in either language. The verbal complex, in which the two verbs form a constituent without the complements is well-suited to account for the absence of adverbs (in a general way) and of subject NPs, if such combinations exclude elements other than verbs (adverbs in particular). This constraint can be captured by the feature [LIGHT +]\footnote{The adverbs admissible in the Spanish verbal complex are light.}, which has been used in Romance languages for other phenomena as well \citep{abeille2000french} (see Section~\ref{GSsection3.3}). Hence, complex predicate constructions in Spanish contain a verbal complex, while they form a flat structure in Italian containing the complement verb and its complements. 

This is illustrated with examples in Figures \ref{GSfigure3}, which all mean `Marco wants to give it to Maria'. The verb takes a VP complement in Figure~\ref{GSfigure3a} in both languages, it is the head of a flat VP in Italian in Figure~\ref{GSfigure3b}, and enters a verbal V-V complex in Spanish in Figure~\ref{GSfigure3c}.

\begin{figure}
\begin{subfigure}[b]{.495\textwidth}
\begin{forest} 
for tree={%
    l sep=10pt}
[S
   [SN
      [Marco\\Marco\\Marco\\Marco, align=center, base=bottom, tier=word]]
   [VP
      [V[vuole\\wants\\quiere\\wants, align=center, base=bottom, tier=word]]
      [N[lo-dare a Maria\\it-give to Maria\\darlo a Mar\'ia\\give-it to Maria, align=center, base=bottom, tier=word, roof]]
]]
\end{forest}
\caption{VP complement}
\label{GSfigure3a}
\end{subfigure}
\hfill
\begin{subfigure}[b]{.495\textwidth}
\begin{forest} 
for tree={%
    l sep=19pt}
[S
   [NP
      [Marco\\Marco, align=center, base=bottom, tier=word]]
   [VP
      [V[lo-vuole\\it-wants, align=center, base=bottom, tier=word]]
      [V[dare\\give, align=center, base=bottom, tier=word]]
      [PP[a Maria\\to Maria, align=center, base=bottom, tier=word, roof]]
      ]]
\end{forest}
\caption{Flat structure}
\label{GSfigure3b}
\end{subfigure}
\hfill
\begin{subfigure}[b]{.5\textwidth}
\begin{forest} 
for tree={%
    l sep=10pt}
[S
   [NP
      [Marco\\Marco, align=center, base=bottom, tier=word]]
   [VP
      [V[V [lo-quiere\\it-wants, align=center, base=bottom, tier=word]] [V[dar\\give, align=center, base=bottom, tier=word]]]
      [PP[a Mar\'ia\\to Maria, align=center, base=bottom, tier=word, roof]]
]]]
\end{forest}
\caption{Verbal complex}
\label{GSfigure3c}
\end{subfigure}
\caption{}
\label{GSfigure3}
\end{figure}

The possibility of the coordination in (\ref{GSexemple23a}) has been viewed as an argument in favor of a complement VP even when there is argument attraction \citep{andrews1999complex}. The data go against such an analysis for Spanish, since the coordination is not acceptable. For Italian, although such sequences as (\ref{GSexemple23a}) can be analyzed as coordinations of VP, they can also be Non Constituent Coordinations (NCC) (\textit{John gives a book to Maria and discs to her brother}). So, the question becomes: why is (\ref{GSexemple23b}) not an acceptable NCC in Spanish? We propose, following \citep{AG2010}, that NCC coordinations are subject to a general constraint in Romance languages: the parallel elements of the coordination must be at the same syntactic level, otherwise the acceptability is degraded. An example is the contrast between (\ref{GSexemple24a}) and (\ref{GSexemple24b}) in Spanish, which is similar to that of (\ref{GSexemple23b}), repeated in (\ref{GSexemple24c}), if the structure is that of a verbal complex.

\eal
\judgewidth{??}
\label{GSexemple24} 
\ex[]{
Juan da [el libro de Proust] [a Mar\'ia] y [el (libro) de Camus] [a Pablo].\jambox*{(\ili{Spanish})}}\label{GSexemple24a}
\ex[??]{ 
Juan da [el libro de Proust] [a Mar\'ia] y [de Camus] [a Pablo].}\label{GSexemple24b}
\ex[*]{ 
\gll [Le              volvi\'o      a  pedir] [un          aut\'ografo] y   [a hacer]         [proposiciones].\\
     \spacebr{}to.him started.again to ask    \spacebr{}an autograph    and \spacebr{}to make \spacebr{}propositions\\}\label{GSexemple24c}
\zl

In (\ref{GSexemple24a}) the NP \textit{el de Camus} is parallel to and at the same level as \textit{el libro de Proust}, the PP a \textit{Pablo} is parallel to and at the same level as \textit{a Mar\'ia}, the NP and the PP are both complements of \textit{da}. But, in (\ref{GSexemple24b}), \textit{de Camus} is parallel to \textit{de Proust}, and not at the same level as \textit{el libro de Proust} or as \textit{a Pablo}: \textit{a Pablo} corresponds to the complement of \textit{da} while \textit{de Camus} corresponds to the complement of the noun \textit{libro}. The acceptability is degraded. 

If the structure of a complex predicate is that of a verbal complex in Spanish, the structure of (\ref{GSexemple24c}) is similar to that of (\ref{GSexemple24b}): \textit{a hacer} corresponds to a \textit{a pedir}, which is the complement V of \textit{volvi\'o} in a V-V constituent, and is not at the same level as \textit{proposiciones} which corresponds to \textit{un aut\'ografo} which is outside the V-V constituent.   

\subsection{Analysis of Romance restructuring verb constructions in HPSG} \label{GSsection3.3}

It has been shown in Section~\ref{GSsection3.1} that the different Romance languages all have complex predicate constructions, and, in Section~\ref{GSsection3.2}, that, although they share some properties (such as clitic climbing, occurrence in bounded dependencies), they also show syntactic differences among themselves (separability of the head and the infinitive or participle in Italian, but not in Spanish, possibility of coordination of the complement verb with its complements in Italian, but not in Spanish). The flexibility of HPSG grammar allows us to describe both the commonalities and the differences: the common behavior follows from the fact that they share the mechanism of argument attraction, which characterizes certain classes of verbs; the differences follow from a different phrase structure: the restructuring verb in Italian enters a flat structure (Figure~\ref{GSfigure3b}), while it enters a verbal complex in Spanish (Figure~\ref{GSfigure3c}). This analysis contrasts with that of \cite{andrews1999complex} in LFG, who propose that complex predicates in Romance languages arise when two verbs have a common domain of grammatical functions, but correspond to just one phrase structure, all these verbs taking a VP complement. It is not clear how they can account for the differences between the two languages.

Two phrase structures rules combining a head with its complements account for the distinction between the flat structure and the verbal complex: the usual head-complements phrase, and a different one, the head-cluster phrase, also used in German (see Section~\ref{GSsection4.1.2}). The difference between the flat structure and the verbal complex is attributed to the feature [LIGHT $\pm$]. 

The head-complements-phrase is defined as follows:

\begin{exe}
        \ex[]{\textit{Head-complements-phrase} (Romance languages) $\Rightarrow$ \\	
        \begin{avm}
		{\[ mother | synsem | loc | cat
		            \[head & \ibox{1} \\
		            comps & \ibox{3}\\
		            light & -\\\]\\
		  head-dtr | synsem | loc | cat
		            \[head & \ibox{1} \\
		            light & +\\
		            comps & \ibox{2} $\bigcirc$ \ibox{3}\\\]\\
		  non-head-dtrs \,\,\, \ibox{2} \normalfont{non-empty list}\\
		\]}
          	\end{avm}\label{GSexemple25}}
\end{exe}


%%%%%%%%%%%%%%%%%%%%%%%%%%%%%%%MAS LARG%%%%%%%%%%%%%%%%%%%%%%%%%%%%%%%%%%

\begin{figure}
    \centering
        {\small
\begin{forest}
 [S
 [\ibox{1} NP\textit{j}
            [Marco, tier=word]]
  [VP \ms{
            light & \normalfont{-}\\
            comps & \liste{  }} 
    [V \ms{
            light & \normalfont{+}\\
            comps & \liste{ \ibox{2}, \ibox{3}, \ibox{4} }\\
            arg-st & \liste{ \ibox{1}, \ibox{2}, \ibox{3}, \ibox{4} }
            }[vuole, tier=word]]
    [\ibox{2} V \ms{
            vform & inf. \\
            comps & \liste{ \ibox{3}, \ibox{4} }\\
            arg-st & \liste{ \normalfont{\textsc{np}\textit{j}}, \ibox{3}, \ibox{4} }
            }[dare, tier=word]]
     [\ibox{3} NP
            [questo libro, tier=word, roof]]
     [\ibox{4} PP
            [a Maria, tier=word, roof]]]]
\end{forest}} \caption{}
    \label{GSfigure4}
\end{figure}

%%%%%%%%%%%%%%%%%%%%%%%%%%%%%%%MAS LARG%%%%%%%%%%%%%%%%%%%%%%%%%%%%%%%%%%



The head-complements-phrase is usually saturated for the expected complements, but not always: list \ibox{3} is usually empty, but does not have to be (see the case of the copula in Section~\ref{GSsection3.4}). An example of the flat structure with a restructuring verb is given in Figure~\ref{GSfigure4}.


In the flat structure the head verb takes as complements the infinitival verb and the canonical complements expected by the infinitive, and combines with them (Figure~\ref{GSfigure4}). The VP, corresponding to the Head-complements-phrase is complement saturated.

The verbal complex corresponds to another kind of Head-complements-phrase, called the Head-cluster phrase, given in (\ref{GSexemple26}) (adapted from \citealt{muller2018clause}).\footnote{This rule is also used in Romanian. As in German, we do not specify the category of the complement (which can be a noun in Spanish, for instance).} 

\begin{exe}
        \ex[]{\textit{Head-cluster phrase}  (Spanish) $\Rightarrow$\\	
        \begin{avm}
		{\[ mother | synsem | loc | cat &
		            \[comps & \ibox{1}\\
		            light & +\\\]\\
		  head-dtr | synsem | loc | cat &
		            \[head & \textit{verb} \\
		            comps & \ibox{1} $\oplus$ \<\ibox{2}\>\\
		            light & +\\\]\\
		  non-head-dtrs & \< \, \[synsem \ibox{2} \[light +\]\] \,\>\\
		\]}\label{GSexemple26}
          	\end{avm}}
\end{exe}

It differs from the usual Head-complements phrase on two accounts: there is only one non-head daughter, and all the constituents are [LIGHT +]. The LIGHT feature \citep{bonami2012phrase} renames the WEIGHT feature proposed in \cite{abeille2000french}, as well as the LEX feature used in German (e.g. \citealt{hinrichs1989flipped, HN94a, kiss1995infinite, Meurers2000, Mueller2002b, hohle2018spuren}). The LIGHT feature has ordering as well as structural consequences \citep{abeille2000french, AG2010}. It is appropriate both for words and phrases. Words can be light or non-light; lexical verbs (finite verbs, participles or infinitives without complements) are light. Most phrases are non-light; in particular, the VP, that is, the phrase which combines with the subject in Romance languages, is non-light.\footnote{Note that the head only phrase is non-light. Hence, the VP which dominates a lexical verb only is non-light.} But some phrases can be light if they are composed of light constituents. Such is the case of the head-cluster phrase. 

The Head-cluster phrase is illustrated in Figure~\ref{GSfigure5}: the phrase \textit{quiere dar} corresponds to the Head-cluster-phrase, while the whole VP (\textit{quiere dar aquel libro a Mar\'ia}) corresponds to the usual Head-complements-phrase in (\ref{GSexemple25}).

%%%%%%%%%%%%%%%%%%%%%%%%%%%%%%%MAS LARG%%%%%%%%%%%%%%%%%%%%%%%%%%%%%%%%%%

\begin{figure}
    \centering
        {\small
\begin{forest}
 [S
 [\ibox{1} NP\textit{j}
            [Marco, tier=word]]
  [VP \ms{
            light & \normalfont{-}\\
            comps & \liste{ } } 
    [V \ms{
            light & \normalfont{+}\\
            comps & \liste{ \ibox{3}, \ibox{4} }} 
    [V \ms{
            light & \normalfont{+}\\
            comps & \liste{ \ibox{2}, \ibox{3}, \ibox{4} }\\
            arg-st & \liste{ \ibox{1}, \ibox{2}, \ibox{3}, \ibox{4} }
            }[quiere, tier=word]]
    [\ibox{2} V \ms{
            light & \normalfont{+} \\
            comps & \liste{\ibox{3}, \ibox{4} }\\
            arg-st & \liste{ \normalfont{\textsc{np}\textit{j}}, \ibox{3}, \ibox{4} }
            }[dar, tier=word]]]
     [\ibox{3} NP, before computing xy={s'+=22pt}
            [aquel libro, tier=word, roof]]
     [\ibox{4} PP, before computing xy={s'+=22pt} 
            [a María, tier=word, roof]]]]
\end{forest}} \caption{}
    \label{GSfigure5}
\end{figure}

%%%%%%%%%%%%%%%%%%%%%%%%%%%%%%%MAS LARG%%%%%%%%%%%%%%%%%%%%%%%%%%%%%%%%%%

Regarding the canonical complements in the verbal complex construction, the requirement is passed up by the verbal complex, according to the description in (\ref{GSexemple26}) (the list \ibox{1} is non empty). The verbal complex itself combines with the canonical complements expected by the infinitive (here, \ibox{3} and \ibox{4}).

More has to be said regarding the clitic \textit{lo} in Italian \textit{Marco lo-vuole dare a Maria} and Spanish \textit{Marco lo-quiere dar a Mar\'ia} in Figures \ref{GSfigure3}. The infinitive is a basic verb: there is no difference between the complements and the arguments (except for the subject); its complement list contains an affixal element (see Section~\ref{GSsection2}). Following Rule (\ref{GSexemple19a}), this element is attracted to the argument list of the head verb, but it is not realized as a complement, as is expected given Principle (\ref{GSexemple9}); the head verb is then a reduced verb (see Figures \ref{GSfigure6}), which is the target of a morphological rule of cliticization, hence the clitic \textit{lo} on the head verb \textit{vuole} or \textit{quiere}. 


%%%%%%%%%%%%%%%%%%%%%%%%%%%%%%%MAS LARG%%%%%%%%%%%%%%%%%%%%%%%%%%%%%%%%%%

\begin{figure}
\begin{subfigure}[b]{\textwidth}
\centering
       {\small
\begin{forest}
  [VP \ms{comps & \tuple{}} 
    [V \ms{
            \normalfont{\textit{reduced verb}}\\
            light & \normalfont{+}\\
            comps & \liste{ \ibox{2}, \ibox{4} }\\
            arg-st & \liste{ \normalfont{\textsc{np}\textit{\textsubscript{j}}}, \ibox{2}, \ibox{3}, \ibox{4} } }[lo-vuole, tier=word]]
    [\ibox{2} V \ms{
            \normalfont{\textit{basic verb}}\\
            light & \normalfont{+} \\
            comps & \liste{ \ibox{3}, \ibox{4} }\\
            arg-st & \liste{ \normalfont{\textsc{np}\textit{\textsubscript{j}}}, \ibox{3}\textit{aff}, \ibox{4} }
            }[dare, tier=word]]
     [\ibox{4} PP, before computing xy={l'-=20pt}
            [a Maria, tier=word, roof]]]
\end{forest}}
\caption{}
\label{GSfigure6a}
\end{subfigure}
\hfill
\begin{subfigure}[b]{\textwidth}
\centering
  {\small
\begin{forest}
  [VP \ms{comps & \tuple{}} 
  [V \ms{light & \normalfont{+}\\
            comps & \ibox{4}} 
    [V \ms{
            \normalfont{\textit{reduced verb}}\\
            light & \normalfont{+}\\
            comps & \liste{ \ibox{2}, \ibox{4} }\\
            arg-st & \liste{ \normalfont{\textsc{np}\textit{\textsubscript{j}}}, \ibox{2}, \ibox{3}, \ibox{4} } }[lo-quiere, tier=word]]
    [\ibox{2} V \ms{
            \normalfont{\textit{basic verb}}\\
            light & \normalfont{+}\\
            comps & \liste{ \ibox{3}, \ibox{4} }\\
            arg-st & \liste{ \normalfont{\textsc{np}\textit{\textsubscript{j}}}, \ibox{3} \textit{aff}, \ibox{4} } }[dar, tier=word]]] 
     [\ibox{4} PP, before computing xy={s'+=15pt}
            [a Mar\'ia, tier=word, roof]]]
\end{forest}}
\caption{}
\label{GSfigure6b}
\end{subfigure}
\caption{}
\label{GSfigure6}
\end{figure}

It remains to ensure that Spanish restructuring verbs are characterized by a verbal complex, and Italian ones by a flat structure. We assume an additional constraint on phrases in Spanish. According to (\ref{GSexemple27}), if the phrase is light, it follows that the non-head daughters are also light, and, conversely, if the phrase is non-light, the non-head daughters are non-light.

\begin{exe}
        \ex[]{	
        \begin{avm}
		\[\normalfont{\textit{phrase}}\\
		light \ibox{1}\] \, $\Rightarrow$ \[non-head-dtrs | light \ibox{1}\]
	\end{avm}\jambox*{(\ili{Spanish})}\label{GSexemple27} }
\end{exe}
The structure of the flat VP does not obey this constraint: the infinitival verb which is a non-head daughter is light, while the other complements are non-light (see Figure~\ref{GSfigure4}). When constraint (\ref{GSexemple27}) applies, he head of a restructuring verb cannot enter a flat structure. 

In order to prevent a verbal complex in Italian (and French), we assume a different additional constraint.

\begin{exe}
        \ex[]{	
        \begin{avm}
		\[\normalfont{\textit{phrase}}\\
		light +\] \, $\Rightarrow$ \[non-head-dtrs \normalfont{list} ([\textsc{head} $\neg$ \textit{verbal})]\]
	\end{avm}\jambox*{(\ili{Italian})}\label{GSexemple28} }
\end{exe}

If the structure included a verbal complex in Italian, it would be light, but this is not possible because constraint (\ref{GSexemple28}) says that the non-daughter in a light phrase cannot be verbal, which excludes the participle or the infinitive.

Romance languages follow the general constraints on ordering in non-head final languages. According to constraint (\ref{GSexemple29}), the verb precedes the complements it subcategorizes for. This is relevant not only for the head of the complex predicate, but also for the participle complement of the tense auxiliary or the infinitive complement of a restructuring verb. Although the latter do not combine with their expected complements, they still subcategorize for them.  

\begin{exe} 
        \ex[]{	
        \begin{avm}
		\[V \[comps \<...\ibox{1}...\>\]\] \, \normalfont{precedes} 
		\[synsem \ibox{1}\]
\end{avm}\jambox*{(\ili{Romance languages})}}\label{GSexemple29}
\end{exe}

\subsection{The complements of the copula in Romance Languages}\label{GSsection3.4}

It is an interesting fact that, while Romance restructuring verbs enter two different structures (the flat structure and the verbal complex), the copula has the same complementation across Romance Languages \citep{abeille2001varieties, AG2010}\footnote{We concentrate on the predicative use of the copula.}. Moreover, this complementation differs both from the flat structure and the verbal complex: the copula takes a non-light complement, which can be saturated or not. 

The complement of the copula is underspecified: it is predicative (noted [PRD +]), but can be an adjective, a noun, a preposition or a passive participle (for the passive construction, see \citealt{AG2002b-u}). We illustrate clitic climbing with the same example in different Romance languages (examples from \citealt{AG2010}).

\begin{exe}
	\ex \label{GSexemple30} 
	\begin{xlist}
        \ex{
		\gll Jean lui \'etait fid\`ele.\\ 
		John to.him was faithful\\\jambox*{(\ili{French})}}\label{GSexemple30a}

		\ex{ 
		Giovanni le era fedele.\jambox*{(\ili{Italian})}}\label{GSexemple30b}
		
		\ex{ 
		Juan le era fiel. \jambox*{(\ili{Spanish})}}\label{GSexemple30c}
		
		\ex{ 
		O Jo\~an era-lhe fiel. \jambox*{(\ili{Portuguese})}}\label{GSexemple30d}
		
		\ex{ 
		En Joan li era fidel. \jambox*{(\ili{Catalan})}}\label{GSexemple30e}
		
		\ex{ 
		Ion \^ii era credincios.\jambox*{(\ili{Romanian})}
		\glt `John was faithful to him'}\label{GSexemple30f}
	\end{xlist}
\end{exe}

The properties of the construction differentiate it clearly from tense auxiliaries and restructuring verbs. For the sake of simplification, we restrict the examples to French, Italian and Spanish. The sequence of the head of the complement with its complements is a constituent, since, for instance, it can be dislocated and pronominalized (\ref{GSexemple31}).

\begin{exe}
	\ex{
[Context] Is John faithful to his friends?}\label{GSexemple31} 
	\begin{xlist}
        \ex[]{
		\gll Fid\`ele \`a ses amis, il l'est plus qu'\`a ses convictions politiques.\\
		Faithful to his friends, he it-is more {than to} his convictions political\\\jambox*{(\ili{French})}}\label{GSexemple31a} 
		
        \ex[]{
		\gll {? Fedele} ai suoi amici, (lo) \`e pi\`u che alle sue idee politiche.\\
		Faithful to.the his friends, (it) is more than to.the his ideas political\\\jambox*{(\ili{Italian})}}\label{GSexemple31b} 
		
		\ex[]{
		\gll Fiel a sus amigos, lo es m\'as que a sus convicciones pol\'iticas.\\
		Faithful to his friends, it is more than to his convictions political\\\jambox*{(\ili{Spanish})}
		\glt `Faithful to his friends, he is, more than to his political ideas'}\label{GSexemple31c} 
	\end{xlist}
\end{exe}

Crucially, the construction differs from that of restructuring verbs in that the dislocated constituent can leave behind its complements (\ref{GSexemple32}).

\begin{exe}
	\ex \label{GSexemple32}
	\begin{xlist}
        \ex[]{
		\gll Fid\`ele, il l'est plus \`a ses amis qu'\`a ses convictions politiques.\\
		Faithful, he it-is more to his friends {than to} his convictions political\\\jambox*{(\ili{French})}}\label{GSexemple32a} 
		
        \ex[]{
		\gll Fedele, lo \`e ai sui amici pi\`u che alle sue idee politiche.\\
		Faithful, it is to.the his friends more than to.the his ideas political\\\jambox*{(\ili{Italian})}}\label{GSexemple32b} 
		
		\ex[]{
		\gll Fiel, lo es m\'as a sus amigos que a sus convicciones pol\'iticas.\\
		Faithful, it is more to his friends than to his political convictions\\\jambox*{(\ili{Spanish})}
		\glt `As for being faithful, he is to his friends more than to his political convictions’
		}\label{GSexemple32c} 
	\end{xlist}
\end{exe}

Similarly, the predicative complement can be extracted with its complements or leave its complements behind. In the latter case, this complement can be cliticized, as shown in (\ref{GSexemple33c}) (compare with examples (\ref{GSexemple17}) and (\ref{GSexemple18}) with restructuring verbs). In (\ref{GSexemple33}), the adjective is extracted (it corresponds to the predicative complement of \textit{\^etre}), as part of a concessive adjunct.


\begin{exe}
	\ex{
[Context] Is he really faithful to his friends?}\label{GSexemple33} 
	\begin{xlist}
        \ex[]{
		Aussi fid\`ele \`a ses amis qu'il soit, il ne perd pas de vue ses int\'er\^ets.\jambox*{(\ili{French})}
		\glt `As faithful to his friends as he is, he does not lose sight of his interests'}\label{GSexemple33a} 
		
        \ex[]{
		\gll Aussi fid\`ele qu'il soit \`a ses amis, il ne perd pas de vue ses int\'er\^ets.\\
		As faithful {as he} is to his friends, he does not lose sight of his interests.\\}\label{GSexemple33b} 
		
		\ex[]{
		\gll Aussi fid\`ele qu'il leur soit, il ne perd pas de vue ses int\'er\^ets.\\
		As faithful {as he} to.them is, he does not lose sight of his interests.\\}\label{GSexemple33c} 
	\end{xlist}
\end{exe}

Moreover, an adverb may intervene between the copula and the adjective, not only in French or Italian, where it is expected (it is possible with tense auxiliaries and restructuring verbs) but also in Spanish, where it is not expected, if the structure is the same as with restructuring verbs. We illustrate this possibility with cliticization, in order to make the contrast with restructuring verbs clearer.

\begin{exe}
	\ex \label{GSexemple34} 
	\begin{xlist}
        \ex[]{
		Rom\'eo lui sera probablement fid\`ele. \jambox*{(\ili{French})}}\label{GSexemple34a}

		\ex[]{ 
		Romeo le sar\`a probabilmente fedele. \jambox*{(\ili{Italian})}}\label{GSexemple34b}
		
		\ex[]{ 
		Romeo le ser\'a probablemente fiel. \jambox*{(\ili{Spanish})}}\label{GSexemple34c}
	\end{xlist}
\end{exe}

The data show that, contrary to restructuring verbs, the copula, in Romance Languages, has only one complementation. \cite{AG2002b-u, AG2010} propose that the copula takes a ``phrasal'' complement, which can be saturated or not. This analysis is implemented by saying that the predicative complement is non-light, whether it is saturated or not, and that it is underspecified with respect to complement saturation or attraction.

\begin{exe}
        \ex[]{Description of the copula in Romance Languages \\
        \begin{avm}
		{\[arg-st \<\ibox{1},
                    \[head [prd +] \\
                    light - \\
                    subj \ibox{1}\\
                    comps \ibox{2}\\\]\,\> \,\,$\oplus$ \ibox{2}\]}
          	\end{avm}}\label{GSexemple35}
\end{exe}

Like tense auxiliaries, the copula is a subject raising verb, hence the identical value \ibox{1} for its subject and that of the complement, which allows it to be empty. Its complement differs from that of a tense auxiliary (\ref{GSexemple7}) on several accounts: it is predicative, which is not the case for tense auxiliaries, and it is non-light; in addition, it is not specified for its category. Being non-light, it can have combined with its complements or some of them, while the complement of the auxiliary is light, hence all its complements are attracted.\footnote{Note that the complements included in a predicative PP are not attracted to the copula in a general way.}  

\begin{figure}
    \centering
\begin{forest}
  [VP \ms{comps & \liste{ }}
 [V \ms{
            subj & \liste{ \ibox{1} }\\
            comps & \liste{ \ibox{3} }\\
            arg-st & \liste{ \ibox{1}, \ibox{3} }
            }[sera\\will.be, align=center, base=bottom]] 
 [\ibox{3} AP \ms{
            head & \normalfont{[\textsc{prd} +]}\\
            light & -\\
            subj & \liste{ \ibox{1} }\\
            comps & \liste{ }}[fid\`ele \`a ses amis\\faithful to his friends, align=center, base=bottom, tier=word, roof]]]
\end{forest}    \label{GSfigure7}
    \caption{}
\end{figure}{}

Figure~\ref{GSfigure8} illustrates a case where the affix complement of the adjective is attracted to the copula. For cliticization and the notion of reduced verb, see Section~\ref{GSsection2}. 

\begin{figure}
    \centering
\begin{forest}
  [VP \ms{comps & \liste{ }}
 [V \ms{
             \normalfont{\textit{reduced-verb}}\\
            subj & \liste{ \ibox{1} }\\
            comps & \liste{ \ibox{3} }\\
            arg-st & \liste{ \ibox{1}, \ibox{3}, \ibox{2} }}[leur-sera\\to.them-will.be, align=center, base=bottom]] 
 [\ibox{3} AP \ms{
            head & \rm [\textsc{prd} +]\\
            light & -\\
            subj & \liste{ \ibox{1} }\\
            comps & \liste{ \ibox{2} \rm \textit{aff} }}[fid\`ele\\faithful, align=center, base=bottom, tier=word]]]
\end{forest}
    \caption{}
    \label{GSfigure8}
\end{figure}

\section{Complex predicates and word order}\label{GSsection4}


In certain languages, a complex verb construction signals itself essentially by properties of word order. This is the case for instance in German \citep{hinrichs1989flipped, HN94a, Kiss94, kiss1995infinite, HN98a, Kathol98b, kathol2000linear, Meurers2000, DM2002, dKM2001a, Mueller2002b, muller2003solving, muller2013copula} and Dutch \citep{Rentier94, BvN98a}, as well as Korean \citep{ Sells1991, CC1998, Yoo2003, Kim2016a-u}. We concentrate on coherent constructions in German, and on auxiliaries and control verbs in Korean.    

\subsection{Verbal complexes in German}\label{GSsection4.1}

\subsubsection{Coherent and incoherent constructions in German}\label{GSsection4.1.1}

Among verbs with an infinitival complement, German distinguishes between coherent and incoherent constructions \citep{gunnar1955studien}. We speak of constructions rather than verbs, because, although the constructions are triggered by lexical properties of verbs, many verbs can be constructed either way. Verbs entering coherent constructions, obligatorily or optionally, belong to different classes: tense auxiliaries (the verbal complement is an infinitive or a participle), modals, subject and object raising, subject and object control verbs, the copulas and predicative constructions, particle verbs (see \citealt{Mueller2002b}).

Coherent and incoherent constructions differ with respect to several properties (separability of the head verb and the infinitive, extraposition of the infinitive with its complements, pied-piping in relative clauses, scope of adjuncts). In incoherent constructions, an adverb such as \textit{nicht} may occur between the two verbs (\ref{GSexemple36a}) (from \citealt{Mueller2002b}), the infinitival phrase can be extraposed (\ref{GSexemple36b}, \ref{GSexemple36c}), and the infinitive is pied-piped with its relative pronoun complement (\ref{GSexemple36d}) (examples from \citealt{HN98a}).

\begin{exe}
	\ex \label{GSexemple36} 
	\begin{xlist}
        \ex[]{
		\gll {... dass} Karl zu schlafen nicht versucht.\\ 
		that Karl to sleep not tries \\\jambox*{(\ili{German})}
		\glt `that Karl does not try to sleep'}\label{GSexemple36a}

        \ex[]{
		\gll {... dass} Peter Maria das Auto zu kaufen \"uberredet.\\ 
		that Peter Maria the car to buy persuades \\
		\glt `that Peter persuades Maria to buy the car'\label{GSexemple36b}}

		\ex[]{ 
		\gll dass Peter Maria \"uberredet, [das Auto zu kaufen].\\
		that Peter  Maria persuades, the car to buy\\\label{GSexemple36c}}		
		
		\ex[]{ 
		\gll Das ist das Auto, das zu kaufen er Peter  \"uberreden wird.\\
		That is the car, which to buy he Peter persuade will\\
		\glt `That is the car, which he will persuade Peter to buy'\label{GSexemple36d}}	
	\end{xlist}
\end{exe}

On the other hand, coherent constructions, of which the combination of the future auxiliary \textit{wird} (\ref{GSexemple37a}) or the raising verb \textit{scheinen} (`to seem’) with an infinitival complement (\ref{GSexemple37d}) are typical examples, do not allow for a non-verbal element between the two verbs (\ref{GSexemple37b}), nor for extraposition of the infinitive with its complements (\ref{GSexemple37c}), (\ref{GSexemple37e}) (examples from \citealt{Mueller2002b}), or pied-piping of the infinitive in relative clauses (\mex{1}g,f).\footnote{The head verb in coherent constructions is italicized.}    

\begin{exe}
	\ex \label{GSexemple37} 
	\begin{xlist}
        \ex[]{
		\gll {... dass} Karl das Buch lesen \textit{wird}. \\ 
		that Karl the book read will \\\jambox*{(\ili{German})}
		\glt `that Karl will read the book'}\label{GSexemple37a}

        \ex[*]{
		\gll {... dass} Karl das Buch lesen nicht wird.\\ 
		that Karl the book read not will \\
		\glt `that Karl will not read the book'}\label{GSexemple37b}

		\ex[*]{ 
		\gll {... dass} Karl wird das Buch lesen.\\
		that   Karl will  the book read\\}\label{GSexemple37c}	
		
		\ex[]{ 
		\gll {... weil} Karl das Buch zu lesen \textit{scheint}.\\
		because Karl the book to read seems\\
		\glt `because Karl seems to read the book'}\label{GSexemple37d}	
		
		\ex[*]{ 
		\gll {... weil} Karl scheint das Buch zu lesen.\\
		because Karl seems the book to read\\}\label{GSexemple37e}
		
		\ex[*]{ 
		\gll Das ist das Buch das lesen Karl wird.\\
		This is the book that read Karl will\\}\label{GSexemple37f}
		
		\ex[*]{ 
		\gll Das ist das Buch das zu lesen Karl scheint.\\
		This is the book that to read Karl seems\\}\label{GSexemple37g}
		
		\end{xlist}
\end{exe}

Scrambling of the complements of the two verbs, or of the subject of the head verb with the complements of the infinitival is possible in a coherent construction. In (\ref{GSexemple38a}) the complements of \textit{sehen (Peter)} and \textit{kaufen (das Auto)} are not interleaved. In (\ref{GSexemple38b}), \textit{Peter}, the complement of \textit{sehen}, occurs between \textit{das Auto}, which is the complement of \textit{kaufen}, and \textit{kaufen} (examples from \citealt{HN98a}).

\begin{exe}
	\ex \label{GSexemple38} 
	\begin{xlist}
        \ex[]{
		\gll {... dass} er Peter das Auto kaufen \textit{sehen} \textit{wird}.\\ 
		that he Peter the car buy see will \\\jambox*{(\ili{German})}
		\glt `that he will see Peter buy the car'}\label{GSexemple38a}

		\ex[]{ 
		\gll {... dass} er das Auto Peter kaufen \textit{sehen}  \textit{wird}.\\
		that he the car Peter buy see will\\}\label{GSexemple38b}
		\end{xlist}
\end{exe}

In the complex predicate approach of this chapter, these data point to the following analysis: incoherent constructions involve a saturated VP complement, while coherent constructions do not: they involve a complex predicate, with a verb attracting the complements of its complement. We assume here a verbal complex for the complex predicate. Figure~\ref{GSfigure9a} represents example (\ref{GSexemple36b}), and Figure~\ref{GSfigure9b} represents example (\ref{GSexemple38b}).

\begin{figure}
\begin{subfigure}[b]{\textwidth}
 \centering
 \caption{Incoherent construction (embedded clause)}
 \begin{forest}
for tree={
        s sep=1.1em, 
        l sep=1.33em,
        l=0,
        calign=midpoint,
        fit=tight}
 [S%, fairly nice empty nodes,
    [NP [Peter\\Peter, base=bottom, tier=word]]
 [ [NP [Maria\\Maria, base=bottom, tier=word]][ [VP [das Auto zu kaufen\\the car to buy, base=bottom, roof, tier=word] ][V[überredet\\persuades, base=bottom, tier=word]]]]]
 \end{forest}
\label{GSfigure9a}
\end{subfigure}
\\
\vspace{20pt}
\begin{subfigure}[b]{\textwidth}
\centering
\caption{Coherent construction (embedded clause)}
 \begin{forest}
for tree={
        s sep=1.1em, 
        l sep=1.33em,
        l=0,
        calign=midpoint,
        fit=tight}
 [S%, fairly nice empty nodes,
    [NP [er\\he, base=bottom, tier=word]]
 [ [NP [das Auto\\the car, base=bottom, roof, tier=word]][ [NP [Peter\\Peter, base=bottom, tier=word] ][ [ [V[kaufen\\buy, base=bottom, tier=word]][V[sehen\\see, base=bottom, tier=word]]] V[V[wird\\will, base=bottom, tier=word]]]]]]
 \end{forest}
\label{GSfigure9b}
\end{subfigure}
\caption{}
\label{GSfigure9}
\end{figure}

\subsubsection{Coherent constructions in HPSG}\label{GSsection4.1.2}

One might wonder whether it is possible to analyze the data in terms of word order instead of structure: a verb governing a coherent construction would trigger a modification of the ordering domain. More precisely, it would induce domain union of the two ordering domains associated with the two verbal projections \citep{Reape94a}, thus allowing the order in (\ref{GSexemple39b}), for instance, while the structure would remain the same with a VP complement (as in Figure~\ref{GSfigure9a}) (chapter ??). The existence of the remote (or long) passive goes against such an analysis \citep{HN94a, Kathol98b, Mueller2002b}. A complex predicate construction can be passivized in such a way that the subject (in the nominative case) of the passive auxiliary corresponds to the object of the active infinitive complement. An (impersonal) passive construction like (\ref{GSexemple39a}) with an infinitival VP containing an accusative object (\textit{den Wagen}) alternates with a coherent construction such as (\ref{GSexemple39b}), with a corresponding nominative (examples from \citealt{Mueller2002b}). 

\begin{exe}
	\ex \label{GSexemple39} 
	\begin{xlist}
        \ex[]{
		\gll {... weil} oft versucht wurde, [den Wagen zu reparieren].\\ 
		because often tried was the car to repair\\
		\glt `many attempts were made to repair the car'}\label{GSexemple39a}

        \ex[]{
		\gll {... weil} der Wagen oft zu reparieren \textit{versucht} \textit{wurde}.\\ 
		because the car often to repair tried was\\}\label{GSexemple39b}

		\ex[]{ 
		\gll Karl darf nicht versuchen zu schlafen.\\
		Karl is.allowed not try to sleep\\
		\glt `Karl is not allowed to try to sleep'\\
		\glt `Karl is allowed to not try to sleep’'}\label{GSexemple39c}	
		
		\ex[]{ 
		Karl darf versuchen, nicht zu schlafen.
		\glt `Karl is allowed to try not to sleep’}\label{GSexemple39d}
		
		\end{xlist}
\end{exe}

In (\ref{GSexemple39a}), the infinitival VP is extraposed. In (\ref{GSexemple39b}), there is no infinitival VP, as shown by the position of the adverb \textit{oft}, which occurs before \textit{zu reparieren}, while modifying \textit{versucht}. In a coherent construction, an adverb can scope over any of the verbs that belong to it. In (\ref{GSexemple39c}), \textit{zu schlafen} is not part of the coherent construction, because it is extraposed; \textit{nicht} can have scope over \textit{darf} or \textit{versuchen}, not over \textit{schlafen}. In (\ref{GSexemple39d}), \textit{nicht} belongs to the extraposed infinitival; accordingly, it can only scope over it. The fact that \textit{oft} can scope over \textit{versucht} in (\ref{GSexemple39b}) shows that they belong to the same coherent construction, which allows for passivization: \textit{versuchen} attracting the complement of \textit{reparieren} can be passivized.

German differs from Romance languages in not distinguishing structurally between the subject and the complements: the subject can be considered as a complement, and introduced by the same rule. The structure of the sentence is usually represented as having binary branching daughters (see Figures \ref{GSfigure9}). The constraint is as follows \citep{muller2018clause}.

\begin{exe}
    \ex[] {\textit{Head-complement-phrase} (German) $\Rightarrow$ \\
    \begin{avm}
      {\[mother | synsem & \[loc | cat | comps \ibox{1} $\oplus$ \ibox{3}\\
      light -\]\\
      head-dtr | synsem & \[loc | cat | comps \ibox{1} $\oplus$ \<\ibox{2}\>\, $\oplus$ \ibox{3}\]\\
      non-head-dtrs & \<[synsem \ibox{2}]\>\]}\label{GSexemple40}
    \end{avm}}
\end{exe}

Following constraint (\ref{GSexemple40}), the head combines with one complement at a time, noted \ibox{2}. The presentation of the list as composed of three parts, with the relevant one in any position, allows for a free order. The lexical verb is [LIGHT +], and the phrase combining the verb with a complement is [LIGHT -].\footnote{The feature LIGHT is the equivalent of LEX used in German studies, although the properties of light elements may differ depending on the language. It does not belong to Local features in (\ref{GSexemple40}), because an extracted constituent may differ from its trace as regards lightness (see chapter ?? for extraction) \citep{muller2018clause}.} The structure of (\ref{GSexemple41}) is exemplified in Figure~\ref{GSfigure10}.

\begin{exe}
	\ex[]{
		\gll \ldots{} weil das Buch jeder kennt\\
		     {}       because the book everybody knows\\
		\glt `because everybody knows the book'}\label{GSexemple41}

\end{exe}

%%%%%%%%%%%%%%%%%%%%%%%%%%%%%%%MAS LARG%%%%%%%%%%%%%%%%%%%%%%%%%%%%%%%%%%

\begin{figure}
    \centering
\begin{forest}
sm edges
 [S 
    [C [weil;because]]
    [V \ms{{\normalfont{\textit{fin}}}, comps \liste{ }, \normalfont{\textsc{light $-$}}} 
        [\ibox{2} NP [das Buch;the book, roof]]    
        [V \ms{{\normalfont{\textit{fin}}}, comps \liste{ \ibox{2} }, \normalfont{\textsc{light $-$}}} 
            [\ibox{1} NP [jeder;everybody]]
            [V \ms{{\normalfont{\textit{fin}}}, comps \liste{\ibox{1}, \ibox{2} }, \normalfont{\textsc{light +}}} [kennt;knows]]]]]
\end{forest}
    \caption{}
    \label{GSfigure10}
\end{figure}


%%%%%%%%%%%%%%%%%%%%%%%%%%%%%%%MAS LARG%%%%%%%%%%%%%%%%%%%%%%%%%%%%%%%%%%

Turning to complex predicates, they form a verbal complex phrase: they cannot be separated by an adverb or an NP (\ref{GSexemple37b}) (\ref{GSexemple37c}). Given the structure of the German sentence with binary branching, illustrated in Figure~\ref{GSfigure9}, this verbal complex only shows structurally when there is a series of verbs attracting the complements of their complements, as in (\ref{GSexemple38}) (see Figure~\ref{GSfigure9b}).

The phrase structure constraint allowing complex predicates is as in (\ref{GSexemple42}) \citep{muller2013copula, muller2018clause}. It is called \textit{the head-cluster-phrase}, rather than \textit{the verbal-complex-phrase}, because it is not specialized for verbal heads.\footnote{Following \cite{HN94a} and \cite{dKM2001a}, but contrary to \cite{muller2018clause}, we mention the lightness of the mother. Müller’s decision is motivated by the fact that infinitive intransitive verbs may be analyzed as argument saturated (and non-light) if their subject is represented as a head feature rather than a complement. However, it leads to formal complications which are best ignored in this presentation. Hence, we assume here, for the sake of simplification, that the subject of the infinitival verb is a complement in German, like the subject of a finite verb.  }  

\begin{exe}
    \ex[] {\textit{Head-cluster-phrase} (German) $\Rightarrow$ \\
    \begin{avm}
      {\[mother | synsem & \[loc | cat | comps \ibox{1}\\
      light +\]\\
      head-dtr | synsem & \[loc | cat | comps \ibox{1} $\oplus$ \<\ibox{2}\>\\ light +\]\\
      non-head-dtrs & \<[synsem \ibox{2} [light +]]\>\]} \end{avm}\label{GSexemple42}}
\end{exe}

We illustrate with sentence (\ref{GSexemple38b}) (... \textit{dass er das Auto Peter kaufen sehen wird}, `that he will see Peter buy the car’), elaborating on Figure~\ref{GSfigure9b}. The description of \textit{werden} (the future auxiliary), a subject raising verb, is as in (\ref{GSexemple43}) (from \citealt{muller2018clause}), and that of \textit{sehen}, an object raising verb and an obligatorily coherent verb, in (\ref{GSexemple44}). The subject is assumed here (for simplification) to be part of the list of complements of infinitives as well as of finite verbs; hence the raised subject of the infinitive complement of \textit{werden} is included in list \ibox{1}, and that of \textit{sehen} is included in list \ibox{2}. The subject is distinguished from the other elements of this list by its semantic role (it is the first semantic argument of the infinitive).     

\begin{exe}
    \ex[] {\textit{werden} (future auxiliary) \\
    \begin{avm}
      {\[head {\normalfont{\textit{verb}}}\\
      comps \ibox{1} $\oplus$ \<v [{\normalfont{\textit{bse}}}, comps \ibox{1}, light +]\>\]}
    \end{avm}}\label{GSexemple43}
\end{exe}

\begin{exe}
    \ex[] {\textit{sehen} (obligatory coherent verb) \\
    \begin{avm}
      {\[head {\normalfont{\textit{verb}}}\\
      comps \ibox{1} $\oplus$ \ibox{2} $\oplus$ \<v [{\normalfont{\textit{bse}}}, comps \ibox{2}, light +]\>\]}
    \end{avm}}\label{GSexemple44}
\end{exe}

Sentence (\ref{GSexemple38b}) is represented in Figure~\ref{GSfigure11}. 


%%%%%%%%%%%%%%%%%%%%%%   LARG    &&&&&&&&&&&&&&&&&&&&&&&

\begin{figure}
    \centering
        {\footnotesize
\begin{forest}
%    fairly nice empty nodes,
 [S, s sep=0pt [C[dass\\that, base=bottom, tier=word]]
    [[\ibox{1} NP [er\\he, base=bottom, tier=word]]
     [[\ibox{2} NP [das Auto\\the car, base=bottom, tier=word, roof]]
     [[\ibox{3} NP [Peter\\Peter, base=bottom, tier=word]]
     [V \ms{head \ibox{4}, comps \liste{ \ibox{1},\ibox{3},\ibox{2} }}
     [\ibox{7} V \ms{head \ibox{5}\\
                comps \liste{ \ibox{1},\ibox{3},\ibox{2} }}
                [\ibox{6} V \ms{vform {\normalfont{\textit{bse}}}\\
                comps \liste{\ibox{3},\ibox{2} }}
                [kaufen\\buy, base=bottom, tier=word]]
                [V \ms{head \ibox{5}\\
                vform {\normalfont{\textit{bse}}}\\
                comps \liste{\ibox{1},\ibox{3},\ibox{2} }  $\oplus$ \ibox{6}}, before computing xy={s'-=7pt} 
                [sehen\\see, base=bottom, tier=word]]]
     [V \ms{head \ibox{4}\\
                    vform {\normalfont{\textit{fin}}}\\
                    comps \liste{ \ibox{1},\ibox{3},\ibox{2} } $\oplus$ \liste{ \ibox{7} }} 
        [wird\\will, base=bottom, tier=word]]]]]]]
 \end{forest}}    
    \caption{}
    \label{GSfigure11}
\end{figure}

%%%%%%%%%%%%%%%%%%%%%%   LARG    &&&&&&&&&&&&&&&&&&&&&&&


\subsubsection{The German copula}\label{GSsection4.1.3}

The copula in German, with an adjectival argument, is also the head of a complex predicate.\footnote{As in Romance Languages, the German copula accepts nominal and prepositional predicative complements. However, they are complement saturated.} The subject of the copula and the complements of the adjectives can be permuted (examples from \citealt{Mueller2002b, muller2013copula}) (see (\ref{GSexemple38}) for coherent verbs):

\begin{exe}
	\ex \label{GSexemple45} 
	\begin{xlist}
        \ex[]{
		\gll {... dass} die Sache dem Minister ganz klar war.\\ 
		that the {matter.\textsc{nom}} the {minister.\textsc{dat}} completely clear was\\}\label{GSexemple45a}

		\ex[]{ 
		\gll {... dass} dem Minister die Sache ganz klar war. \\
		that the {minister.\textsc{dat}} the {matter.\textsc{nom}} completely clear was\\
		\glt `that the matter was completely clear to the minister'}\label{GSexemple45b}
		\end{xlist}
\end{exe}

Adverbs can have different scopings: in (\ref{GSexemple46}), \textit{immer} can modify the modal or the adjective. This follows if there is just one coherent construction, both the modal and the copula being the head of a complex predicate (see Section~\ref{GSsection4.1.2}, example (\ref{GSexemple39b}) for coherent verbs).

\begin{exe}
	\ex[]{
		\gll {... weil} der Mann ihr immer true sein wollte\\
		because the {man-\textsc{nom}} {her.\textsc{dat}} always faithful be wanted.to\\
		\glt `because the man always wanted to be faithful to her'\\
		\glt `because the man wanted to be faithful to her forever'}\label{GSexemple46}

\end{exe}

\cite{Mueller2002b} also shows that the copula does not take a saturated AP complement. Contrary to a construction with an incoherent verb, this AP cannot be extraposed (\ref{GSexemple47b}), or pied piped with a relative pronoun (\ref{GSexemple47d}) (compare with (\ref{GSexemple36c}), (\ref{GSexemple36d})).   

\begin{exe}
	\ex \label{GSexemple47} 
	\begin{xlist}
        \ex[]{
		\gll Karl ist auf seinem Sohn stolz gewesen.\\ 
		Karl is on his son proud been\\
		\glt `Karl was proud of his son'}\label{GSexemple47a}

		\ex[*]{ 
		Karl ist gewesen auf seinem Sohn stolz.}\label{GSexemple47b}

        \ex[]{
		\gll der Sohn, auf den Karl stolz gewesen ist.\\ 
		the son, on whom Karl proud been is\\
		\glt `the son, of whom Karl was proud'}\label{GSexemple47c}
		     
		 \ex[*]{
		\gll der Sohn, auf den stolz Karl gewesen ist\\ 
		the son on whom proud Karl been is\\}\label{GSexemple47d}
		\end{xlist}
\end{exe}

In addition, the German copula, like the Romance copula, is a subject raising verb: the semantic properties of the subject depend on the adjective (a human is proud or faithful, and a matter is clear, cf. Karl's faithfulness, the clarity of the matter); moreover, the sentence can be subjectless: 

\begin{exe}
	\ex[]{
		\gll Am Montag ist schulfrei.\\
		At.the Monday is school.free\\
		\glt `There is no school on Monday'}\label{GSexemple48}

\end{exe}

The description of the German copula, restricted to its predicative use, and to its syntactic part, is as follows:\footnote{\cite{muller2013copula} generalizes the description of the copula to English and Danish. Given that list \ibox{1} can be empty, the description is also valid for the copula taking a saturated complement (an NP or a PP complement in German, as well as for English and Danish, where the copula does not enter a complex predicate). However, NP and PP complements of the copula are independently specified as complement saturated in German, and more generally in English and Danish.}

\ea
\label{GSexemple49}
    \begin{avm}
      {\[{\normalfont{\textit{sein}}} \\
      head {\normalfont{\textit{verb}}}\\
      comps \ibox{1} $\oplus$ \<\ \[head [prd +]\\
      comps \ibox{1}\]\,\>\]}
    \end{avm}
\z

It differs from the Romance copula in not specifying the lightness of its predicative complement. The COMPS list includes the subject, while subject and complements are distinguished in Romance Languages.


\subsection{Argument attraction with Korean auxiliaries and control verbs}\label{GSsection4.2}

\subsubsection{Scrambling in Korean}\label{GSsection4.2.1}
 
Korean resembles German in that a complex predicate signals itself mainly by its word order properties. The examples as well as the analysis are from \cite{CC1998} (see also \citealt{Sells1991, Yoo2003, Kim2016a-u}).

Korean auxiliaries semantically resemble aspectual or modal verbs rather than tense auxiliaries: they include such verbs as \textit{iss-} `to be in the process/state of', \textit{chiwu-} `to do resolutely’, \textit{siph-} `to want', but also the verb of negation \textit{anh-} `not'. Scrambling with auxiliaries is illustrated in (\ref{GSexemple50}). There is no evidence of scrambling in (\ref{GSexemple50a}): the subject \textit{Maryka} starts the sentence, and the complement of the verb \textit{ilkko} immediately precedes it. However, in (\ref{GSexemple50b}), the subject of the head verb \textit{issta} occurs between the complement of \textit{ilkko} and the verb \textit{ilkko} itself. 

\begin{exe}
	\ex \label{GSexemple50} 
	\begin{xlist}
        \ex[]{
		\gll Mary-ka ku chayk-ul ilk-ko iss-ta.\\ 
		{Mary-\textsc{nom}} the {book-\textsc{acc}} {read-\textsc{part}} {be.in.the.process.of-\textsc{decl}}\\
		\glt `Mary is in the process of reading the book'}\label{GSexemple50a}

        \ex[]{
		\gll Ku chayk-ul Mary-ka ilk-ko iss-ta.\\ 
		the {book-\textsc{acc}} {Mary-\textsc{nom}} {read-\textsc{part}} {be.in.the.process.of-\textsc{decl}}\\}\label{GSexemple50b}
		   
		\end{xlist}
\end{exe}

Control verbs such as \textit{seltukha-} `to persuade’, \textit{cisi-} `to order' (object control), \textit{yaksokha-} `to promise',  \textit{sito-} `to try' (subject control), as well as causative \textit{ha-}, allow complements of the embedded verb to be interleaved with their own arguments: the subject of the head verb \textit{Maryka} occurs between the complement of the embedded verb and this verb in (\ref{GSexemple51b}) (\ref{GSexemple51d}).  

\begin{exe}
	\ex \label{GSexemple51} 
	\begin{xlist}
        \ex[]{
		\gll Mary-ka John-hantley [ku chayk-ul ilk-ula-ko] seltukha-yss-ta.\\ 
		{Mary-\textsc{nom}} {John-\textsc{dat}} the {book-\textsc{acc}} {read-\textsc{part}-\textsc{part}} {persuade-\textsc{past}-\textsc{decl}}\\
		\glt `Mary persuaded John to read the book'}\label{GSexemple51a}
		
	    \ex[]{
		\gll Ku chayk-ul Mary-ka John-hantley ilk-ula-ko 				seltukha-yss-ta.\\ 
		the {book-\textsc{acc}} {Mary-\textsc{nom}} {John-\textsc{dat}}  {read-\textsc{part}-\textsc{part}} {persuade-\textsc{past}-\textsc{decl}}\\}\label{GSexemple51b}

        \ex[]{
		\gll Mary-ka John-hantley [ku chayk-ul pilyecwu-kess-tako] yaksokha-yss-ta.\\ 
		{Mary-\textsc{nom}} {John-\textsc{dat}} the {book-\textsc{acc}} {lend-\textsc{part}-\textsc{part}} {promise--\textsc{past}-\textsc{decl}}\\
		\glt `Mary promised John to lend the book'\\}\label{GSexemple51c}
		   
	    \ex[]{
		\gll Ku chayk-ul Mary-ka John-hantley pilyecwu-kess-tako yaksokha-yss-ta.\\ 
		the {book-\textsc{acc}} {Mary-\textsc{nom}} {John-\textsc{dat}}  {lend-\textsc{part}-\textsc{part}} {promise--\textsc{past}-\textsc{decl}}\\}\label{GSexemple51d}
		\end{xlist}
\end{exe}

A priori, these data could be explained in two ways: either the auxiliary or the control verb always takes a VP complement, and scrambling is due to linearization, in which case the domains of the two verbs are unioned \citep{Reape94a}; or there is a complex predicate: the complement of the embedded verb (the book) is attracted by the auxiliary or the control verb. Assuming a flat structure for the Korean sentence, as does \cite{CC1998}\footnote{However, see \cite{Kim2016a-u} for a binary branching structure.} the head-subject-complements-phrase is as in (\ref{GSexemple52}), the subject occurs at the same level as the complements, and they can be reordered.

\begin{exe}
    \ex[] {\textit{Head-subject-complements-phrase} (Korean) $\Rightarrow$ \\
    \begin{avm}
      {\[mother | synsem & \[cat \[light -\\
                                subj / \< \>\\
                                comps / \< \>\]\]\\
      head-dtr | synsem & \[cat \[light +\\
                                subj \ibox{1}\\
                                comps \ibox{2}\]\]\\
      non-head-dtrs & {\rm (\ibox{1} $\bigcirc$ \ibox{2})} non empty list\]}
    \end{avm}
    }\label{GSexemple52}
\end{exe}

However, the same argument can be levelled against an analysis which appeals to linearization, as above for German (Section~\ref{GSsection4.1.2}): so-called long passivization is possible, which cannot be accounted for by appeal to linearization and domain union.

Certain auxiliary and control verbs can be passivized, so that the expected complement of the embedded verb becomes the subject of the construction. (\ref{GSexemple53a}) exemplifies the active sentence, and (\ref{GSexemple53b}), the passive one. In (\ref{GSexemple53a}), \textit{malssengmanhun solul} (`the troublesome cow’) is the complement of the embedded verb \textit{phala}. In (\ref{GSexemple53b}), \textit{malssengmanhun soka} is the subject of the passivized head verb \textit{chiwe ciessta}. In the same way, while \textit{ku cengchaykul} is the complement of the embedded verb \textit{sihaynghalako} in (\ref{GSexemple54a}), \textit{ku cengchayki} is the subject of the head passive verb \textit{cisitoyessta} in (\ref{GSexemple54b}).  

\begin{exe}
	\ex \label{GSexemple53} 
	\begin{xlist}
        \ex[]{
		\gll Ku mongpwu-ka malssengmanhun so-lul phal-a 		chiw-ess-ta.\\ 
		The {farmer-\textsc{nom}} troublesome {cow-\textsc{acc}} {sell-\textsc{part}} {do.resultely-\textsc{past}-\textsc{decl}}\\
		\glt `The farmer resolutely sold the troublesome cow'}\label{GSexemple53a}
		
	    \ex[]{
		\gll Malssengmanhun so-ka (ku nongpwu-eyuyhay) phal-a chiw-e ci-ess-ta.\\ 
		Troublesome {cow-\textsc{nom}} the {farmer-by} {sell-\textsc{part}} {do.resolutely-\textsc{part}} {\textsc{pass}-\textsc{past}-\textsc{decl}}\\
		\glt `The troublesome cow was resolutely sold (by the farmer)'}\label{GSexemple53b}
		\end{xlist}
\end{exe}


\begin{exe}
	\ex \label{GSexemple54} 
	\begin{xlist}
        \ex[]{
		\gll Nay-ka Mary-hantley ku cengchayk-ul sihayngha-lako cisiha-yss-ta.\\ 
		{I-\textsc{nom}} {Mary-\textsc{dat}} the {policy-\textsc{acc}} {carry.out-\textsc{part}} {order-\textsc{past}-\textsc{decl}}\\
		\glt `I ordered Mary to carry out the policy'}\label{GSexemple54a}
		
	    \ex[]{
		\gll Ku cengchayk-i	naey-uyhayse Mary-hantley sihayngha-lako cisitoy-ess-ta.\\ 
		The {policy-\textsc{nom}}  {I-by} {Mary-\textsc{dat}} {carry.out-\textsc{part}} {be.ordered-\textsc{past-decl}}\\
		\glt (lit. the policy was ordered by me for Mary to carry out)\\
		\glt `Mary was ordered by me to carry out the policy'}\label{GSexemple54b}
		\end{xlist}
\end{exe}

Since passivization only affects the complement of the verb which is itself passivized, it follows that \textit{malssengmanhun solul} is the complement of the auxiliary in (\ref{GSexemple53a}) (\textit{chiwesta}), and \textit{ku cengchaykul} the complement of the control verb in (\ref{GSexemple54a}) (\textit{cisihayssta}).\footnote{In addition, \cite{Yoo2003} shows that some variation regarding the case of the attracted complement can be explained if the auxiliary and the verbal complement form a verbal complex.}

\subsubsection{The Structures of Complex Predicates in Korean}\label{GSsection4.2.2}
 
As shown by \cite{CC1998}, in Korean, the structure of a complex predicate is different depending on whether the head verb is an auxiliary or a control verb. Auxiliaries are the head of a verbal complex (as in German complex predicates), while control verbs can either take a saturated VP complement, or be the head of a flat structure (like restructuring verbs in Italian). Thus, complex predicates in Korean confirm the observation made in Romance Languages, although the complex predicate manifests itself by different properties: argument attraction is not correlated with a specific structure. 

A characteristic property of Korean auxiliary constructions is that nothing can separate the two verbs: no parenthetical expression, such as \textit{hayekan}, can intervene (\ref{GSexemple55}) (examples from \citealt{CC1998}). In addition, the embedded verb cannot occur by itself. Thus, while an NP may occur after the head verb in a construction called afterthought (\ref{GSexemple56a}), this is not possible for the embedded verb \textit{mekko} alone (\ref{GSexemple56b}) or with its complement (\ref{GSexemple56c}). This behavior follows if the two verbs form a verbal complex (see Section~\ref{GSsection3.2}).

\begin{exe}
	\ex \label{GSexemple55} 
	\begin{xlist}
        \ex[]{
		\gll Mary-ka hayekan sakwa-lul mek-ko iss-ta. \\ 
		{Mary-\textsc{nom}} anyway {apple-\textsc{acc}} {eat-\textsc{part}} {be.in.the.process.of-\textsc{part}}\\
		\glt `Anyway, Mary is eating an apple'}\label{GSexemple55a}

        \ex[*]{
		Mary-ka sakwa-lul mek-ko hayekan issta.}\label{GSexemple55b}
		   
		\end{xlist}
\end{exe}

\begin{exe}
	\ex \label{GSexemple56} 
	\begin{xlist}
        \ex[]{
		\gll Mary-ka mek-ko iss-ta, sakwa-lul. \\ 
		{Mary-\textsc{nom}} {eat-\textsc{part}} {be.in.the.process.of-\textsc{decl}} {apple-\textsc{acc}} \\
		\glt `Mary is in the process of eating an apple'}\label{GSexemple56a}

        \ex[*]{
		\gll Mary-ka sakwa-lul iss-ta, mek-ko.\\ 
		{Mary-\textsc{nom}} {apple-\textsc{acc}} {be.in.the.process.of-\textsc{decl}} {eat-\textsc{part}}\\
		\glt `Mary is in the process of eating an apple'}\label{GSexemple56b}

        \ex[*]{
		\gll Mary-ka iss-ta, sakwa-lul mek-ko.\\ 
		{Mary-\textsc{nom}} {be.in.the.process.of-\textsc{decl}} {apple-\textsc{acc}} {eat-\textsc{part}}\\}\label{GSexemple56c}
		\end{xlist}
\end{exe}

The Korean verbal-complex-phrase is shown in (\ref{GSexemple57}). This structure characterizes the class of auxiliaries. This is captured by the head feature [AUX +], which differentiates auxiliaries from control verbs, which are [AUX-]. Auxiliaries attract both the subject (list \ibox{1}) and the complements of their verbal complement (list \ibox{3}). Note the use of `$\bigcirc$', the shuffle operator, which allows reordering of the elements of the two lists.

\begin{exe}
    \ex[] {\textit{Verbal-complex-phrase} (Korean) $\Rightarrow$ \\
    \begin{avm}
      {\[mother | synsem | cat \[head 
                                \[\normalfont{\textit{verb}}\\
                                  aux +\]\\
                            light +\\
                            subj\, \<\ibox{1}\>\\
                            comps \ibox{3}\]\\
      head-dtr | synsem | cat \[aux +\\
                            light +\\
                            subj\, \<\ibox{1}\>\\
                            comps \<\ibox{2}\>\, $\bigcirc$  \ibox{3}\]\\
      non-head-dtrs \<\ibox{2} \[\normalfont{\textit{verb}}\\
                                light +\\
                                subj\,  \<\ibox{1}\>\\
                                comps \ibox{3}\]\,\> \]}
    \end{avm}}\label{GSexemple57}
\end{exe}

The auxiliary \textit{iss-} in (\ref{GSexemple55a}), (\ref{GSexemple56a}) takes as its complement a light verb which is constrained to end in \textit{–ko} (different auxiliaries put different restrictions on their verbal complement, \citealt{Yoo2003}), from which it attracts all arguments. 

We follow \cite{CC1998} in assuming that the sentence in Korean has a flat structure, which is allowed by the Head-subject-complements-phrase in (\ref{GSexemple52}). Sentence (\ref{GSexemple50a}) is represented in Figure~\ref{GSfigure12}.

%%%%%%%%%%%%%%%%%%%%%%%%%%%%%%%MAS LARG%%%%%%%%%%%%%%%%%%%%%%%%%%%%%%%%%%
\begin{figure}
    \centering
\begin{forest}
 [S [\ibox{2} NP
            [Mary-ka\\Mary-\textsc{nom}, base=bottom, tier=word]]
 [\ibox{3} NP
            [ku chayk-ul\\the book-\textsc{acc}, base=bottom, tier=word, roof]]
  [V \ms{aux & \normalfont{+}\\
            light & \normalfont{+}\\
            subj & \liste{ \ibox{2} }\\
            comps & \liste{ \ibox{3} }} 
    [\ibox{4} V \ms{light & \normalfont{+}\\
            comps & \liste{ \ibox{3} }\\
            subj & \liste{ \ibox{2} }
            }[ilk-ko\\read-\textsc{part}, align=center, base=bottom, tier=word]]
        [V \ms{aux & \normalfont{+}\\
            comps & \liste{ \ibox{4}, \ibox{3} }\\
            subj & \liste{ \ibox{2} } 
            }[iss-ta\\be.in.the.process.of-\textsc{part}, align=center, base=bottom, tier=word]]]] \end{forest} \caption{}
    \label{GSfigure12}
\end{figure}{}

%%%%%%%%%%%%%%%%%%%%%%%%%%%%%%%MAS LARG%%%%%%%%%%%%%%%%%%%%%%%%%%%%%%%%%%

Nothing prevents the verbal complex from embedding a verbal complex: it suffices that the complement of the auxiliary be an auxiliary itself. This is illustrated in (\ref{GSexemple58}).

\begin{exe}
	\ex \label{GSexemple58} 
	\begin{xlist}
        \ex[]{
		\gll Mary-ka ku chayk-ul ilk-e po-ko iss-ta.\\ 
		{Mary-\textsc{nom}} the {book-\textsc{acc}} {read-\textsc{part}} {try-\textsc{part}} {be.in.the.process.of-\textsc{decl}}\\
		\glt `Mary is giving the book a trial reading'}\label{GSexemple58a}
		
	    \ex[]{
		\gll Ku chayk-ul Mary-ka ilk-e po-ko iss-ta.\\ 
		the {book-\textsc{acc}} {Mary-\textsc{nom}} {read-\textsc{part}} {try-\textsc{part}} {be.in.the.process.of-\textsc{decl}}\\}\label{GSexemple58b}
		
		\end{xlist}
\end{exe}

The structure of (\ref{GSexemple58a}), with a series of two auxiliaries, is represented in Figure~\ref{GSfigure13}. \textit{Issta} takes as its complement the verbal complex \textit{ilke poko}, whose head is \textit{poko}. \textit{Poko}, being an auxiliary like \textit{issta}, takes as its complement the verb \textit{ilke}, attracting its subject and complements, which are transmitted to the verbal complex \textit{ilke poko}; \textit{ilke poko} saturates the verbal complement expected by \textit{issta}, and transmits the subject and complements to the head auxiliary (see (\ref{GSexemple57})).


%%%%%%%%%%%%%%%%%%%%%%%%%%%%%%%MAS LARG%%%%%%%%%%%%%%%%%%%%%%%%%%%%%%%%%%

\begin{figure}
    \centering
    {\footnotesize
\begin{forest}
 [S [\ibox{1} NP [Mary-ka\\Mary-\textsc{nom}, base=bottom, tier=word]]
 [\ibox{2} NP [ku chayk-ul\\the book-\textsc{acc}, base=bottom, tier=word, roof]]
  [V \ms{aux & \normalfont{+}\\
            light & \normalfont{+}\\
            subj & \liste{ \ibox{1} }\\
            comps & \liste{ \ibox{2} }} 
    [\ibox{4} V \ms{light & \normalfont{+}\\
            aux & \normalfont{+}\\
            subj & \liste{ \ibox{1} }\\
            comps & \liste{ \ibox{2} } } [\ibox{3} V \ms{
            light & \normalfont{+}\\
            subj & \liste{ \ibox{1} }\\
            comps & \liste{ \ibox{2} } } 
            [ilk-e\\read-\textsc{part}, align=center, base=bottom, tier=word]]
            [V \ms{light & \normalfont{+}\\
            aux & \normalfont{+}\\
            subj & \liste{ \ibox{1} }\\
            comps & \liste{ \ibox{3}, \ibox{2} } } 
            [poko\\try-\textsc{part}, align=center, base=bottom, tier=word]]]
    [V \ms{light & \normalfont{+}\\
        aux & \normalfont{+}\\
        subj & \liste{ \ibox{1} }\\
        comps & \liste{ \ibox{4}, \ibox{2} }
        }[issta\\be.in.the.process.of-\textsc{part}, align=center, base=bottom, tier=word]]]] \end{forest}}
    \caption{}
    \label{GSfigure13}
\end{figure}

%%%%%%%%%%%%%%%%%%%%%%%%%%%%%%%MAS LARG%%%%%%%%%%%%%%%%%%%%%%%%%%%%%%%%%%

Contrary to auxiliaries, control verbs such as \textit{seltukha-} (\ref{GSexemple51a}), (\ref{GSexemple51b}) or \textit{cisi-} (\ref{GSexemple54}) can be separated from their verbal complement, for instance by an adverb as in (\ref{GSexemple59a}). They also allow for the infinitive and its complement to form a VP constituent as in (\ref{GSexemple59b}), where it is an afterthought. Thus, control verbs are analyzed in the same way as Italian restructuring verbs: they either take a saturated VP complement (\ref{GSexemple59a}) (\ref{GSexemple59b}), or are the head of a complex predicate (\ref{GSexemple59c}) (\ref{GSexemple59d}). They contrast with Korean Raising verbs which only take a VP complement.

\begin{exe}
	\ex \label{GSexemple59} 
	\begin{xlist}
        \ex[]{
		\gll Mary-ka ku mwuncey-lul phwulye-ko (kkuncilkikey)		sito-hayss-ta.\\ 
		{Mary-\textsc{nom}} the {problem-\textsc{acc}} {solve-\textsc{part}} ceaselessly {try-\textsc{past}-\textsc{decl}}\\
		\glt `Mary tried (ceaselessly) to solve the problem'}\label{GSexemple59a}
		
	    \ex[]{
		\gll Mary-ka sito-hayess-ta, [ku mwuncey-lul phwulye-ko].\\ 
		{Mary-\textsc{nom}} {try-\textsc{part-decl}} the {problem-\textsc{acc}} {solve-\textsc{part}}\\}\label{GSexemple59b}
		
	    \ex[]{
		\gll Ku chayk-ul Mary-ka ilku-lako sito-hayss-ta.\\ 
		The {book-\textsc{acc}} {Mary-\textsc{nom}} {read-\textsc{part}} {try-\textsc{past}-\textsc{decl}}\\
		\glt `Mary tried to read the book'}\label{GSexemple59c}
		
       \ex[]{
		\gll Ku chayk-ul Mary-ka ilku-lako John-hantley  seltukha-yss-ta.\\ 
		The {book-\textsc{acc}} {Mary-\textsc{nom}} {read-\textsc{part}} {John-\textsc{dat}} {persuade-\textsc{past}-\textsc{decl}}\\
		\glt `Mary persuaded John to read the book'}\label{GSexemple59d}
		\end{xlist}
\end{exe}




The scrambling data, together with the possibility of long passivization (\ref{GSexemple54}), show that there is a complex predicate. The subject of the head verb occurs between the complement of the infinitive and the infinitive in (\ref{GSexemple59c}), (\ref{GSexemple59d}). More precisely, there is no verbal complex in this case: the two verbs do not have to be contiguous, but can be separated, for instance by a complement as in (\ref{GSexemple59d}) \textit{(John-hantley)}. Thus, they are the head of a flat structure as in Figure~\ref{GSfigure14} corresponding to (\ref{GSexemple59d}). The head of verbal complexes in Korean is [AUX +], see (\ref{GSexemple57}). Since control verbs are [AUX –], they cannot enter verbal complexes, and the structure for complex predicates whose head is a control verb is a flat structure, corresponding to the Head-subject-complements-phrase in (\ref{GSexemple52}). 


%%%%%%%%%%%%%%%%%%%%%%%%%%%%%%%MAS LARG%%%%%%%%%%%%%%%%%%%%%%%%%%%%%%%%%%

\begin{figure}
    \centering
    {\small
\begin{forest}
 [S
 [\ibox{1} NP [Ku chayk-ul\\The book-\textsc{acc}, base=bottom, tier=word, roof]]
 [NP [Mary-ka\\Mary-\textsc{nom}, base=bottom, tier=word]]
  [\ibox{3} V \ms{
            light & \normalfont{+}\\
            comps & \liste{ \ibox{1} }} 
    [ilku-lako\\read-\textsc{part}, align=center, base=bottom, tier=word]] 
  [\ibox{2} NP[John-hantley\\John-\textsc{dat}, align=center, base=bottom, tier=word]]
  [\ibox{3} V \ms{
            light & \normalfont{+}\\
            comps & \liste{ \ibox{2}, \ibox{3}, \ibox{1} }} 
    [seltukha-yss-ta\\persuade-\textsc{past}-\textsc{decl}, align=center, base=bottom, tier=word]]] \end{forest}}
    \caption{}
    \label{GSfigure14}
\end{figure}{}

%%%%%%%%%%%%%%%%%%%%%%%%%%%%%%%MAS LARG%%%%%%%%%%%%%%%%%%%%%%%%%%%%%%%%%%

Control verbs which may be the head of a complex predicate are subject or object control verbs. They are the target of an argument attraction rule, in a way parallel to Italian Restructuring verbs. 

\begin{exe}
	\ex {Argument Attraction Rules for Korean control verbs}\label{GSexemple60} 
	\begin{xlist}
        \ex[]{Subject control verbs \\
	{\small
        \begin{avm}
{\[head & \[{\normalfont{\textit{verb}}}\\
		            aux-\]\\
        light & +\\
        subj & \<np{\normalfont{\textit{\textsubscript{i}}}}\>\\
        comps & \<v \[subj & \<np{\normalfont{\textit{\textsubscript{i}}}}\>\\
                    comps & \<\> \] \,\> \]} \end{avm}
          	$\Rightarrow$
        \begin{avm}
		{\[comps & \<v
		            \[light & +\\
		            comps & \ibox{2}\] \,\> \,$\oplus$ \ibox{2}\]}
          	\end{avm}}}
	
	    \ex[]{Object control verbs \\
    {\footnotesize
        \begin{avm}
		{\[head & \[{\normalfont{\textit{verb}}}\\
		            aux-\]\\
        light & +\\
        subj & \<np{\normalfont{\textit{\textsubscript{i}}}}\>\\
        comps & \<np{\normalfont{\textit{\textsubscript{j}}}}, v \[subj & \<np{\normalfont{\textit{\textsubscript{j}}}}\>\\
                    comps & \<\> \] \,\>\]}
          	\end{avm}
          	$\Rightarrow$
        \begin{avm}
		{\[comps & \<np{\normalfont{\textit{\textsubscript{j}}}}, v
		            \[light & +\\
		            comps & \ibox{2}\] \,\> \,$\oplus$ \ibox{2}\]}
          	\end{avm}}} \end{xlist}
\end{exe}

An example with the verb \textit{seltukha-} is given in Figure~\ref{GSfigure14}, which represents (\ref{GSexemple59d}).


The head comes last in Korean, except in the afterthought construction (\ref{GSexemple59b}), which requires an additional mechanism. This is true in the verbal complex, in the embedded VP and in the flat structure, as well. The alternative ordering for (\ref{GSexemple59a}) \textit{\*Maryka phwulyeko ku mwunceylul sitohayssta}, where the verb \textit{phwulye-ko} precedes its complement \textit{ku mwuncey-lul} is thus impossible. In (\ref{GSexemple59d}), the NP \textit{John-hantley} can follow the complement verb \textit{ilkulako} because it is the complement of the head verb \textit{seltukhayssta}, not of \textit{ilkulako}.  

Constraint (\ref{GSexemple61}) mirrors constraint (\ref{GSexemple29}) for Romance Languages.

\begin{exe} 
        \ex[]{
        \begin{avm}\label{GSexemple61}
		\[synsem \ibox{1}\] \,\, \normalfont{precedes}\,  \[comps\, \<...\ibox{1}...\>\,\]
	\end{avm}\jambox*{(\ili{Korean})}}
\end{exe}

\cite{CC1998} extends the possibility of argument attraction to adjuncts, as well as to constructions with a S complement, although in the latter case the data are somewhat marginal. The behavior of adjuncts is easily accounted for, if adjuncts can be treated as complements \citep{bouma2001satisfying} by a verb. For the second case, Chung proposes a flat structure, in which the subject, the complements and the (lexical) verbs are all sisters. In this analysis, the definition of a complex predicate, which relies on a syntactic relation between words can then be maintained (but see \citealt{lee2001argument} for a different proposal based on linearization).

\section{Light verb constructions in Persian: syntax and morphology, syntax and semantics}\label{GSsection5}

Light verb constructions constitute the third guise of complex predicates. They are characterized semantically: the verb and the second predicate constitute together a semantic predicate. For instance, the French expression combining a semantically light verb and a noun \textit{faire une proposition} `to make a proposal’ is close to \textit{proposer} `to propose’. They have been studied in HPSG for Korean \citep{Ryu:93, lee2001argument, choi2001mixed, Kim2016a-u}. We focus here on Persian light verb constructions, which form a rich class and tend to replace simplex verbs.   
    
\subsection{What are complex predicates in Persian?}\label{GSsection5.1}

Persian verbs (simplex verbs) constitute a small closed class of about 250 members, only around 100 of which are commonly used. Speakers resort to complex predicates, sequences of a light verb and a preverbal element, belonging to different categories (adjective, noun, particule, prepositional phrase). Following \cite{bonami2010persian}, \cite{pollet2012grammaire}, such sequences are `multi-word expressions', that is, they are made up of several words, which, together, form a lexeme. 

Several properties show that the elements are independent syntactic units \citep{Karimi-Doostan97a, megerdoomian2000beyond, pollet2012grammaire}. We concentrate on noun + verb combinations. All inflection is prefixed or suffixed on the verb, as is the negation in (\ref{GSexemple62}), and never on the noun. The two elements can be separated by the future auxiliary, or even by clearly syntactic constituents, like the complement PP in (\ref{GSexemple62}). Both the noun and the verb can be coordinated, as shown in (\ref{GSexemple63}) and (\ref{GSexemple64}) respectively. The noun can be extracted, as in the topicalization in (\ref{GSexemple65}), where the sign --indicates where the non extracted noun would have occurred. Complex predicates can be passivized. In this case, the nominal element of the complex predicate (\textit{tohmat} in (\ref{GSexemple66a})) becomes the subject of the passive construction (\ref{GSexemple66b}), as does the object of a transitive construction. The nominal part of the complex predicate is italicized in the examples.

\begin{exe}
	\ex[]{
		\gll \textit{Dast} be gol-h\=a na-zan\\
		Hand to {flower-\textsc{pl}} {\textsc{neg}-hit}\\
		\glt `Don't touch the flowers'}\label{GSexemple62}
\end{exe}

\begin{exe}
	\ex[]{
		\gll Mu-h\=a=\=a\v s=r\=a [\textit{boros} \textit{y\=a} \textit{\v s\=ane}] zad\\
		Hair-\textsc{pl=3sg=ra} brush or comb hit\\
		\glt `(S)he brushed or combed her hair'}\label{GSexemple63}
\end{exe}

\begin{exe}
	\ex[]{
		\gll Omid \textit{sili} [zad va xord]\\
		Omid slap hit and strike\\
		\glt `Omid gave and received slaps'}\label{GSexemple64}
\end{exe}

\begin{exe}
	\ex[]{
		\gll \textit{Dast} goft=am [be gol-h\=a -- na-zan]\\
		Hand {said-\textsc{1sg}} to {flower-\textsc{pl}} {} {\textsc{neg}-hit}\\
		\glt `I told you not to touch the flowers'}\label{GSexemple65}
\end{exe}

\begin{exe}
	\ex \label{GSexemple66} 
	\begin{xlist}
        \ex[]{
		\gll Maryam be Omid \textit{tohmat} zad\\ 
		Maryam to Omid slander hit\\
		\glt `Maryam slandered Omid'}\label{GSexemple66a}
		
	    \ex[]{
		\gll be Omid \textit{tohmat} zade \v sod\\ 
		to Omid  slander {hit.\textsc{pst-ptcp}} become\\
		\glt `Omid was slandered'}\label{GSexemple66b}
		
		\end{xlist}
\end{exe}

There is evidence that the verb and the noun in a complex predicate share one argument structure. In (\ref{GSexemple67a}), the verb \textit{d\=adan} takes two complements, the noun \textit{\=ab} and the PP \textit{[be b\=aq\v ce]}, while in (\ref{GSexemple67b}) the combination of \textit{d\=adan} and the noun \textit{\=ab} takes a direct objet, which is marked with \textit{-r\=a}: in (\ref{GSexemple67b}), the noun \textit{\=ab} and the verb \textit{d\=ad} form a complex predicate.  

\begin{exe}
	\ex \label{GSexemple67} 
	\begin{xlist}
        \ex[]{
		\gll Maryam [be b\=aq\v ce] \=ab d\=ad\\ 
		Maryam to garden water {give.\textsc{3sg}} \\
		\glt `Maryam watered the garden'}\label{GSexemple67a}
		
	    \ex[]{
		\gll Maryam [b\=aq\v ce=r\=a] \textit{\=ab} d\=ad\\ 
		Maryam {garden.\textsc{ra}} water {give.\textsc{3sg}} \\
		\glt `Maryam watered the garden'}\label{GSexemple67b}
		
		\end{xlist}
\end{exe}

Other properties show that the combination of the two elements, here a noun and a verb, behaves like a lexeme \citep{bonami2010persian}. Such combinations feed lexeme formation rules: for instance, the suffix \textit{–i} forms adjectives from verbs: \textit{xordan} `eat' > \textit{xordani} `edible', and the same is possible with complex predicates (\ref{GSexemple68}); perfect participles can be converted into adjectives, and this applies to complex predicates (\ref{GSexemple69}) (see also Section~\ref{GSsection5.2}).

\begin{exe}
	\ex \label{GSexemple68} 
	\begin{xlist}
        \ex[]{
		\gll dust d\=a\v stan > dustd\=a\v stani\\ 
		friend {have `love'} {} lovely\\}\label{GSexemple68a}
		
	    \ex[]{
		\gll xat xordan > xatxordani\\ 
		scratch {strike `be scratched'} {} scratchable\\}\label{GSexemple68b}
		
		\end{xlist}
\end{exe}

\begin{exe}
	\ex \label{GSexemple69} 
	\begin{xlist}
        \ex[]{
		\gll dast xordan > dastxorde\\ 
		hand {strike `to be sullied'} {} `sullied'\\}\label{GSexemple69a}
		
	    \ex[]{
		\gll xat xordan > xatxorde\\ 
		scratch {strike `be scratched'} {} `scratched'\\}\label{GSexemple69b}
		
		\end{xlist}
\end{exe}

The meaning of the complex predicate is often a specialization of the predictable meaning of the combination: \textit{dast d\=adan} lit. hand give means `shake hands', \textit{\v c\=aqu zadan} lit. knife hit means `stab', \textit{\v s\=ane zadan} lit. comb hit means `comb'. Each specialized meaning has to be learned in the same way as that of a lexeme. Analogy often plays a role in the creation of new lexemes, and this is also true of complex predicates. For instance, the family of complex predicates expressing manners of communication goes from \textit{telegr\=am zadan} `telegraph' where hitting \textit{(zadan)} is involved to cases where hitting is not clearly involved: \textit{telefon zadan} `phone’, \textit{imeyl zadan} `email', \textit{esemes zadan} `text' etc.   

These complex predicates raise two problems, a morpho-syntactic one and a semantic one. To solve them, we rely crucially on the same property of HPSG as in the preceding syntactic cases, that is, the view of heads as sharing information with their expected complements. 

\subsection{Complex predicates and derivational processes}\label{GSsection5.2}

Although Persian complex predicates are combinations of words, they feed some derivational rules; see Section~\ref{GSsection5.1}, examples (\ref{GSexemple68}), (\ref{GSexemple69}). We analyze here the case of agent nominalization, studied in \cite{muller2010persian}.\footnote{Müller’s analysis is couched in a slightly different framework.} The example he examines is especially interesting in that the nominalization does not exist independently of the complex predicate: as shown in (\ref{GSexemple70}), although no agent noun can be derived from the verb \textit{kon} `do', an agent noun can be derived from the complex predicate formed with the verb \textit{kon} and the adjective \textit{b\=az} `open'.


\begin{exe}
	\ex{\label{GSexemple70} 
	\begin{xlist}
        \ex[]{
		kon (`do'), * kon-ande (`do-er')}\label{GSexemple70a}
		\ex[]{ 
		b\=az kon (open + do `open'), b\=az-konande (`opener')}\label{GSexemple70b}
	\end{xlist}}
\end{exe}

The lexeme \textit{b\=az-konande} can be analyzed as a compound lexeme to which the suffix \textit{–ande} is added, indicating agent nominalization. Compound lexemes are made of two lexemes. A simple rule for noun-noun compounds is given in (\ref{GSexemple71}) \citep{bonami2018lexeme}, where the elements of the compound are the value of the feature MDTRS (morphological daughters):\footnote{For a similar approach to Morphology in HPSG, see \cite{orgun1996sign}, \cite{riehemann1998type}, \cite{koenig1999lexical}, \cite{sag2003syntactic}.}

\begin{exe}
        \ex[]{
        \begin{avm}\label{GSexemple71}
		{\[{\normalfont{\textit{lexeme}}} \\
		phon \ibox{1} $\oplus$ \ibox{2}\\ 
		synsem | cat |  head  \normalfont{\textit{noun}}\\
        mdtrs \< \,\[{\normalfont{\textit{lexeme}}} \\
                   phon \ibox{1} \\ 
                    synsem | head {\normalfont{\textit{noun}}} \] \,, \[{\normalfont{\textit{lexeme}}} \\
                   phon \ibox{2} \\ 
                    synsem | head {\normalfont{\textit{noun}}} \] \,\>\\\]} \end{avm}}
\end{exe}

Similarly, a noun can be formed from the elements of the complex predicate, in this instance an adjective and a verb. The verb \textit{kon} in the complex predicate \textit{b\=az kon} is described in (\ref{GSexemple72}). It expects a subject NP, the agent, and two complements, an adjective and an NP, the latter being attracted from the adjective. The content of the adjective is included in the content of the verb, as the nucleus of the caused \textit{soa} (`make something be adj').

\begin{exe}
        \ex[]{\begin{avm}\label{GSexemple72}
		{\[cat \normalfont{\textit{verb}}\\
		  arg-st \<np{\normalfont{\textsubscript{\textit{k}}}}, np{\normalfont{\textsubscript{\textit{j}}}}, {\normalfont{\textit{Adj}}} \[prd +  \\
			  arg-st \<np{\normalfont{\textsubscript{\textit{j}}}}\>\\
			  cont \ibox{1} \[\normalfont{\textit{open-relation}} \\
		            theme \normalfont{\textit{j}}\\\]\] \,\>\\
		  cont \[\normalfont{\textit{soa}}\\
		        nucleus \[\normalfont{\textit{cause-relation}}\\
		            causer \normalfont{\textit{k}}\\
		            soa-arg | nucleus \ibox{1}\\ \]\\\]\\\]}\end{avm}}
\end{exe}

The compound lexeme \textit{b\=az-konande} is made of the adjective and the verb, which are the morphological daughters, very similar to what they are in the complex predicate. The verbal element is expecting two complements, an adjective and an NP, and they have the same semantics as in the complex predicate: the verb denotes a cause relation taking as argument the adjective content, and the adjective content is a relation taking the nominal complement as its argument (SS abbreviates SYNSEM). 

\begin{exe}
        \ex[]{
            {\footnotesize
        \begin{avm}\label{GSexemple73}
		{\[\rm\textit{lexeme}\\
		phon \ibox{1} $\oplus$ \ibox{2} $\oplus$ \< \rm\textit{ande}\> \\ 
		synsem \[cat \[head \rm\textit{noun}\\
					comps \<\ibox{3} np\>\]\\
			       cont [ind \rm\textit{k}] \]\\
        mdtrs \<\,\[phon \ibox{1} \\ 
                    ss \ibox{4} \[ cat \rm\textit{adjective}\\
                    			arg-st \< \ibox{3} np\rm\textsubscript{\textit{j}}\> \\
					cont \ibox{5} \rm\textit{adj-rel (j)}\] 
                    \] \,, \[phon \ibox{2} \\ 
                    ss \[ cat \rm\textit{verb}\\
                    			arg-st \<np{\rm\textsubscript{\textit{k}}}, \ibox{3}, \ibox{4}\>\\
					cont | nucl \[\rm\textit{cause-rel}\\
								causer \rm\textit{k}\\
								soa-arg | nucl \ibox{5}\]\]\]\,\>\\\]}\end{avm}}}
\end{exe}

The agent noun keeps as a complement the NP expected by the verb, as illustrated in (\ref{GSexemple74}).   

\begin{exe}
	\ex[]{
		\gll [dar-e quti] b\=az-konande\\
		{lid-\textsc{ez}} box opener\\
		\glt `a bottle opener'}\label{GSexemple74}

\end{exe}

This compound nominalization is accompanied with the appropriate changes: the noun denotes the causer, the first argument of the verb m-daughter, and a derivational suffix \textit{(-ande)} is appended to the sequence of the two elements. Nothing in the rule requires that the agent noun (\textit{*kon-and\=er}) exist independently of the elements of the complex predicate. Hence, the data in (\ref{GSexemple70a}) are accounted for.

\subsection{The Semantics of light verb constructions}\label{GSsection5.3}

The nouns found in complex predicates are used either as referential, or as part of a complex predicate, where they are analyzed as predicative complements, participating in its semantics (with the feature [PRD +]). We assume that such nouns have two realizations, as [PRD +] and as [PRD -]. 

These complex predicates do not have a homogeneous semantics. The general idea is that the verb serves to turn a noun into a verb \citep{bonami2010persian}, but there is a spectrum, going from a (relatively) semantically compositional combination, to idioms whose semantics is not predictable from the components. Complex predicates exploit different schemas, which can be extended to new nouns, describing new situations. We will exemplify certain common cases, drawing on the detailed study of \textit{zadan} in \cite{pollet2012grammaire}. The uses of \textit{zadan} as a light verb are numerous and varied. We will not try to investigate them exhaustively; rather, we indicate different patterns for the combination of this verb with the noun. 

The semantics of a complex predicate is often a specialization of that of the simplex verb. For instance, \textit{lagad zadan} (kick hit) means `kick', and \textit{sili zadan} (slap hit) means `slap'. 

\begin{exe}
	\ex[]{
		\gll Ol\=aq be Omid lagad zad\\
		Donkey to Omid kick hit\\
		\glt `The donkey kicked Omid'}\label{GSexemple75}
\end{exe}

Within a hierarchical organization of the lexicon (chapter ?????), the content of the simplex verb is higher and less specialized than that of the predicative noun. Thus, the content of the complex predicate can be simply that of the noun, if it unifies with it. This is reminiscent of the way \cite{Wechsler1995c} represents the import of a PP with a verb like \textit{talk}; the verb content itself is represented as an \textit{soa} with one participant, the talker; the verb can take a number of PP complements (headed by \textit{to, about…}), and its content is then unified with that of the PP. The result is a description of an \textit{soa} with more information than is present in the verb alone. Similarly, here, the content of \textit{lagad} is more specialized than that of \textit{zadan}; hence, it takes over if it is unified with it. The complement of the complex predicate may be an NP or a PP headed by \textit{be} (the preposition is optional). 

\begin{exe}
        \ex[]{
        \begin{avm}\label{GSexemple76}
		{\[\rm \textit{zadan1-lxm}\\
		  cat \rm \textit{verb}\\
		  arg-st \<np{\rm\textsubscript{\textit{k}}}, {\rm \textit{(be)}} np{\rm \textsubscript{\textit{m}}}, n 
		  	\[cat [prd +]\\
			  cont \ibox{1}\]\,\>\\
		  cont \[\rm \textit{soa}\\
		        nucleus \ibox{1} \[\rm \textit{kick-relation}\\
		            agent \rm \textit{k}\\
		            undergoer \rm \textit{m}\]\]\]}\end{avm}}
\end{exe}

Another case where the combination gives more information than the simplex verb is when this verb takes as its predicative complement a noun denoting an instrument crucially involved in the situation \citep{bonami2010persian}. Such are, in different domains, \textit{\v c\=aqu zadan} (knife hit) `stab', \textit{telefon zadan} (phone hit) `phone', \textit{pi\=ano zadan} (piano hit) `play the piano'. We illustrate here \textit{\v s\=ane zadan} (comb hit) `comb'.    

\begin{exe}
	\ex[]{
		\gll Maryam {mu-h\=a=ya\v s=r\=a} \textit{\v s\=ane} zad\\
		Maryam {hair-\textsc{pl=3sg=ra}} comb hit\\
		\glt `Maryam combed her hair'}\label{GSexemple77}
\end{exe}

\begin{exe}
        \ex[]{\begin{avm}\label{GSexemple78}
		{\[\rm \textit{zadan2-lexeme}\\
		synsem \[cat {\normalfont{\textit{verb}}}\\
		  arg-st \<np{\rm \textsubscript{\textit{k}}}, np{\rm \textsubscript{\textit{m}}}, n 
		  	\[cat [prd +]\\
			  cont \ibox{2}\]\,\>\\
		  cont \ibox{2} \[\rm \textit{soa}\\
		  	sit \ibox{1} \\
		        nucleus \[\rm \textit{comb-relation}\\
		            agent \rm \textit{k}\\
		            undergoer \rm \textit{m}\]\]\]\\
		 background \,\{\rm\textbf{involves} (\ibox{1}, $\exists$ $x$ [\textbf{comb} $(x) \,\wedge$ \textbf{use} (\ibox{1}, $k, x$)]\}\]}
          	\end{avm}}
\end{exe}

Although the complex predicate takes on the content of the predicative complement, the semantics does not rely on unification as in the preceding case, but exploits a schema based on information present in the background: the existence of an object (the [PRD -] correspondent of the predicative complement), and the fact that, in the situation, such an object is used.  

Further from a compositional or recoverable meaning is the use of \textit{zadan} or more precisely \textit{xod=r\=a zadan} (self zadan) with a series of nouns denoting illnesses, handicaps or problematic states (like stupidity, ignorance…): it means `to pretend, feign' the illness or state in question.

\begin{exe}
	\ex[]{
		\gll Maryam {xod=r\=a} be div\=anegi zad\\
		Maryam {self=\textsc{ra}} to madness hit\\
		\glt `Maryam feigned madness'}\label{GSexemple79}
\end{exe}

This use of \textit{zadan} may be seen as an extension of its use with nouns denoting some sort of deceit, such as \textit{gul zadan} (deceit hit) `to deceive’: as in (\ref{GSexemple76}), the noun imposes its content to the combination, with a metaphorical use of the verb, retaining from the physical violence meaning of \textit{zadan} (`hit’) the idea of an action to the detriment of someone. Nevertheless, nothing in the actual combination in (\ref{GSexemple79}) indicates deception. Not all nouns for illnesses are acceptable, only those which cannot be really verified in the situation: a state of fatigue, but not a heart attack. We group them as \textit{internal-problematic-states}. Here the combination of the verb and the noun is standard, in that the noun is a semantic argument of the verb, but the meaning of the verb is unpredictable.

\begin{exe}
        \ex[]{{\small
        \begin{avm}\label{GSexemple80}
		{\[\rm \textit{zadan3-lexeme}\\
		cat \rm \textit{verb}\\
		  arg-st \<np\rm \textsubscript{\textit{k}}, pro\rm \textsubscript{\textit{k}}, \textsc{pp} \[cat \[pform be\\
		  prd + \\ \] \\
		  cont \ibox{1} | nucleus \[\rm \textit{internal-problematic-state} \\
		            experiencer \rm \textit{k}\\\]\\\] \,\>\\
		  cont \[\rm \textit{soa}\\
		        nucleus \[\rm \textit{pretend-relation}\\
		            agent \rm \textit{k}\\
		            theme \ibox{1}\\ \]\\\]\\\]}
          	\end{avm}}}
\end{exe}

Note that, contrary to \textit{zadan1-lexeme}, with which \textit{be} is optional, the \textit{zadan3-lexeme} requires the complement to be in fact a PP, headed by \textit{be}. We assume that the preposition \textit{be} (frequent in the complement of a complex predicate) is contentless and shares syntactic (the [PRD $\pm$] value) and semantic information with its complement, the predicative N ([CONT \ibox{1}]); this is indicated by treating \textit{be} as the value of the feature PREP(OSITION) FORM \citep[Chapter~3]{pollard1987information}.  

Finally, we turn to an idiom: \textit{dast zadan} (hand hit) meaning `start'. The combination may mean, in a more recoverable way `to touch' with PP complements denoting concrete objects (as in (\ref{GSexemple62})), or `to applaud' with a PP complement denoting a person (\ref{GSexemple81a}). However, it means `to start' with a PP complement denoting an event as in (\ref{GSexemple81b}).

\begin{exe}
	\ex \label{GSexemple81} 
	\begin{xlist}
        \ex[]{
		\gll Bar\=a=ya\v s xeyli dast zad-im\\ 
		{For=\textsc{3sg}} {a lot} hand {hit-\textsc{1pl}}\\
		\glt `We applauded him a lot'}\label{GSexemple81a}
		
	    \ex[]{
		\gll K\=argar-\=an be e'tes\=ab dast zad-an\\ 
		{Worker-\textsc{pl}} to strike hand {hit-\textsc{3pl}}\\
		\glt `The workers went on strike'}\label{GSexemple81b}
		
		\end{xlist}
\end{exe}

To represent the idiom, we resort to the feature LID (lexical identifier) which is associated with lexemes in the lexicon, and contains morpho-syntactic as well as semantic information, and allows the verb to select a specific form \citep{Sag2007a, Sag2012a}. Thus, the noun \textit{dast} in the idiom has a LID value \textit{dast}. The preposition \textit{be}, which heads the other complement, is the same as in \textit{zadan-3}: it identifies its content with that of its complement.

The description of \textit{zadan-4}, which occurs in the idiom \textit{dast zadan} `to start' is given in (\ref{GSexemple82}). The predicative noun complement being specified with the LID value \textit{dast}, it is only in combination with the noun \textit{dast} that \textit{zadan} acquires this meaning.


\begin{exe}
        \ex[]{
        \begin{avm}\label{GSexemple82}
		{\[\rm \textit{zadan4-lexeme}\\
		  cat \rm \textit{verb}\\
		  arg-st \<np{\normalfont{\textsubscript{\textit{k}}}}, pp \[pform be\\
		  cont \ibox{1}\] \,, n 
		  	\[cat [prd +]\\
			  lid \rm \textit{dast}\]\,\>\\
		  cont \[\rm \textit{soa}\\
		        nucleus \[\rm \textit{start-relation}\\
		            agent \rm \textit{k}\\
		            soa-arg \ibox{1} | nucleus \rm \textit{event-relation}\]\]\]}\end{avm}}
\end{exe}

\section{Conclusion}\label{GSsection6}

Following the usual definition of complex predicates in HPSG, as a series of (at least) two predicates, of which one is the head attracting the complements of the other, we have studied them in different languages, Romance Languages, German, Korean, and Persian. These languages illustrate three ways in which argument attraction (or composition) manifests itself: bounded dependencies (such as clitic climbing), flexible word order, mixing the arguments of the two predicates, and special semantic combination, which builds a lexeme out of the two predicates (particularly from the verb and the noun in light verb constructions). 

HPSG is well equipped to represent this phenomenon. The feature structure associated with a predicate specifies which complements it is waiting for, and the feature structure associated with a phrase allows it to be non-saturated regarding its complements, a possibility exploited by a number of verbs, which are or can be the head of a complex predicate: the phenomenon is lexically driven. Certain verbs have two entries, one which takes a saturated complement, one which is the head of a complex predicate; but a head can be itself flexible, accepting a complement which is saturated, partially saturated or not saturated at all: this is the case of the copula in Romance Languages.

Crucially, the mechanism of argument attraction is not tied with a specific syntactic structure; on the contrary, it is compatible with different structures: it is shown that the properties of a verbal complex (where the two predicates form a syntactic unit by themselves) differ from those of a flat structure (where the two predicates form a unit with the complements). The structures can characterize a language as opposed to another one (Spanish contrasts with Italian), but they can also be present within the same language (Korean auxiliaries contrast with Korean control verbs).

Similarly, the mechanism of argument attraction does not induce a specific semantic combination: it is compatible with a compositional semantics (as in a verb + adjective combination in Persian, or modal verb + infinitive complement in Romance Languages), as well as a variety of senses specific to the combination of the verb with a class of complements. The description in HPSG can exploit the hierarchical organization of the lexicon and the mechanism of unification (as with combinations specializing the meaning of the verb in Persian), the richness of the feature structure, appealing to a background feature (as when the noun corresponds to an instrument implied in the action), or to a special feature allowing to point to a specific form (for representing idioms).    

} % avmoptions

\section*{Abbreviations}

CAUS (causative), DAT (dative), DECL (particule indicating a declarative), 
NEG (negation), NOM (nominative), PART (particule), PASS (passive), PL (plural), PROGR
(progressive), SG (singular), PST-PTCP (past participle); = indicates an enclitic.

\section*{Acknowledgements}

We thank Anne Abeill\'e, Gabriela B\^ilb\^ie, Olivier Bonami, Caterina Donati,
  Jean-Pierre Koenig, Kil Soo Ko, Paola Monachesi, Stefan Müller, Tsuneko Nakazawa, and Stephen
  Wechsler. 

{\sloppy
	\printbibliography[heading=subbibliography,notkeyword=this]
}
\end{document}
