\documentclass[output=paper]{langsci/langscibook} 
\author{%
	Berthold Crysmann\affiliation{U Paris \& CNRS}%
	\lastand Gerald Penn\affiliation{University of Toronto}%
}
\title{Clitics}

% \chapterDOI{} %will be filled in at production

\epigram{Change epigram in chapters/03.tex or remove it there }
\abstract{Change the  abstract in chapters/03.tex }
\maketitle



\begin{document}
\label{chap-clitics}

\section{The notion of clitics}

% 
\subsection{Diagnostic criteria \citep{Zwicky83}}

\begin{itemize}
\item Criteria A--F
  
\item Coordination Criterion \citep{Miller92} 
\end{itemize}

\section{Lexical clitics}

\subsection{French}

\begin{itemize}
\item Clitic climbing by argument composition \citep{Miller97}
\item \textit{en}-cliticisation \citep{Sag:Godard:93}

\end{itemize}

\subsection{Romance}

\begin{itemize}
\item Restructuring verbs in Italian \citep{monachesi_p99book}
  
\item Romanian 
  
\end{itemize}

\subsection{Beyond Romance}

Augustinova: Bulgaria \citet{avgustinova_t97}

Augustinova \& Oliva: Czech

\section{Syntactic clitics}

\begin{itemize}
\item Serbo-Croat \citep{Penn:99}

\item Czech \citep{avgustinova_t-oliva_k95}
\item Polish pronominal clitics \citep{kupsc_a00,kupsc_a99}

\end{itemize}

\section{Clitics at the morphology--syntax interface}
% not cross-cutting.  These are "the other"

French: Miller's 1992 diss (GPSG)

Miller and Sag's 1993 CELR analysis 

Polish:
Borsley: halfway between composition and inflection

Kupsc and Tseng: trigger features go up, marking features come down
--- see below

\subsection{Edge inflection}

\begin{itemize}
\item English possessive \citep{zwicky_a87}
% fox's mother vs. foxes' mother  

\item Trigger features and marking features
  \citep{miller_p-halpern_a93}
% syntactic trigger features, which don't percolate along right edge

\item French \citep{Tseng02,Bonami14c}
\end{itemize}

\subsection{Mobile affixes}

\begin{itemize}
\item European Portuguese \citep{crysmann_b03book}
  
\item Sorani Kurdish endoclitics \citep{Bonami08f,Walther12}

\end{itemize}

\subsection{Polish clitic auxiliaries: edge inflection or mobile affixes?}

\citet{borsley_r99} \citet{Kupsc05} \citet{crysmann_b09degruyter}

% \section*{Abbreviations}
% \begin{tabularx}{.45\textwidth}{lQ}
% ... & \\
% ... & \\
% \end{tabularx}
% \begin{tabularx}{.45\textwidth}{lQ}
% ... & \\
% ... & \\
% \end{tabularx}

% \section*{Acknowledgements}
% \citet{Nordhoff2018} is useful for compiling bibliographies

 

\printbibliography[heading=subbibliography,notkeyword=this] 
\end{document}

%%% Local Variables:
%%% mode: latex
%%% TeX-master: "../main-clitics"
%%% End:
