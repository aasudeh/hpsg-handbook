%% -*- coding:utf-8 -*-
\documentclass[output=paper
                ,modfonts
                ,nonflat
	        ,collection
	        ,collectionchapter
	        ,collectiontoclongg
 	        ,biblatex
                ,babelshorthands
                ,newtxmath
                ,draftmode
                ,colorlinks, citecolor=brown
]{./langsci/langscibook}

\IfFileExists{../localcommands.tex}{%hack to check whether this is being compiled as part of a collection or standalone
  % add all extra packages you need to load to this file 

\usepackage{graphicx}
\usepackage{tabularx}
\usepackage{amsmath} 
\usepackage{tipa}      % Davis Koenig
\usepackage{multicol}
\usepackage{lipsum}


\usepackage{./langsci/styles/langsci-optional} 
\usepackage{./langsci/styles/langsci-lgr}
%\usepackage{./styles/forest/forest}
\usepackage{./langsci/styles/langsci-forest-setup}
\usepackage{morewrites}

\usepackage{tikz-cd}

\usepackage{./styles/tikz-grid}
\usetikzlibrary{shadows}


%\usepackage{pgfplots} % for data/theory figure in minimalism.tex
% fix some issue with Mod https://tex.stackexchange.com/a/330076
\makeatletter
\let\pgfmathModX=\pgfmathMod@
\usepackage{pgfplots}%
\let\pgfmathMod@=\pgfmathModX
\makeatother

\usepackage{subcaption}

% Stefan Müller's styles
\usepackage{./styles/merkmalstruktur,german,./styles/makros.2e,./styles/my-xspace,./styles/article-ex,
./styles/eng-date}

\selectlanguage{USenglish}

\usepackage{./styles/abbrev}

\usepackage{./langsci/styles/jambox}

% Has to be loaded late since otherwise footnotes will not work

%%%%%%%%%%%%%%%%%%%%%%%%%%%%%%%%%%%%%%%%%%%%%%%%%%%%
%%%                                              %%%
%%%           Examples                           %%%
%%%                                              %%%
%%%%%%%%%%%%%%%%%%%%%%%%%%%%%%%%%%%%%%%%%%%%%%%%%%%%
% remove the percentage signs in the following lines
% if your book makes use of linguistic examples
\usepackage{./langsci/styles/langsci-gb4e} 

% Crossing out text
% uncomment when needed
%\usepackage{ulem}

\usepackage{./styles/additional-langsci-index-shortcuts}

%\usepackage{./langsci/styles/langsci-avm}
\usepackage{./styles/avm+}


\renewcommand{\tpv}[1]{{\avmjvalfont\itshape #1}}

% no small caps please
\renewcommand{\phonshape}[0]{\normalfont\itshape}

\regAvmFonts

\usepackage{theorem}

\newtheorem{mydefinition}{Def.}
\newtheorem{principle}{Principle}

{\theoremstyle{break}
%\newtheorem{schema}{Schema}
\newtheorem{mydefinition-break}[mydefinition]{Def.}
\newtheorem{principle-break}[principle]{Principle}
}

% This avoids linebreaks in the Schema
\newcounter{schema}
\newenvironment{schema}[1][]
  {% \begin{Beispiel}[<title>]
  \goodbreak%
  \refstepcounter{schema}%
  \begin{list}{}{\setlength{\labelwidth}{0pt}\setlength{\labelsep}{0pt}\setlength{\rightmargin}{0pt}\setlength{\leftmargin}{0pt}}%
    \item[{\textbf{Schema~\theschema}}]\hspace{.5em}\textbf{(#1)}\nopagebreak[4]\par\nobreak}%
  {\end{list}}% \end{Beispiel}

%% \newcommand{schema}[2]{
%% \begin{minipage}{\textwidth}
%% {\textbf{Schema~\theschema}}]\hspace{.5em}\textbf{(#1)}\\
%% #2
%% \end{minipage}}

%\usepackage{subfig}





% Davis Koenig Lexikon

\usepackage{tikz-qtree,tikz-qtree-compat} % Davis Koenig remove

\usepackage{shadow}




\usepackage[english]{isodate} % Andy Lücking
\usepackage[autostyle]{csquotes} % Andy
%\usepackage[autolanguage]{numprint}

%\defaultfontfeatures{
%    Path = /usr/local/texlive/2017/texmf-dist/fonts/opentype/public/fontawesome/ }

%% https://tex.stackexchange.com/a/316948/18561
%\defaultfontfeatures{Extension = .otf}% adds .otf to end of path when font loaded without ext parameter e.g. \newfontfamily{\FA}{FontAwesome} > \newfontfamily{\FA}{FontAwesome.otf}
%\usepackage{fontawesome} % Andy Lücking
\usepackage{pifont} % Andy Lücking -> hand

\usetikzlibrary{decorations.pathreplacing} % Andy Lücking
\usetikzlibrary{matrix} % Andy 
\usetikzlibrary{positioning} % Andy
\usepackage{tikz-3dplot} % Andy

% pragmatics
\usepackage{eqparbox} % Andy
\usepackage{enumitem} % Andy
\usepackage{longtable} % Andy
\usepackage{tabu} % Andy


% Manfred's packages

%\usepackage{shadow}

\usepackage{tabularx}
\newcolumntype{L}[1]{>{\raggedright\arraybackslash}p{#1}} % linksbündig mit Breitenangabe


% Jong-Bok

%\usepackage{xytree}

\newcommand{\xytree}[2][dummy]{Let's do the tree!}

% seems evil, get rid of it
% defines \ex is incompatible with gb4e
%\usepackage{lingmacros}

% taken from lingmacros:
\makeatletter
% \evnup is used to line up the enumsentence number and an entry along
% the top.  It can take an argument to improve lining up.
\def\evnup{\@ifnextchar[{\@evnup}{\@evnup[0pt]}}

\def\@evnup[#1]#2{\setbox1=\hbox{#2}%
\dimen1=\ht1 \advance\dimen1 by -.5\baselineskip%
\advance\dimen1 by -#1%
\leavevmode\lower\dimen1\box1}
\makeatother


% YK -- CG chapter

%\usepackage{xspace}
\usepackage{bm}
\usepackage{bussproofs}


% Antonio Branco, remove this
\usepackage{epsfig}

% now unicode
%\usepackage{alphabeta}



% Berthold udc
%\usepackage{qtree}
%\usepackage{rtrees}

\usepackage{pst-node}

  %add all your local new commands to this file

\makeatletter
\def\blx@maxline{77}
\makeatother


\newcommand{\page}{}



\newcommand{\todostefan}[1]{\todo[color=orange!80]{\footnotesize #1}\xspace}
\newcommand{\todosatz}[1]{\todo[color=red!40]{\footnotesize #1}\xspace}

\newcommand{\inlinetodostefan}[1]{\todo[color=green!40,inline]{\footnotesize #1}\xspace}


\newcommand{\spacebr}{\hspaceThis{[}}

\newcommand{\danish}{\jambox{(\ili{Danish})}}
\newcommand{\english}{\jambox{(\ili{English})}}
\newcommand{\german}{\jambox{(\ili{German})}}
\newcommand{\yiddish}{\jambox{(\ili{Yiddish})}}
\newcommand{\welsh}{\jambox{(\ili{Welsh})}}

% Cite and cross-reference other chapters
\newcommand{\crossrefchaptert}[2][]{\citet*[#1]{chapters/#2}, Chapter~\ref{chap-#2} of this volume} 
\newcommand{\crossrefchapterp}[2][]{(\citealp*[#1][]{chapters/#2}, Chapter~\ref{chap-#2} of this volume)}
% example of optional argument:
% \crossrefchapterp[for something, see:]{name}
% gives: (for something, see: Author 2018, Chapter~X of this volume)

\let\crossrefchapterw\crossrefchaptert



% Davis Koenig

\let\ig=\textsc
\let\tc=\textcolor

% evolution, Flickinger, Pollard, Wasow

\let\citeNP\citet

% Adam P

%\newcommand{\toappear}{Forthcoming}
\newcommand{\pg}[1]{p.#1}
\renewcommand{\implies}{\rightarrow}

\newcommand*{\rref}[1]{(\ref{#1})}
\newcommand*{\aref}[1]{(\ref{#1}a)}
\newcommand*{\bref}[1]{(\ref{#1}b)}
\newcommand*{\cref}[1]{(\ref{#1}c)}

\newcommand{\msadam}{.}
\newcommand{\morsyn}[1]{\textsc{#1}}

\newcommand{\nom}{\morsyn{nom}}
\newcommand{\acc}{\morsyn{acc}}
\newcommand{\dat}{\morsyn{dat}}
\newcommand{\gen}{\morsyn{gen}}
\newcommand{\ins}{\morsyn{ins}}
\newcommand{\loc}{\morsyn{loc}}
\newcommand{\voc}{\morsyn{voc}}
\newcommand{\ill}{\morsyn{ill}}
\renewcommand{\inf}{\morsyn{inf}}
\newcommand{\passprc}{\morsyn{passp}}

%\newcommand{\Nom}{\msadam\nom}
%\newcommand{\Acc}{\msadam\acc}
%\newcommand{\Dat}{\msadam\dat}
%\newcommand{\Gen}{\msadam\gen}
\newcommand{\Ins}{\msadam\ins}
\newcommand{\Loc}{\msadam\loc}
\newcommand{\Voc}{\msadam\voc}
\newcommand{\Ill}{\msadam\ill}
\newcommand{\INF}{\msadam\inf}
\newcommand{\PassP}{\msadam\passprc}

\newcommand{\Aux}{\textsc{aux}}

\newcommand{\princ}[1]{\textnormal{\textsc{#1}}} % for constraint names
\newcommand{\notion}[1]{\emph{#1}}
\renewcommand{\path}[1]{\textnormal{\textsc{#1}}}
\newcommand{\ftype}[1]{\textit{#1}}
\newcommand{\fftype}[1]{{\scriptsize\textit{#1}}}
\newcommand{\la}{$\langle$}
\newcommand{\ra}{$\rangle$}
%\newcommand{\argst}{\path{arg-st}}
\newcommand{\phtm}[1]{\setbox0=\hbox{#1}\hspace{\wd0}}
\newcommand{\prep}[1]{\setbox0=\hbox{#1}\hspace{-1\wd0}#1}

%%%%%%%%%%%%%%%%%%%%%%%%%%%%%%%%%%%%%%%%%%%%%%%%%%%%%%%%%%%%%%%%%%%%%%%%%%%

% FROM FS.STY:

%%%
%%% Feature structures
%%%

% \fs         To print a feature structure by itself, type for example
%             \fs{case:nom \\ person:P}
%             or (better, for true italics),
%             \fs{\it case:nom \\ \it person:P}
%
% \lfs        To print the same feature structure with the category
%             label N at the top, type:
%             \lfs{N}{\it case:nom \\ \it person:P}

%    Modified 1990 Dec 5 so that features are left aligned.
\newcommand{\fs}[1]%
{\mbox{\small%
$
\!
\left[
  \!\!
  \begin{tabular}{l}
    #1
  \end{tabular}
  \!\!
\right]
\!
$}}

%     Modified 1990 Dec 5 so that features are left aligned.
%\newcommand{\lfs}[2]
%   {
%     \mbox{$
%           \!\!
%           \begin{tabular}{c}
%           \it #1
%           \\
%           \mbox{\small%
%                 $
%                 \left[
%                 \!\!
%                 \it
%                 \begin{tabular}{l}
%                 #2
%                 \end{tabular}
%                 \!\!
%                 \right]
%                 $}
%           \end{tabular}
%           \!\!
%           $}
%   }

\newcommand{\ft}[2]{\path{#1}\hspace{1ex}\ftype{#2}}
\newcommand{\fsl}[2]{\fs{{\fftype{#1}} \\ #2}}

\newcommand{\fslt}[2]
 {\fst{
       {\fftype{#1}} \\
       #2 
     }
 }

\newcommand{\fsltt}[2]
 {\fstt{
       {\fftype{#1}} \\
       #2 
     }
 }

\newcommand{\fslttt}[2]
 {\fsttt{
       {\fftype{#1}} \\
       #2 
     }
 }


% jak \ft, \fs i \fsl tylko nieco ciasniejsze

\newcommand{\ftt}[2]
% {{\sc #1}\/{\rm #2}}
 {\textsc{#1}\/{\rm #2}}

\newcommand{\fst}[1]
  {
    \mbox{\small%
          $
          \left[
          \!\!\!
%          \sc
          \begin{tabular}{l} #1
          \end{tabular}
          \!\!\!\!\!\!\!
          \right]
          $
          }
   }

%\newcommand{\fslt}[2]
% {\fst{#2\\
%       {\scriptsize\it #1}
%      }
% }


% superciasne

\newcommand{\fstt}[1]
  {
    \mbox{\small%
          $
          \left[
          \!\!\!
%          \sc
          \begin{tabular}{l} #1
          \end{tabular}
          \!\!\!\!\!\!\!\!\!\!\!
          \right]
          $
          }
   }

%\newcommand{\fsltt}[2]
% {\fstt{#2\\
%       {\scriptsize\it #1}
%      }
% }

\newcommand{\fsttt}[1]
  {
    \mbox{\small%
          $
          \left[
          \!\!\!
%          \sc
          \begin{tabular}{l} #1
          \end{tabular}
          \!\!\!\!\!\!\!\!\!\!\!\!\!\!\!\!
          \right]
          $
          }
   }



% %add all your local new commands to this file

% \newcommand{\smiley}{:)}

% you are not supposed to mess with hardcore stuff, St.Mü. 22.08.2018
%% \renewbibmacro*{index:name}[5]{%
%%   \usebibmacro{index:entry}{#1}
%%     {\iffieldundef{usera}{}{\thefield{usera}\actualoperator}\mkbibindexname{#2}{#3}{#4}{#5}}}

% % \newcommand{\noop}[1]{}



% Rui

\newcommand{\spc}[0]{\hspace{-1pt}\underline{\hspace{6pt}}\,}
\newcommand{\spcs}[0]{\hspace{-1pt}\underline{\hspace{6pt}}\,\,}
\newcommand{\bad}[1]{\leavevmode\llap{#1}}
\newcommand{\COMMENT}[1]{}


% Andy Lücking gesture.tex
\newcommand{\Pointing}{\ding{43}}
% Giotto: "Meeting of Joachim and Anne at the Golden Gate" - 1305-10 
\definecolor{GoldenGate1}{rgb}{.13,.09,.13} % Dress of woman in black
\definecolor{GoldenGate2}{rgb}{.94,.94,.91} % Bridge
\definecolor{GoldenGate3}{rgb}{.06,.09,.22} % Blue sky
\definecolor{GoldenGate4}{rgb}{.94,.91,.87} % Dress of woman with shawl
\definecolor{GoldenGate5}{rgb}{.52,.26,.26} % Joachim's robe
\definecolor{GoldenGate6}{rgb}{.65,.35,.16} % Anne's robe
\definecolor{GoldenGate7}{rgb}{.91,.84,.42} % Joachim's halo

\makeatletter
\newcommand{\@Depth}{1} % x-dimension, to front
\newcommand{\@Height}{1} % z-dimension, up
\newcommand{\@Width}{1} % y-dimension, rightwards
%\GGS{<x-start>}{<y-start>}{<z-top>}{<z-bottom>}{<Farbe>}{<x-width>}{<y-depth>}{<opacity>}
\newcommand{\GGS}[9][]{%
\coordinate (O) at (#2-1,#3-1,#5);
\coordinate (A) at (#2-1,#3-1+#7,#5);
\coordinate (B) at (#2-1,#3-1+#7,#4);
\coordinate (C) at (#2-1,#3-1,#4);
\coordinate (D) at (#2-1+#8,#3-1,#5);
\coordinate (E) at (#2-1+#8,#3-1+#7,#5);
\coordinate (F) at (#2-1+#8,#3-1+#7,#4);
\coordinate (G) at (#2-1+#8,#3-1,#4);
\draw[draw=black, fill=#6, fill opacity=#9] (D) -- (E) -- (F) -- (G) -- cycle;% Front
\draw[draw=black, fill=#6, fill opacity=#9] (C) -- (B) -- (F) -- (G) -- cycle;% Top
\draw[draw=black, fill=#6, fill opacity=#9] (A) -- (B) -- (F) -- (E) -- cycle;% Right
}
\makeatother


% pragmatics
\newcommand{\speaking}[1]{\eqparbox{name}{\textsc{\lowercase{#1}\space}}}
\newcommand{\name}[1]{\eqparbox{name}{\textsc{\lowercase{#1}}}}
\newcommand{\HPSGTTR}{HPSG$_{\text{TTR}}$\xspace}

\newcommand{\ttrtype}[1]{\textit{#1}}
% \newcommand{\avmel}{\q<\quad\q>} %% shortcut for empty lists in AVM
\newcommand{\ttrmerge}{\ensuremath{\wedge_{\textit{merge}}}}
\newcommand{\Cat}[2][0.1pt]{%
  \begin{scope}[y=#1,x=#1,yscale=-1, inner sep=0pt, outer sep=0pt]
   \path[fill=#2,line join=miter,line cap=butt,even odd rule,line width=0.8pt]
  (151.3490,307.2045) -- (264.3490,307.2045) .. controls (264.3490,291.1410) and (263.2021,287.9545) .. (236.5990,287.9545) .. controls (240.8490,275.2045) and (258.1242,244.3581) .. (267.7240,244.3581) .. controls (276.2171,244.3581) and (286.3490,244.8259) .. (286.3490,264.2045) .. controls (286.3490,286.2045) and (323.3717,321.6755) .. (332.3490,307.2045) .. controls (345.7277,285.6390) and (309.3490,292.2151) .. (309.3490,240.2046) .. controls (309.3490,169.0514) and (350.8742,179.1807) .. (350.8742,139.2046) .. controls (350.8742,119.2045) and (345.3490,116.5037) .. (345.3490,102.2045) .. controls (345.3490,83.3070) and (361.9972,84.4036) .. (358.7581,68.7349) .. controls (356.5206,57.9117) and (354.7696,49.2320) .. (353.4652,36.1439) .. controls (352.5396,26.8573) and (352.2445,16.9594) .. (342.5985,17.3574) .. controls (331.2650,17.8250) and (326.9655,37.7742) .. (309.3490,39.2045) .. controls (291.7685,40.6320) and (276.7783,24.2380) .. (269.9740,26.5795) .. controls (263.2271,28.9013) and (265.3490,47.2045) .. (269.3490,60.2045) .. controls (275.6359,80.6368) and (289.3490,107.2045) .. (264.3490,111.2045) .. controls (239.3490,115.2045) and (196.3490,119.2045) .. (165.3490,160.2046) .. controls (134.3490,201.2046) and (135.4934,249.3212) .. (123.3490,264.2045) .. controls (82.5907,314.1553) and (40.8239,293.6463) .. (40.8239,335.2045) .. controls (40.8239,353.8102) and (72.3490,367.2045) .. (77.3490,361.2045) .. controls (82.3490,355.2045) and (34.8638,337.3259) .. (87.9955,316.2045) .. controls (133.3871,298.1601) and   (137.4391,294.4766) .. (151.3490,307.2045) -- cycle;
\end{scope}%
}


% KdK
\newcommand{\smiley}{:)}

\renewbibmacro*{index:name}[5]{%
  \usebibmacro{index:entry}{#1}
    {\iffieldundef{usera}{}{\thefield{usera}\actualoperator}\mkbibindexname{#2}{#3}{#4}{#5}}}

% \newcommand{\noop}[1]{}

% chngcntr.sty otherwise gives error that these are already defined
%\let\counterwithin\relax
%\let\counterwithout\relax

% the space of a left bracket for glossings
\newcommand{\LB}{\hspaceThis{[}}

\newcommand{\LF}{\mbox{$[\![$}}

\newcommand{\RF}{\mbox{$]\!]_F$}}

\newcommand{\RT}{\mbox{$]\!]_T$}}





% Manfred's

\newcommand{\kommentar}[1]{}

\newcommand{\bsp}[1]{\emph{#1}}
\newcommand{\bspT}[2]{\bsp{#1} `#2'}
\newcommand{\bspTL}[3]{\bsp{#1} (lit.: #2) `#3'}

\newcommand{\noidi}{§}

\newcommand{\refer}[1]{(\ref{#1})}

%\newcommand{\avmtype}[1]{\multicolumn{2}{l}{\type{#1}}}
\newcommand{\attr}[1]{\textsc{#1}}

\newcommand{\srdefault}{\mbox{\begin{tabular}{c}{\large <}\\[-1.5ex]$\sqcap$\end{tabular}}}

%% \newcommand{\myappcolumn}[2]{
%% \begin{minipage}[t]{#1}#2\end{minipage}
%% }

%% \newcommand{\appc}[1]{\myappcolumn{3.7cm}{#1}}


% Jong-Bok


% clean that up and do not use \def (killing other stuff defined before)
%\if 0
\def\DEL{\textsc{del}}
\def\del{\textsc{del}}

\def\conn{\textsc{conn}}
\def\CONN{\textsc{conn}}
\def\CONJ{\textsc{conj}}
\def\LITE{\textsc{lex}}
\def\lite{\textsc{lex}}
\def\HON{\textsc{hon}}

\def\CAUS{\textsc{caus}}
\def\PASS{\textsc{pass}}
\def\NPST{\textsc{npst}}
\def\COND{\textsc{cond}}



\def\hd-lite{\textsc{head-lex construction}}
\def\NFORM{\textsc{nform}}

\def\RELS{\textsc{rels}}
\def\TENSE{\textsc{tense}}


%\def\ARG{\textsc{arg}}
\def\ARGs{\textsc{arg0}}
\def\ARGa{\textsc{arg}}
\def\ARGb{\textsc{arg2}}
\def\TPC{\textsc{top}}
\def\PROG{\textsc{prog}}

\def\pst{\textsc{pst}}
\def\PAST{\textsc{pst}}
\def\DAT{\textsc{dat}}
\def\CONJ{\textsc{conj}}
\def\nominal{\textsc{nominal}}
\def\NOMINAL{\textsc{nominal}}
\def\VAL{\textsc{val}}
\def\val{\textsc{val}}
\def\MODE{\textsc{mode}}
\def\RESTR{\textsc{restr}}
\def\SIT{\textsc{sit}}
\def\ARG{\textsc{arg}}
\def\RELN{\textsc{rel}}
\def\REL{\textsc{rel}}
\def\RELS{\textsc{rels}}
\def\arg-st{\textsc{arg-st}}
\def\xdel{\textsc{xdel}}
\def\zdel{\textsc{zdel}}
\def\sug{\textsc{sug}}
\def\IMP{\textsc{imp}}
\def\conn{\textsc{conn}}
\def\CONJ{\textsc{conj}}
\def\HON{\textsc{hon}}
\def\BN{\textsc{bn}}
\def\bn{\textsc{bn}}
\def\pres{\textsc{pres}}
\def\PRES{\textsc{pres}}
\def\prs{\textsc{pres}}
\def\PRS{\textsc{pres}}
\def\agt{\textsc{agt}}
\def\DEL{\textsc{del}}
\def\PRED{\textsc{pred}}
\def\AGENT{\textsc{agent}}
\def\THEME{\textsc{theme}}
\def\AUX{\textsc{aux}}
\def\THEME{\textsc{theme}}
\def\PL{\textsc{pl}}
\def\SRC{\textsc{src}}
\def\src{\textsc{src}}
\def\FORM{\textsc{form}}
\def\form{\textsc{form}}
\def\GCASE{\textsc{gcase}}
\def\gcase{\textsc{gcase}}
\def\SCASE{\textsc{scase}}
\def\PHON{\textsc{phon}}
\def\SS{\textsc{ss}}
\def\SYN{\textsc{syn}}
\def\LOC{\textsc{loc}}
\def\MOD{\textsc{mod}}
\def\INV{\textsc{inv}}
\def\L{\textsc{l}}
\def\CASE{\textsc{case}}
\def\SPR{\textsc{spr}}
\def\COMPS{\textsc{comps}}
%\def\comps{\textsc{comps}}
\def\SEM{\textsc{sem}}
\def\CONT{\textsc{cont}}
\def\SUBCAT{\textsc{subcat}}
\def\CAT{\textsc{cat}}
\def\C{\textsc{c}}
\def\SUBJ{\textsc{subj}}
\def\subj{\textsc{subj}}
\def\SLASH{\textsc{slash}}
\def\LOCAL{\textsc{local}}
\def\ARG-ST{\textsc{arg-st}}
\def\AGR{\textsc{agr}}
\def\PER{\textsc{per}}
\def\NUM{\textsc{num}}
\def\IND{\textsc{ind}}
\def\VFORM{\textsc{vform}}
\def\PFORM{\textsc{pform}}
\def\decl{\textsc{decl}}
\def\loc{\textsc{loc   }}
% \def\   {\textsc{  }}

\def\NEG{\textsc{neg}}
\def\FRAMES{\textsc{frames}}
\def\REFL{\textsc{refl}}

\def\MKG{\textsc{mkg}}

\def\BN{\textsc{bn}}
\def\HD{\textsc{hd}}
\def\NP{\textsc{np}}
\def\PF{\textsc{pf}}
\def\PL{\textsc{pl}}
\def\PP{\textsc{pp}}
\def\SS{\textsc{ss}}
\def\VF{\textsc{vf}}
\def\VP{\textsc{vp}}
\def\bn{\textsc{bn}}
\def\cl{\textsc{cl}}
\def\pl{\textsc{pl}}
\def\Wh{\ital{Wh}}
\def\ng{\textsc{neg}}
\def\wh{\ital{wh}}
\def\ACC{\textsc{acc}}
\def\AGR{\textsc{agr}}
\def\AGT{\textsc{agt}}
\def\ARC{\textsc{arc}}
\def\ARG{\textsc{arg}}
\def\ARP{\textsc{arc}}
\def\AUX{\textsc{aux}}
\def\CAT{\textsc{cat}}
\def\COP{\textsc{cop}}
\def\DAT{\textsc{dat}}
\def\DEF{\textsc{def}}
\def\DEL{\textsc{del}}
\def\DOM{\textsc{dom}}
\def\DTR{\textsc{dtr}}
\def\FUT{\textsc{fut}}
\def\GAP{\textsc{gap}}
\def\GEN{\textsc{gen}}
\def\HON{\textsc{hon}}
\def\IMP{\textsc{imp}}
\def\IND{\textsc{ind}}
\def\INV{\textsc{inv}}
\def\LEX{\textsc{lex}}
\def\Lex{\textsc{lex}}
\def\LOC{\textsc{loc}}
\def\MOD{\textsc{mod}}
\def\MRK{{\nr MRK}}
\def\NEG{\textsc{neg}}
\def\NEW{\textsc{new}}
\def\NOM{\textsc{nom}}
\def\NUM{\textsc{num}}
\def\PER{\textsc{per}}
\def\PST{\textsc{pst}}
\def\QUE{\textsc{que}}
\def\REL{\textsc{rel}}
\def\SEL{\textsc{sel}}
\def\SEM{\textsc{sem}}
\def\SIT{\textsc{arg0}}
\def\SPR{\textsc{spr}}
\def\SRC{\textsc{src}}
\def\SUG{\textsc{sug}}
\def\SYN{\textsc{syn}}
\def\TPC{\textsc{top}}
\def\VAL{\textsc{val}}
\def\acc{\textsc{acc}}
\def\agt{\textsc{agt}}
\def\cop{\textsc{cop}}
\def\dat{\textsc{dat}}
\def\foc{\textsc{focus}}
\def\FOC{\textsc{focus}}
\def\fut{\textsc{fut}}
\def\hon{\textsc{hon}}
\def\imp{\textsc{imp}}
\def\kes{\textsc{kes}}
\def\lex{\textsc{lex}}
\def\loc{\textsc{loc}}
\def\mrk{{\nr MRK}}
\def\nom{\textsc{nom}}
\def\num{\textsc{num}}
\def\plu{\textsc{plu}}
\def\pne{\textsc{pne}}
\def\pst{\textsc{pst}}
\def\pur{\textsc{pur}}
\def\que{\textsc{que}}
\def\src{\textsc{src}}
\def\sug{\textsc{sug}}
\def\tpc{\textsc{top}}
\def\utt{\textsc{utt}}
\def\val{\textsc{val}}
\def\LITE{\textsc{lex}}
\def\PAST{\textsc{pst}}
\def\POSP{\textsc{pos}}
\def\PRS{\textsc{pres}}
\def\mod{\textsc{mod}}%
\def\newuse{{`kes'}}
\def\posp{\textsc{pos}}
\def\prs{\textsc{pres}}
\def\psp{{\it en\/}}
\def\skes{\textsc{kes}}
\def\CASE{\textsc{case}}
\def\CASE{\textsc{case}}
\def\COMP{\textsc{comp}}
\def\CONJ{\textsc{conj}}
\def\CONN{\textsc{conn}}
\def\CONT{\textsc{cont}}
\def\DECL{\textsc{decl}}
\def\FOCUS{\textsc{focus}}
\def\FORM{\textsc{form}}
\def\FREL{\textsc{frel}}
\def\GOAL{\textsc{goal}}
\def\HEAD{\textsc{head}}
\def\INDEX{\textsc{ind}}
\def\INST{\textsc{inst}}
\def\MODE{\textsc{mode}}
\def\MOOD{\textsc{mood}}
\def\NMLZ{\textsc{nmlz}}
\def\PHON{\textsc{phon}}
\def\PRED{\textsc{pred}}
%\def\PRES{\textsc{pres}}
\def\PROM{\textsc{prom}}
\def\RELN{\textsc{pred}}
\def\RELS{\textsc{rels}}
\def\STEM{\textsc{stem}}
\def\SUBJ{\textsc{subj}}
\def\XARG{\textsc{xarg}}
\def\bse{{\it bse\/}}
\def\case{\textsc{case}}
\def\caus{\textsc{caus}}
\def\comp{\textsc{comp}}
\def\conj{\textsc{conj}}
\def\conn{\textsc{conn}}
\def\decl{\textsc{decl}}
\def\fin{{\it fin\/}}
\def\form{\textsc{form}}
\def\gend{\textsc{gend}}
\def\inf{{\it inf\/}}
\def\mood{\textsc{mood}}
\def\nmlz{\textsc{nmlz}}
\def\pass{\textsc{pass}}
\def\past{\textsc{past}}
\def\perf{\textsc{perf}}
\def\pln{{\it pln\/}}
\def\pred{\textsc{pred}}


%\def\pres{\textsc{pres}}
\def\proc{\textsc{proc}}
\def\nonfin{{\it nonfin\/}}
\def\AGENT{\textsc{agent}}
\def\CFORM{\textsc{cform}}
%\def\COMPS{\textsc{comps}}
\def\COORD{\textsc{coord}}
\def\COUNT{\textsc{count}}
\def\EXTRA{\textsc{extra}}
\def\GCASE{\textsc{gcase}}
\def\GIVEN{\textsc{given}}
\def\LOCAL{\textsc{local}}
\def\NFORM{\textsc{nform}}
\def\PFORM{\textsc{pform}}
\def\SCASE{\textsc{scase}}
\def\SLASH{\textsc{slash}}
\def\SLASH{\textsc{slash}}
\def\THEME{\textsc{theme}}
\def\TOPIC{\textsc{topic}}
\def\VFORM{\textsc{vform}}
\def\cause{\textsc{cause}}
%\def\comps{\textsc{comps}}
\def\gcase{\textsc{gcase}}
\def\itkes{{\it kes\/}}
\def\pass{{\it pass\/}}
\def\vform{\textsc{vform}}
\def\CCONT{\textsc{c-cont}}
\def\GN{\textsc{given-new}}
\def\INFO{\textsc{info-st}}
\def\ARG-ST{\textsc{arg-st}}
\def\SUBCAT{\textsc{subcat}}
\def\SYNSEM{\textsc{synsem}}
\def\VERBAL{\textsc{verbal}}
\def\arg-st{\textsc{arg-st}}
\def\plain{{\it plain}\/}
\def\propos{\textsc{propos}}
\def\ADVERBIAL{\textsc{advl}}
\def\HIGHLIGHT{\textsc{prom}}
\def\NOMINAL{\textsc{nominal}}

\newenvironment{myavm}{\begingroup\avmvskip{.1ex}
  \selectfont\begin{avm}}%
{\end{avm}\endgroup\medskip}
\def\pfix{\vspace{-5pt}}


\def\jbsub#1{\lower4pt\hbox{\small #1}}
\def\jbssub#1{\lower4pt\hbox{\small #1}}
\def\jbtr{\underbar{\ \ \ }\ }


%\fi

  %% hyphenation points for line breaks
%% Normally, automatic hyphenation in LaTeX is very good
%% If a word is mis-hyphenated, add it to this file
%%
%% add information to TeX file before \begin{document} with:
%% %% hyphenation points for line breaks
%% Normally, automatic hyphenation in LaTeX is very good
%% If a word is mis-hyphenated, add it to this file
%%
%% add information to TeX file before \begin{document} with:
%% %% hyphenation points for line breaks
%% Normally, automatic hyphenation in LaTeX is very good
%% If a word is mis-hyphenated, add it to this file
%%
%% add information to TeX file before \begin{document} with:
%% \include{localhyphenation}
\hyphenation{
A-la-hver-dzhie-va
anaph-o-ra
affri-ca-te
affri-ca-tes
Atha-bas-kan
Chi-che-ŵa
com-ple-ments
Da-ge-stan
Dor-drecht
er-klä-ren-de
Ginz-burg
Gro-ning-en
Jon-a-than
Ka-tho-lie-ke
Ko-bon
krie-gen
Le-Sourd
moth-er
Mül-ler
Nie-mey-er
Prze-piór-kow-ski
phe-nom-e-non
re-nowned
Rie-he-mann
un-bound-ed
}

% why has "erklärende" be listed here? I specified langid in bibtex item. Something is still not working with hyphenation.


% to do: check
%  Alahverdzhieva

\hyphenation{
A-la-hver-dzhie-va
anaph-o-ra
affri-ca-te
affri-ca-tes
Atha-bas-kan
Chi-che-ŵa
com-ple-ments
Da-ge-stan
Dor-drecht
er-klä-ren-de
Ginz-burg
Gro-ning-en
Jon-a-than
Ka-tho-lie-ke
Ko-bon
krie-gen
Le-Sourd
moth-er
Mül-ler
Nie-mey-er
Prze-piór-kow-ski
phe-nom-e-non
re-nowned
Rie-he-mann
un-bound-ed
}

% why has "erklärende" be listed here? I specified langid in bibtex item. Something is still not working with hyphenation.


% to do: check
%  Alahverdzhieva

\hyphenation{
A-la-hver-dzhie-va
anaph-o-ra
affri-ca-te
affri-ca-tes
Atha-bas-kan
Chi-che-ŵa
com-ple-ments
Da-ge-stan
Dor-drecht
er-klä-ren-de
Ginz-burg
Gro-ning-en
Jon-a-than
Ka-tho-lie-ke
Ko-bon
krie-gen
Le-Sourd
moth-er
Mül-ler
Nie-mey-er
Prze-piór-kow-ski
phe-nom-e-non
re-nowned
Rie-he-mann
un-bound-ed
}

% why has "erklärende" be listed here? I specified langid in bibtex item. Something is still not working with hyphenation.


% to do: check
%  Alahverdzhieva

  \bibliography{../Bibliographies/stmue,
                ../localbibliography,
../Bibliographies/formal-background,
../Bibliographies/understudied-languages,
../Bibliographies/phonology,
../Bibliographies/case,
../Bibliographies/evolution,
../Bibliographies/agreement,
../Bibliographies/lexicon,
../Bibliographies/np,
../Bibliographies/negation,
../Bibliographies/argst,
../Bibliographies/binding,
../Bibliographies/complex-predicates,
../Bibliographies/coordination,
../Bibliographies/relative-clauses,
../Bibliographies/udc,
../Bibliographies/processing,
../Bibliographies/cl,
../Bibliographies/dg,
../Bibliographies/islands,
../Bibliographies/diachronic,
../Bibliographies/gesture,
../Bibliographies/semantics,
../Bibliographies/pragmatics,
../Bibliographies/information-structure,
../Bibliographies/idioms,
../Bibliographies/cg,
../Bibliographies/udc}

  \togglepaper[8]
}{}
\author{Frank Van Eynde\affiliation{University of Leuven}}
\title{Nominal structures}

% \chapterDOI{} %will be filled in at production

%\epigram{Change epigram in chapters/03.tex or remove it there }
\abstract{This chapter shows how nominal structures are treated in HPSG. 
The introduction puts the discussion in the broader framework of the NP vs. DP debate and 
differentiates three HPSG treatments, i.e. the specifier treatment, the DP treatment and the functor treatment. 
They are each presented in some detail and applied to the analysis of ordinary nominals. 
A comparison reveals that the DP treatment does not mesh as well with the monostratal 
surface-oriented nature of the HPSG framework as the other treatments. 
Then it is shown how the specifier treatment and the functor treatment deal with nominals that have 
idiosyncratic properties, such as the gerund, the Big Mess Construction and irregular 
P+NOM combinations.}

\begin{document}
\maketitle
\label{chap-np}

\section{Introduction}


We use the term \emph{nominal} in a theory-neutral way as standing for a noun
and its phrasal projection. In this broad sense all of the bracketed strings in 
(\ref{rb}) are nominals. 

\begin{exe} 
\ex\label{rb}  
{}[the [red [box]]] is empty
\end{exe} 
  
\noindent
For the analysis of nominals there are two main approaches in generative grammar. 
One treats the noun (N) as the head all the way through. In that analysis the 
largest bracketed string in (\ref{rb}) is an NP. 
The other makes a distinction between the nominal core, 
consisting of a noun with its complements and modifiers, if any,  
and a functional outer layer, comprising determiners, quantifiers and numerals. 
In that analysis the noun is the head of \emph{red box}, while the determiner is 
the head of \emph{the red box}, so that the category of the latter is DP. 

The first approach, henceforth called the \emph{NP approach}, prevailed in 
generative grammar up to and including the Government and Binding model
\citep{Chomsky81}. One of its modules, the categorial component, 
consists of phrase structure rules, such as those in (\ref{ps0}). 

\begin{exe} 
\ex\label{ps0}   
\begin{xlist}
\ex  VP ~ $\rightarrow$ ~ V ~ NP 
\ex  NP ~ $\rightarrow$ ~ Det ~ Nom
\end{xlist} 
\end{exe}

\noindent
They are required to ``meet some variety of X-bar theory'' \citep[5]{Chomsky81}. 
The original variety is that of \citet{Chomsky70}. It consists of the following
cross-categorial rule schemata:

\begin{exe} 
\ex\label{xbar} 
\begin{xlist} 
\ex\label{xbar1}   X$'$ ~ $\rightarrow$ ~ X ~  ... 
\ex\label{xbar2}   X$''$ ~ $\rightarrow$ ~ [Spec, X$'$] ~ X$'$ 
\end{xlist} 
\end{exe} 

\noindent
X$'$ stands for the combination of a head X and its complements,
where X is N, A or V, and X$''$ stands for the combination of 
X$'$ and its specifier ``where \mbox{[Spec,N$'$]} will be analyzed as the determiner'' 
\citep[210]{Chomsky70}. 
X-bar theory was further developed in \citet{Jackendoff77}, who added a
schema for the addition of adjuncts and who extended the range of 
X with P, the category of adpositions. A monostratal version of X-bar theory is 
developed in Generalized Phrase Structure Grammar (GPSG). Its application to nominals 
is exemplified in Figure \ref{sis}, quoted from \citet[126]{GPSG85}. 
The top node is the double-bar category N$''$, which 
consists of the determiner and the single-bar category N$'$. 
The AP and the relative clause are adjoined to N$'$, and 
the lowest N$'$ consists of the noun and its PP complement.

\begin{figure}
\centering
\begin{forest}
sm edges
[ N$''$
  [Det [that]]
  [N$'$
    [AP [very tall,roof]]
    [N$'$
      [N$'$
        [N [sister]]
        [PP [of Leslie,roof]]]
      [{S[+R]} [who we met,roof]]]]]
\end{forest}
\caption{\label{sis}An instance of the NP approach} 
\end{figure}
 
The second approach, henceforth called the \emph{DP approach}, results from an
extension of the range of X in (\ref{xbar}) to the functional categories. 
This was motivated by the fact that some of the phrase structure rules, 
such as (\ref{ps1}), do not fit the X-bar mould. 

\begin{exe} 
\ex\label{ps1}   S ~ $\rightarrow$ ~ NP ~ Aux ~ VP
\end{exe}   

\noindent
To repair this, the category Aux, which contained both auxiliaries and 
inflectional verbal affixes \citep{Chomsky57}, was renamed as I(nfl) and treated as the head of S. 
More specifically, I(nfl) was claimed to combine with a VP complement, yielding I$'$, 
and the latter was claimed to combine with an NP specifier (the subject), yielding I$''$
(formerly S).
For the analysis of nominals such an overhaul did not at first seem necessary, 
since the relevant PS rules did fit the X-bar mould, but it took place nonetheless, 
mainly in order to capture similarities between nominal and clausal structures. 
These are especially conspicuous in gerunds, nominalized infinitives and nominals 
with a deverbal head, and were seen as evidence for the claim that determiners have their 
own phrasal projection, just like the members of I(nfl) \citep{Abney87}. 
More specifically, members of D were claimed to take an N$''$ complement (formerly Nom), 
yielding D$'$, and the latter was claimed to have a specifier sister, as in Figure \ref{abn}.
The DP approach was also taken on board in other frameworks, 
such as Word Grammar \citep{Hudson90} and Lexical Functional Grammar \citep[99]{Bresnan00}. 

\begin{figure}
\centering
\begin{forest}
sm edges
[D$''$ 
	[D$'$
		[D [that]]
		[N$''$
			[N$'$ 
				[N [sister]]
				[PP [of Leslie,roof]]]]]]
\end{forest}
\caption{\label{abn} An instance of the DP approach} 
\end{figure}
     
Turning now to Head-Driven Phrase Structure Grammar, we find three different treatments.  
The first and oldest can be characterized as a lexicalist version of the NP approach, 
more specifically of its monostratal formulation in GPSG.  
It is first proposed in \citet{ps} and further developed in \citet{ps2} and 
\citet{GS00}. We henceforth call it the \emph{specifier treatment}, 
after the role which it assigns to the determiner. 
The second is a lexicalist version of the DP approach.  
It is first proposed in \citet{Netter94} and further developed in \citet{Netter96a}
and \citet{NerbonneMullen00}. We will call it the \emph{DP treatment}. 
The third adopts the NP approach, but neutralizes the distinction between adjuncts and specifiers, 
treating them both as functors. It is first proposed in \citet{VanEynde98a} and 
\citet{Allegranza98} and further developed in \citet{VanEynde03}, \citet{VanEynde06} 
and \citet{Allegranza06}. It is also adopted in Sign-Based Construction Grammar \citep{Sag2012}. 
We will call it the \emph{functor treatment}. This chapter presents the three treatments and 
compares them wherever this seems appropriate.  

We first focus on ordinary nominals (Section~\ref{ordi}) and then on nominals with idiosyncratic 
properties (Section~\ref{idio}). For exemplification we use English and a number of other Germanic 
and Romance languages, including Dutch, German, Italian and French.  
We assume familiarity with the Typed Feature Structure notation and with such basic notions 
as unification, inheritance and token-identity.\footnote{This chapter does not treat relative clauses,
since they are the topic of a separate chapter \crossrefchapterp{relative-clauses}.}
    
\section{Ordinary nominals} 
\label{ordi} 

We use the term \emph{ordinary nominal} for a nominal that contains a noun, 
any number of complements and/or adjuncts and at most one determiner. 
This section shows how such nominals are analyzed in respectively the 
specifier treatment (Section~\ref{spect}), the DP treatment (Section~\ref{dpt}) and 
the functor treatment (Section~\ref{funct}). 

    
\subsection{The specifier treatment} 
\label{spect} 


The specifier treatment adopts the same distinction between heads, complements, 
specifiers and adjuncts as X-bar theory, but its integration 
in a monostratal lexicalist framework inevitably leads to non-trivial differences,
as will be demonstrated in this section. 
The presentation is mainly based on \citet{ps2} and \citet{GS00}. 
We first discuss the syntactic structure (Section~\ref{syns}) and the semantic composition (Section~\ref{semco}) 
of nominals, and then turn to nominals with a phrasal specifier (Section~\ref{phrsp}). 


\subsubsection{Syntactic structure}
\label{syns} 

Continuing with the same example as in Figure \ref{abn}, 
a relational noun, such as \emph{sister}, selects a PP as its complement 
and a determiner as its specifier, as spelled out in 
the following \textsc{category} value:  

\begin{exe} 
\ex\label{n}
\begin{avm}
[\type{category}         \\ 
 head  & \type{noun}     \\
 spr   & <Det>            \\
 comps & <PP\[\type{of}\]> ]
\end{avm}
\end{exe}

\noindent
The combination with a matching PP, as in \emph{sister of Leslie},   
is subsumed by the \type{head-complements-phrase} type, as defined in \crossrefchaptert{properties}, 
and yields a nominal with an empty \textsc{comps} list.  
Similarly, the combination of this nominal with a matching determiner, as in \emph{that sister of Leslie},    
is subsumed by the \type{head-specifier-phrase} type,
and yields a nominal with an empty \textsc{spr} list, as spelled out in Figure \ref{les}. 

\begin{figure}
\centering
\begin{forest}
sm edges
[{[\textsc{head} \avmbox{1} \type{noun}, \textsc{spr} \eliste, \textsc{comps} \eliste]}
		[\avmbox{2} [that]]
		[{[\textsc{head} \avmbox{1}, \textsc{spr} \liste{\avmbox{2}}, \textsc{comps} \eliste]} 
			[{[\textsc{head} \avmbox{1}, \textsc{spr} \liste{\avmbox{2} Det}, \textsc{comps} \liste{\avmbox{3} PP}]} [sister]]
			[\avmbox{3} [of Leslie,roof]]]]	
\end{forest}
\caption{\label{les} Adnominal complements and specifiers}
\end{figure}

Since the noun is the head of \emph{sister of Leslie} and since the latter is 
the head of \emph{that sister of Leslie}, the Head Feature Principle implies 
that the phrase as a whole shares the \textsc{head} value of the noun (\iboxb{1}). 
The valence features, \textsc{comps} and \textsc{spr}, have a double role. 
On the one hand, they register the degree of saturation of the nominal; 
in this role they supersede the bar levels of X-bar theory. 
On the other hand, they capture co-occurrence restrictions, 
such as the fact that the complement of \emph{sister} be a PP, rather than an NP or a clause.

In contrast to complements and specifiers, adjuncts are not selected by their 
head sister. Instead, they are treated as selectors of their head sisters. 
To model this \citet[55--57]{ps2} employs the feature \textsc{mod(ified)}. 
It is part of the \textsc{head} value of the substantive parts-of-speech, 
i.e. noun, verb, adjective and preposition. Its value is of type \type{synsem} 
in the case of adjuncts and of type \type{none} otherwise.

\begin{exe} 
\ex   \type{substantive}: \begin{avm} 
                          [mod \type{synsem} $\vee$ \type{none}]  
                          \end{avm} 
\end{exe} 

\noindent
Attributive adjectives, for instance, select a nominal head sister 
which lacks a specifier, as spelled out in (\ref{rd}). 

\begin{exe} 
\ex\label{rd}
\begin{avm}
[\type{category}                              \\
 head [\type{adjective}                       \\
       mod|loc|category [head & \type{noun}   \\
                         spr  & \type{nelist}]]]
\end{avm}
\end{exe}

\noindent
The token-identity of the \textsc{mod(ified)} value of the adjective
with the \textsc{synsem} value of its head sister is part of the 
definition of type \type{head-adj(unct)-phr(ase)}, as defined in \crossrefchaptert{properties}. 
The requirement that the \textsc{spr} value of the selected nominal be a non-empty list
blocks the addition of adjectives to nominals which contain a 
determiner, as in *\emph{tall that bridge}.\footnote{This constraint is 
overruled in the Big Mess Construction, see Section~\ref{bime}.} 
Since the \textsc{mod(ified)} feature is part of the \textsc{head} value, 
it follows from the Head Feature Principle that it is shared between an adjective 
and the AP which it projects. As a consequence, the \textsc{mod(ified)} value of 
\emph{very tall} is shared with that of \emph{tall}, as shown in Figure \ref{lea}. 

\begin{figure}
\centering
\begin{forest}
sm edges
[{[\textsc{head} \avmbox{1} \type{noun}, {\sc spr} \eliste]}
	[\avmbox{2} [that]]
	[{[\textsc{head} \avmbox{1}, \textsc{spr} \liste{\avmbox{2}}]}
		[{[\textsc{head} \avmbox{4}]}
			[{[\textsc{head} \type{adv}]} [very]]
			[{[\textsc{head} \avmbox{4} [\type{adj} \textsc{mod} \avmbox{3}]]} [tall]]]
		[{\avmbox{3} [\textsc{head} \avmbox{1}, \textsc{spr} \liste{\avmbox{2} Det}]}, before drawing tree={x+=2em} [sister of Leslie, roof, before drawing tree={x+=2em}]]]]
\end{forest}
\caption{\label{lea} Adnominal modifiers}
\end{figure}
 
For languages in which attributive adjectives show number and gender agreement 
with the nouns they modify, the selected nominal is required to have specific 
number and gender values. The Italian \emph{grossa} `big', for instance, 
selects a singular feminine nominal and is, hence, compatible with a noun like 
\emph{scatola} `box', but not with the plural \emph{scatole} `boxes' nor with 
the masculine \emph{libro} `book' or \emph{libri} `books'.\footnote{This is an 
instance of concord, as defined in \crossrefchaptert{agreement}.}  


\subsubsection{Semantic composition}
\label{semco} 


Given the monostratal nature of HPSG, semantic representations 
do not constitute a separate level of representation, but take the form 
of attribute value pairs that are added to the syntactic representations.   
Phrase formation and semantic composition are, hence, modeled in tandem.  
Technically, the \textsc{content} feature is declared for the same type of objects 
as the \textsc{category} feature, as spelled out in (\ref{local}). 

\begin{exe} 
\ex\label{local}  \type{local}: \begin{avm} 
                   [category & \type{category}         \\
                    content  & \type{semantic-object}]
                   \end{avm} 
\end{exe} 

\noindent 
In the case of nominals the value of the \textsc{content} feature is of 
type \type{scope-object}, a subtype of \type{semantic-object} \citep[122]{GS00}. 
A scope-object is an index-restriction pair in which the index stands for 
entities and in which the restriction is a set of facts which constrain the 
denotation of the index, as in the \textsc{content} value of the noun \emph{box}:    

\begin{exe} 
\ex\label{red} 
\begin{avm}
[\type{scope-object}   \\
 index @1 \type{index} \\
 restr \{[\type{box}   \\
            arg @1 ]\}]
\end{avm} 
\end{exe}

\noindent
This is comparable to the representations which are canonically used in 
Predicate Logic (PL), such as \{x~$|$~\type{box}(x)\}, where x stands for 
the entities that the predicate \emph{box} applies to. In contrast to 
PL variables, HPSG indices are sorted with respect to person, number 
and gender. This provides the means to model the type of agreement that 
is called \emph{index agreement} in \crossrefchaptert{agreement}.

\begin{exe} 
\ex  \type{index}: \begin{avm}
                     [person & \type{person} \\
                      number & \type{number} \\
                      gender & \type{gender}] 
                     \end{avm} 
\end{exe} 

\textsc{content} values of attributive adjectives are also of type \type{scope-object}. 
When combined with a noun, as in \emph{red box}, the resulting representation 
is one in which the indices of the adjective and the noun are identical, as in 
(\ref{redbox}).\footnote{This is an example of intersective modification. 
The semantic contribution of other types of adjectives, such as  
\emph{alleged} and \emph{fake}, are modeled differently \citep[330--331]{ps2}.}   

\begin{exe} 
\ex\label{redbox} 
\begin{avm}
[\type{scope-object}     \\
 index @1                \\
 restr \{ [\type{red}    \\
             arg @1 ] ,
            [\type{box}  \\
             arg @1 ]\}]
\end{avm}
\end{exe}

\noindent
Also this is comparable to the PL practice of representing such 
combinations with one variable to which both predicates apply, as in 
\{x~$|$~\type{red}(x) \& \type{box}(x)\}. What triggers the index sharing is 
the \textsc{mod(ified)} value of the adjective, as illustrated by the \textsc{avm} of 
\emph{red} in (\ref{reddd}) \citep[55]{ps2}.\footnote{Boxed Greek capitals, 
such as \iboxb{\Sigma}, are used to indicate structure sharing for objects of 
type \type{set} \citep{GS00}.} 

\begin{exe} 
\ex\label{reddd}
\begin{avm}
[category|head [\type{adjective}                            \\
                mod|loc|content [\type{scope-object}        \\
                                 index @1                   \\
                                 restr \avmbox{$\Sigma$} ]] \\
 content [index @1                                          \\
          restr \{[\type{red}                               \\
                   arg @1 ]\} $\cup$ \avmbox{$\Sigma$} ]]
\end{avm}
\end{exe}

\noindent
The adjective selects a scope-object, shares its index and adds its own 
restriction to those that are already present. The resulting \textsc{content} 
value is then shared with the mother.

To model the semantic contribution of determiners, \citet[135--136]{GS00} 
make a distinction between scope-objects that contain a quantifier 
(\type{quant-rel}), and those that do not (\type{parameter}). 
In terms of this distinction, the addition of a quantifying determiner to a nominal, 
as in \emph{every red box}, triggers a shift from \type{parameter} to \type{quant-rel}. 
To capture this the specifier treatment employs the feature \textsc{spec(ified)}. 
It is part of the \textsc{head} value of the determiners, and its value is of type 
\type{semantic-object} \citep[362]{GS00}.\footnote{In \citet[45]{ps2} the \textsc{spec(ified)}
feature was also assigned to other function words, such as complementizers, 
and its value was of type \type{synsem}.}   

\begin{exe} 
\ex   \type{determiner}: \begin{avm} [spec \type{semantic-object}] \end{avm}  
\end{exe} 

\noindent
In the case of \emph{every}, the \textsc{spec} value is an object of 
type \type{parameter}, but its own \textsc{content} value is a subtype of 
\type{quant-rel} and this quantifier is put in store, to be retrieved 
at the place where its scope is determined, as illustrated by the AVM
of \emph{every} in (\ref{every}) \citep[204]{GS00}.  

\begin{exe} 
\ex\label{every} 
\begin{avm}
[category|head [\type{determiner}        \\
                spec [\type{parameter}   \\
                      index @1           \\
                      restr \avmbox{$\Sigma$} ]] \\
 content @2 [\type{every-rel}          \\
             index @1                  \\
             restr \avmbox{$\Sigma$} ] \\
 store \{ @2 \}]
\end{avm}
\end{exe}

\noindent 
Notice that the addition of the \textsc{spec} feature yields an analysis in which the determiner 
and the nominal select each other: the nominal selects 
its specifier by means of the valence feature \textsc{spr} and the determiner selects the nominal 
by means of \textsc{spec}.  


\subsubsection{Nominals with a phrasal specifier} 
\label{phrsp}


Specifiers of nominals tend to be single words, but they can also take the form 
of a phrase. The bracketed phrase in [\emph{the Queen of England's}] \emph{sister},
for instance, is in complementary distribution with the possessive
determiner in \emph{her sister} and has a comparable semantic contribution.   
For this reason it is treated along the same lines. More specifically, the 
possessive marker \emph{'s} is treated as a determiner that takes an NP as its specifier, 
as shown in Figure \ref{cousin} \citep[51--54]{ps2} and \citep[193]{GS00}.\footnote{The treatment 
of the phonologically reduced \emph{'s} as the head of a phrase is comparable to 
the treatment of the homophonous word in \emph{he's ill} as the head of a VP.
Notice that the possessive \emph{'s} is not a genitive affix, for if it were, it 
would be affixed to the head noun \emph{Queen}, as in *\emph{the Queen's of England sister}, see  
\citet[199]{SagWasow03}.}

\begin{figure}
\centering
\begin{forest}
sm edges
[{[\textsc{head} \avmbox{1} \type{noun}, \textsc{spr} \eliste]}
	[{\avmbox{2} [\textsc{head} \avmbox{3} \type{det}, \textsc{spr} \eliste]}
		[{\avmbox{4} [\textsc{head} \type{noun}, \textsc{spr} \eliste]} [the Queen of England, roof]]
		[{[\textsc{head} \avmbox{3}, \textsc{spr} \liste{\avmbox{4}}]} ['s]]]
	[{[\textsc{head} \avmbox{1}, \textsc{spr} \liste{\avmbox{2}}]} [sister]]]
\end{forest}
\caption{\label{cousin} Phrasal specifiers }  
\end{figure}

In this anaysis the specifier of \emph{sister} is a DetP that is headed by \emph{'s} 
and the latter takes the NP \emph{the Queen of England} as its specifier.\footnote{Since the 
specifier of \emph{'s} is an NP, it may in turn contain a specifier that is headed 
by \emph{'s}, as in \emph{John's uncle's car}.}
Semantically, \emph{'s} relates the index of its specifier (the possessor) to the index
of the nominal that it selects (the possessed), as spelled out in (\ref{poss}).\footnote{The
terms \emph{possessor} and \emph{possessed} are meant to be understood in a broad not-too-literal 
sense \citep{Nerbonne92}.}     

\begin{exe} 
\ex\label{poss}
\begin{avm}
[category [head [\type{determiner}                \\
                 spec [\type{parameter}           \\
                       index @1                   \\
                       restr \avmbox{$\Sigma$} ]] \\
           spr <[index @3]> ]                     \\
 content @2 [\type{the-rel}                       \\
               index @1                           \\
               restr \{[\type{poss-rel}           \\
                          possessor @3            \\
                          possessed @1 ]\} $\cup$ \avmbox{$\Sigma$} ] \\
 store \{ @2 \}]  
\end{avm}
\end{exe}

\noindent
The assignment of \type{the-rel} as the \textsc{content} value captures 
the definiteness of the resulting NP. Notice that this analysis contains a DetP, 
but in spite of that, it is not an instance of the DP approach, since the 
determiner does not head the nominal as a whole, but only its specifier. 


\subsection{The DP treatment} 
\label{dpt} 


An HPSG version of the DP approach has been developed in \citet{Netter94} and 
\citet{Netter96a}. We sketch the main characteristics of this treatment in Section~\ref{compl}
and discuss some problems for it in Section~\ref{prob}. 


\subsubsection{Functional complementation and functional completeness} 
\label{compl} 

\begin{figure}
\centering
\begin{forest}
sm edges
[{[\textsc{head} \avmbox{1} \type{det}, \textsc{comps} \eliste]}
	[{[\textsc{head} \avmbox{1}, \textsc{comps} \liste{\avmbox{4} NP}]} [that]]
	[{\avmbox{4} [\textsc{head} \avmbox{2} \type{noun}, \textsc{comps} \eliste]}
		[{[\textsc{head} \avmbox{2}, \textsc{comps} \liste{\avmbox{3} PP}]} [sister]]
		[\avmbox{3} [of Leslie, roof]]]]
\end{forest}
\caption{\label{net} Propagation of the \textsc{head} and \textsc{comps} values}
\end{figure}

The combination of a noun with its complements and its adjuncts is analyzed in much the 
same way as in the specifier treatment. The addition of the determiner, though, is modeled differently.  
It is not the nominal that selects the determiner as its specifier, but rather the determiner that 
selects the nominal as its complement.
More specifically, it selects the nominal by means of the valence feature \textsc{comps} and the 
result of the combination is a DP with an empty \textsc{comps} list, as in Figure \ref{net}.  
In this analysis there is no need for the valence feature \textsc{spr}.  
This looks like a gain of generalization, but in practice it is offset by the 
introduction of a distinction between functional 
complementation and ordinary complementation. To model it \citet[307--308]{Netter94} differentiates 
between major and minor \textsc{head} features: 

\begin{exe} 
\ex    \begin{avm}
       [head [major [n \type{boolean}   \\
                     v \type{boolean} ] \\
              minor [fcompl \type{boolean}]]]
       \end{avm} 
\end{exe} 

\noindent
The \textsc{major} attribute includes the boolean features N and V, where 
nouns are [+N, --V], adjectives [+N, +V], verbs [--N, +V] and prepositions [--N, --V]. 
Besides, [+N] categories also have the features \textsc{case}, \textsc{number} and \textsc{gender}. 
Typical of functional complementation is that the functional head shares the 
\textsc{major} value of its complement \citep[311--312]{Netter94}. 

\begin{exe} 
\ex\label{maj} Functional Complementation: In a lexical category of type \type{func-cat} the value of its \textsc{major} 
      attribute is token identical with the \textsc{major} value of its complement. 
\end{exe} 

\noindent
Since determiners are of type \type{func-cat}, they share the \textsc{major} value of their 
nominal complement and since that value is 
also shared with the DP (given the Head Feature Principle), it follows that the resulting 
DP is [+N,--V] and that its \textsc{case}, \textsc{number} and \textsc{gender} values are 
identical to those of its nominal non-head daughter. 
Nouns, by contrast, are not of type \type{func-cat} and, hence, do not share the 
\textsc{major} value of their complement. The noun \emph{sister} 
in Figure \ref{net}, for instance, does not share the part-of-speech of its PP complement. 

The \textsc{minor} attribute is used to model properties which a functional head 
does {\em not} share with its complement. It includes \textsc{fcompl}, a feature which 
registers whether a projection is functionally complete or not. Its value is positive for 
determiners, negative for singular count nouns and underspecified for plurals and mass nouns.  
Determiners take a nominal complement with a negative \textsc{fcompl} value, but their 
own \textsc{fcompl} value is positive and since they are the head, they share this value with 
the mother, as in Figure \ref{netter}. 
In this analysis, a nominal is complete, if it is both saturated 
(empty \textsc{comps} list) and functionally complete (positive \textsc{fcompl}), as 
spelled out in (\ref{min}) \citep[312]{Netter94}. 

\begin{exe} 
\ex\label{min} Functional Completeness Constraint: Every maximal projection is marked  
      as functionally complete in its \textsc{minor} feature. 
\end{exe}

\begin{figure}
\centering
\begin{forest}
sm edges
[{\avmbox{2} [\textsc{major} \avmbox{1} [--V, +N[\type{sing}]], \textsc{minor$|$fcompl} +]}
		[{\avmbox{2} [\textsc{major} \avmbox{1}, \textsc{minor$|$fcompl} +]} [that]]
		[{[\textsc{major} \avmbox{1}, \textsc{minor$|$fcompl} --]} [sister of Leslie, roof]]]
\end{forest}
\caption{\label{netter} Propagation of the \textsc{head} values}
\end{figure}


\subsubsection{Two problems for the DP treatment}  
\label{prob}


Given the definition of functional complementation in (\ref{maj})
determiners share the \textsc{major} value of the nominals which they select
and are, hence, nominal themselves, i.e. [+N, --V].
However, while this makes sense for determiners with (pro)nominal properties,
such as the English demonstrative \emph{that}, 
it is rather implausible for determiners with adjectival properties,
such as the German interrogative \emph{welch-} `which' and 
the Italian demonstrative \emph{questo} `this', which show the same variation for 
number, gender and case as the adjectives and which are subject to 
the same requirement on concord with the noun as the adnominal adjectives. 
Since such determiners have more in common with adjectives than with (pro)nouns,  
it would be more plausible to treat them as members of [+N, +V].  
The problem also affects the associated agreement features, i.e. \textsc{case}, 
\textsc{number} and \textsc{gender}. If a determiner 
is required to share the values of these features with its nominal complement,
as spelled out in (\ref{maj}), then we get implausible results for nominals in 
which the determiner and the noun do not show agreement.    
In the Dutch \emph{'s lands hoogste bergen} `the country's highest mountains', 
for instance, the selected nominal (\emph{hoogste bergen}) is plural and non-genitive, 
while the selecting determiner (\emph{'s lands}) is singular and genitive.  
The assumption that the latter shares the case and number of its nominal sister 
is, hence, factually wrong.  

Another problem concerns the assumption ``that all substantive categories will 
require the complement they combine with to be both saturated and 
functionally complete'' \citep[311]{Netter94}. Complements of verbs and 
prepositions must, hence, be positively specified for \textsc{fcompl}. 
This, however, is contradicted by the existence of 
prepositions which require their complement to be functionally incomplete.
The Dutch \emph{te} and \emph{per}, for instance, are not compatible with nominals 
that contain a determiner, also if the nominal is singular and count, 
as in \emph{te (*het/een) paard} `on horse' and \emph{per (*de/een) trein} `by train'.   
In this respect, they differ from most of the other Dutch prepositions, which 
require their nominal complement to have a determiner if it is singular and count,   
as in \emph{ze viel van *(het) paard} `she fell from *(the) horse' 
and \emph{ze zit op *(de) trein naar Londen} `she is on *(the) train to London'.  
This shows that there are prepositions which require their complement to be 
functionally complete, such as \emph{van} and \emph{op}, 
and prepositions which requitre it to be functionally incomplete, such as 
\emph{per} and \emph{te}. This distinction, though, cannot be made in 
an analysis that does not allow functionally incomplete complements. 


\subsection{The functor treatment} 
\label{funct}


The functor treatment adopts the NP approach, but in contrast to the specifier treatment 
it does not model specification and adjunction in different terms, and it does not adopt 
the distinction between substantive (or lexical) categories and functional 
categories.\footnote{The term \emph{functor} is also used in Categorial (Unification) Grammar, 
where it has a very broad meaning, subsuming the nonhead daughter in combinations of a 
head with a specifier or an adjunct, and the head daughter otherwise, 
see \citet{Bouma88}. This broad notion is also adopted in 
\citet{Reape94}. We adopt a more restrictive version in which functors 
are nonhead daughters which lexically select their head sister.}  
The presentation in this section is mainly based on \citet{VanEynde06} 
and \citet{Allegranza06}. We first discuss the motivation which underlies the adoption 
of the functor treatment (Section~\ref{motiv}) and then present its basic
properties (Section~\ref{sec-basics}). After that we turn to nominals with a 
phrasal specifier (Section~\ref{sec-phrasal-spec}) 
and to the hierarchy of \textsc{marking} values (Section~\ref{sec-without-spec}).    


\subsubsection{Motivation} 
\label{motiv}


The distinction between specifiers and adjuncts is usually motivated by 
the assumption that the former are obligatory and non-stackable, while the latter  
are optional and stackable. In practice, though, this distinction 
is blurred by the fact that many nominals are well-formed without specifier.
Bare plurals and singular mass nouns, for instance, are routinely used without 
specifier in English, and many other languages allow singular count nouns without 
specifier too. The claim that specifiers are obligatory is, hence, to be taken 
with a large pinch of salt. The same holds for their non-stackability. 
Italian possessives, for instance, are routinely preceded by an article, as in 
\emph{il nostro futuro} `the our future' and \emph{un mio amico} `a friend of mine'.     
The same holds for the Greek demonstratives, which are canonically preceded by the 
definite article. Also English has examples of this kind, as in \emph{his every wish}.    

Similar remarks apply to the distinction between lexical and functional categories. 
It plays a prominent role in the specifier and the DP treatment, both of which treat the 
determiners as members of a separate functional category Det, that is distinct from 
such lexical categories as N, Adj and Adv.
In practice, though, it turns out that the class of determiners is quite heterogeneous in 
terms of part-of speech. \citep{VanEynde06}, for instance, demonstrates that the 
Dutch determiners come in (at least) two kinds. On the one hand, there are those 
which show the same inflectional variation and the same concord with the noun as the 
prenominal adjectives: They take the affix \emph{-e} in combination with plural 
and singular non-neuter nominals, but not in combination with singular 
neuter nominals, as shown for the adjective \emph{zwart} in (\ref{wit}), 
for the possessive determiner \emph{ons} `our' in (\ref{ons}) and for the 
interrogative determiner \emph{welk} `which' in (\ref{welk}).\footnote{If the adjective 
is preceded by a definite determiner, 
it also takes the affix in singular neuter nominals. This phenomenon is treated 
in Section~\ref{sec-without-spec}.} 

\begin{exe} 
\ex\label{wit} 
\begin{xlist}
\ex
\gll  zwarte muren      \\
      black wall.\textsc{pl} \\
\ex
\gll  zwarte verf \\
      black paint.\textsc{sg.fem} \\
\ex
\gll  zwart zand \\
      black sand.\textsc{sg.neu} \\
\end{xlist}
\ex\label{ons}
\begin{xlist}
\ex
\gll onze ouders     \\
     our parent.\textsc{pl}  \\
\ex
\gll onze muur     \\
     our wall.\textsc{sg.mas}  \\
\ex
\gll ons huis     \\
     our house.\textsc{sg.neu}  \\
\end{xlist}
\ex\label{welk}
\begin{xlist} 
\ex
\gll welke boeken  \\
     which book.\textsc{pl} \\
\ex
\gll welke man                 \\
     which man.\textsc{sg.mas}  \\
\ex
\gll welk boek   \\
     which book.\textsc{sg.neu} \\
\end{xlist}
\end{exe} 

\noindent
On the other hand, there are determiners which are inflectionally invariant and which do 
do not show concord with the noun, such as the interrogative \emph{wiens} `whose' 
and the quantifying \emph{wat} `some'. 

\begin{exe} 
\ex\label{wiens}
\begin{xlist} 
\ex
\gll  wiens ouders \\ 
      whose parent.\textsc{pl} \\
\ex
\gll  wiens muur \\ 
      whose wall.\textsc{sg.mas} \\
\ex
\gll  wiens huis \\ 
      whose house.\textsc{sg.neu} \\
\end{xlist}
\ex\label{wat}
\begin{xlist}
\ex
\gll  wat boeken  \\
      some book.\textsc{pl} \\
\ex
\gll  wat verf                  \\
      some paint.\textsc{sg.fem} \\
\ex
\gll  wat zand  \\
      some sand.\textsc{sg.neu} \\
\end{xlist} 
\end{exe} 

\noindent
In that respect, they are like nouns that appear in prenominal position,
as in \emph{aluminium tafels} `aluminium tables' and \emph{de maximum lengte}
`the maximum length'. There are, hence, determiners with adjectival 
properties and determiners with nominal properties.   
The distinction is also relevant for other languages. The Italian 
possessives of the first and second person, for instance, 
show the same alternation for number and gender as the adjectives
and are subject to the same constraints on NP-internal concord, as illustrated 
for \emph{nostro} `our' in (\ref{nostr}). 

\begin{exe}  
\ex\label{nostr} 
\begin{xlist}
\ex 
\gll  il nostro futuro \\
      the our future.\textsc{sg.mas} \\ 
\ex 
\gll  la nostra scuola  \\
      the our school.\textsc{sg.fem} \\ 
\ex 
\gll  i nostri genitori \\
      the our parent.\textsc{pl.mas} \\ 
\ex 
\gll  le nostre scatole \\
      the our box.\textsc{pl.fem} \\ 
\end{xlist}
\end{exe} 

\noindent
By contrast, the possessive of the third person plural, \emph{loro} `their',
does not show any inflectional variation and does not show concord with the noun.

\begin{exe} 
\ex 
\begin{xlist}
\ex 
\gll  il loro futuro \\   
      the their future.\textsc{sg.mas}  \\ 
\ex 
\gll  la loro scuola  \\   
      the their school.\textsc{sg.fem}   \\ 
\ex 
\gll  i loro genitori \\   
      the their parent.\textsc{pl.mas} \\ 
\ex 
\gll  le loro scatole \\   
      the their box.\textsc{pl.fem} \\ 
\end{xlist} 
\end{exe}

\noindent
Confirming evidence for the distinction is provided by the fact that 
\emph{loro} is also used as a personal pronoun, whereas the other 
possessive determiners are not. 

\begin{exe} 
\ex 
\gll   Enrico ha dato una scatola a loro/*nostro. \\
       Enrico has given a box to them/*our \\
\trans `Enrico gave them a box.'
\end{exe} 

\noindent
In this context one has to use the pronoun \emph{noi} `us' instead. 
Besides, there are determiners with adverbial properties. 
\citet{Abeilleetal04}, for instance, 
assign adverbial status to the quantifying determiner in the French 
\emph{beaucoup de farine} `much flour', and the same could be argued to 
be plausible for such determiners as the English \emph{enough} and its 
Dutch equivalent \emph{genoeg}. 
In sum, there is evidence that the class of determiners is categorially 
heterogeneous and that a treatment which acknowledges this is potentially 
simpler and less stipulative than one which introduces a separate functional 
category for them. 
 

\subsubsection{Basics} 
\label{sec-basics}


\begin{figure}
\centering
\begin{forest}
%sm edges
[\type{headed-phrase}
	[\type{head-argument-phrase}
		[\type{head-comps-phr}]
		[\type{head-subj-phr}]
		[...]]
	[\type{head-nonarg-phrase}
		[\type{head-functor-phr}]
		[\type{head-indep-phr}]]]	
\end{forest}
\caption{\label{typ} Hierarchy of headed phrases}
\end{figure}

Technically, the elimination of the distinction between specifiers and adjuncts
implies that the \textsc{spr} feature is dropped.\footnote{Intriguingly, Noam
Chomsky has recently argued that there is no need for the notion of specifier in 
Transformational Grammar: ``There is a large and instructive literature 
on problems with Specifiers, but if the reasoning here is correct, they do not
exist and the problems are unformulable.'' \citet[43]{Chomsky13}.}  
Likewise, the elimination of the distinction between lexical and 
functional categories implies that there is no longer any need 
for separate selection features for them; \textsc{mod(ified)} and \textsc{spec(ified)}
are dropped and replaced by the more general \textsc{select}.  

To spell out the functor treatment in more detail we start from the 
hierarchy of headed phrases in Figure \ref{typ}. The basic distinction is
that between \type{head-argument-phrase} and \type{head-nonargument-phrase}. 
In the former the head daughter selects its non-head sister(s) by means of 
valence features, such as \textsc{comps} and \textsc{subj} (but not \textsc{spr}!), 
and it is their values that register the degree of saturation of the phrase, 
as shown for \textsc{comps} in Section~\ref{syns}.  
In head-nonargument phrases the degree of saturation is registered  
by the \textsc{marking} feature. It is declared for objects of type \type{category}, 
along with the \textsc{head} and valence features.\footnote{The \textsc{marking} feature  
is introduced in \citet[46]{ps2} to model the combination of a complementizer 
and a clause.} Its value is shared with the head daughter in head-argument phrases
and with the non-head daughter in head-nonargument phrases, as spelled out in 
(\ref{mark1}) and (\ref{mark2}) respectively. 

\begin{exe}
\ex\label{mark1} 
\type{head-argument-phrase}  ~ \impl ~ 
\begin{avm}
[synsem|loc|category|marking @1 \type{marking}  \\
 head-dtr|synsem|loc|category|marking @1] 
\end{avm}
\ex\label{mark2} 
\type{head-nonarg-phrase} ~ \impl ~
\begin{avm}
[synsem|loc|category|marking @1 \type{marking}   \\
 dtrs <[synsem|loc|category|marking @1]~, @2 > \\
 head-dtr @2 ]
\end{avm}
\end{exe}

\noindent
At a finer-grained level there is a distinction between two subtypes of 
\type{head-nonargument-phrase}. There is the type, called \type{head-functor-phrase},  
in which the non-head daughter selects its head sister. This selection is modeled 
by the \textsc{select} feature. Its value is an object of type \type{synsem} and is 
required to match the \textsc{synsem} value of the head daughter, as spelled out 
in (\ref{hefu}).  

\begin{exe}
\ex\label{hefu} 
\type{head-functor-phrase} ~ \impl ~ 
\begin{avm}
[dtrs <[synsem|loc|category|head|select @1]~, @2 > \\
 head-dtr @2 [synsem @1]]
\end{avm}
\end{exe} 

\noindent
The other subtype, called \type{head-independent-phrase}, subsumes combinations in 
which the nonhead daughter does not select its head sister.\footnote{This type is 
introduced in \citet[130]{VanEynde98a}. It will be used in Section~\ref{idio} to deal with 
idiosyncratic nominals, such as the Big Mess Construction and the Binominal Noun Phrase 
Construction.} In that case the \textsc{select} value of the nonhead daughter is of type 
\type{none}, as spelled out in (\ref{hein}). 

\begin{exe}
\ex\label{hein} 
\type{head-independent-phrase} ~ \impl ~\\
\begin{avm}
[dtrs <[synsem|loc|category|head|select \type{none}]~, @1> \\
 head-dtr @1 ]
\end{avm}
\end{exe}    

\begin{figure}
\centering
\oneline{%
\begin{forest}
sm edges
[{[\textsc{head} \avmbox{1} \type{noun} , \textsc{mark} \avmbox{2} \type{marked}]}
	[{[\textsc{head$|$sel} \avmbox{4} , \textsc{mark} \avmbox{2}]} [that]]
	[{\avmbox{4} [\textsc{head} \avmbox{1}, \textsc{mark} \avmbox{5} \type{unmarked}]}
		[{[\textsc{head$|$sel} \avmbox{3} , \textsc{mark} \avmbox{5}]} [long]]
		[{\avmbox{3} [\textsc{head} \avmbox{1}, \textsc{mark} \avmbox{5}]} [bridge]]]]
\end{forest}}
\caption{\label{markyy} Adnominal functors}
\end{figure}

An illustration of the functor treatment is given in Figure \ref{markyy}. 
The combination of the noun with the adjective is an instance of \type{head-functor-phrase}, 
in which the adjective selects an unmarked nominal (\iboxb{3}),  
shares its \textsc{marking} value (\iboxb{5}), and, being a non-argument, 
shares it with the mother as well. 
The combination of the resulting nominal with the demonstrative is also 
an instance of \type{head-functor-phrase},
in which the demonstrative selects an unmarked nominal (\iboxb{4}), 
but -- differently from the adjective -- its \textsc{marking} value is of type 
\type{marked}, and this value is shared with the mother (\iboxb{2}).    
This accounts for the well-formedness of \emph{that narrow long bridge} with stacked 
adjectives, as well as for the ill-formedness of 
*\emph{long that bridge} and *\emph{the that bridge}, since 
adnominal adjectives and articles are not compatible with a marked nominal.  
Whether an adnominal functor is marked or unmarked is subject to cross-linguistic variation. 
The Italian possessives, for instance, are unmarked and can, hence, be preceded 
by an article, as in \emph{il mio cane} `the my dog', but   
their French equivalents are marked: \emph{(*le) mon chien} `(*the) my dog'. 

In this treatment, determiners, understood as words that are in complementary distribution with 
the articles, are marked selectors of an unmarked nominal. Since this definition does not 
make reference to a specific part of speech, it is well equipped to deal with the categorial 
heterogeneity of the determiners. The English demonstrative \emph{that}, for instance, 
can be treated as a pronoun, not only when it is used in nominal position, as in 
\emph{I like that}, but also when it is used adnominally, as in \emph{I like that bike}.   
What captures the difference between these uses is not the part-of-speech but  
the \textsc{select} value: while the adnominal \emph{that} selects an unmarked nominal, 
its nominal counterpart does not select anything.      

        
\subsubsection{Nominals with a phrasal functor} 
\label{sec-phrasal-spec}


To illustrate how the treatment deals with phrasal functors we 
take the nominal \emph{a hundred pages}. Since the indefinite article is not 
compatible with a plural noun like \emph{pages} we assume that this phrase  
has a left branching structure in which the indefinite article selects 
the unmarked singular noun \emph{hundred} -- its plural counterpart is \emph{hundreds} --  
and in which the resulting NP selects the unmarked plural noun 
\emph{pages}, as spelled out in Figure \ref{glorie}. 
 
\begin{figure}
\centering
\oneline{%
\begin{forest}
sm edges
[{[\textsc{head} \avmbox{1} \type{noun} , \textsc{mark} \avmbox{2}]}
	[{[\textsc{head} \avmbox{4} [\type{noun} \textsc{sel} \avmbox{3}], \textsc{mark} \avmbox{2}]}
		[{[\textsc{head$|$sel} \avmbox{5} , \textsc{mark} \avmbox{2} \type{marked}]} [a]]
		[{\avmbox{5} [\textsc{head} \avmbox{4} , \textsc{mark} \type{unmarked}]} [hundred]]]
	[{\avmbox{3} [\textsc{head} \avmbox{1} , \textsc{mark} \type{unmarked}]} [pages]]]
\end{forest}}
\caption{\label{glorie} Phrasal functors }
\end{figure}

The \textsc{head} value of the entire NP is identified with that 
of \emph{pages} (\iboxb{1}), which accounts a.o. for the fact that it is plural:
\emph{a hundred pages are/*is missing}. 
Its \textsc{marking} value is identified with that of \emph{a hundred} 
(\iboxb{2}). The latter selects an unmarked plural nominal (\iboxb{3}) and 
since it is itself a head-functor-phrase, its \textsc{head} value is shared with 
that of the numeral \emph{hundred} (\iboxb{4}) and its \textsc{marking} value 
with that of the article (\iboxb{2}). Moreover, the latter selects an unmarked 
singular nominal (\iboxb{5}).
 
This treatment provides an account for the difference between 
the well-formed \emph{those two hundred pages} and the 
ill-formed *\emph{those a hundred pages}. The former is licensed since numerals 
like \emph{two} and \emph{hundred} are unmarked, while the latter is not, since 
the article is marked and since it shares that value with \emph{a hundred pages}.  


\subsubsection{The hierarchy of MARKING values} 
\label{sec-without-spec}


The distinction between marked and unmarked nominals in the functor treatment 
largely coincides with the distinction between nominals with an empty and a 
non-empty \textsc{spr} value in the specifier treatment. However, while  
the latter simply captures the difference between nominals with and without 
determiner, the former can be used to capture finer-grained distinctions.  
To illustrate the need for such finer-grained distinctions 
let us take another look at the attributive adjectives of Dutch. 
As already pointed out in Section~\ref{sec-basics}, they take the form without affix in  
singular neuter nominals, as in \emph{zwart zand} `black sand'. A complication, 
though, is that they canonically take the form with the affix if the nominal is  
introduced by a definite determiner, as in \emph{het zwarte zand} `the black sand'. 
This has consequences for the status of nominals with a singular neuter head: 
\emph{zwart zand} and *\emph{zwarte zand}, for instance, are both unmarked, 
but while the former is well-formed as it is, the latter is only 
well-formed if it is preceded by a definite determiner. 
To model this \citet{VanEynde06} differentiates between two types 
of \type{unmarked} nominals, as shown in Figure \ref{bare}. 

\begin{figure}
\centering
\begin{forest}
%sm edges
[\type{marking}
	[\type{unmarked}
		[\type{incomplete}]
		[\type{bare}]]
	[\type{marked}]]		
\end{forest}
\caption{\label{bare} Hierarchy of \textsc{marking} values} 
\end{figure}

Employing the more specific subtypes, the adjectives without affix which select a singular 
neuter nominal have the \textsc{marking} value \type{bare}, while their declined counterparts 
which select a singular neuter nominal have the value \type{incomplete}. 
Since this \textsc{marking} value is shared with the mother, the \textsc{marking} value 
of \emph{zwart zand} is \type{bare}, while that of *\emph{zwarte zand} is \type{incomplete}. 
The fact that the latter must be preceded by a definite determiner
is modeled in the \textsc{select} value of the determiner: 
while definite determiners select an unmarked nominal, which implies that 
they are compatible with both bare and incomplete nominals,
non-definite determiners select a bare nominal and are, hence, not compatible 
with an incomplete one, as in *\emph{een zwarte huis} `a black house'. 
The \textsc{marking} feature is, hence, useful to differentiate bare 
nominals from incomplete nominals.
  
In a similar way, one can make finer-grained distinctions in the hierarchy of  
\type{marked} values to capture co-occurrence restrictions between determiners and 
nominals, as in the functor treatment of the Italian determiner system of 
\citet{Allegranza06}. See also the treatment of nominals with idiosyncratic properties 
in Section~\ref{idio}. 


\subsection{Conclusion} 


This section has presented the three main treatments of nominal structures in HPSG. 
They are all surface-oriented and monostratal, and they are very similar in their 
treatment of the semantics of the nominals. 
The differences mainly concern the treatment of the determiners and the adjuncts. 
In terms of the dichotomy between NP and DP approaches, the specifier and the functor 
treatment side with the former, while the DP treatment sides with the latter. 
Overall, the NP treatments turn out to be more amenable to integration  
in a monostratal surface-oriented framework than the DP treatment, see also \citet{MuellerHeadless}. 
Of the two NP treatments
the specifier treatment is closer to early versions of X-bar theory and GPSG.   
The functor treatment is closer to versions of Categorial (Unification) Grammar, and 
has also been adopted in Sign-Based Construction Grammar \citep[155--157]{Sag2012}.

 

\section{Idiosyncratic nominals}
\label{idio}


This section focusses on the analysis of nominals with idiosyncratic properties. 
Since their analysis often requires a relaxation of the strictly lexicalist approach 
of early HPSG, we first introduce some basic notions of Constructional HPSG (Section~\ref{cohp}). Then we present analyses of nominals with a verbal core (Section~\ref{geru}), 
of the Big Mess Construction (Section~\ref{bime}) and of idiosyncratic [P + Nom] combinations 
(Section~\ref{prep}). Finally, we provide pointers to analyses of other nominals with 
idiosyncratic properties (Section~\ref{other}). 


\subsection{Constructional HPSG} 
\label{cohp}


The lexicalist approach of early HPSG can be characterized as one in which the 
properties of phrases are mainly determined by properties of the constituent words 
and only to a small extent by properties of the combinatory operations. 
\citet[391]{ps2}, for instance, employ no more than seven types 
of combinations, including those which were exemplified in Section~\ref{syns}, 
i.e. head-complements, head-adjunct and head-specifier.\footnote{The remaining four 
are head-subject, head-subject-complements, head-marker and head-filler.}   
Over time, though, the radical lexicalism gave way to an 
approach in which the properties of the combinatory operations  
play a larger role. The small inventory of highly abstract phrase types got 
replaced by a finer-grained hierarchy in which the types contain more specific 
and -- if need be -- idiosyncratic constraints. This development started in \citet{Sag97}, 
was elaborated in \citet{GS00}, and gained momentum afterward. 
Characteristic of Constructional HPSG is the use of a bidimensional hierarchy 
of phrasal signs. In such a hierarchy the phrases are not only partitioned 
in terms of \textsc{headedness}, but also in terms of a second dimension, called  
\textsc{clausality}, as in Figure \ref{bidim}. 

\begin{figure}
\centering
\begin{forest}
%sm edges
[\type{phrase}
	[\textsc{headedness}
		[\type{headed-phrase}
			[\type{head-subj-phr}, name=A]
			[...]]
		[\type{non-headed-phr}]]
	[\textsc{clausality}
		[\type{clause}
			[\type{declarative-cl} [\type{decl-head-subj-cl}, name=B]]
			[...]]
		[\type{non-clause}]]]
\draw (A.south) -- (B.north);
\end{forest}
\caption{\label{bidim} Bidimensional hierarchy of clauses}  
\end{figure}

The types in the \textsc{clausality} dimension are associated with constraints,
in much the same way as the types in the \textsc{headedness} dimension.  
Clauses, for instance, are required to denote an object of type \type{message} 
\citep[41]{GS00}.

\begin{exe}
\ex \type{clause} ~ \impl ~ 
\begin{avm}
[synsem|loc|content \type{message}] 
\end{avm}
\end{exe}

\noindent
At a finer-grained level, the clauses are partitioned into 
declarative, interrogative, imperative, exclamative and relative
clauses, each with their own constraints. 
Interrogative clauses, for instance, have a \textsc{content} value of type 
\type{question}, which is a subtype of \type{message}, and 
indicative declarative clauses have a \textsc{content} value of type 
\type{proposition}, which is another subtype of \type{message}.

Exploiting the possibilities of multiple inheritance one can 
define types which inherit properties from more than one supertype. 
The type \type{declarative-head-subject-clause}, for instance, inherits 
the properties of \type{head-subject-phrase}, on the one hand, and 
\type{declarative-clause}, on the other hand. Besides, it may 
have properties of its own, such as the fact that its head daughter 
is a finite verb \citep[43]{GS00}. 
This combination of multiple inheritance and specific   
constraints on maximal phrase types is also useful for the analysis of 
nominals with idiosyncratic properties, as will be shown in Sections \ref{geru} 
and \ref{bime}. 


\subsection{Nominals with a verbal core} 
\label{geru}


Ordinary nominals have a nominal core, but there are also nominals  
with a verbal core, such as gerunds and nominalized infinitives. They are 
of special interest, since they figure prominently in the argumentation 
that triggered the shift from the NP approach to the DP approach in Transformational 
Grammar. Some examples of gerunds are given in (\ref{gen})--(\ref{opt}), 
quoted from \citet[1290]{Quirketal85}. 

\begin{exe} 
\ex\label{gen}  {}[Brown's deftly painting his daughter] is a delight to watch. 
\ex\label{acc}  I dislike [Brown painting his daughter]. 
\ex\label{opt}  Brown is well known for [painting his daughter].
\end{exe}

\noindent
The bracketed phrases have the external distribution of an NP, 
taking the subject position in (\ref{gen}), 
the complement position of a transitive verb in (\ref{acc}) and 
the complement position of a preposition in (\ref{opt}). 
The internal structure of these phrases, though, shows a mixture of nominal and verbal 
characteristics. 
Typically verbal are the presence of an NP complement in (\ref{gen})--(\ref{opt}), 
of an adverbial modifier in (\ref{gen}) and of an accusative subject in (\ref{acc}). 
Typically nominal is the presence of the possessive in (\ref{gen}). 

\begin{figure}
\centering
\begin{forest}
%sm edges
[\type{part-of-speech}
	[\type{noun}
		[\type{proper-noun}]
		[\type{common-noun}]
		[\type{gerund}, name=A2]]
	[\type{relational}, name=A1
		[\type{verb}]
		[\type{adjective}]]]
\draw (A1.south) -- (A2.north);
\end{forest}
\caption{ \label{ger} The gerund as a mixed category }
\end{figure} 

To model this mixture of nominal and verbal properties \citep[65]{Malouf00} 
develops an analysis along the lines of the specifier treatment, in which 
the hierarchy of part-of-speech values is given more internal structure, as in 
Figure \ref{ger}. 
Instead of treating \type{noun}, \type{verb}, \type{adjective} etc. as 
immediate subtypes of \type{part-of-speech}, they are grouped in terms of 
intermediate types, such as \type{relational}, which subsumes a.o. verbs and adjectives, 
they are partitioned in terms of subtypes, such as \type{proper-noun}
and \type{common-noun}, and they are extended with types that inherit properties 
of more than one supertype, such as \type{gerund}, which is a subtype of both 
\type{noun} and \type{relational}. Beside the inherited properties the gerund has 
some properties of its own. These are spelled out in a lexical rule 
which derives gerunds from the homophonous present participles \citep[66]{Malouf00}.

\begin{exe}
\ex 
\begin{avm} 
[head  & [\type{verb}          \\
          vform \type{prp}]    \\
 subj  & <@1 np>               \\
 comps & @A                    \\
 spr   & < >]
\end{avm} \impl \begin{avm} 
                            [head  & \type{gerund}  \\
                             subj  & <@1>           \\
                             comps & @A             \\
                             spr   & <@1>] 
                            \end{avm}
\end{exe}

\noindent
This rule says that gerunds take the same complements 
as the present participles from which they are derived 
(\iboxb{A}).\footnote{Boxed Roman capitals stand for objects of type \type{list}, 
as in \citet{GS00} and \citet{SagWasow03}.}  
Their compatibility with adverbial modifiers follows from the 
fact that adverbs typically modify objects of type \type{relational},
which is a supertype of \type{gerund}. 
The availability of different options for realizing the subject is 
captured by the inclusion of the subject requirement of the present 
participle in both the \textsc{subj} list and the \textsc{spr} list of the gerund
(\iboxb{1}). To model the two options \citet[15]{Malouf00} employs the  
bidimensional hierarchy of phrase types in Figure \ref{bido}. 

\begin{figure}
\centering
\begin{forest}
%sm edges
[\type{phrase}
	[\textsc{headedness}
		[\type{headed-phr}
			[\type{head-subj-phr} [\type{nonfin-head-subj-cx}, name=A2]]
                        [...]			
                        [\type{head-spr-phr} [\type{noun-poss-cx}, name=B2]]]]			
	[\textsc{clausality}
		[\type{clause}, name=A1]
		[\type{non-clause}, name=B1]]]
\draw (A1.south) -- (A2.north);
\draw (B1.south) -- (B2.north);
\end{forest}
\caption{\label{bido} Bidimensional hierarchy of gerundial phrases } 
\end{figure}

The combination with an accusative subject is subsumed by \type{nonfin-head-subj-cx}, 
which is a subtype of \type{head-subject-phrase} and \type{clause}. 
Its defining properties are spelled out in (\ref{acccx}) \citep[16]{Malouf00}. 

\begin{exe}
\ex\label{acccx} 
\type{nonfin-head-subj-cx} ~ \impl ~ 
\begin{avm} 
[synsem|loc|category|head|root --                \\
 dtrs <[synsem|loc|category|head [\type{noun}   \\
                                  case \type{acc}]], @1 > \\
 head-dtr @1 ] 
\end{avm}
\end{exe} 

\noindent
This construction type subsumes combinations of a non-finite head with 
an accusative subject, as in (\ref{acc}). When the non-finite head is a gerund, 
the \textsc{head} value of the resulting clause is \type{gerund} 
and since that is a subtype of \type{noun}, the clause is also a nominal phrase. 
This accounts for the fact that its external distribution is that of an NP.  
By contrast, the combination with a possessive subject is subsumed by 
\type{noun-poss-cx}, which is a subtype of \type{head-specifier-phrase} and 
\type{non-clause} \citep[16]{Malouf00}.\footnote{Malouf treats 
the English possessive as a genitive, differently from \citet{SagWasow03}.}   

\begin{exe} 
\ex\label{gencx} 
\type{noun-poss-cx} ~ \impl ~ 
\begin{avm} 
[synsem|loc [category|head \type{noun}                    \\
             content \type{scope-object}]                 \\
 dtrs <[synsem|loc|category|head [\type{noun}             \\
                                  case \type{gen}]], @1 > \\
 head-dtr @1 ] 
\end{avm}
\end{exe}
 
\noindent
This construction subsumes combinations of a nominal and a 
possessive specifier, as in \emph{Brown's house}, and since 
since \type{noun} is a supertype of \type{gerund},  
it also subsumes combinations with the gerund, as in (\ref{gen}). 

In sum, Malouf's analysis of the gerund involves a reorganization of the 
part-of-speech hierarchy, a lexical rule and the addition of two construction types.     



\subsection{The Big Mess Construction} 
\label{bime}  


In ordinary nominals determiners precede attributive adjectives. Changing the order 
yields ill-formed combinations, such as *\emph{long that bridge} and *\emph{very tall every man}. 
However, this otherwise illegitimate order is precisely what we find in 
the Big Mess Construction (BMC), a term coined by \citet{Berman74}.  

\begin{exe}
\ex\label{bigme}
\begin{xlist}
\ex   It's [so good a bargain] I can't resist buying it.
\ex   [How serious a problem] is this?
\end{xlist}
\end{exe} 

\noindent
The idiosyncratic order in (\ref{bigme}) is required if the nominal is introduced 
by the indefinite article, and if the preceding AP is introduced by one of a small 
set of degree markers, including \emph{so, as, how, this, that} and \emph{too}. 


\subsubsection{A specifier treatment} 


A specifier treatment of the BMC is provided in \citet[201]{GS00}. It adopts  
a left branching structure, as in [[[\emph{so good}] \emph{a}] \emph{bargain}], 
in which \emph{so good} is the specifier of the indefinite article and in which 
\emph{so good a} is the specifier of \emph{bargain}. 
This is comparable to the treatment of the possessive in 
[[[\emph{the Queen of England}] \emph{'s}] \emph{sister}] in Section~\ref{phrsp}.  
However, while there is evidence that \emph{the Queen of England's} is a constituent,
since it may occur independently, as in (\ref{crown}), there is no evidence that 
\emph{so good a} is a constituent, as shown in (\ref{brgn}).

\begin{exe} 
\ex\label{crown}  This crown is [the Queen of England's].
\ex\label{brgn}   That bargain is [so good (*a)]. 
\end{exe} 

\noindent
Instead, there is evidence that the article forms a constituent with the following noun, 
since it also precedes the noun when the AP is in postnominal position, as in (\ref{barga}). 

\begin{exe} 
\ex\label{barga}  We never had [a bargain] [so good as this one].
\end{exe} 

\noindent
It is, hence, preferable to assign a structure in which the AP and the NP are sisters, as in  
[[\emph{so good}] [\emph{a bargain}]]. 


\subsubsection{A functor treatment} 


A structure in which the AP and the NP are sisters 
is adopted in \citet{VanEynde07}, \citet{KimSells11}, \citet{KaySag12}, 
\citet{ArnoldSadler14} and \citet{VanEynde18}, all of which are functor treatments. 
They also share the assumption that the combination is an NP and that its head daughter is 
the lower NP. The structure of the latter is spelled out in Figure \ref{aprob}. 
The article has a \textsc{marking} value of type \type{a} which is a subtype of \type{marked} and which it
shares with the mother.\footnote{The \textsc{marking} value of the article looks similar to its 
\textsc{phonology} value, but it is not the same. The \textsc{phonology} values of \emph{a} and \emph{an}, 
for instance, are different, but their \textsc{marking} value is not.} 

\begin{figure}
\centering
\begin{forest}
sm edges
[{[\textsc{head} $\avmbox{1}$ \type{noun}, \textsc{mark} $\avmbox{2}$ \type{a}]}
		[{[\textsc{head$|$sel} $\avmbox{3}$ , \textsc{mark} $\avmbox{2}$]} [a]]
		[{$\avmbox{3}$ [\textsc{head} $\avmbox{1}$, \textsc{mark} \type{unmarked}]} [bargain]]]
\end{forest}
\caption{\label{aprob} The lower NP }
\end{figure}

\begin{figure}
\centering
\begin{forest}
sm edges
[{[\textsc{head} $\avmbox{3}$ \type{adj}, \textsc{mark} $\avmbox{1}$ \type{marked}]}
		[{[\textsc{head$|$sel} $\avmbox{2}$ , \textsc{mark} $\avmbox{1}$]} [so]]
		[{$\avmbox{2}$ [\textsc{head} $\avmbox{3}$, \textsc{mark} \type{unmarked}]} [good]]]
\end{forest}
\caption{\label{sohow} The marked AP }
\end{figure}

The AP is also treated as an instance of the head-functor type 
in \citet{VanEynde07}, \citet{KimSells11} and \citet{VanEynde18}. 
The adverb has a \textsc{marking} value of type \type{marked}, 
so that the AP is marked as well, as shown in Figure \ref{sohow}.   
In combination with the fact that the article selects an unmarked nominal, 
this accounts for the ill-formedness of (\ref{ster}). 

\begin{exe}
\ex\label{ster}
\begin{xlist}
\ex   [*] {It's a so good bargain I can't resist buying it.} 
\ex   [*] {A how serious problem is it?}   
\end{xlist}
\end{exe}

\noindent
By contrast, adverbs like \emph{very} and \emph{extremely} are unmarked,
so that the APs which they introduce are admissible in this position, as in (\ref{st}).  

\begin{exe}
\ex\label{st}
\begin{xlist}
\ex  This is a very serious problem. 
\ex  We struck an extremely good bargain. 
\end{xlist} 
\end{exe} 

To model the combination of the AP with the lower NP it may at first seem 
plausible to treat the AP as a functor which selects  
an NP that is introduced by the indefinite article. This, however, has 
unwanted consequences: given that \textsc{select} is a \textsc{head} feature, 
its value is shared between the AP and the adjective, so that the latter 
has the same \textsc{select} value as the AP, erroneously licensing such 
combinations as *\emph{good a bargain}. To avoid this \citet{VanEynde18} models 
the combination in terms of a special type of phrase, called \type{big-mess-phrase}, 
whose place in the hierarchy of phrase types is defined in Figure \ref{prot}. 

\begin{figure}
\centering
\begin{forest}
%sm edges
[\type{phrase}
	[\textsc{headedness}
		[\type{headed-phrase}
			[\type{head-nonargument-phr}
				[\type{head-functor-phr} [\type{regular-nominal-phrase}, name=B2]]
				[\type{head-independent-phr}, name=A1]]]]
	[\textsc{clausality}
		[\type{non-clause}
			[\type{nominal-parameter}
			[\type{intersective-modification}, name=B1 [\type{big-mess-phrase}, name=A2]]]]]]
\draw (A1.south) -- (A2.north);
\draw (B1.south) -- (B2.north);
\end{forest}
\caption{ \label{prot} Bidimensional hierarchy of nominals} 
\end{figure}

The types in the \textsc{headedness} dimension are a subset of those in Figure \ref{typ}.  
The types in the \textsc{clausality} dimension mainly capture semantic and 
category-specific properties, in analogy with the hierarchy of clausal phrases 
in \citet{GS00}. One of the non-clausal phrase types is \type{nominal-parameter}: 

\begin{exe}
\ex\label{param} 
\type{nominal-parameter} ~ \impl ~
\begin{avm}
[synsem|loc [category|head \type{noun}                              \\
             content [\type{parameter}                               \\
                      index @1                                       \\
                      restr \ibox{\Sigma_{1}} $\cup$ \ibox{\Sigma_{2}}]] \\
 dtrs <[synsem|loc|content|restr \ibox{\Sigma_{1}} ] ~, @2>     \\
 head-dtr @2 [synsem|loc|content [\type{parameter}                  \\
                                    index @1                         \\
                                    restr \ibox{\Sigma_{2}} ]]]
\end{avm}
\end{exe}

\noindent
The mother shares its index with the head daughter (\iboxb{1}) and 
its \textsc{restr(iction)} value is the union of the \textsc{restr} values 
of the daughters (\iboxb{\Sigma_{1}} and \iboxb{\Sigma_{2}}). 
In the hierarchy of non-clausal phrases, this type contrasts amongst others with 
the quantified nominals, which have a \textsc{content} value of type 
\type{quant-rel} \citep[203--205]{GS00}. A subtype of \type{nominal-parameter} is  
\type{intersective-modification}, as defined in (\ref{mononom}).  

\begin{exe}
\ex\label{mononom} 
\type{intersective-modification} ~ \impl ~ 
\begin{avm}
[synsem|loc|content|index @1      \\
 dtrs <[synsem|loc|content|index @1 ] ~, @2 > \\
 head-dtr @2  ]
\end{avm}
\end{exe}

\noindent 
This constraint requires the mother to share its index also with the 
non-head daughter. It captures the intuition that the 
noun and its non-head sister apply to the same entities, as in 
the case of \emph{red box}.\footnote{Another subtype of \type{nominal-parameter} 
is \type{inverted-predication}, which subsumes amongst others 
the Binominal Noun Phrase Construction and certain types of apposition,
see Section~\ref{other}.}  

Maximal types inherit properties of one of the types of headed phrases,
on the one hand, and of one of the non-clausal phrase types, on the other hand.  
Regular nominal phrases, for instance, such as \emph{red box}, are subsumed 
by a type, called \type{regular-nominal-phrase}, that inherits the 
constraints of \type{head-functor-phrase}, on the one hand, and 
\type{intersective-modification}, on the other hand.  
Another maximal type is \type{big-mess-phrase}. 
Its immediate supertype in the \textsc{clausality} hierarchy is the same 
as for the regular nominal phrases, i.e. \type{intersective-modification}, 
but the one in the \textsc{headedness} hierarchy is different: 
being a subtype of \type{head-independent-phrase}, 
its non-head daughter does not select the head-daughter. Its \textsc{select} 
value is, hence, of type \type{none}. 
Beside the inherited properties the BMC has some properties of its own.   
They are spelled out in (\ref{bigmess2}).

\begin{exe}
\ex\label{bigmess2} 
\type{big-mess-phr} ~ \impl ~ 
\begin{avm}
 [ dtrs <[\type{head-functor-phrase}                             \\
          synsem|loc|category [head    & \type{adjective}      \\
                               marking & \type{marked}]] , @1>  \\
 head-dtr @1 [\type{regular-nominal-phrase}                    \\
                synsem|loc|category|marking \type{a}]]
\end{avm}
\end{exe}

\noindent
The head daughter is required to be a regular nominal phrase 
whose \textsc{marking} value is of type \type{a}, and the non-head daughter 
is required to be an adjectival head-functor phrase
with a \textsc{marking} value of type \type{marked}. 
This licenses APs which are introduced by a marked adverb, 
as in \emph{so good a bargain} and \emph{how serious a problem}, 
while it excludes unmarked APs, as in 
*\emph{good a bargain} and *\emph{very big a house}.
Iterative application is not licensed, since (\ref{bigmess2}) requires the 
head daughter to be of type \type{regular-nominal-phrase}, which is incompatible with the type 
\type{big-mess-phrase}. This accounts for the fact that a big
mess phrase cannot contain another big mess phrase, as in
*\emph{that splendid so good a bargain}.

A reviewer remarked that this analysis allows combinations like 
\emph{so big a red expensive house}, suggesting that it should not. 
We are not sure, though, that this combination is ill-formed.
Notice, for instance, that the sentences in (\ref{shrub}), 
quoted from \citet[116]{Zwicky95} and \citet[42]{Troseth09} respectively, 
are well-formed. 

\begin{exe} 
\ex\label{shrub} 
\begin{xlist} 
\ex  How big a new shrub from France were you thinking of buying? 
\ex  That's as beautiful a little black dress as I've ever seen.  
\end{xlist} 
\end{exe} 

In sum, the analysis of the Big Mess Phrase involves the addition of 
a type to the bidimensional hierarchy of phrase types, whose properties 
are partly inherited from its supertypes and partly idiosyncratic.      


\subsection{Idiosyncratic P+NOM combinations} 
\label{prep}


When an ordinary nominal combines with a preposition, the result is a PP. 
The French \emph{de} `of', for instance, heads a PP in 
\emph{je viens de Roubaix} `I come from Roubaix'. 
In \emph{beaucoup de farine} `much flour', by contrast, \emph{de} has 
a rather different role, as argued in \citet{Abeilleetal04}. 
Similar contrasts can be found in other languages. The English \emph{of}, for instance, 
heads a PP in \emph{the dog of the neighbors}, but its role in \emph{these sort of problems} 
is rather different, as argued in \citet{Maekawa15}.  


\subsubsection{A specifier treatment} 


In their specifier treatment of \emph{beaucoup de farine} `much flour'  
\citet{Abeilleetal04} treat \emph{de} as a weak head. 
Typical of a weak head is that it shares 
nearly all properties of its complement, as spelled out in (\ref{de}).

\begin{exe} 
\ex\label{de} 
\begin{avm} 
[category [head & @1                                    \\
           subj & @A                                    \\
           spr  & @B                                     \\
           comps & <[category [head & @1                 \\
                               subj & @A                 \\
                               spr  & @B                 \\
                               comps & < >]              \\
                               marking & \type{unmarked} \\
                     content @4 ]>                       \\ 
           marking & \type{de}]                          \\
 content @4 ] 
\end{avm}
\end{exe} 

\noindent
\emph{de} has the same values for \textsc{head}, \textsc{subj}, \textsc{spr} and 
\textsc{content} as its nominal complement. 
The only difference concerns the \textsc{marking} value: \emph{de} requires an 
unmarked complement, but its own \textsc{marking} value is of type \emph{de}. 
When combined with a noun, such as \emph{farine} `flour', \emph{de} is itself a noun 
that selects a specifier and that denotes a parameter.
Since it shares its \textsc{marking} value with the mother, the latter is 
compatible with specifiers that require a nominal that is introduced by \emph{de}, 
such as \emph{beaucoup} `much'/`many', whose AVM is given in 
(\ref{coup}).\footnote{In this AVM, quoted from \citet[18]{Abeilleetal04}, 
the value of \textsc{spec} is of type \type{synsem}, as in \citet{ps2}, and not of type 
\type{semantic-object}, as in \citet{GS00}.}  

\begin{exe} 
\ex\label{coup} 
\begin{avm} 
[category|head [\type{adverb}                                 \\
                spec|loc [category [head    & \type{noun}     \\
                                    spr     & <X>             \\
                                    marking & \type{de}]      \\
                          content [index @1                    \\
                                   restr \avmbox{$\Sigma$} ]]] \\ 
 content @2 [\type{beaucoup-rel}         \\
               index @1                  \\
               restr \avmbox{$\Sigma$} ] \\
 store \{ @2 \} ]
\end{avm}
\end{exe} 

\noindent
The selected nominal is required to be unsaturated for \textsc{spr} and to have a 
\textsc{marking} value of type \type{de}. The determiner itself is treated as an adverb that  
shares the index and the restrictions of its nominal head sister. Conversely, the latter
also selects its specifier via its \textsc{spr} value, following the mutual 
selection regime of the specifier treatment, see Section~\ref{semco}.

Technically, there is a similarity between the weak head treatment and the 
definition of functional complementation in \citet{Netter94}, see Section~\ref{compl}. 
In both cases the head, whether weak or functional, inherits the properties of its 
complement, except for those that are captured by a special feature, such as  
\textsc{marking} or \textsc{minor}. There are also differences, of course, for 
while \citet{Netter94} applies the functional head analysis to the determiners,  
\citet{Abeilleetal04} apply the weak head analysis to \emph{de}, but not to 
\emph{beaucoup}. That is why the former is an instance of the DP approach and the 
latter of the NP approach.  


\subsubsection{A functor treatment} 


In a functor treatment of \emph{beaucoup de farine} `much flour' both \emph{de} and 
\emph{beaucoup} are functors. The former selects a nominal of type \type{bare} and 
has a \textsc{marking} value of \type{de} which it shares with the mother. 
The latter selects a nominal with the \textsc{marking} value 
\type{de} and has a \textsc{marking} value of type \type{marked} which it 
shares with the NP as a whole, as spelled out in Figure \ref{beau}. 

\begin{figure}
\centering
\oneline{%
\begin{forest}
sm edges
[{[\textsc{head} \avmbox{1} \type{noun} , \textsc{mark} \avmbox{2} \type{marked}]}
	[{[\textsc{head$|$sel} \avmbox{4} , \textsc{mark} \avmbox{2}]} [beaucoup]]
	[{\avmbox{4} [\textsc{head} \avmbox{1}, \textsc{mark} \avmbox{5} \type{de}]}
		[{[\textsc{head$|$sel} \avmbox{3} , \textsc{mark} \avmbox{5}]} [de]]
		[{\avmbox{3} [\textsc{head} \avmbox{1} , \textsc{mark} \type{bare}]} [farine]]]]
\end{forest}}
\caption{\label{beau} A prepositional functor}
\end{figure}

Since the noun is the head daughter of \emph{de farine}, the latter's 
part-of-speech, valence and meaning are shared directly with \emph{farine}, 
rather than via the AVM of \emph{de}. This has the advantage of ironing out
some of the quirks that the weak head treatment implies, such as the assumption 
that \emph{de} is not a preposition when it is a weak head, but rather something 
which belongs to the same part-of-speech as its complement, so that it is 
a noun in \emph{beaucoup de farine} and a verb in (\ref{sorti}), where it takes 
an infinitival VP as its complement.   

\begin{exe} 
\ex\label{sorti}   
\gll   De sortir un peu te ferait du bien. \\
       to go.out a bit you would.do {} good \\
\trans `Going out a bit would do you some good.'
\end{exe}  

\noindent
Similar remarks apply to its valence and its meaning. While it has a non-empty 
\textsc{spr} value and a \textsc{content} value of type \type{scope-object} in 
\emph{beaucoup de farine} it has a non-empty \textsc{subj} value and a 
\textsc{content} value of type \type{state-of-affairs} in (\ref{sorti}). 

In some cases, these quirks lead to analyses that are empirically inadequate. 
An example is discussed in \citet{Maekawa15}, who provides  
an analysis of English nominals of the \emph{kind}/\emph{type}/\emph{sort} variety.  
A typical property of these nominals is that the determiner may show agreement with the 
rightmost noun, as in \emph{these sort of problems} and \emph{those kind of pitch changes},
rather than with the noun that it immediately precedes. 
To model this Maekawa considers the option of treating \emph{of} and the immediately 
preceding noun as weak heads, but dismisses it, since it has the unwanted effect of treating 
\emph{kind}/\emph{type}/\emph{sort} as plural. 
As an alternative, he develops an analysis in which \emph{of} and 
the preceding noun are functors \citep[149]{Maekawa15}. This yields a plural nominal, but 
without the side-effect of treating \emph{kind}/\emph{type}/\emph{sort} as plural. 



\subsection{Other nominals with idiosyncratic properties} 
\label{other}


There are many more types of nominals with idiosyncratic properties,
but the alotted time and space are not sufficient to provide a full survey here. 
Instead, we mention some that have been analyzed in HPSG terms and add pointers 
to the relevant literature.    

Comparable to the nominals with a verbal core, such as gerunds and nominalized 
infinitives, are nominals with an adjectival core, as in \emph{the very poor} and 
\emph{the merely skeptical}. They are described and provided with an HPSG analysis 
in \citet{ArnoldSpencer2015}.

A much studied nominal with idiosyncratic properties is the Binominal Noun Phrase 
Construction (BNPC), exemplified in (\ref{climb}). 

\begin{exe}
\ex\label{climb}
\begin{xlist}
\ex  She blames it on [her nitwit of a husband]. 
\ex  She had [a skullcracker of a headache]. 
\end{xlist}
\end{exe}

\noindent
In contrast to ordinary [NP--\emph{of}--NP] sequences, 
as in \emph{an employee of a Japanese company}, where the 
first nominal is the head of the entire NP, and where the second 
nominal is part of its PP adjunct, the BNPC shows a pattern
in which the relation between the nominals is a predicative one: 
her husband is claimed to be a nitwit, and the headache is claimed to be 
like a skullcracker. HPSG treatments of the BNPC are provided in
\citet{KimSells14} and \citet{VanEynde18}. The latter uses 
the phrase type hierarchy in Figure \ref{prot}, defining the BNPC as 
a maximal type that inherits from \type{head-independent-phrase} and 
\type{inverted-predication}. To capture the intuition that the second
nominal is the head of the entire NP, the preposition \emph{of} is 
treated as a functor that selects a nominal head, as in Maekawa's treatment of 
the preposition in \emph{these sort of problems}, see Section~\ref{prep}. 

Another special case is apposition. It comes in (at least) two types, known as 
close apposition and loose apposition. Relevant examples are given in (\ref{appo}). 

\begin{exe} 
\ex\label{appo} 
\begin{xlist} 
\ex  [My brother Richard] is a soldier.  
\ex  [Sarajevo, the capital of Bosnia,] is where WW I began.
\end{xlist} 
\end{exe}

\noindent
Both types are compared and analyzed in \citet{Kim12} and \citet{Kim14}. 
\citet{VanEyndeKim16} provides an analysis of loose apposition in the 
Sign-Based Construction Grammar framework. 

Idiosyncratic are also the nominals with an extracted determiner, as in 
the French (\ref{combi}) and the Dutch (\ref{gelu}). 

\begin{exe} 
\ex\label{combi}  
\gll   Combien as-tu lu [\_\_ de livres en latin]?  \\
       how.many have-you read {} of books in Latin  \\ 
\trans `How many books have you read in Latin?' 
\ex\label{gelu}
\gll   Wat zijn dat [\_\_ voor vreemde geluiden]? \\
       what are that {} for strange noises        \\ 
\trans `What kind of strange noises are those?'  
\end{exe} 

\noindent
The French example is analyzed in \citet[20--21]{Abeilleetal04} and the Dutch one in 
\citet[47--50]{VanEynde04}. Other kinds of discontinuous NPs are treated in \citet{DeKuthy2002a}. 


\section{Conclusion} 


This chapter has provided a survey of how nominals are analyzed in HPSG. 
Over time three treatments have taken shape, i.e. the specifier treatment, 
the DP treatment and the functor treatment. 
Each was presented and applied to ordinary nominals in Section~\ref{ordi}. 
A comparison showed that the treatments which adopt the NP approach fit in better with 
the surface-oriented monostratal character of HPSG than the DP treatment. 

We then turned to nominals with idiosyncratic properties in Section~\ref{idio}.
Since their analysis often requires a relaxation of the strictly lexicalist stance of early HPSG, 
we first introduced some basic notions of Constructional HPSG and then applied them to 
such idiosyncratic nominals as the gerund, the Big Mess Construction and irregular P+NOM combinations.  
Some of these analyses adopt the specifier treatment, others the functor treatment. 
When both are available, as in the case of the Big Mess Construction and  
irregular P+NOM combinations, the functor treatment seems more plausible. 
Finally, we have added pointers to relevant literature for other nominals with idiosyncratic properties,
such as those with an adjectival core, the Binominal Noun Phrase Construction, apposition 
and discontinuous NPs. 

 
%\section*{Abbreviations}
\section*{Acknowledgements}


For their comments on earlier versions of this chapter I would like to thank Bob Borsley, Stefan
Müller, Liesbeth Augustinus and two anonymous reviewers. My contributions to the HPSG framework span
nearly three decades now, starting with work on a EU-financed project to which
also Valerio Allegranza, Doug Arnold and Louisa Sadler contributed \citep{VanEyndeSchmidt98}. 
Beside the EU-funding, it was conversations with Dani\`ele Godard and Ivan Sag that convinced me 
of the merits and the potential of the HPSG approach.  
I became a regular at the annual HPSG conferences by 1998 and started teaching it in Leuven around 
the same time. A heart-felt thanks goes to the colleagues who over the years have created such a stimulating
environment to work in.        

{\sloppy
\printbibliography[heading=subbibliography,notkeyword=this] 
}
\end{document}


\if0

\subsubsection{The Dutch nominalized infinitive: phrasal conversion } 
\label{nominf}

The closest equivalent of the gerund in Dutch is the nominalized infinitive. 
Some examples are given in (\ref{schenk})--(\ref{mobil}). 

\begin{exe} 
\ex\label{schenk} 
\gll   [geld wegschenken] maakt vrouwen gelukkig  \\
       [money donate.\textsc{inf}] makes women happy \\
\trans `Donating money makes women happy.' 
\ex\label{mobil}   
\gll   voor [het op diepte houden van de Vlaamse kusthavens] dient gebaggerd te worden  \\  
       for [the on depth keep.\textsc{inf} of the Flemish coast.ports] needs dredged to be \\
\trans `Dredging is necessary to keep the Flemish coastal harbors accessible.'  
\end{exe} 

\noindent
Also here the bracketed phrases have the external distribution of an NP, 
taking the subject position in (\ref{schenk}) and 
the complement position of a transitive verb in (\ref{mobil}). 
And also here, the internal structure shows a mix of nominal and verbal
characteristics. 
Typically nominal are the presence of the article and the postnominal PP
complement in (\ref{mobil}). 
Typically verbal are the presence of the direct object complement in (\ref{schenk})
and the predicative PP complement in (\ref{mobil}). 
To model this \citet{VanEynde19} makes a distinction between the verbal core 
and the nominal fringe of a nominalized infinitive, as in the structure of 
(\ref{mobil}), spelled out in Figure \ref{kust}.  

\begin{figure}
	\centering
	\begin{forest}
sm edges
[{[\textsc{head} \avmbox{1} \type{noun}, \textsc{comps} \eliste]}
	[N [het]]
	[{[\textsc{head} \avmbox{1}, \textsc{comps} \eliste]}
		[{[\textsc{head} \avmbox{1}, \textsc{comps} \liste{\avmbox{2} PP$_{j}$}]}
			[{[\textsc{head} \avmbox{4} \type{verb}, \textsc{comps} \liste{NP$_{j}$}]}
			[\avmbox{3} [op diepte, roof]]
			[{[\textsc{head} \avmbox{4}, \textsc{comps} \liste{NP$_{j}$, \avmbox{3} PP$_{k}$}]} [houden]]]]
		[\avmbox{2}, before drawing tree={x+=3em} [van de kusthavens, roof, before drawing tree={x+=3em}]]]]
	\end{forest}
	\caption{\label{kust} Dutch nominalized infinitive}
\end{figure}   

The infinitive is treated as unambiguously verbal at the lexical level and 
remains verbal when combined with its predicative PP complement, but then it  
is converted into a nominal projection and combined with a postnominal PP[\type{van}]
complement and the definite article.\footnote{Since the article is a pronominal 
determiner, it is assigned the category N in the functor treatment.}     
The conversion is modelled in terms of a non-headed phrase type, 
since it does not comply with the Head Feature Principle: 
the \textsc{head} value of the mother is not shared with the daughter.  
More specifically, there is a subtype of the non-headed phrases, called \type{convert-phr}, 
whose defining characteristic is that they have a single daughter. 
In that respect they differ from coordinate phrases, which have at least two daughters.  

\begin{exe} 
\ex\label{conv} 
\begin{avm} 
[\type{convert-phr} \\
 dtrs <X> ]
\end{avm}
\end{exe}

\noindent
The conversion which we observe in the Dutch nominalized infinitive is modeled 
in terms of a phrase type that inherits properties of \type{convert-phr} and 
\type{non-clause}. Its properties are spelled out in (\ref{nominf}). 

\begin{figure}
\centering
\begin{forest}
%sm edges
[\type{phrase}
	[\textsc{headedness}
		[\type{non-headed-phr}
			[\type{coord-phr}]
			[\type{convert-phr} [\type{nom-inf-phr}, name=A2]]
			[...]]]
	[\textsc{clausality}
		[\type{clause}]
		[\type{non-clause}, name=A1]]]
\draw (A1.south) -- (A2.north);
\end{forest}
\caption{\label{bido6} Nominalized infinitives in the bidimensional hierarchy  } 
\end{figure}

\begin{exe} 
\ex\label{nominf} 
\begin{avm} 
[\type{nom-inf-phr}                                      \\
 synsem|cat [head [\type{noun}                           \\
                   number \type{singular}              \\
                   gender \type{neuter}]               \\
             subj < >                                 \\
             comps <(pp$_{j}$)> $\oplus$ <(pp$_{i}$)> $\oplus$ @A \\ 
             marking \type{unmarked}]                     \\           
 dtrs <[synsem|cat [head|vform \type{inf}               \\
                      subj <np$_{j}$>                         \\
                      comps <(np$_{i}$)> $\oplus$ @A ]]>]
\end{avm}
\end{exe}

\noindent
The daughter is an infinitive with a non-empty \textsc{subj} list and 
a possibly empty \textsc{comps} list.
The mother is an unmarked singular neuter nominal that inherits the 
unsaturated complement requirements of its daughter, 
albeit with the twist that NP complements become PP complements. 
The subject requirement of the infinitive is added to the \textsc{comps} list 
of the nominal and becomes a PP too. It is made optional, since 
it is often left unexpressed, as in (\ref{schenk})--(\ref{mobil}). If present,  
it is introduced by \emph{van} `of' or \emph{door} `by', as in 
(\ref{van}) and (\ref{door}) respectively. 

\begin{exe} 
\ex\label{van} 
\gll   het trage afsterven van de koraalriffen   \\
       the slow die.\textsc{inf} of the coral.reefs \\
\trans `the slow dying of the coral reefs' 
\ex\label{door}
\gll   het uitschakelen van Chelsea door Real Madrid      \\
       the eliminate.\textsc{inf} of Chelsea by Real Madrid  \\ 
\trans `the elimination of Chelsea by Real Madrid'  
\end{exe} 

For a full treatment that also covers the semantic type shift that 
accompanies the syntactic conversion, we refer to \citet{VanEynde19}.



There is a difference, though, in the treatment of nominals 
whose first argument is realized as a possessive, as in 
\emph{his painting of this house}. In this case, the first argument is not put on the 
\textsc{spr} list of the noun. Instead, the possessive selects an unmarked nominal with an optional 
PP[\type{by}] complement on its \textsc{comps} list, and shares the index of that PP, 
as spelled out in Figure \ref{possy}. 

\begin{figure}
\centering
\oneline{%
\begin{forest}
sm edges
[{[\textsc{head} \avmbox{1} [\type{noun} \textsc{sel} \type{none}], \textsc{mark} \avmbox{2} \type{marked}]}
	[{[\textsc{head} [\type{pron} \textsc{sel} \avmbox{4}], \textsc{mark} \avmbox{2}]$_{i}$} [his]]
	[{\avmbox{4} [\textsc{head} \avmbox{1}, \textsc{comps} \liste{(PP[\type{by}]$_{i}$)}, \textsc{mark} \avmbox{5} \type{unmarked}]}
		[{[\textsc{head} \avmbox{1}, \textsc{comps} \liste{ (PP[\type{by}]$_{i}$), \avmbox{3}}, \textsc{mark} \avmbox{5}]} [painting]]
		[{\avmbox{3} [\textsc{head} \type{prep}, \textsc{comps} \eliste]} [of this house, roof]]]]
\end{forest}}
\caption{\label{possy} Deverbal nominals }
\end{figure}

\fi
