\documentclass[output=paper]{langsci/langscibook} 
\author{%
	Stephen Wechsler\affiliation{The University of Texas}%
	\and Jean-Pierre Koenig\affiliation{University at Buffalo}%
	\lastand Anthony Davis\affiliation{Southern Oregon University}%
}
\title{Argument structure and linking}

% \chapterDOI{} %will be filled in at production

\epigram{}%We are up against one of the great sources of philosophical bewilderment: we must try to find a substance for a substantive.}
\abstract{What we cannot speak about, we must pass over in silence.}
\maketitle


\begin{document}
\label{chap-arg-st}

\section{Introduction}


When a verb or other predicator is composed with the phrases or pronominal affixes expressing its semantic arguments, the grammar must specify the mapping between the semantic participant roles and syntactic dependents of that verb.  For example, the grammar of English indicates that the subject of \word{eat} fills the eater role and the object of \word{eat}  fills the role of the thing eaten.  In HPSG this mapping is usually broken down into two simpler mappings by positing an intermediate representation called \argst (`argument structure').  The first mapping connects the participant roles within the semantic \content with the elements of the \argst feature; here we will call the theory of this mapping \emph{linking theory} (see Section \ref{linking-sec}).  The second mapping connects those \argst list elements to the elements of the \val lists, namely \comps (`complements') and \subj (`subject'; or \spr, `specifier'); we will refer to this second mapping as \emph{argument realization} (see Section \ref{sec:arg-st}).\footnote{Some linguists, such as \citet{LevinandRappaport2005}, use the term `argument realization' more broadly, to encompass linking as well.}  These two mappings are illustrated with the simplified lexical sign for the verb \word{eat} in (\ref{eat}). 


\begin{exe} 
	\label{eat}
\ex	Lexical sign for the verb \word{eat}\\
{\avmoptions{center}
\begin{avm}
\[phon & \< $\textrm{eat}$ \> \\
%TD 17 July: maybe /i:t/, or orth rather than phon
valence & \[ subj \ \ & \<  {\@1} \> \\ 
comps \ \ & \< {\@2}  \> \] \\ 
arg-st & \< {\@1}\textsc{np}$_i$ , {\@2}np$_j$ \> \\	
content  \ \ & \textbf{eat}($i, j$)\] 
\end{avm}}
\end{exe}

	
\noindent
In (\ref{eat}), `NP' abbreviates a feature structure representing syntactic and semantic information about a nominal phrase.  The variables $i$ and 
$j$ are the referential indices for the eater and eaten arguments, respectively, of the \textbf{eat} relation.  The semantic information in 
NP$_i$ semantically restricts the value or referent of $i$. 

The \argst feature plays an important  
role in HPSG grammatical theory.  In addition to regulating the mapping from semantic arguments to grammatical relations, \argst is the locus of the theories of anaphoric binding and other construal relations such as control and raising.  (This chapter focuses on the function of \argst  in semantic mapping, with some discussion of binding and other construal relations only insofar as they interact with that mapping.  A more detailed look at binding is presented in Chapter \ref{BINDINGCHAPTER}.)   

In HPSG, verb diathesis alternations, voice alternations, and derivational processes such as category conversions are all captured within the lexicon (see Section \ref{alternations} and Chapter \ref{LEXICONCHAPTER}).  The different variants of a word are grammatically related either through lexical rules or by means of the lexical type hierarchy.  HPSG grammars explicitly capture paradigmatic relations between word variants, making HPSG a \textit{lexical approach to argument structure}, in the sense of \citet{MWArgSt}.
This fundamental property of lexicalist theories contrasts with many transformational approaches, where such relationships are treated as syntagmatically related through operations on phrasal structures representing sentences and other syntactic constituents.  Arguments for the lexical approach are reviewed in Section \ref{lexicalapproach}.  

Within the HPSG framework presented here, we will formulate and address a number of empirical questions: 

\begin{itemize}
\item We know that a verb's meaning influences its valence requirements, (via the \argst list, on this theory). 
 What are the principles governing the mapping from \content to \argst?  Are some aspects of \argst idiosyncratically stipulated for individual verbs?  What aspects of the semantic \content  bear on the value of \argst, and what aspects do not?  (For example, what is the role of modality?)  
\item How are argument alternations defined with respect to our formal system?  For each alternation we may ask which of the following it involves: a shuffling of the \argst list;  a change in the mapping from \argst to \val; or  a change in the \content, with a concomitant change in the \argst?  
\end{itemize}

\noindent
These questions will be addressed below in the course of presenting the theory.  We begin by considering \argst itself (Section \ref{sec:arg-st}), followed by the mapping from \argst to \val (Section \ref{valence-sec}) and the mapping from \content to \argst (Section \ref{linking-sec}).  
The remaining sections address further issues relating to argument structure: the nature of argument alternations, extending the \argst attribute to include additional elements, whether \argst is a universal feature of languages, and a comparison of the lexicalist view of argument structure presented here with phrasal approaches.


\section{The representation of argument structure in HPSG}

\label{sec:arg-st}

In the earliest versions of HPSG, the selection of dependent phrases was specified in the \subcat feature of the head word (\citet{ps}, \citet[ch. 1--8]{ps2}).  The value of \subcat is a list of items, each of which corresponds to the \synsem value of a complement or subject.  Following are \subcat features for an intransitive verb, a transitive verb, and a transitive verb with obligatory PP complement:


\begin{exe} 
\ex \label{subcats}
\begin{xlist}
\ex	\word{laugh}:  $[$ \subcat $\langle$ \textsc{np} $\rangle ]$
\ex    \word{eat}:  $[$ \subcat $\langle$ \textsc{np}, \textsc{np} $\rangle ]$
\ex    \word{put}:  $[$ \subcat $\langle$ \textsc{np}, \textsc{np}, \textsc{pp} $\rangle ]$
\end{xlist}
\end{exe}

\noindent
Phrase structure rules in the form of immediate dominance schemata identify a certain daughter node
as the head daughter (\textsc{head-dtr}) and others as complement daughters (\textsc{comp-dtrs}).
In keeping with the \emph{Subcategorization Principle}, here paraphrased from \citew[\page 34]{ps2}, list items are effectively ‘cancelled’ from the \subcat list as complement phrases are joined with the selecting head:

\begin{exe}
\ex Subcategorization Principle: In a headed phrase, the \subcat value of the \headdtr (`head daughter') is the concatenation of the phrase's \subcat list with the list of \synsem values of the \compsdtrs (`complement daughters').
\end{exe}

\noindent
Phrasal positions are distinguished by their saturation level: `VP' is defined as a verbal projection whose \subcat list contains a single item, corresponding to the subject; and `S' is defined as a verbal projection whose  \subcat list is empty. 

The `subject' of a verb, a distinguished dependent with respect to construal processes such as binding, control, and raising, was then defined as the first item in the  \subcat list, hence the last item with which the verb combines.   However, defining `subject' as the last item to combine with the head proved inadequate \citep[Ch. 9]{ps2}.  There are many cases where the dependent displaying subject properties need not be the last item added to the head projection.  For example, in German the construal subject is a nominal in nominative case \citep{Reis82}, but the language allows subjectless clauses containing only a dative or genitive non-subject NP.  If that oblique NP is the only NP dependent to combine with the verb then it is \emph{ipso facto} the last NP to combine.  

Consequently, the \subcat list was split into two valence lists, a \subj list of length zero or one for subjects, and a \comps list for complements.  Nonetheless, certain grammatical phenomena, such as binding and other construal processes, must still be defined on a single list comprising both subject and complements (Manning \& Sag 1999). Additionally, some syntactic arguments are unexpressed or realized by affixal pronouns, rather than as subject or complement phrases.  
The new list containing all the syntactic arguments of a predicator was named \textsc{arg-st} (`argument structure').  

In clauses without implicit or affixal arguments, the \textsc{arg-st} is the concatenation of  \textsc{subj} and \textsc{comps} respectively.  For example, the \subcat list for \word{put} in (\ref{subcats}c) is replaced with the following:

\begin{exe} 
	\label{put}
\ex	%\textit{put}: \\
\begin{avm}
[ phon & < \textrm{put} > \\
valence & [ subj \ \ & <  \ {@1} \ > \\ 
comps \ \ & < \  {@2}, {@3} \  > ] \\ 
arg-st & < \ {@1}np , {@2}np, {@3}pp \  > ] 
\end{avm}
\end{exe}

\noindent
The idealization according to which \argst is the concatenation of \subj and \comps is canonized as the \emph{Argument Realization Principle} (ARP) \citep[PAGE]{SWB2003a}.  Exceptions to the ARP, that is, dissociations between \textsc{valence} and \argst, are discussed in Section \ref{argst-sec} below.  

A predicator's \textsc{valence} lists indicate its requirements for syntactic concatenation with phrasal dependents (Section \ref{valence-sec}). 
\argst, meanwhile, provides syntactic information about the expression of semantic roles and is related, via linking theory, to the lexical semantics of the word (Section \ref{argst-sec}).  
The \argst list contains specifications for the union of the verb's syntactic local phrasal dependents (the subject and complements, whether they are semantic arguments, raised phrases, or expletives) and its syntactic arguments that are not realized locally, whether they are unbounded dependents, affixal, or unexpressed. 

Figure \ref{fig:over} provides a schematic representation of %``typical'' 
linking and argument realization in HPSG.   Linking principles govern the mapping of of participant roles in a predicator's \content to %direct 
syntactic arguments on \argst.   Argument realization is shown in this figure only for mapping to \val; affixal and null arguments are not depicted.
Here, the semantic roles are just arbitrary labels, but we discuss in Section (\ref{linking-sec}) how they can be systematically related to lexical entailments of predicators.
The \argst and \val lists in this figure contain only arguments linked to participant roles, but in Section \ref{extended-arg-st} we note motivations for extending \argst to include additional elements.
And in Section (\ref{valence-sec}), we examine cases where the relationship between \argst and \val violates the ARP.

\begin{figure}[htbp!]
\begin{tabular}{p{3cm}p{4.5cm}p{4cm}}
	Semantics (\attrib{content}) & {\avmoptions{center}\begin{avm}
 \[\asort{$\textit{sem-rel-1}$}
 sem-role-1	& \@1 \\
 sem-role-2 & \@2 \\
 soa & \[\asort{$\textit{sem-rel-2}$}
 sem-role-3 & \@3 \\
 sem-role-4 & \@4 \] \]
 \end{avm}} & Structured representation of semantic relations, their roles and the participants filling them\\
  & \Large{\begin{tikzcd}
 {} \arrow[Leftrightarrow]{d}{\,\,\,\text{\textit{Linking principles}}}
               \\ 
               {}
\end{tikzcd} } &  \\
& \\
	Argument structure (\attrib{arg-st}) & {\avmoptions{center}\begin{avm}\[arg-st &
 \<\textsc{xp}$_{1}$, xp$_{2}$, \ldots \textsc{xp}$_{n}$ \> \]	
 \end{avm}} & List of syntactic arguments
\\
 & \Large{\begin{tikzcd}
 {} \arrow[Leftrightarrow]{d}{\,\,\,\text{\textit{Argument Realization Principle}}}
               \\ 
               {}
\end{tikzcd} } & \\
	Syntax (\attrib{valence}) & {\avmoptions{center}\begin{avm}
 \[spr & \<\,\> \\
 subj & \<xp$_{1}$ \> \\
 comps & \<\textsc{xp}$_{2}$, \ldots \textsc{xp}$_{n}$ \> \] 	
 \end{avm}}
& Lists of locally realized dependents
\end{tabular}
\caption{\label{fig:over}How linking works in HPSG}	
\end{figure}



\section{Argument realization: The mapping from \argst to \val lists}
\label{valence-sec}


\subsection{Variation in the expression of arguments}
\label{express-sec}

The \val feature is responsible for composing a verb with its phrasal dependents, but this is just one of the ways that semantic arguments of a verb are expressed in natural language.  Semantic arguments can be expressed in various linguistic forms: as local syntactic dependents (\subj and \comps), as affixes, or displaced in unbounded dependency constructions (\textsc{slash}). 

Affixal arguments can be illustrated with the first person singular Spanish verb \textit{hablo} `speak.\textsc{1sg}', as in \ref{hablo}.


%
\begin{exe} 
\ex	\label{hablo}
\begin{xlist}
\ex 		\gll Habl-o espa\~{n}ol.  \\
		speak-\textsc{1sg} Spanish  \\
		\glt `I speak Spanish.'
\ex \textit{hablo}`speak.\textsc{1sg}': \\
{\avmoptions{center}
\begin{avm}
\[ phon & \< $\text{hablo}$ \> \\
valence & \[ subj & \<  \ \  \> \\ 
comps & \< \@2   \> \] \\ 
arg-st & \< np:\[$ppro$ \\ index \[pers & $1st$ \\ num & $sg$ \] \]  , \@2np  \>  \]
\end{avm}}
\end{xlist}
\end{exe}

\noindent
The \textit{-o} suffix contributes the first person singular pronominal subject content to the verb form (the morphological process is not shown here; see Chapter MORPHOLOGY-CHAPTER).  The pronominal subject  appears on the \argst list and hence is subject to the binding theory.  But it does not appear in \subj , if no subject NP appears in construction with the verb.   

A lexical sign whose \argst list that is just the concatenation of its \subj and \comps lists conforms to the Argument Realization Principle (ARP); such signs are called \word{canonical signs} by \citet{Boumaetal2001}.  Non-canonical signs, which violate the ARP, have been approached in two ways.  In one approach, a lexical rule takes as input a canonical entry and derives a non-canonical one by removing items from the \val lists, while adding an affix or designating an item as an unbounded dependent by placement on the \textsc{slash} list.  
In the other approach, a feature of each \argst list item specifies whether the item is subject to the ARP (hence mapped to a \val list), or ignored by it (hence expressed in some other way).  
See the chapter on the lexicon for more detail and \citet{MillerandSag1997} for a treatment of French clitics as affixes. 

A final case to consider is null anaphora, in which a semantic argument is simply left unexpressed and receives a definite pronoun-like interpretation.  Japanese \textit{mi-} `see' is transitive but the object NP can be omitted as in (\ref{jap}).

\begin{exe}
	\ex\label{jap}
		\gll Naoki-ga mi-ta.  \\
		Naoki-\ig{nom} see-\ig{past}  \\
		\glt `Naoki saw it/him/her/*himself.'
\end{exe} 

\noindent
Null anaphors of this kind typically arise in discourse contexts similar to those that license ordinary weak pronouns, and the unexpressed object often has the (Principle B) obviation effects characteristic of overt pronouns, as shown in (\ref{jap}).  But HPSG eschews the use of silent formatives like `small \textit{pro}' when there is no evidence for such items, such as local interactions with the phrase structure.  Instead, null anaphors of this kind are present in \argst but absent from \val lists.  \argst is directly linked to the semantic \content and is the locus of binding theory, so the presence of a syntactic argument on the \argst list but not a \val list
 accounts for null anaphora.  To account for obviation, the \argst list item, when unexpressed, receives the binding feature of ordinary (non-reflexive) pronouns, usually \textit{ppro}.  This language-specific option can be captured in a general way by \val and \argst defaults in the lexical hierarchy for verbs.   

\subsection{The syntax of \argst and its relation to \textsc{valence}}
\label{argst-sec}


The \argst ordering represents a preliminary syntactic structuring of the set of argument roles.  In that sense it functions as an interface between the lexical semantics of the verb, and the expressions of dependents as described in Section \ref{valence-sec}.  Its role thus bears some relation to the initial stratum in Relational Grammar, \textit{argument structure} (including intrinsic classifications) in LFG Lexical Mapping Theory, D-structure in Government/Binding theory, and the Merge positions of arguments in Minimalism, assuming in the last two cases the Uniform Thematic Alignment Hypothesis \citep{Baker1988} or something similar.  However, it also differs from all of those in important ways.  

Semantic constraints on \argst are explored in Section \ref{linking-sec} below.  But \argst is not only structured by semantic distinctions between the arguments but also  by syntactic  ones.  Specifically, the list ordering represents relative syntactic \textit{obliqueness} of arguments.   The least oblique argument is the subject (\subj), followed by the complements (\comps).  Following \citet{Manning1996} term arguments (direct arguments, i.e. objects) are  assumed to be less oblique than `oblique' arguments (adpositional and oblique case marked phrases), followed finally by predicate and clausal complements.  The transitive ordering relation on the \argst list is called \textit{o-command} (`obliqueness command'): the  subject list item o-commands those of the complements; an object list item o-commands those of any obliques; and so on.
  
Voice alternations like the passive, which are defined on the \argst list, illustrate the ordering of terms before obliques on the \argst list.   Passivization alters the syntactic properties of \argst list items: the initial item of the active, normally mapped to \subj of the active, is an oblique (\textit{by} phrase) or unexpressed argument in the passive.  Given that terms precede obliques in the list order, any term arguments must o-command the passive oblique, so passive effectively reorders the initial item in \argst to a list position following any terms.   

\begin{exe}
\ex \label{passive}
\begin{xlist}
\ex Susan gave Mary a book.
\ex Mary was given a book by Susan.
 \end{xlist}
 \end{exe}

\begin{exe}
\ex \label{pasargst}
\begin{xlist}
\ex \textit{give} (active): $[ \textsc{arg-st}  < \textsc{np}_i, \textsc{np}_j, \textsc{np}_k > ]$
\ex \textit{given} (passive): $[ \textsc{arg-st}  < \textsc{np}_j, \textsc{np}_k, \textsc{pp}[by]_i > ]$
 \end{xlist}
 \end{exe}


Relative obliqueness conditions a number of syntactic processes and phenomena, including anaphoric binding.  The o-command relation replaces the c-command in the Principles A, B, and C of Chomsky's \citeyearpar{Chomsky81a} configurational theory of binding.  For example, HPSG's Principle B states that an ordinary pronoun cannot be o-commanded by its coargument antecedent, which accounts for the pronoun obviation observed in the English sentence \textit{Naoki$_i$ saw him$_{*i/j}$}, and also accounts for obviation in the Japanese sentence (\ref{jap}) above.  

Relative obliqueness also conditions the accessibility hierarchy of \citet{KeenanandComrie1977}, according to which a language allowing relativization of some type of dependent also allows relativization of any less oblique than it.  Hence if a language has relative clauses at all, it has subject relatives; if it allows obliques to relativize then it also allows subject and object relatives; and so on.  
Similar implicational universals apply to verb agreement with subjects, objects, and obliques \citet{greenberg:1966}.  

Returning now to argument realization, we saw above that the rules for the selection of the subject from among the verb's arguments are also stated on the \argst list.  In a `canonical' realization the subject is the first list item, o-commanding all of its coarguments.  

\subsection{Syntactic ergativity}
The autonomy of \argst from the \val lists is further illustrated by cross-linguistic variation in the mapping between them.  As just noted, in English and many other languages the initial item in \argst maps to the subject.  However,  languages with so-called \textit{syntactically ergative} clauses have been analyzed as following a different mapping rule.  Crucially, the \argst ordering in those languages is still supported by independent evidence from properties such as binding and NP versus PP categorial status of arguments.
Balinese (Austronesian), as analyzed by \citet{Wechsler+Arka:1998}, is such a language.  In the morphologically unmarked, and most common voice, called `Objective voice' (OV), the subject is any term \textit{except} the \argst-initial one.   

Balinese canonically has SVO order, regardless of the verb's voice form \citep{Artawa1994, Wechsler+Arka:1998}.  The preverbal NPs in  (\ref{bal1}) are the surface subjects and the postverbal ones are complements.  When the verb appears in the unmarked OV verb, a non-initial term is the subject, as in (\ref{bal1-a}).    But verbs in `Agentive Voice' (AV) select as their subject the \argst{}-initial item, as in (\ref{bal1-b}).  

\begin{exe}
	\ex\label{bal1}
\begin{xlist}
\ex \label{bal1-a}	\gll Bawi adol ida.  \\
		pig OV.sell 3sg   \\
		\glt `He/She sold a pig.'
\ex	\label{bal1-b}\gll Ida ng-adol bawi.  \\
		3sg AV-sell pig   \\
		\glt `He/She sold a pig.'
\end{xlist}
\end{exe} 

\noindent
A ditransitive verb such as the benefactive applied form of \textit{beli} `buy' in (\ref{bal2}), has three term arguments on its \argst list.  The subject can be either term that is non-initial in \argst{}:

\begin{exe}
	\ex\label{bal2}
\begin{xlist}
\ex 	\gll Potlote ento beli-ang=a I Wayan.  \\
		pencil-DEF that OV.buy-APPL=3 Art Wayan   \\
		\glt `(s)he bought Wayan the pencil.'
\ex 	\gll I Wayan beli-ang=a potlote ento.   \\
		Art Wayan OV.buy-APPL=3 pencil-DEF that   \\
		\glt `(s)he bought Wayan the pencil.'
\end{xlist}
\end{exe} 

\noindent
Unlike the passive voice, which reorders the \argst list, the Balinese OV does not affect \argst list order. 
Thus the agent argument can bind a coargument reflexive pronoun (but not vice versa), regardless of whether the verb is in OV or AV form:

\begin{exe}
	\ex\label{bal3}
\begin{xlist}
\ex 	\gll Ida ny-ingakin ragan idane. \\
		3sg AV-see self\\
		\glt ‘(s)he saw himself/herself’
\ex 	\gll {Ragan idane} cingakin ida. \\
		self OV.see 3sg \\
		\glt ‘(s)he saw himself/herself’
\end{xlist}
\end{exe} 

\noindent
The `seer' argument o-commands the `seen', with the AV versus OV voice forms regulating subject selection:

\begin{exe} 
	\label{avsee}
\ex	Agentive Voice form of `see': \\
{\avmoptions{center}
\begin{avm}
\[ phon & \< $nyinkagin$ \> \\
valence & \[ subj & \<  \@1  \> \\ 
comps & \< \@2   \> \] \\ 
arg-st & \< \@1np$_{i}$  , \@2np$_j$ \> \\
content  & \[\asort{see-rel}   seer & i \\ seen & j \]
  \] 
\end{avm}}
\end{exe}

\begin{exe} 
	\label{ovsee}
\ex	Objective Voice form of `see': \\
{\avmoptions{center}
\begin{avm}
\[ phon & \< $cinkagin$ \> \\
valence & \[ subj  & \<  \@2 \> \\ 
comps & \< \@1 \> \] \\ 
arg-st & \< \@1np$_{i}$  , \@2np$_j$ \> \\
content  & \[ \asort{see-rel}   seer & i \\ seen & j \]
  \] 
\end{avm}}
\end{exe}

\noindent
Languages like Balinese illustrate the autonomy of \argst .  
Although the agent binds the patient in both (\ref{bal3})a and b,  the binding conditions cannot be stated directly on the thematic hierarchy.  For example, in HPSG a raised argument appears on the \argst list of the raising verb even though that verb assigns no thematic role to that list item.  But a raised subject can bind a coargument reflexive in Balinese (this is comparable to English \textit{John seems to himself to be ugly}).  Anaphoric binding in Balinese raising constructions thus behaves as predicted by the \argst based theory \citep{Wechsler1999}.  
In conclusion, neither \val nor \content provides the right representation for defining binding conditions, but \argst fills the bill.  

Syntactically ergative languages that have been analyzed as using an alternative mapping between \argst and \val include Tagalog, Inuit, some Mayan languages, Chukchi, Toba Batak, Tsimshian languages, and Nad{\"e}b \citep{Manning1996,Manning+Sag:1999}.  

Interestingly, while the GB/Minimalist configurational binding theory may be defined on analogues of \val or \content,  those theories lack any analogue of \argst.  This leads to special problems for such theories in accounting for binding in many Austronesian languages like Balinese.  In transformational theories since \citet{Chomsky81a}, anaphoric binding conditions are usually stated with respect to the A-positions (`argument positions').  A-positions are analogous to HPSG \val list items, with relative c-command in the configurational structure corresponding to relative list ordering in HPSG, in the simplest cases.  Meanwhile, to account for data similar to (\ref{bal3}), where agents asymmetrically bind patients, Austronesian languages like Balinese were said to define binding on the `thematic structure' encoded in d-structure or Merge positions, where agents asymmetrically c-command patients regardless of their surface positions \citep{Guilfoyle+etal:1992}.  But the interaction with raising shows that neither of those levels is appropriate as the locus of binding theory \citep{Wechsler1999}.\footnote{To account for (\ref{bal3}b) under the configurational binding theory, the subject position must be an A-bar position; but to account for binding by a raised subject, it must be an A-position.  See \citet{Wechsler1999}. } 

\subsection{Symmetrical objects}
We have thus far tacitly assumed a total ordering of elements on the \argst list, but \citet{AMM2013a}, \citet{Ackermanetal2017} propose a partial ordering for certain so-called `symmetrical object' languages.  In Moro (Kordofanian), the two term complements of a ditransitive verb have exactly the same object properties.  Relative linear order of the theme and goal arguments is free, as shown by the two translations of (\ref{moro}) (from \citealt[9]{Ackermanetal2017}):


\begin{exe}
	\ex\label{moro}
\gll   \'{e}-g-a-nat\textipa{S}-\'{o} \'{o}r\'{a}\textipa{N}  \textipa{N}e\textipa{R}\'{a}  \\
        1\textsc{sb.sm-cl}g-\textsc{main}-give-\textsc{pfv}    \textsc{cl}g.man \textsc{cl}\textipa{N}.girl \\
\glt `I gave the girl to the man.’ / `I gave the man to the girl.’
\end{exe} 

\noindent
More generally, the two objects have identical object properties with respect to occurrence in post-predicate position, case marking, realization by an object marker, and ability to undergo passivization \citep[9]{Ackermanetal2017}.

\citet{Ackermanetal2017} propose that the two objects are unordered on the \argst list.  This allows for two different mappings to the \comps list, as shown here:

\begin{exe} 
\ex		\label{moro-avm1}
\begin{xlist}
\ex Goal argument as primary object: \\
{\avmoptions{center}
\begin{avm}
\[ valence & \[ subj & \<  \@1 \> \\ 
comps & \< \@2 , \@3  \> \] \\ 
arg-st & \<  \@1np$_{i}$  ,\{ \@2np$_{j}$ , \@3np$_{k}$ \}   \> \\
content  & \[ \tp{give-rel}  \\ agent & i \\ goal & j \\ theme & k \]
  \] 
\end{avm}}
\ex Theme argument as primary object: \\
{\avmoptions{center}
\begin{avm}
\[ valence & \[ subj & \<  \@1 \> \\ 
comps & \< \@3 , \@2 \> \] \\ 
arg-st & \< \@1np$_{i}$  , \{ \@2np$_{j}$ , \@3np$_{k}$  \} \> \\
content  & \[ \tp{give-rel}  \\ agent & i \\ goal & j \\ theme & k \]
  \] 
\end{avm}}
\end{xlist}
\end{exe}

\noindent
The primary object properties, which are associated with the initial term argument of \comps, can go with either the goal or theme argument. 

To summarize this section, while the relationship between \argst, \subj, and \comps lists was originally conceived as a straightforward one, enabling binding principles to maintain their simple form by defining \argst as the concatenation of the other two, the relationship was soon loosened.
Non-canonical relationships between \argst and the \val lists are invoked in accounts of several core syntactic phenomena.
Arguments not realized overtly in their canonical positions, due to extraction, cliticization, or pro-drop (null anaphora), appear on \argst but not in any \val list.
Accounts of syntactic ergativity in HPSG involve variations in the mapping between \argst and \val lists; in particular, the element of \subj is not, in such languages, the first element of \argst.
Modifications of \argst play a role in treatments of passivization, where its expected first element is suppressed, and in languages with multiple, symmetric objects, where a partial rather than total ordering of \argst elements has been postulated.
Thus \argst has now acquired an autonomous % independent 
status within HPSG, and is not merely a predictable rearrangement of information present elsewhere in lexical entries.


\section{Linking: the mapping between semantics and \argst}
\label{linking-sec}


\subsection{HPSG approaches to linking}

The term \textit{linking} refers to the mapping specified in a lexical entry between participant roles in the semantics and their syntactic representations on the \argst list.  
Early HPSG grammars stipulated the linking of each verb:  semantic \content values with predicator-specific attributes like \textsc{devourer} and \textsc{devoured} were mapped to the subject and object, respectively, of the verb \textit{devour}.  But linking observes prevailing patterns, e.g. if one argument of a transitive verb in active voice has an agentive role, it will map to the subject, not the object.
Thus these early accounts were unsatisfying, as they lead to purely stipulative accounts of linking, specified verb by verb.
Beginning with \citet{Wechsler1995b} and \citet{Davis1996}, researchers formulated linking principles stated on more general semantic properties holding across verbs.  

Within the history of linguistics there have been three general approaches to modeling the lexico-semantic side of linking: thematic role types 
(P\={a}\d{n}ini ca 400 B.C., \citealt{Fillmore1968}); lexical decomposition \citep{FoleyandvanValin1984,RappaportandLevin1998}; and the proto-roles approach \citep{Dowty91a}.   In developing linking theories within the HPSG framework \citet{Wechsler1995b} and \citet{Davis1996} employed a kind of lexical decomposition that also incorporated some elements of the proto-roles approach.  The reasons for preferring this over the alternatives are discussed in Section \ref{thetaroles} below.  

Wechsler's \citeyear{Wechsler1995b} linking theory constrains the relative order of pairs arguments on the \argst list according to semantic relations entailed between them.  For example, his \emph{notion rule} states that if one participant in an event is entailed to have a mental notion of another, then the first must precede the second on the \argst list.  The \textit{conceive-pred} type is defined by the following type declaration (based on \cite[127]{Wechsler1995b}, with formal details adjusted for consistency with current usage):

\begin{exe}
	\ex\label{conceive}
	\textit{conceive-pred:}  
	{\avmoptions{center}
	\begin{avm} 
		\[arg-st  &  \<  \textsc{np}$_i$, \textsc{np}$_j$ \> \\
		content  & \[\asort{conceive-rel}  
		conceiver & $i$ \\
		conceived & $j$ \] 
		\]
	\end{avm}
	}
\end{exe}

This accounts for a host of linking facts in verbs as varied as \word{like}, \word{enjoy}, \word{invent}, \word{claim}, and \word{murder}, assuming these verbs belong to the type \textit{conceive-pred}.  
It explains the well known contrast between experiencer-subject \word{fear} and experiencer-object \word{frighten} verbs:  \word{fear} entails that its subject has some notion of its object, so \word{The tourists feared the lumberjacks} entails that the tourists are aware of the lumberjacks.  But the object of \word{frighten} need not have a notion of its subject: in \word{The lumberjacks frightened the tourists (by cutting down a large tree that crashed right in front of them}, the tourists may not be aware of the lumberjacks' existence.  

Two other linking rules appeared in Wechsler (1995).  One stated that `affected themes,' that is, participants that are entailed to undergo a change, map to the object, rather than subject, of a transitive verb.  Another pertained to stative transitive verbs entailing a part-whole relation between the two participants, such as \textit{include} and \textit{contain}: the whole maps to the subject and the part to the object.   

The linking constraints do not rely on a total ordering of thematic roles, nor on an exhaustive asssignment of thematic role types to every semantic role in a predicator. Instead, a small set of partial orderings of semantic roles, based on lexical entailments, suffices to account for the linking patterns of a wide range of verbs. 
This insight was adopted in a slightly different guise in work by \citet{Davis1996,Davis2001} and \cite{DavisandKoenig2000b}, who develop a more elaborated representation of lexical semantics, with which simple linking constraints can be stated.
The essence of this approach is to posit a small number of dyadic semantic relations such as \textit{act-und-rel} (`actor-undergoer relation') with attributes  \attrib{act(or)} and \attrib{und(ergoer)} that serve as intermediaries between semantic roles and syntactic arguments (akin to the notion of Generalized Semantic Roles discussed in \citealt{VanValin1999}).  

What are the truth conditions of \textit{act-und-rel}?  
Following \citet{Fillmore1977}, \citet{Dowty91a}, and \citet{Wechsler1995b}, Davis and Koenig note that many of the pertinent lexical entailments come in related pairs.
For instance, one of Dowty's entailments is that one participant causally affects another, and of course the other is entailed to be causally affected.
Another involves the entailments in Wechsler's notion rule (\ref{conceive}); one participant is entailed to have a notion of another. 
These entailments of paired participant types characterize classes of verbs (or other predicators), and can then be naturally represented as dyadic relations in \attrib{content}.  Collecting those entailments we arrive at a disjunctive statement of truth conditions:

\begin{exe}
\ex \textbf{act-und-rel}($x,y$) is true iff x causes a change in y, or x has a notion of y.
\end{exe}

\noindent
We can designate the $x$ participant  in the pair as the value of \attrib{actor} (or \attrib{act}) and $y$ as the value of \attrib{undergoer} (or \attrib{und}), in a relation of type \type{act-und-rel}.   Semantic arguments that are \attrib{actor} or \attrib{undergoer} will then bear at least one of the entailments characteristic of \attrib{actor}s or \attrib{undergoer}s \citep[72]{DavisandKoenig2000b}. This then simplifies the statement of linking constraints for all of these paired participant types.
\citet{Davis1996} and \citet{KoenigandDavis2001} argue that this obviates counting the relative number of proto-agent and proto-patient entailments, as advocated by \citet{Dowty91a}.

The linking constraints \ref{act-vb-linking} and \ref{und-vb-linking} state that 
a verb whose semantic \content is of type \emph{act-und-rel} will be constrained to link the \attrib{act} participant to the the first element of the verb's \argst list (its subject), and the \attrib{und} participant to the second element of the verb's \argst list (this is analogous to Wechsler's constraints based on partial orderings).  

These linking constraints can be viewed as parts of the definition of lexical types, as in \citet{Davis2001}, where (\ref{act-vb-linking}) defines a particular class of lexemes (or words).\footnote{Alternatively, (\ref{act-vb-linking}) (and other linking constraints) can be recast as implicational constraints on lexemes or words 
\citep{KoenigandDavis2003}.    (\ref{act-vb-linking-alt}) is an implicational constraint indicating that a word whose semantic content includes an \textsc{actor} role must map that role to the initial item in the \argst list. 

\begin{exe}
	\ex\label{act-vb-linking-alt}
	{\avmoptions{center}
	\begin{avm}
		\[content$|$key & \[actor & \@1 \] 
		\]
		$\Rightarrow$
		\[ 
		arg-st & \<\textsc{np}$_{\@1}$,  \ldots \>
		\]
	\end{avm}
	}
\end{exe} }   

\begin{exe}
	\ex\label{act-vb-linking}
	{\avmoptions{center}
	\begin{avm}
		\[content$|$key & \[actor & \@1 \] \\
		arg-st & \<\textsc{np}$_{\@1}$,  \ldots \>
		\]
	\end{avm}
	}
\end{exe}

\begin{exe}
	\ex\label{und-vb-linking}
	{\avmoptions{center}
	\begin{avm}
		\[content$|$key & \[undergoer & \@2 \] \\
		arg-st & \<\ldots, \textsc{np}$_{\@2}$,  \ldots \>
		\]
	\end{avm}
	}
\end{exe}

\begin{exe}
	\ex\label{emb-act-vb-linking}
	{\avmoptions{center}
	\begin{avm}
		\[content$|$key & \[\asort{cause-possess-rel} 
									soa & \[actor & \@3 \] \] \\
		arg-st & \<\textit{synsem}\> $\oplus$ \<\textsc{np}$_{\@3}$,  \ldots \>
		\]
	\end{avm}
	}
\end{exe}


\noindent
The first constraint, in (\ref{act-vb-linking}), links the value of \attrib{act} (when not embedded within another attribute) to the first element of \argst.
The second, in (\ref{und-vb-linking}), merely links the value of \attrib{und} (again, when not embedded within another attribute) to some NP on \argst.
Given this understanding of how the values of \attrib{actor} and \attrib{undergoer} are determined, these constraints cover the linking patterns of a wide range of transitive verbs: \word{throw} (\attrib{act} causes motion of \attrib{und}), \word{slice} (\attrib{act} causes change of state in \attrib{und}), \word{frighten} (\attrib{act} causes emotion in \attrib{und}), \word{imagine} (\attrib{act} has a notion of \attrib{und}), \word{traverse} (\attrib{act} ``measures out'' \attrib{und} as an incrememtnal theme), and \word{outnumber} (\attrib{act} is superior to \attrib{und} on a scale).

The third constraint, in (\ref{emb-act-vb-linking}), links the value of an \attrib{act} attribute embedded within a \attrib{soa} attribute to an NP that is second on \argst.
This constraint accounts for the linking of the (primary) object of ditransitives.
In English, these verbs (\word{give}, \word{hand}, \word{send}, \word{earn}, \word{owe}, etc.) involve (prospective) causing of possession \citep{Pinker1989,Goldberg1995}, and the possessor is represented as the value of the embedded \attrib{act} in (\ref{emb-act-vb-linking}).
There could be additional constraints of a similar form in langages with a wider range of ditransitive constructions; conversely, such a constraint might be absent in languages that lack ditransitives entirely.
As mentioned earlier in this section, the range of subcategorization options varies somewhat from one language to another.

The \attrib{key} attribute in (\ref{act-vb-linking}) -- (\ref{emb-act-vb-linking}) also requires explanation.
The formulation of linking constraints here employs the architecture used in \citet{KoenigandDavis2006}, in which the semantics represented in \content values is expressed as a set of \emph{elementary predications}, formalized within Minimal Recursion Semantics \citep{Copestakeetal2001,Copestakeetal2005}.
Each elementary predication is a simple relation, but the relationships among them may be left unspecified.
For linking, one of the elementary predications is designated the \attrib{key}, and it serves as the locus of linking.
This allows us to %sidestep thorny questions of lexical semantics and linking, such as 
indicate the linking of participants that play multiple roles in the denoted situation. 
%, or whether all aspects of a particpant's involvement in an situation type are properly represented in \content.
The \attrib{key} selects one relation as the ``focal point,'' and the other elementary predications are then irrelevant as far as linking is concerned. 
The choice of \attrib{key} then becomes an issue demanding consideration; we will see in the discussion of argument alternations in Section \ref{alternations} how this choice might account for some alternation phenomena.

Note too that these linking constraints are treated as constraints on classes in the lexical hierarchy (see Chapter \ref{LEXICONCHAPTER}).
One consequence of this fact merits brief mention.
Constraint (\ref{und-vb-linking}), which links the value of \attrib{und} to some NP on \argst, is a specification of one class of verbs.
Not all verbs (and certainly not all other predicators, such as nominalizations) with a \content value containing an \attrib{und} value realize it as an NP.
Verbs obeying this constraint include the transitive verbs noted above, and intransitive ``unaccusative'' verbs such as \word{fall} and \word{persist}.
But some verbs with both \attrib{act} and \attrib{und} attributes in their \content are intransitive, such as \word{impinge (on)}, \word{prevail (on}, and \word{tinker (with)}.
Interactions with other constraints, such as the requirement that verbs (in English, at least) have an NP subject, determine the range of observed linking patterns.

These linking constraints also assume that the proto-role attributes \attrib{actor}, \attrib{undergoer}, and \attrib{soa} are appropriately matched to entailments, as described above.
Other formulations are possible, such as that of \citet{KoenigandDavis2003}, where the participant roles pertinent to each lexical entailment are represented in \content by corresponding, distinct attributes.

In addition to the linking constraints, there may be some very general well-formedness conditions on linking.  We rarely find verbs that obligatorily map one semantic role to two distinct members of the \argst list that are both expressed overtly.  A verb meaning `eat', but with that disallowed property, could appear in a ditransitive sentence like (\ref{rah-a}), with the meaning that Pat ate dinner, and his dinner was a large steak.  

\begin{exe}
\label{rah-a} 
	\ex *Pat ate dinner a large steak.
\end{exe}

\noindent
Typically semantic arguments map to at most one (overtly expressed) \argst list item \citep[262-268]{Davis2001}.  

\subsection{Linking oblique arguments}

%We have not yet discussed 
In this section we discuss linking of oblique arguments, that is, PP's and oblique case marked NP's.
In some instances, a verb's selection of a particular preposition appears at least partly arbitrary;
it is hard to explain why we \word{hanker after} and \word{yearn for}, but we don't \word{*yearn after}.
In these cases, the choice of preposition may be stipulated by the individual lexical entry.
But as \citet{Gawron1986} and \citet{Wechsler1995} have shown, many prepositions are semantically meaningful.
\word{For} in the above-mentioned cases, and in \word{look for}, \word{wait for}, and \word{aim for} is surely not a lexical accident.
And in the cases like \word{cut with}, \word{with} is used in an instrumental sense, denoting a \type{use-rel} relation, as with verbs that either allow (\word{eat}) or require (\word{cut}) an instrument.
\citet{Davis1996,Davis2001} adopts the position of Gawron and Wechsler in his treatment of linking to PPs
As an example of this kind of account, the linking type in (\ref{with-linking}) characterizes a verb selecting a \word{with}-PP. 
The PP argument is linked from the \rels list rather the \attrib{key}. 

\begin{exe}\ex\label{with-linking}
{\avmoptions{center}\begin{avm}
\[content & \[key & \@1 \\
              relations & \<\@1, 
                                   \@2\[\asort{use-rel} \\
                                    act & a \\
                                    und & u  \\
                                    soa & s \], 
                                 \ldots \> \] \\
    arg-st & \< \ldots, \textsc{pp}$_{with}$:\@2 \ldots  \>                          
                           \]
                                             \end{avm} }
\end{exe} 


Apart from the details of individual linking constraints, we have endeavored here to describe how linking can be modeled in HPSG using the same kinds of constraints used ubiquitously in the framework.
Within the hierarchical lexicon, constraints between semantically defined classes and syntactically defined ones, can furnish an account of linking patterns, and there is no resort to additional mechanisms such as a thematic hierarchy or numerical comparison of entailments.


\subsection{To what extent does meaning predict linking?}


The framework outlined above allows us to address the following question: how much of linking is strictly determined by semantic factors, and how much is left open to 
lexically arbitrary subcategorization specifications, or perhaps subject to other factors?

Subcategorization--- the position and nature of \argst elements, in HPSG terms-- is evidently driven to a great extent by semantics,
but debate continues about how much, and which components of semantics are involved.
%Taking a viewpoint on one end of the decompositional spectrum, there are many elements of lexical semantics that could appear on \argst.
%For example, denominal verbs like \word{pocket}, \word{chisel}, and \word{saddle}; verbs of sound emission like \word{laugh} and \word{rumble}, and verbs with entailed participants like \word{spit}, \word{bleed}, and \word{urinate} all involve participant types that typically are syntactically unrealized.
%Reviving syntactic representations from Generative Semantics, some  Minimalist accounts have advocated exactly that (see the chapter on HPSG and Minimalism for more discussion) and so do linking theories based on the kind of conceptual structure developed in \citet{Jackendoff1990}, to some extent .
%Within HPSG too, there has been an increasing tendency to include ``non-core'' semantic participants--- adjuncts, adverbials, secondary predications, and implicit arguments such as null anaphors--- on the \argst list.
%We will examine this in more detail in Section \ref{extended-arg-st}.
%Even with regard to core, syntactically realized, arguments, we can ask how closely subcategorization mirrors lexical semantics.
%Here as well, 
Views have ranged from the strict, highly constrained relationship in which lexical semantics essentially determines syntactic argument structure, to a looser one in which some elements of subcategorization may be stipulated.
Among  the first camp are those who espouse the Uniformity of Theta Assignment Hypothesis  proposed in \citet[46]{Baker1988} or \citet{Baker1997}, which maintains that ``identical thematic relationships between items are represented by identical structural relationships'' in the syntax.
With regard to the source of diathesis alternations, \citet[12-13]{Levin1993} notes that ``studies of these properties suggest that argument structures might in turn be derivable to a large extent from the meaning of words'', and accordingly ``pursues the hypothesis of semantic determinism seriously to see just how far it can be taken.''

Others, including \citet{ps} (Section 5.3) and \citet{Davis2001} (Section 5.1), have expressed caution, pointing out cases where subcategorization and diathesis alternations seem to be at least partly arbitrary.
Pollard \& Sag note contrasts like these:

\begin{exe}
\ex \label{ps-subcat-ex}
\begin{xlist}
\ex    Sandy spared/*deprived Kim a second helping.
\ex    Sandy *spared/deprived Kim of a second helping.
\citep[ex. 214--215]{ps}
\end{xlist}
\end{exe}

\noindent
And Davis provides these pairs of semantically similar verbs with differing subcategorization requirements:

\begin{exe}
\ex \label{ard-subcat-ex}
\begin{xlist}
\ex    Few passengers waited for/awaited the train.
\ex    Homer opted for/chose a chocolate frosted donut.
\ex    The music grated on/irritated the critics.
\citep[ex. 5.4]{Davis2001}
\end{xlist}
\end{exe}

Other cases where argument structure seems not to mirror semantics precisely include raising constructions, in which one of a verb's direct arguments bears no semantic role to it at all.
Similarly, overt expletive arguments cannot be seen as deriving from some participant role in a predicator's semantics.
Like the examples above, these phenomena suggest that some aspects of subcategorization are specified independently of semantics.

Another point against strict semantic determination of argument structure comes from cross-linguistic observations of subcategorization possibilities.
It is evident, for example, that not all languages display the same range of direct argument mappings.
Some lack ditransitive constructions entirely (Halkomelem), some allow them across a limited semantic range (English), some quite generally (Georgian), and a few permit tritransitives (Kinyarwanda and Moro).
\citet{Gerdts1992} surveys about twenty languages and describes consistent patterns like these.
The range of phenomena such as causative and applicative formation in a language is constrained by what she terms its ``relational profile;'' this includes, in HPSG terms, the number of direct NP arguments permitted on its \argst lists.
Again, it is unclear that underlying semantic differences across languages in the semantics of verbs meaning `give' or `write' would be responsible for these general patterns.

Summarizing, there is much evidence tempting us to derive the contents of \argst solely from lexical semantics.
If this ultimately proves feasible, then \argst serves more as a convenient interface notion with little possibility of independently expressing strictly syntactic aspects of subcategorization.
This view, however satisfying it might be, does not accord with our current best understanding of the syntactic and semantic evidence.
In the following sections we delve into some of the nuances that make linking more than a simple rendering of lexical semantics.
We begin by noting a point on which HPSG accounts of linking differ from many others--- the absence of traditional thematic roles.


\subsection{HPSG and thematic roles}
\label{thetaroles}

The \textsc{arg-st} list constitutes the syntactic side of the mapping between semantic roles and syntactic dependents.  As \argst is merely an ordered list of arguments, without any semantic ``labels,'' it contains no counterparts to thematic roles, such as \textsc{agent}, \textsc{patient}, \textsc{theme}, or \textsc{goal}.  Thematic roles like these, however, have been a mainstay of linking in generative grammar since \citet{Fillmore1968} and have antecedents going back to (P\={a}\d{n}ini.
Ranking them in a \emph{thematic hierarchy}, and labeling each of a predicator's semantic roles with a unique thematic role, then yields an ordering of roles analogous to the ordering on the \argst list.  Indeed, it would not be difficult to import this kind of system into HPSG, as a means of determining the order of elements on the \argst list.  However, HPSG researchers have generally avoided using a thematic hierarchy, for reasons we now briefly set out.

\citet{Fillmore1968} and many others thereafter have posited a small set of disjoint, thematic roles, with each semantic role of a predicator assigned exactly one thematic role.
Thematic hierarchies depend on these properties for a consistent linking theory.
But they do not hold up well to formal scrutiny.
\citet{Jackendoff1987} and \citet{Dowty91a} note (from somewhat different perspectives) that numerous verbs have arguments not easily assigned a thematic role from the typically posited inventory (e.g., the objects of \word{risk}, \word{blame}, and \word{avoid}), that more than one argument might sensibly be assigned the same role (e.g., the subjects and objects of \word{resemble}, \word{border}, and some alternants of commercial transaction verbs), and that multiple roles can be sensibly assigned to a single argument (the subjects of verbs of volitional motion are like both an \textsc{agent} and a \textsc{theme}).
In addition, consensus on the inventory of thematic roles has proven elusive, and some, notoriously \textsc{theme}, have resisted clear definition.
Work in formal semantics, including \citet{LadusawandDowty1988}, \citet{Dowty1989}, \citet{Landman2000}, and \cite{Schein2002}, casts doubt on the prospects of assigning formally defined thematic roles to all of a predicator's arguments, at least in a manner that would allow them to play a crucial part in linking.
Thematic role types seem to pose problems, and there are alternatives that avoids those problems.  As \citet{Carlson1998} notes about thematic roles, ``It is easy to conceive of how to write a lexicon, a syntax, a morphology, a semantics, or a pragmatics without them.''

%
%Moreover, it is not clear how an ordered hierarchy of thematic roles could be modeled elegantly within HPSG's typed feature structure formalism, though it is certainly not impossible to achieve in some fashion.  Equally unclear is how to compute and compare total numbers of proto-agent and proto-patient entailments for each of a predicator's semantic roles, as \citet{Dowty91a} suggests, because no mechanisms are available for numerical calculations. As a constraint-based formalism, HPSG has led researchers to seek other ways of incorporating insights from these approaches into the framework.

\subsection{\content decomposition and \argst}

%ok, I see.  I agree, this should be discussed.  So, how does this sound as a list of criteria for inclusion in the decomposition?  Include something in the decomposition (in CONTENT) only if needed in order to:
%
%-- account for sublexical scope, e.g. result states (the Dowty 1979 stuff)
%-- account in a Pinker / Davis & Koenig way for argument alternations  
%-- account for linking in our system
%
%Basically this is everything the grammar needs to see.  Anything else?     
%
%So I'll say this, and I'll briefly contrast it with Hale & Keyser, Jackendoff.  Sound ok?  

Instead of thematic role types, lexical decomposition is typically used in HPSG to model the semantic side of the linking relation.  The word meaning represented by the \content value is decomposed into a set of elementary predications that share arguments, as described in Section \ref{linking-sec} above.  Lexical decompositions cannot be directly observed, but the decompositions are justified indirectly by the roles they play in the grammar.  Decompositions play a role in at least the following processes:

\begin{itemize}
\item  \textit{Linking.}  As described in Section \ref{linking-sec}, linking constraints are stated on semantic relations like \textit{act-und-rel} (`actor-undergoer relation'), so those relations must be called out in the \content field.
\item \textit{Sublexical scope.}  Certain modifiers can scope over a part of the situation denoted by a verb \citep{Dowty:1979a}.  

\begin{exe}
\label{again}
\ex John sold the car, and then he bought it again.
\end{exe}

In this sentence the adverb \textit{again} either adds the presupposition that John bought it before, or, in the more probable interpretation, it adds the presupposition that \textit{the result of buying the car} obtained previously.  The result of buying a car is owning it, so this sentence presupposes that John previously owned the car. Thus the decomposition of the verb \textit{buy} in (\ref{buy-lex}) below includes the \textit{possess-rel} (`possession relation') holding between the buyer and the goods.  This is available for modification by adverbials like \textit{again}.
\item \textit{Argument alternations.}  Some argument alternations can be modeled as highlighting of different portions of a single lexical decomposition.  See Section \ref{alternations}.  
\end{itemize}  

\noindent
In general, sublexical decompositions are included in the \content field only insofar as they are visible to the grammar for processes like these.  

The \argst feature lies at the syntax side of the linking relation.  Much like the \content field, the \argst items are justified only insofar as they are visible to the syntax.  Many \argst list items are obviously justified by being explicitly expressed as subject and complement phrases or as affixal pronouns.  Certain implicit arguments appear if they are subject to the binding theory as applied to the \argst list (as discussed in Section \ref{express-sec} above).  

Implicit arguments can also participate in the syntax, and therefore appear on the \argst list, by acting as controllers of adjunct clauses.  For example, English  rationale clauses like the infinitival phrase in (\ref{mka}) are controlled by the agent argument in the clause, \textit{the hunter} in this 
example.  The implicit agent of a short passive can likewise control the rationale clause as shown in(\ref{mkb}).  But the middle in (\ref{mkc}) lacks an implicit agent that is capable of controlling, even though native speakers assume that some agent must have caused the gun to load.  This contrast was observed by \citet{KeyserandRoeper1984} and confirmed in experimental work by \citet{MaunerandKoenig2000}.  

\begin{exe}
\ex\label{mk}
\begin{xlist}
\ex\label{mka}
The shotgun was loaded quietly by the hunter
to avoid the possibility of frightening off the deer.
\ex\label{mkb}The shotgun was loaded quietly
to avoid the possibility of frightening off the deer.
\ex\label{mkc}*The shotgun had loaded quietly
to avoid the possibility of frightening off the deer.
\end{xlist}
\end{exe}



\noindent
If the syntax of control  is specified such that the controller of the rationale clause is an (agent) argument on the \argst list of the verb, then this contrast is captured by assuming that the agent appears on the \argst list of the passive verb but not the middle.


\subsection{Modal transparency}
Another observation concerning lexical entailments and linking was developed by \citet{KoenigandDavis2001}, who point out that linking appears to ignore modal elements of lexical semantics, even when those elements invalidate entailments (expanding on an observation implicit in \citealt{Goldberg1995}).
For instance, there are various English verbs displaying linking patterns like the ditransitive verbs of possession transfer \word{give} and \word{hand}, but which denote situations in which the transfer need not, or does not, take place.
Thus, \word{offer} describes a situation where the transferor is willing to effect the transfer, \word{owe} one in which the transferor should effect the transfer but has not yet, \word{promise} describes a situation where the transferor commits to effect the transfer, and \word{deny} one in which the transferor does not effect the contemplated transfer. 
Koenig and Davis argue that modal elements should be clearly separated in \content values from the representations of predicators and their arguments.  (\ref{promise-sem}) exemplifies this factoring out of sublexical modal information from core situational information. This pattern of linking functioning independently of sublexical modal information applies not only to these ditransitive cases, but also to verbs involving possession (cf. \word{own} and \word{obtain}, vs. \word{lack}, \word{covet} and \word{lose}), perception (\word{see} vs. \word{ignore} and \word{overlook}), and carrying out an action (\word{manage} vs. \word{fail} and \word{try}).  Whatever the role of lexical entailments in linking, then, the modal information should be factored out, since  the entailments canonically driving, e.g., the ditransitive linking patterns of verbs like \word{give} and \word{hand}, do not hold of \word{offer}, \word{owe}, or \word{deny}. The constraints in (\ref{act-vb-linking})-(\ref{emb-act-vb-linking}) need only been minimally altered to target the value of \attrib{sit-core} within the representation of relation.

\begin{exe}
\ex\label{promise-sem} The lexical semantic representation of \word{promise} 
\citep[101]{KoenigandDavis2001}
{\avmoptions{center}
\begin{avm} 
\[\asort{$\textit{promise-sem}$ $\wedge$ $\textit{cause-possess-sem}$}
  sit-core &  \@3\[\asort{$\textit{cause-possess-rel}$}
                actor & \@1 \\
               undergoer & \@2 \\
                soa & \[\textup{sit-core} & \@5\[\asort{$\textit{have-rel}$}
                                       actor & \@2 \\
                                       undergoer & \@4\]\]\]\\
   modal-base & \< \[\asort{$\textit{deontic-mb}$ $\wedge$ $\textit{condit-satis-mb}$}
                soa  & \@3 \] \>\] \end{avm}
}
 \end{exe}
 
 



\section{The semantics and linking of argument alternations}
\label{alternations}

A verb can often occur in varied syntactic contexts, as \word{find} does in (\ref{chair}); these are termed \emph{valence alternations} or \emph{diathesis alternations}, in reference to their different argument structuress.
\citet{Levin1993} lists around 50 kinds of alternations in English, and there are still more, including the alternation illustrated in (\ref{chair}).  

\begin{exe}
	\ex\label{chair}
	\begin{xlist}
	\ex\label{chair-a} I found that the chair was comfortable
	\ex\label{chair-b} I found the chair to be comfortable
	\ex\label{chair-c} I found the chair comfortable	
	\end{xlist}
\end{exe}

Another well studied alternation, the locative alternation, is exemplified by the two uses of \word{spray} in (\ref{spray}).

\begin{exe}
\ex \label{spray}
\begin{xlist}
\ex \label{spraya} \textit{spray$_{loc}$}: Joan sprayed the paint onto the statue.
\ex \label{sprayb} \textit{spray$_{with}$}: Joan sprayed the statue with paint.
 \end{xlist}
 \end{exe}


It is typically assumed that these two different uses of \textit{spray} in (\ref{spray}) have slightly different meanings, with the statue being in some sense more affected in the \word{with} alternant.
This exemplifies the ``holistic'' effect of direct objecthood, which we will return to.
Here, we will examine how semantic differences between alternant relate to their linking patterns.
The semantic side of linking has often been devised with an eye to syntax (e.g., \citet{Pinker1989}, and see \citet{KoenigandDavis2006} for more examples).
There is a risk of stipulation here, without independent evidence for these semantic differences.
In the case of locative alternations, though, the meaning difference between (\ref{spraya}) and (\ref{sprayb}) is easily stated (and Pinker had the right intuition), as (\ref{sprayb}) entails (\ref{spraya}), bot not conversely.
Informally, (\ref{spraya}) describes a particular kind of caused motion situation, while  (\ref{sprayb}) describes a situation in which this kind of caused motion additionally results in a caused change of state.
The difference is depicted in the two structures in (\ref{spray-sem}).

\begin{exe}\ex\label{spray-sem}
\begin{xlist}
\ex \label{spray-sema} \textsc{cause (Joan, go (paint, to (statue)))}
\ex \label{spray-semb} \textsc{act-on (Joan, statue, by (cause (Joan, go (paint, to (statue)))))}
\end{xlist}
\end{exe}

This description of the semantic difference between sentences (\ref{spraya}) and (\ref{sprayb}) provides a strong basis for predicting their different argument structures.
But we still need to explain how linking principles give rise to this difference.
Pinker's account rests on semantic structures like (\ref{spray-sem}), in which depth of embedding reflects sequence of causation, with ordering on \argst stemming from depth of semantic embedding, a strategy adopted in \citet{Davis1996,Davis2001}.
This is one reasonable alternative, although the resulting complexity of some of the semantic rpresentations raises valid questions about what independent evidence supports them.
An alternative appears in \citet{KoenigandDavis2006}, who borrow from Minimal Recursion Semantics (see the chapter on Semantics for an introduction to MRS).
MRS ``flattens'' semantic relationss, rather than embedding them in one another, so the arrangment of these \emph{elementary predications}, as they are temed, is of less import.
They posit a \attrib{relations} (or \rels) attribute that collects a set of elementary predications, each representing some part of the predicator's semantics.
A \attrib{key} attribute specifies a particular member of \rels as the relevant one for linking (of direct syntactic arguments). 
In the case of (\ref{sprayb}) the \attrib{key} is the caused change of state description.
These MRS-style representations of the two alternants of \word{spray}, with different \attrib{key} values, are shown in (\ref{spray-on}) and (\ref{spray-with}).

\begin{exe}
\ex\label{spray-on}
{\avmoptions{center}\begin{avm}\[key & \@5 \\
                    relations & \<\@5\[\asort{$\textit{spray-ch-of-loc-rel}$} 
                                    act & \@1 \\
                                    und & \@4 \\
                                    soa & \[\asort{\textit{ch-of-loc-rel}} 
                                                fig & \@4\]\] \>\]
                  \end{avm}}
\end{exe}

\begin{exe}\ex\label{spray-with}
{\avmoptions{center}\begin{avm}
\[key & \@3\[\asort{$\textit{spray-ch-of-st-rel}$}
                                    act & \@1 \\
                                    und & \@2  \\
                                    soa & \[\asort{ch-of-st-rel} 
                                                und & \@2\] 
                      \] \\
                   relations & \<\@3, 
                                   \[\asort{$\textit{use-rel}$} \\
                                    act & \@1 \\
                                    und & \@4  \\
                                    soa & \@3 \], 
                                 \[\asort{$\textit{spray-ch-of-loc-rel}$} 
                                    act & \@1 \\
                                    und & \@4 \\
                                    soa & \[\asort{ch-of-loc-rel} 
                                                fig & \@4\]\] \>
                           \]
                                             \end{avm} }
\end{exe}                  
                  
Generalizing from this example, one possible characterization of valence alternations, implicit in \citet{KoenigandDavis2006}, is as systematic relations between two sets of lexical entries in which the \rels of any pair of related entries are in a subset/subset relation (a weaker version of that definition would merely require an overlap between the \rels values of the two entries). 
Consider another case; (\ref{caus-inch}) illustrates the causative-inchoative alternation, where the intransitive alternant describes only the change of state, while the transitive one ascribes a explicit causing agent.

\begin{exe}
\ex\label{caus-inch}
\begin{xlist}
	\ex\label{caus-inch-a}John broke the window.
	\ex\label{caus-inch-b}The window broke.
\end{xlist}	
\end{exe}

Under a MRS representation, the change of state relation is a separate member of \rels; it is also included in the \rels of the transitive alternant, which contains a cause relation as well.
Again, the \rels value of one member of each pair of related entries is a subset of the \rels value of the other.

Many other alternations linvolve one argument shifting from direct to oblique.
Some English examples include conative, locative preposition drop, and \word{with} preposition drop alternations, as shown in (\ref{altex}):

\begin{exe}\ex\label{altex}
\begin{xlist}
\ex \label{altexa} Rover clawed (at) Spot. 
\ex \label{altexb} Bill hiked (along) the Appalachian Trail.
\ex \label{altexc} Burns debated (with) Smithers.
\end{xlist}
\end{exe}

It is well known that the direct objects in these alternations seem to be ``affected'' more than their oblique counterparts.
So if Rover clawed Spot, we infer that Spot was subjected to direct contact with Rover's claws and may have been injured by them, while if Rover merely clawed \word{at} Spot, no such inference can be made.
Similarly, to say the one has hiked the Appalachian Trail suggests that one has hiked its entire length, not merely hiked along some portion of it.
This holistic effect is not so evident in cases like (\ref{altexc}), though the direct object variant suggests that a formally organized debate took place, while the \word{with} variant could just describe an informal discussion.
How might these varying intuitions related to ``affectedness'' relate to lexical semantic representations like those in (\ref{spray-on}) and (\ref{spray-with})?
\citet{Beavers2010} provides one analytical advance in this direction, similar to the subset relationship between \rels values described above.
He generalizes from affectedness to strength of entailments, where one semantic role's entailments are stronger than another's if and only if the set of entailments characterizing the second role also hold of the first.
That is, what is true for any particpant that bears the first role will be true for any particpant that bears the second, but not necessarily the converse.
His \emph{Morphosyntactic Alignment Principle} then relates this to linking, as stated in (\ref{beavers-map}), where an ``L-thematic role'' is a linguistically relevant semantic role:

\begin{exe}
\ex\label{beavers-map}
When participant $x$ may be realized as either a direct or oblique argument of verb V, it bears L-thematic role $R$ as a direct argument and L-thematic role $Q\subseteq_{M}R$ as an oblique.
\citep[848]{Beavers2010}
\end {exe}

\noindent
Here, $Q\subseteq_{M}R$ means that $Q$ is a ``minimally weaker'' role than $R$; in other words, there is no role $P$ in the predicator such that $Q⊂P⊂R$.
Thus, the substantive claim is essentially that the MAP rules out ``verbs where the alternating participant has \textsc{MORE} lexical entailments as an oblique than the corresponding object realization'' \citep[849]{Beavers2010}.

The entailments Beavers employs differ somewhat from those we have discussed here, involving quantized change, nonquantized change, potential for change (where change can refer to change in location, possession, state, or something more abstract), furnishing the clear ordering by strength that is central to his proposal.
But they do resemble entailments of semantic relations we have represented as elementary predications, such as incremental theme, change of state, and possession, along with the modal effects described in \citet{KoenigandDavis2001}.
Thus the notion of a stronger role in Beavers' analysis has a rough analog in terms of whether a particular elementary predication is present in the semantics of a particular alternant.
And only if an elementary predication is present, can it be designated as the \attrib{key}, and its roles linked directly.
For example, in (\ref{spray-on}), there is nothing representing affectedness of the location, while in (\ref{spray-with}), there is, and it is designated as the \attrib{key}.
As noted earlier, the semantics in (\ref{spray-with}) represents this additional entailment borne by the location argument.
However, we are not aware of any simple, general way to represent Beavers' MAP within the EP-based model of \citet{KoenigandDavis2006}.
Indeed, there is an aspect of Beavers' view that seems more in accord with numerical comparison approaches such as those of \citet{Dowty91a} and \citet{AckermanandMoore2001}, in that role strength is determined by the number of entailments that hold of it relative to others.

Having outlined the semantic basis of the different linking patterns of alternating verbs, we briefly take up three other issues.
First is the question of how the alternants are related to one another.
Second is how \attrib{key} selection has been used to account not just for alternants of the same verb, but for (nearly) synonymous verbs whose semantics contain the same set of elementary predications.
Third is whether passives, which arguably do not differ semantically from their active counterparts, should be assimilated with other alternations or treated distinctly, as a kind of non-canonical lexical item.

The hypothesis pursued in \citet{Davis1996,Davis2001}  is that 
most alternations are the consequence of classes of lexical entries having
two related meanings. This follows researchers such as \citet{Pinker1989} and \citet{Levin1993} in modeling subcategorization alternations as underlyingly meaning alternations. 
This change in meaning is crucial in \citet{KoenigandDavis2006} \attrib{key} shifts as well. In some cases, the value of the \rels attribute of the two valence alternates differ (as in the two alternates of \word{spray} in \word{spray/load} alternation we discussed earlier).
In some cases, the alternation might be different construals of the same event for some verbs, but not others, as \citet{RappaportandLevin2008} claim for the English ditransitive alternations, which adds the meaning of transfer for verbs like \word{send}, but not for verbs like \word{promise}; a \attrib{key} change would be involved (with the addition of a \type{cause-possess-rel}) for the first verb only. But \attrib{key} shifts and diathesis alternations do not always involve a change in meaning. The same elementary predications can be present in as the \attrib{content} values of two alternants, with each alternant designating a different elementary predication as the \attrib{key}. 

Koenig \& Davis propose this not only for cases in which there is no obvious meaning difference betwen two alternants of a verb, but also for different verbs that appear to be truth-conditionally equivalent, one famous example being the verbs of commercial exchange \word{buy} and \word{sell} (but see \citet[387-388]{VanValin1999}, \citet[20]{LevinandRappaport2005}, and \citet{Wechsler:2005} for arguments that \word{buy} and \word{sell} are not equivalent). Koenig and Davis argue that a commercial event involves two reciprocal actions, an exchange of goods (which involves giving goods and obtaining goods) and an exchange of money (which involves giving money and obtaining money). Individual verbs might select  one or the other these four relations, thus accounting for the differences in subject and object selection. As shown in (\ref{buy-lex}) and (\ref{sell-lex}), each of these verbs contains four elementary predications: one \type{exch-give-rel} and one \type{obtain-rel} for the transfer of goods, and one of each for the counter-transfer of money. \word{Buy} designates the \type{obtain-rel} representing the transfer of goods as the \attrib{key}, while \word{sell} designates the \type{exch-give-rel} representing the transfer of goods as the \attrib{key}.  Other verbs, such as \word{pay} or \word{charge}, choose elementary predications representing the counter-transfer as the \attrib{key}.  In all cases, the same linking constraints apply between the  \attrib{key} and the \argst list, yielding the different argument realizations of these verbs while preserving their underlying semantic commonality. The relevant portions of the entries for \word{buy} and \word{sell} in (\ref{buy-lex}) and (\ref{sell-lex}) below illustrate: critically, the \attrib{key} relation for \word{buy} is not the same as that for \word{sell}.


\begin{exe}
\ex\label{buy-lex}
A representation of the relevant parts of the lexical entry for \word{buy}: \\
{\avmoptions{center}
\begin{avm}\[content & \[key &\@7 \\
                         relations & \<\@7\[act & \@1 (buyer) \\
                          und & \@2 \\
                          soa & \@5\[\asort{$\textit{possess-rel}$} 
                                    act & \@1 \\
                                    und & \@2 (goods)\]\],
                        \[act & \@1 (buyer) \\
                          und & \@4 \\
                    soa & \@6\[\asort{$\textit{possess-rel}$}
                                    act & \@3 (seller) \\ 
                                    und & \@4 (money)\]\] \\
                              \[\asort{$\textit{obtain-rel}$}
                   act & \@3 (seller) \\
                          und & \@4 \\
                    soa & \@6\[\asort{$\textit{possess-rel}$} 
                                    act & \@3 (seller) \\ 
                                    und & \@4 (money)\]\], 
\@8\[\asort{$\textit{exch-give-rel}$}
                    act & \@3 (seller) \\
                          und & \@2 \\
                    soa & \@5\[\asort{$\textit{possess-rel}$} 
                                    act & \@1 \\ 
                                    und & \@2 (goods)\]\]
                                                         \> \] \\
             arg-st & \<NP:\@1, NP:\@2, PP(from):\@8\>\]
\end{avm}}	
\end{exe}

\begin{exe}
\ex\label{sell-lex}	
A representation of the relevant parts of the lexical entry for \word{sell}: \\
{\avmoptions{center}
\begin{avm}\[content & \[key &\@7 \\
                         relations & \<\@7\[act & \@1 (seller) \\
                                          und & \@2 \\
                    soa & \@6\[\avmspan{\textit{possess-rel}} \\
                                    act & \@3(buyer) \\ 
                                    und & \@2 (goods)\]\],
                             \[act & \@1 (seller) \\
                          und & \@4 \\
                    soa & \@6\[\avmspan{\textit{possess-rel}} \\
                                    act & \@1 \\ 
                                    und & \@4 (money)\]\]\> \] \\
             arg-st & \<NP:\@1, NP:\@2, PP(to):\@7\>\]
\end{avm}}
\end{exe}


As a final example of  semantic alternations in the fine-grained meaning possibilities of verbs, we consider here the source-final product alternation exemplified in (\ref{carve}) where the direct object can be either the final product or the material source of the final product. Davis proposes that the (\ref{carve-a}) sentences involve an alternation between the two meanings of entries represented in (\ref{carve-sem}). We adapt \citet{Davis2001} to make it consistent with \citet{KoenigandDavis2006} and also treat the alternation as an alternation of meaning of \emph{entries}. Note that in the meaning alternation described in (\ref{carve-sem}), we use, informally, a double-headed arrow. One of the potential drawbacks of a lexical rule approach to valence alternations is that it requires selecting one or the other alternant as basic and the other as derived (e.g., is the inchoative or the causative basic?). This is not always an easy decision, as \citet{Goldberg1995} or \citet{LevinandRappaport1994} have pointed out. Sometimes, morphology provides a clue, although in different languages the clues may point in different directions.  French, and other Romance languages, use a ``reflexive'' clitic as a detransitivizing affix.  In English, though, there is no obvious ``basic'' form or directionality. It is to avoid committing ourselves to a directionality in the meaning relation described in (\ref{carve-sem}) that we eschews treating it as a lexical rule.

\begin{exe}
\ex\label{carve}
\begin{xlist}
	\ex\label{carve-a} Kim made/carved/sculpted/crafted a toy (out of the wood).
	\ex\label{carve-b} Kim made/carved/sculpted/crafted the wood into a toy.
\end{xlist}
\end{exe}

\begin{exe}
	\ex\label{carve-sem}
{\avmoptions{center}
\begin{avm}
\[
content \[key & \@3\[\asort{affect-incr-th-rel} 
act & \@1 \\
soa & \@2(final product)\] \\
rels & \< \@3, \@4 \> \] \]	
\end{avm} \\ \vspace{.25in}
\Huge{$\leftrightarrow$} \\
\begin{avm}
\[
content \[key &
\@4\[\asort{affect-incr-th-rel}
act & \@1 \\
und & (source material) \\
soa & \[\asort{affect-incr-th-rel} 
		act & \@1 \\
		soa & \@2(final product)\] \] \\
rels & \< \@3, \@4 \>
\] \]
\end{avm}		
}
\end{exe}
 

Although most diathesis alternations can be modeled as alternations in meaning and as \attrib{key} shifts, some arguably cannot. We discuss the active/passive alternation here, but impersonals, as well as raising structures exemplified in (\ref{chair}) are good candidates too. 
The semantic relations of actives and long passives, as in (\ref{act-pass}), are practically identical and the difference between the two alternates is pragmatic in nature. Arguably, then, passives are a degenerate case of the subset relationship between \rels attributes, where the \rels values of the two entries are identical and so are the two entries' \attrib{key}.
But this raises the question of whether linking in passives violates the constraints in (\ref{act-vb-linking})--(\ref{emb-act-vb-linking}), especially (\ref{act-vb-linking}, which links the value of \attrib{act} to the first element of \argst.

\begin{exe}
	\ex\label{act-pass}
	\begin{xlist}
		\ex\label{act-pass-a}Fido dug a couple of holes.
		\ex\label{act-pass-b}A couple of holes were dug by Fido.
	\end{xlist}
\end{exe}

One typical HPSG method for modeling valence alternations like passives is through lexical rules (see Chapter \ref{CHAPTERONLEXICON}) with one alternant serving as input and the other as output; the main effect of the lexical rule in such an approach is to alter the \argst of the input, going from the \argst of \word{give} to that of \word{given} in (\ref{pasargst}). Critically, we must assume that the output cannot be subject to linking constraint (\ref{act-vb-linking}), since the actor argument is not linked to the first member of the \argst list.
A simplified representation of what such a rule would look like is provided in (\ref{pass-lr}) where we assume that the input to the rule must be transitive.


\begin{exe}
\ex\label{pass-lr}
{\avmoptions{center}
\begin{avm}
	\[\asort{$\textit{trans-vb}$}
	head & $\textit{verb}$ \\
		arg-st & \< \textsc{NP}$_{\@1}$, NP$_ {\@2}$\> $\oplus$ \@3$\textit{list}$
	\]
	$\mapsto$
	\[head & \[vform & pass \] \\
	arg-st & \<NP$_{\@2}$\> $\oplus$\@3 $\oplus$
	\<PP[$_{by}$]$_{\@1}$\>
	\]
	\end{avm}
}

\end{exe}


To sum up, in contrast to most meaning-driven alternations, valence alternations like the active/passive are  modeled through the use of lexical rules that alter the \argst of ``base'' entries. Which alternations pattern with active/passive and require positing lexical rules that alter a ``base'' entry's \argst list is, as of yet, not settled. 
We turn now to roles that are putatively present semantically, but not realized syntactically at all. 


\section{Extended \argst}
\label{sec:extended-arg-st}

Most of this chapter focuses on cases where semantic roles linked to the \argst list are arguments of the verb's core meaning. But in quite a few cases, complements (or even subjects) of a verb are not part of this basic meaning; consequently, the \argst list must be extended to include elements beyond the basic meaning. We consider three cases here, illustrated in (\ref{fish})--(\ref{adj}).  

Resultatives, illustrated in (\ref{fish}), express an effect, which is caused by an action of the type denoted by the basic meaning of the verb. The verb \textit{fischen} `to fish' is a simple intransitive verb (\ref{fish}a) that does not entail that any fish were caught, or any other specific effect of the fishing.  

\begin{exe}
\ex\label{fish}
\begin{xlist}
\ex
\gll dass er  fischt\\
     that he  fishes\\
\glt `that he is fishing'
\ex 
\gll dass er ihn leer fischt\\
     that he it empty fishes\\
\glt `that he is fishing it empty'
\ex 
\gll wegen der Leerfischung der Nordsee \\
     because.of the empty.fishing of.the North.See.\textsc{gen} \\
\glt `because of the North Sea being fished empty'
\end{xlist}
\end{exe}

\noindent
 In (\ref{fish}b) we see a resultative construction, with an object NP and a secondary predicate AP.  The meaning is that he is fishing, causing it (the body of water) to become empty of fish.  \citet{Mueller:2002} posits a German lexical rule applying to the verb that augments the \argst list with an NP and AP, and adds the causal semantics to the \content (see \cite{Wechsler2005result} for a similar analysis of English resultatives).     The existence of deverbal nouns like \textit{Leerfischung} `fishing empty', which takes the body of water as an argument in genitive case (see \ref{fish}c) confirms that the addition of the object is a lexical process, as noted by \citet{Mueller:2002}.  

%any effect or change of state due to the hammering. Since there is evidence th
Romance clause-union structures as in (\ref{faire}) have long been analyzed as cases where the complements of the complements of a clause-union verb (\word{faire} in (\ref{faire})) are complements of the clause-union verb itself \citep{Aissen1979}. 

\begin{exe}
\ex \label{faire}
\gll Johanna a fait manger les enfants. \\
Johanna have made eat the children \\
\glt `Johanna had the children eat.'
\end{exe}

\noindent
Within HPSG, the ``union'' of the two verbs' dependents is modeled via the composition of \argst lists of the clause union verb, following \citet{HinrichsandNakazawa1994} (this is a slight simplification, see Chapter \ref{CHAPTERONCOMPLEXPREDICATES} for details). 

Sentence (\ref{adj}) illustrates a slightly different point, namely that some semantic modifiers, such as \word{souvent} in (\ref{adj}), can be realized as complements, and thus should be added as members of \argst (or members of the \deps list if one countenances such an additional list). 

\begin{exe}
\ex\label{adj}
 \gll
	Mes amis m’ ont souvent aidé. \\
	My friends me have often helped \\
	\glt `My friends often helped me.'
\end{exe}

\noindent
\citet{AbeilleandGodard1997} have argued that many adverbs including negative adverbs and negation in French are complements of the verb and \citet{KimandSag2002} extended that view to some uses of negation in English. In contrast to resultatives, which affect the meaning of the verb, or to clause union, where one verb co-opts the argument structure of another verb, what is added to the \argst list in these cases is typically considered a semantic adjunct and a modifier in HPSG (thus it selects the verb or VP via the \attrib{mod} attribute). 

Another case of an adjunct that behaves like a complement is found in (\ref{mourir}).  The clitic \word{en} expressing the cause of death   is not normally an argument of the verb \textit{mourir} `die', but rather an adjunct \citep{KoenigandDavis2006}: % discuss examples such as (\ref{mourir}).

\begin{exe}
	\ex\label{mourir}
	\gll Il en est mort \\
	 He of.it is dead.\textsc{perf.past} \\
	 \glt `He died of it' (Koenig \& Davis 2006, ex. 12a)
\end{exe}

\noindent
On the widespread assumption (at least within HPSG) that pronominal clitics are verbal affixes (see Miller \& Sag 1997), the adjunct cause of the verb \word{mourir} must be represented within the entry for \word{mourir}, so as to trigger affixation by \word{en}.  Bouma, Malouf \& Sag (2001) discuss such cases and other cases where ``adverbials'' as they call them, can be part of a verb's lexical entry. To avoid mixing those adverbials with the argument structure list (and have to address their relative obliqueness with syntactic arguments of verbs), they introduce yet an additional list, the dependents list (abbreviated as \deps) which includes the \argst list but also a list of adverbials. Each adverbial selects for the verb on whose \deps list it appears as argument, as shown in (\ref{deps}). But, of course, not all verb modifiers can be part of the \deps list and Bouma, Malouf \& Sag discuss at length some of the differences between the two kinds of ``adverbials.''

\begin{exe}
\ex\label{deps}	\type{verb} $\Rightarrow$
{\avmoptions{center}
\begin{avm}
	\[cont$|$key & \@2 \\
	  head & \@3 \\
	  deps & \@1 $\oplus$ $\textit{list}$(\[\textup{\textsc{mod}} & \[\textup{\textsc{head}} & \@3 \\
	  									 \textup{key} & \@2\]\]) \\
	  arg-st & \@1
	\]
\end{avm}}
\end{exe}

Although the three cases we have outlined result in an extended \argst, the ways in which this extension arises differ. In the case of resultatives, the extension results partly or wholly from changing the meaning in a way similar to \citet{RappaportandLevin1998}: by adding a causal relation, the effect argument of this causal relation is added to the membership in the base \argst list (see Section \ref{alternations} for a definition of the attributes \attrib{key} and \attrib{rels}; suffice it to say for now that a \type{cause-rel} is added to the list of relations that are the input of the rule). The entries of the clause union verbs are simply stipulated to include on their \argst lists the syntatic arguments of their (lexical) verbal arguments. Finally, (negative) adverbs that select for a verb (VP) are added to the \argst of the verb they select. A simplified representation of all three processes is provided in (\ref{res-lr})-(\ref{neg-lr}). 


\begin{exe}
	\ex\label{res-lr}
	{\avmoptions{center}
	\begin{avm}
		\[key & \@2 \\ rels & \@1\<\ldots \@2 \ldots \>\]
	\end{avm}
	$\mapsto$
	\begin{avm}
		\[key & \@3$\textit{cause-rel}$ \\ rels & \@1 $\oplus$ \@3\]
	\end{avm}
	}
\end{exe}

\begin{exe}
\ex\label{faire-ent}
	{\avmoptions{center}
	\begin{avm}
		\[arg-st & \<\ldots, \[head & $\textit{verb}$ \\
								arg-st & \@1\> \] $\bigcirc$ \@1\]
	\end{avm}
	}
\end{exe} 

\begin{exe}
	\ex\label{neg-lr}
	{\avmoptions{center}
	\begin{avm}
		\[arg-st & \@1\]
	\end{avm}
	$\mapsto$
	\begin{avm}
		\[arg-st & \@1 $\bigcirc$ \<\textsc{adv}$_{neg}$\>\]
	\end{avm}
	}
\end{exe}


 

\section{Is \argst universal?}

In this section, we briefly consider the question of whether something akin to Figure \ref{fig:over} offers a satisfying account of the grammatical encoding of semantic arguments  across all languages.
Because of its role in accounting for the syntax of basic clauses, the presence of an \argst list on lexical entries comes with expectations about the syntactic realization of semantic roles.
In recent work, \citet{KoenigandMichelson2015a} argue these expectations are not universally borne out, based on data from Oneida (Norther Iroquoian). The only grammatical reflex of semantic arguments in Oneida, they argue, is inflectional: the referencing of semantic arguments by so-called pronominal prefixes, which are better thought of as agreement markers \`a la \citet{Evans2002}. 
Koenig and Michelson distinguish between grammatical and syntactic arguments.
Grammatical arguments include not only syntactic arguments (that is, those on \argst) but also inflectional referencing of semantic roles. 
Some ordering analogous to linking in other languages is present in Oneida, because the semantic roles are not arbitrarily associated with agreement morphemes, but this can be captured in an ordered list of semantic indices , called \attrib{infl-str} in this proposal. \attrib{infl-str} is part of the morphological information relevant to word-internal inflectional processes, what  \citet{Anderson1992} calls `Morphosyntactic Representation' in his treatment of Georgian agreement markers. The ordering of semantic indices on \attrib{infl-str} insures that the predicator is properly inflected. For example,  the prefix \word{lak-} occurs if a third singular masculine proto-agent argument is acting on a first singular proto-patient argument as in \word{lak-hlo·lí-he\textipa{P}}  `he tells me’ (habitual aspect), whereas  the prefix \word{li-} occurs if a first singular proto-agent argument is acting on a third masculine singular argument, as in \word{li-hlo·lí-he\textipa{P}} `I tell him’ (habitual aspect).


In (\ref{KD-03}) and (\ref{KM-ex}) we show the distinction between grammatical arguments realized as syntactic arguments (as in most languages) and those that are not (as in Oneida).
Here, Koenig and Michelson follow the encoding of linking constraints as implicational constraints, as in \citet{KoenigandDavis2003}, although nothing critical hinges on that choice. 

\begin{exe}
	\ex\label{KD-03}
	{\avmoptions{center}
	\begin{avm}\[content & \[\asort{$\textit{cause-rel}$}
												causer & \@1\] \\
                   arg-st & \<np, \ldots \>\]\end{avm}
$\Rightarrow$ \begin{avm}\[
                           arg-st & \<np:\@1, \ldots\>\]
\end{avm} 
	}
\end{exe}
\begin{exe}
\ex\label{KM-ex}	
{\avmoptions{center}\begin{avm}\[content & \[\asort{$\textit{cause-rel}$}
                                              causer & \@1anim\] \]\end{avm}
$\Rightarrow$ \begin{avm}\[
                           infl-str & \<\@1, \ldots\>\]
\end{avm}}
\end{exe}

(\ref{KD-03}) constrains the association between a cause and a \type{synsem} member of the \argst list; (\ref{KM-ex}) constrains the  semantic \type{index} of the cause to be the first member of \attrib{infl-str}. 

If a language like Oneida (Northern Iroquoian) only includes an ordering of semantic indices for inflectional purposes and constraints such as (\ref{KM-ex}) and no \argst list, a number of predictions follow, which Koenig and Michelson claim are borne out. Briefly summarizing their evidence, the relation between semantic arguments and external phrases, when they occur, is not necessarily one of co-indexing, no binding constraints exist between external phrases (e.g., condition C violations can be found), there are no valence alternations, and no syntactic constraints on extraction. In other words, Oneida contains none of the evidence supporting the presence of an \argst list and an ordering of syntactic arguments along an obliqueness hierarchy we have discussed in this chapter. 
The \argst list may thus not a universal attribute of words, though present in the overwhelming majority of languages.
Linking, understood as constraints between semantic roles and members of the \argst list, is then but one possibility; constraints that relate semantic roles to an \attrib{infl-str} list of semantic indices is also an option.
In languages that exclusively exploit that latter possibility, syntax is indeed simpler.


\section{The lexical approach to argument structure}
\label{lexicalapproach}

We end this chapter with a necessarily brief comparison between the approach to argument structure we describe in this chapter with other approaches to argument structure that have developed since the 1990's.
This chapter describes a \textit{lexical approach to argument structure}, which is typical of research in HPSG.  The basic tenet of such approaches is that lexical items include argument structures, which represent essential information about potential argument selection and expression, but
abstract away from the actual local phrasal structure.  In contrast, \emph{phrasal approaches}, which
are common both in Construction Grammar and in transformational approaches such as Distributed Morphology, reject such lexical argument structures.   Let us briefly review the reasons for preferring a lexical approach. (This section is drawn from \citet{Mueller+Wechsler:2014}, which may be consulted for more detailed and extensive argumentation). 

In phrasal approaches to argument structure, components of a verb's apparent meaning are actually `constructional meaning' contributed directly by the phrasal structure.  The linking constraints of the sort discussed above are then said to arise from the interaction of the verb meaning with the constructional meaning.  For example, agentive arguments tend to be realized as subjects, not objects, of transitive verbs.  On the theory presented above, that generalization is captured by the linking constraint (\ref{act-vb-linking}), which states that the \textsc{actor} argument of an \textit{act-und-rel} (`actor-undergoer relation') is mapped to the initial item in the \argst list.  In a phrasal approach, the agentive semantics is directly associated with the subject position in the phrase structure.  In transformational theories, a silent `light verb' (usually called `little \textit{v}') heads a projection in the phrase structure and assigns the agent role to its specifier (the subject).  In constructional theories, the phrase structure itself assigns the agent role.  In either type of phrasal approach, the agentive component of the verb meaning is actually expressed by the phrasal structure into which the verb is inserted.  


The lexicalist’s predicate argument structure provides essential information for a verb's potential
combination with argument phrases.   If a given lexical entry could only  combine with the particular set of phrases specified in a single \val feature, then the lexical and phrasal approaches would be difficult to distinguish: whatever information the lexicalist specifies for each \val list item could, on the phrasal view, be specified instead for the phrases realizing those list items.  
But crucially, the verb need not immediately combine with its specified
arguments.  Alternatively it can meet other fates: it can serve as the input to a lexical rule; it
can combine first with a modifier in an adjunction structure; it can be coordinated with another
word with the same predicate argument structure; instead of being realized locally, one or more of
its arguments can be effectively transferred to another head’s valence feature (raising or argument
transfer); or arguments can be saved for expression in some other syntactic position (partial
fronting).   Here we consider two of these, lexical rules and coordination.  
 
The predicate argument structure is abstract: it does not directly encode the phrase structure or
precedence relations between this verb and its arguments. This abstraction captures the commonality
across different syntactic expressions of the arguments of a given root.

\begin{exe}
\ex \label{nibble}
\begin{xlist}
\ex  The rabbits were nibbling the carrots.  
\ex  The carrots were being nibbled (by the rabbits).
\ex  a large, partly nibbled, orange carrot 
\ex  the quiet, nibbling, old rabbits
\ex  the rabbit's nibbling of the carrots
\ex  The rabbit gave the carrot a nibble.  
\ex  The rabbit wants a nibble (on the carrot).  
\ex  The rabbit nibbled the carrot smooth.
\end{xlist}
\end{exe}

\noindent
Verbs undergo  morpholexical operations like passive
(\ref{nibble}d), as well as antipassive, causative, and applicative in other languages.  They have cognates in
other parts of speech such as adjectives  (\ref{nibble}c,d) and nouns  (\ref{nibble}e,f,g).  
Verbs have been argued to form complex predicates with resultative secondary predicates (\ref{nibble}h), and with serial verbs in other languages.   

The same root lexical entry \emph{nibble}, with the same meaning, appears in all of these contexts.
The effects of lexical rules together with the rules of syntax dictate the proper argument
expression in each context.  For example, if we call the first two arguments in an \argst list
(such as the one in (\ref{nibble}) above) Arg1 and Arg2, respectively, then in an active
transitive sentence Arg1 is the subject and Arg2 the object; in the passive, Arg2 is the subject and
the referential index of Arg1 is optionally assigned to a \emph{by}-phrase.  The same rules of
syntax dictate the position of the subject, whether the verb is active or passive.  When adjectives
are derived from verbal participles, whether active (\emph{a nibbling rabbit}) or passive (\emph{a
  nibbled carrot}), the rule is that whichever role would have been expressed as the subject of the
verb is assigned by the participial adjective to the referent of the noun that it modifies, see \citet{Bresnan:1982passive} 
and \citet[Chapter~3]{BATW2015a}.  
The phrasal approach, in which the agent role is assigned to the subject position, is too rigid.  

Nor could this be solved by associating each syntactic environment with a different meaningful phrasal construction: an active construction with agent role in the subject position; a passive construction with agent in the \textit{by}-phrase position; etc.  The problem for that view is that that one lexical rule can feed another.  In the example above, the output of the verbal passive rule (see (\ref{nibble}d)) feeds the adjective formation rule (see (\ref{nibble}e)).  
 
A verb can also be coordinated with another verb with the same valence requirements.  The two verbs then share their dependents.  This causes problems for the phrasal view, especially when a given dependent receives different semantic roles from the two verbs.  For example, in an influential phrasal analysis, \citet{hale+keyser:1993}
derived denominal verbs like \textit{to saddle} through noun incorporation out of a structure akin to
[PUT a saddle ON x].  Verbs with this putative derivation routinely coordinate and share
dependents with verbs of other types: 

\begin{exe}
\ex Realizing the dire results of such a capture and that he was the only one to prevent it, he quickly
[saddled and mounted] his trusted horse and with a grim determination began a journey that would
become legendary.\footnote{\url{http://www.jouetthouse.org/index.php?option=com_content&view=article&id=56&Itemid=63},
  21.07.2012}  
\end{exe}

\noindent
Under the phrasal analysis the two verbs place contradictory demands on a single phrase structure.  But on the lexical analysis, this is simple V$^0$ coordination.   
 
To summarize, a lexical argument structure is an abstraction or generalization over various occurrences of the verb in syntactic contexts. To be sure, one key use of that argument structure is simply to indicate what sort of phrases the verb must (or can) combine with, and the result of semantic composition; if that were the whole story then the phrasal theory would be viable. But it is not. As it turns out, this lexical valence structure, once abstracted, can alternatively be used in other ways: among other possibilities, the verb (crucially including its valence structure) can be coordinated with other verbs that have a similar valence structure; or it can serve as the input to lexical rules specifying a new word bearing a systematic relation to the input word.   The phrasal approach prematurely commits to a single phrasal position for the realization of 
a semantic argument.  In contrast, a lexical argument structure gives a word the appropriate flexibility to account for the full range of expressions found in natural language.   
 
 
%\section*{Abbreviations}
%\section*{Acknowledgements}

%
\printbibliography[heading=subbibliography,notkeyword=this] 
\end{document}
