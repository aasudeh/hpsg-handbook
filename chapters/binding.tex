% !BIB TS-program = biber
% !TeX TS-program = xelatexmk

\documentclass[output=paper
                ,modfonts
                ,nonflat
	        ,collection
	        ,collectionchapter
	        ,collectiontoclongg
 	        ,biblatex
                ,babelshorthands
                ,newtxmath
                ,draftmode
                ,colorlinks, citecolor=brown
]{./langsci/langscibook}

\IfFileExists{../localcommands.tex}{%hack to check whether this is being compiled as part of a collection or standalone
  % add all extra packages you need to load to this file 

\usepackage{graphicx}
\usepackage{tabularx}
\usepackage{amsmath} 
\usepackage{tipa}      % Davis Koenig
\usepackage{multicol}
\usepackage{lipsum}


\usepackage{./langsci/styles/langsci-optional} 
\usepackage{./langsci/styles/langsci-lgr}
%\usepackage{./styles/forest/forest}
\usepackage{./langsci/styles/langsci-forest-setup}
\usepackage{morewrites}

\usepackage{tikz-cd}

\usepackage{./styles/tikz-grid}
\usetikzlibrary{shadows}


%\usepackage{pgfplots} % for data/theory figure in minimalism.tex
% fix some issue with Mod https://tex.stackexchange.com/a/330076
\makeatletter
\let\pgfmathModX=\pgfmathMod@
\usepackage{pgfplots}%
\let\pgfmathMod@=\pgfmathModX
\makeatother

\usepackage{subcaption}

% Stefan Müller's styles
\usepackage{./styles/merkmalstruktur,german,./styles/makros.2e,./styles/my-xspace,./styles/article-ex,
./styles/eng-date}

\selectlanguage{USenglish}

\usepackage{./styles/abbrev}

\usepackage{./langsci/styles/jambox}

% Has to be loaded late since otherwise footnotes will not work

%%%%%%%%%%%%%%%%%%%%%%%%%%%%%%%%%%%%%%%%%%%%%%%%%%%%
%%%                                              %%%
%%%           Examples                           %%%
%%%                                              %%%
%%%%%%%%%%%%%%%%%%%%%%%%%%%%%%%%%%%%%%%%%%%%%%%%%%%%
% remove the percentage signs in the following lines
% if your book makes use of linguistic examples
\usepackage{./langsci/styles/langsci-gb4e} 

% Crossing out text
% uncomment when needed
%\usepackage{ulem}

\usepackage{./styles/additional-langsci-index-shortcuts}

%\usepackage{./langsci/styles/langsci-avm}
\usepackage{./styles/avm+}


\renewcommand{\tpv}[1]{{\avmjvalfont\itshape #1}}

% no small caps please
\renewcommand{\phonshape}[0]{\normalfont\itshape}

\regAvmFonts

\usepackage{theorem}

\newtheorem{mydefinition}{Def.}
\newtheorem{principle}{Principle}

{\theoremstyle{break}
%\newtheorem{schema}{Schema}
\newtheorem{mydefinition-break}[mydefinition]{Def.}
\newtheorem{principle-break}[principle]{Principle}
}

% This avoids linebreaks in the Schema
\newcounter{schema}
\newenvironment{schema}[1][]
  {% \begin{Beispiel}[<title>]
  \goodbreak%
  \refstepcounter{schema}%
  \begin{list}{}{\setlength{\labelwidth}{0pt}\setlength{\labelsep}{0pt}\setlength{\rightmargin}{0pt}\setlength{\leftmargin}{0pt}}%
    \item[{\textbf{Schema~\theschema}}]\hspace{.5em}\textbf{(#1)}\nopagebreak[4]\par\nobreak}%
  {\end{list}}% \end{Beispiel}

%% \newcommand{schema}[2]{
%% \begin{minipage}{\textwidth}
%% {\textbf{Schema~\theschema}}]\hspace{.5em}\textbf{(#1)}\\
%% #2
%% \end{minipage}}

%\usepackage{subfig}





% Davis Koenig Lexikon

\usepackage{tikz-qtree,tikz-qtree-compat} % Davis Koenig remove

\usepackage{shadow}




\usepackage[english]{isodate} % Andy Lücking
\usepackage[autostyle]{csquotes} % Andy
%\usepackage[autolanguage]{numprint}

%\defaultfontfeatures{
%    Path = /usr/local/texlive/2017/texmf-dist/fonts/opentype/public/fontawesome/ }

%% https://tex.stackexchange.com/a/316948/18561
%\defaultfontfeatures{Extension = .otf}% adds .otf to end of path when font loaded without ext parameter e.g. \newfontfamily{\FA}{FontAwesome} > \newfontfamily{\FA}{FontAwesome.otf}
%\usepackage{fontawesome} % Andy Lücking
\usepackage{pifont} % Andy Lücking -> hand

\usetikzlibrary{decorations.pathreplacing} % Andy Lücking
\usetikzlibrary{matrix} % Andy 
\usetikzlibrary{positioning} % Andy
\usepackage{tikz-3dplot} % Andy

% pragmatics
\usepackage{eqparbox} % Andy
\usepackage{enumitem} % Andy
\usepackage{longtable} % Andy
\usepackage{tabu} % Andy


% Manfred's packages

%\usepackage{shadow}

\usepackage{tabularx}
\newcolumntype{L}[1]{>{\raggedright\arraybackslash}p{#1}} % linksbündig mit Breitenangabe


% Jong-Bok

%\usepackage{xytree}

\newcommand{\xytree}[2][dummy]{Let's do the tree!}

% seems evil, get rid of it
% defines \ex is incompatible with gb4e
%\usepackage{lingmacros}

% taken from lingmacros:
\makeatletter
% \evnup is used to line up the enumsentence number and an entry along
% the top.  It can take an argument to improve lining up.
\def\evnup{\@ifnextchar[{\@evnup}{\@evnup[0pt]}}

\def\@evnup[#1]#2{\setbox1=\hbox{#2}%
\dimen1=\ht1 \advance\dimen1 by -.5\baselineskip%
\advance\dimen1 by -#1%
\leavevmode\lower\dimen1\box1}
\makeatother


% YK -- CG chapter

%\usepackage{xspace}
\usepackage{bm}
\usepackage{bussproofs}


% Antonio Branco, remove this
\usepackage{epsfig}

% now unicode
%\usepackage{alphabeta}



% Berthold udc
%\usepackage{qtree}
%\usepackage{rtrees}

\usepackage{pst-node}

  %add all your local new commands to this file

\makeatletter
\def\blx@maxline{77}
\makeatother


\newcommand{\page}{}



\newcommand{\todostefan}[1]{\todo[color=orange!80]{\footnotesize #1}\xspace}
\newcommand{\todosatz}[1]{\todo[color=red!40]{\footnotesize #1}\xspace}

\newcommand{\inlinetodostefan}[1]{\todo[color=green!40,inline]{\footnotesize #1}\xspace}


\newcommand{\spacebr}{\hspaceThis{[}}

\newcommand{\danish}{\jambox{(\ili{Danish})}}
\newcommand{\english}{\jambox{(\ili{English})}}
\newcommand{\german}{\jambox{(\ili{German})}}
\newcommand{\yiddish}{\jambox{(\ili{Yiddish})}}
\newcommand{\welsh}{\jambox{(\ili{Welsh})}}

% Cite and cross-reference other chapters
\newcommand{\crossrefchaptert}[2][]{\citet*[#1]{chapters/#2}, Chapter~\ref{chap-#2} of this volume} 
\newcommand{\crossrefchapterp}[2][]{(\citealp*[#1][]{chapters/#2}, Chapter~\ref{chap-#2} of this volume)}
% example of optional argument:
% \crossrefchapterp[for something, see:]{name}
% gives: (for something, see: Author 2018, Chapter~X of this volume)

\let\crossrefchapterw\crossrefchaptert



% Davis Koenig

\let\ig=\textsc
\let\tc=\textcolor

% evolution, Flickinger, Pollard, Wasow

\let\citeNP\citet

% Adam P

%\newcommand{\toappear}{Forthcoming}
\newcommand{\pg}[1]{p.#1}
\renewcommand{\implies}{\rightarrow}

\newcommand*{\rref}[1]{(\ref{#1})}
\newcommand*{\aref}[1]{(\ref{#1}a)}
\newcommand*{\bref}[1]{(\ref{#1}b)}
\newcommand*{\cref}[1]{(\ref{#1}c)}

\newcommand{\msadam}{.}
\newcommand{\morsyn}[1]{\textsc{#1}}

\newcommand{\nom}{\morsyn{nom}}
\newcommand{\acc}{\morsyn{acc}}
\newcommand{\dat}{\morsyn{dat}}
\newcommand{\gen}{\morsyn{gen}}
\newcommand{\ins}{\morsyn{ins}}
\newcommand{\loc}{\morsyn{loc}}
\newcommand{\voc}{\morsyn{voc}}
\newcommand{\ill}{\morsyn{ill}}
\renewcommand{\inf}{\morsyn{inf}}
\newcommand{\passprc}{\morsyn{passp}}

%\newcommand{\Nom}{\msadam\nom}
%\newcommand{\Acc}{\msadam\acc}
%\newcommand{\Dat}{\msadam\dat}
%\newcommand{\Gen}{\msadam\gen}
\newcommand{\Ins}{\msadam\ins}
\newcommand{\Loc}{\msadam\loc}
\newcommand{\Voc}{\msadam\voc}
\newcommand{\Ill}{\msadam\ill}
\newcommand{\INF}{\msadam\inf}
\newcommand{\PassP}{\msadam\passprc}

\newcommand{\Aux}{\textsc{aux}}

\newcommand{\princ}[1]{\textnormal{\textsc{#1}}} % for constraint names
\newcommand{\notion}[1]{\emph{#1}}
\renewcommand{\path}[1]{\textnormal{\textsc{#1}}}
\newcommand{\ftype}[1]{\textit{#1}}
\newcommand{\fftype}[1]{{\scriptsize\textit{#1}}}
\newcommand{\la}{$\langle$}
\newcommand{\ra}{$\rangle$}
%\newcommand{\argst}{\path{arg-st}}
\newcommand{\phtm}[1]{\setbox0=\hbox{#1}\hspace{\wd0}}
\newcommand{\prep}[1]{\setbox0=\hbox{#1}\hspace{-1\wd0}#1}

%%%%%%%%%%%%%%%%%%%%%%%%%%%%%%%%%%%%%%%%%%%%%%%%%%%%%%%%%%%%%%%%%%%%%%%%%%%

% FROM FS.STY:

%%%
%%% Feature structures
%%%

% \fs         To print a feature structure by itself, type for example
%             \fs{case:nom \\ person:P}
%             or (better, for true italics),
%             \fs{\it case:nom \\ \it person:P}
%
% \lfs        To print the same feature structure with the category
%             label N at the top, type:
%             \lfs{N}{\it case:nom \\ \it person:P}

%    Modified 1990 Dec 5 so that features are left aligned.
\newcommand{\fs}[1]%
{\mbox{\small%
$
\!
\left[
  \!\!
  \begin{tabular}{l}
    #1
  \end{tabular}
  \!\!
\right]
\!
$}}

%     Modified 1990 Dec 5 so that features are left aligned.
%\newcommand{\lfs}[2]
%   {
%     \mbox{$
%           \!\!
%           \begin{tabular}{c}
%           \it #1
%           \\
%           \mbox{\small%
%                 $
%                 \left[
%                 \!\!
%                 \it
%                 \begin{tabular}{l}
%                 #2
%                 \end{tabular}
%                 \!\!
%                 \right]
%                 $}
%           \end{tabular}
%           \!\!
%           $}
%   }

\newcommand{\ft}[2]{\path{#1}\hspace{1ex}\ftype{#2}}
\newcommand{\fsl}[2]{\fs{{\fftype{#1}} \\ #2}}

\newcommand{\fslt}[2]
 {\fst{
       {\fftype{#1}} \\
       #2 
     }
 }

\newcommand{\fsltt}[2]
 {\fstt{
       {\fftype{#1}} \\
       #2 
     }
 }

\newcommand{\fslttt}[2]
 {\fsttt{
       {\fftype{#1}} \\
       #2 
     }
 }


% jak \ft, \fs i \fsl tylko nieco ciasniejsze

\newcommand{\ftt}[2]
% {{\sc #1}\/{\rm #2}}
 {\textsc{#1}\/{\rm #2}}

\newcommand{\fst}[1]
  {
    \mbox{\small%
          $
          \left[
          \!\!\!
%          \sc
          \begin{tabular}{l} #1
          \end{tabular}
          \!\!\!\!\!\!\!
          \right]
          $
          }
   }

%\newcommand{\fslt}[2]
% {\fst{#2\\
%       {\scriptsize\it #1}
%      }
% }


% superciasne

\newcommand{\fstt}[1]
  {
    \mbox{\small%
          $
          \left[
          \!\!\!
%          \sc
          \begin{tabular}{l} #1
          \end{tabular}
          \!\!\!\!\!\!\!\!\!\!\!
          \right]
          $
          }
   }

%\newcommand{\fsltt}[2]
% {\fstt{#2\\
%       {\scriptsize\it #1}
%      }
% }

\newcommand{\fsttt}[1]
  {
    \mbox{\small%
          $
          \left[
          \!\!\!
%          \sc
          \begin{tabular}{l} #1
          \end{tabular}
          \!\!\!\!\!\!\!\!\!\!\!\!\!\!\!\!
          \right]
          $
          }
   }



% %add all your local new commands to this file

% \newcommand{\smiley}{:)}

% you are not supposed to mess with hardcore stuff, St.Mü. 22.08.2018
%% \renewbibmacro*{index:name}[5]{%
%%   \usebibmacro{index:entry}{#1}
%%     {\iffieldundef{usera}{}{\thefield{usera}\actualoperator}\mkbibindexname{#2}{#3}{#4}{#5}}}

% % \newcommand{\noop}[1]{}



% Rui

\newcommand{\spc}[0]{\hspace{-1pt}\underline{\hspace{6pt}}\,}
\newcommand{\spcs}[0]{\hspace{-1pt}\underline{\hspace{6pt}}\,\,}
\newcommand{\bad}[1]{\leavevmode\llap{#1}}
\newcommand{\COMMENT}[1]{}


% Andy Lücking gesture.tex
\newcommand{\Pointing}{\ding{43}}
% Giotto: "Meeting of Joachim and Anne at the Golden Gate" - 1305-10 
\definecolor{GoldenGate1}{rgb}{.13,.09,.13} % Dress of woman in black
\definecolor{GoldenGate2}{rgb}{.94,.94,.91} % Bridge
\definecolor{GoldenGate3}{rgb}{.06,.09,.22} % Blue sky
\definecolor{GoldenGate4}{rgb}{.94,.91,.87} % Dress of woman with shawl
\definecolor{GoldenGate5}{rgb}{.52,.26,.26} % Joachim's robe
\definecolor{GoldenGate6}{rgb}{.65,.35,.16} % Anne's robe
\definecolor{GoldenGate7}{rgb}{.91,.84,.42} % Joachim's halo

\makeatletter
\newcommand{\@Depth}{1} % x-dimension, to front
\newcommand{\@Height}{1} % z-dimension, up
\newcommand{\@Width}{1} % y-dimension, rightwards
%\GGS{<x-start>}{<y-start>}{<z-top>}{<z-bottom>}{<Farbe>}{<x-width>}{<y-depth>}{<opacity>}
\newcommand{\GGS}[9][]{%
\coordinate (O) at (#2-1,#3-1,#5);
\coordinate (A) at (#2-1,#3-1+#7,#5);
\coordinate (B) at (#2-1,#3-1+#7,#4);
\coordinate (C) at (#2-1,#3-1,#4);
\coordinate (D) at (#2-1+#8,#3-1,#5);
\coordinate (E) at (#2-1+#8,#3-1+#7,#5);
\coordinate (F) at (#2-1+#8,#3-1+#7,#4);
\coordinate (G) at (#2-1+#8,#3-1,#4);
\draw[draw=black, fill=#6, fill opacity=#9] (D) -- (E) -- (F) -- (G) -- cycle;% Front
\draw[draw=black, fill=#6, fill opacity=#9] (C) -- (B) -- (F) -- (G) -- cycle;% Top
\draw[draw=black, fill=#6, fill opacity=#9] (A) -- (B) -- (F) -- (E) -- cycle;% Right
}
\makeatother


% pragmatics
\newcommand{\speaking}[1]{\eqparbox{name}{\textsc{\lowercase{#1}\space}}}
\newcommand{\name}[1]{\eqparbox{name}{\textsc{\lowercase{#1}}}}
\newcommand{\HPSGTTR}{HPSG$_{\text{TTR}}$\xspace}

\newcommand{\ttrtype}[1]{\textit{#1}}
% \newcommand{\avmel}{\q<\quad\q>} %% shortcut for empty lists in AVM
\newcommand{\ttrmerge}{\ensuremath{\wedge_{\textit{merge}}}}
\newcommand{\Cat}[2][0.1pt]{%
  \begin{scope}[y=#1,x=#1,yscale=-1, inner sep=0pt, outer sep=0pt]
   \path[fill=#2,line join=miter,line cap=butt,even odd rule,line width=0.8pt]
  (151.3490,307.2045) -- (264.3490,307.2045) .. controls (264.3490,291.1410) and (263.2021,287.9545) .. (236.5990,287.9545) .. controls (240.8490,275.2045) and (258.1242,244.3581) .. (267.7240,244.3581) .. controls (276.2171,244.3581) and (286.3490,244.8259) .. (286.3490,264.2045) .. controls (286.3490,286.2045) and (323.3717,321.6755) .. (332.3490,307.2045) .. controls (345.7277,285.6390) and (309.3490,292.2151) .. (309.3490,240.2046) .. controls (309.3490,169.0514) and (350.8742,179.1807) .. (350.8742,139.2046) .. controls (350.8742,119.2045) and (345.3490,116.5037) .. (345.3490,102.2045) .. controls (345.3490,83.3070) and (361.9972,84.4036) .. (358.7581,68.7349) .. controls (356.5206,57.9117) and (354.7696,49.2320) .. (353.4652,36.1439) .. controls (352.5396,26.8573) and (352.2445,16.9594) .. (342.5985,17.3574) .. controls (331.2650,17.8250) and (326.9655,37.7742) .. (309.3490,39.2045) .. controls (291.7685,40.6320) and (276.7783,24.2380) .. (269.9740,26.5795) .. controls (263.2271,28.9013) and (265.3490,47.2045) .. (269.3490,60.2045) .. controls (275.6359,80.6368) and (289.3490,107.2045) .. (264.3490,111.2045) .. controls (239.3490,115.2045) and (196.3490,119.2045) .. (165.3490,160.2046) .. controls (134.3490,201.2046) and (135.4934,249.3212) .. (123.3490,264.2045) .. controls (82.5907,314.1553) and (40.8239,293.6463) .. (40.8239,335.2045) .. controls (40.8239,353.8102) and (72.3490,367.2045) .. (77.3490,361.2045) .. controls (82.3490,355.2045) and (34.8638,337.3259) .. (87.9955,316.2045) .. controls (133.3871,298.1601) and   (137.4391,294.4766) .. (151.3490,307.2045) -- cycle;
\end{scope}%
}


% KdK
\newcommand{\smiley}{:)}

\renewbibmacro*{index:name}[5]{%
  \usebibmacro{index:entry}{#1}
    {\iffieldundef{usera}{}{\thefield{usera}\actualoperator}\mkbibindexname{#2}{#3}{#4}{#5}}}

% \newcommand{\noop}[1]{}

% chngcntr.sty otherwise gives error that these are already defined
%\let\counterwithin\relax
%\let\counterwithout\relax

% the space of a left bracket for glossings
\newcommand{\LB}{\hspaceThis{[}}

\newcommand{\LF}{\mbox{$[\![$}}

\newcommand{\RF}{\mbox{$]\!]_F$}}

\newcommand{\RT}{\mbox{$]\!]_T$}}





% Manfred's

\newcommand{\kommentar}[1]{}

\newcommand{\bsp}[1]{\emph{#1}}
\newcommand{\bspT}[2]{\bsp{#1} `#2'}
\newcommand{\bspTL}[3]{\bsp{#1} (lit.: #2) `#3'}

\newcommand{\noidi}{§}

\newcommand{\refer}[1]{(\ref{#1})}

%\newcommand{\avmtype}[1]{\multicolumn{2}{l}{\type{#1}}}
\newcommand{\attr}[1]{\textsc{#1}}

\newcommand{\srdefault}{\mbox{\begin{tabular}{c}{\large <}\\[-1.5ex]$\sqcap$\end{tabular}}}

%% \newcommand{\myappcolumn}[2]{
%% \begin{minipage}[t]{#1}#2\end{minipage}
%% }

%% \newcommand{\appc}[1]{\myappcolumn{3.7cm}{#1}}


% Jong-Bok


% clean that up and do not use \def (killing other stuff defined before)
%\if 0
\def\DEL{\textsc{del}}
\def\del{\textsc{del}}

\def\conn{\textsc{conn}}
\def\CONN{\textsc{conn}}
\def\CONJ{\textsc{conj}}
\def\LITE{\textsc{lex}}
\def\lite{\textsc{lex}}
\def\HON{\textsc{hon}}

\def\CAUS{\textsc{caus}}
\def\PASS{\textsc{pass}}
\def\NPST{\textsc{npst}}
\def\COND{\textsc{cond}}



\def\hd-lite{\textsc{head-lex construction}}
\def\NFORM{\textsc{nform}}

\def\RELS{\textsc{rels}}
\def\TENSE{\textsc{tense}}


%\def\ARG{\textsc{arg}}
\def\ARGs{\textsc{arg0}}
\def\ARGa{\textsc{arg}}
\def\ARGb{\textsc{arg2}}
\def\TPC{\textsc{top}}
\def\PROG{\textsc{prog}}

\def\pst{\textsc{pst}}
\def\PAST{\textsc{pst}}
\def\DAT{\textsc{dat}}
\def\CONJ{\textsc{conj}}
\def\nominal{\textsc{nominal}}
\def\NOMINAL{\textsc{nominal}}
\def\VAL{\textsc{val}}
\def\val{\textsc{val}}
\def\MODE{\textsc{mode}}
\def\RESTR{\textsc{restr}}
\def\SIT{\textsc{sit}}
\def\ARG{\textsc{arg}}
\def\RELN{\textsc{rel}}
\def\REL{\textsc{rel}}
\def\RELS{\textsc{rels}}
\def\arg-st{\textsc{arg-st}}
\def\xdel{\textsc{xdel}}
\def\zdel{\textsc{zdel}}
\def\sug{\textsc{sug}}
\def\IMP{\textsc{imp}}
\def\conn{\textsc{conn}}
\def\CONJ{\textsc{conj}}
\def\HON{\textsc{hon}}
\def\BN{\textsc{bn}}
\def\bn{\textsc{bn}}
\def\pres{\textsc{pres}}
\def\PRES{\textsc{pres}}
\def\prs{\textsc{pres}}
\def\PRS{\textsc{pres}}
\def\agt{\textsc{agt}}
\def\DEL{\textsc{del}}
\def\PRED{\textsc{pred}}
\def\AGENT{\textsc{agent}}
\def\THEME{\textsc{theme}}
\def\AUX{\textsc{aux}}
\def\THEME{\textsc{theme}}
\def\PL{\textsc{pl}}
\def\SRC{\textsc{src}}
\def\src{\textsc{src}}
\def\FORM{\textsc{form}}
\def\form{\textsc{form}}
\def\GCASE{\textsc{gcase}}
\def\gcase{\textsc{gcase}}
\def\SCASE{\textsc{scase}}
\def\PHON{\textsc{phon}}
\def\SS{\textsc{ss}}
\def\SYN{\textsc{syn}}
\def\LOC{\textsc{loc}}
\def\MOD{\textsc{mod}}
\def\INV{\textsc{inv}}
\def\L{\textsc{l}}
\def\CASE{\textsc{case}}
\def\SPR{\textsc{spr}}
\def\COMPS{\textsc{comps}}
%\def\comps{\textsc{comps}}
\def\SEM{\textsc{sem}}
\def\CONT{\textsc{cont}}
\def\SUBCAT{\textsc{subcat}}
\def\CAT{\textsc{cat}}
\def\C{\textsc{c}}
\def\SUBJ{\textsc{subj}}
\def\subj{\textsc{subj}}
\def\SLASH{\textsc{slash}}
\def\LOCAL{\textsc{local}}
\def\ARG-ST{\textsc{arg-st}}
\def\AGR{\textsc{agr}}
\def\PER{\textsc{per}}
\def\NUM{\textsc{num}}
\def\IND{\textsc{ind}}
\def\VFORM{\textsc{vform}}
\def\PFORM{\textsc{pform}}
\def\decl{\textsc{decl}}
\def\loc{\textsc{loc   }}
% \def\   {\textsc{  }}

\def\NEG{\textsc{neg}}
\def\FRAMES{\textsc{frames}}
\def\REFL{\textsc{refl}}

\def\MKG{\textsc{mkg}}

\def\BN{\textsc{bn}}
\def\HD{\textsc{hd}}
\def\NP{\textsc{np}}
\def\PF{\textsc{pf}}
\def\PL{\textsc{pl}}
\def\PP{\textsc{pp}}
\def\SS{\textsc{ss}}
\def\VF{\textsc{vf}}
\def\VP{\textsc{vp}}
\def\bn{\textsc{bn}}
\def\cl{\textsc{cl}}
\def\pl{\textsc{pl}}
\def\Wh{\ital{Wh}}
\def\ng{\textsc{neg}}
\def\wh{\ital{wh}}
\def\ACC{\textsc{acc}}
\def\AGR{\textsc{agr}}
\def\AGT{\textsc{agt}}
\def\ARC{\textsc{arc}}
\def\ARG{\textsc{arg}}
\def\ARP{\textsc{arc}}
\def\AUX{\textsc{aux}}
\def\CAT{\textsc{cat}}
\def\COP{\textsc{cop}}
\def\DAT{\textsc{dat}}
\def\DEF{\textsc{def}}
\def\DEL{\textsc{del}}
\def\DOM{\textsc{dom}}
\def\DTR{\textsc{dtr}}
\def\FUT{\textsc{fut}}
\def\GAP{\textsc{gap}}
\def\GEN{\textsc{gen}}
\def\HON{\textsc{hon}}
\def\IMP{\textsc{imp}}
\def\IND{\textsc{ind}}
\def\INV{\textsc{inv}}
\def\LEX{\textsc{lex}}
\def\Lex{\textsc{lex}}
\def\LOC{\textsc{loc}}
\def\MOD{\textsc{mod}}
\def\MRK{{\nr MRK}}
\def\NEG{\textsc{neg}}
\def\NEW{\textsc{new}}
\def\NOM{\textsc{nom}}
\def\NUM{\textsc{num}}
\def\PER{\textsc{per}}
\def\PST{\textsc{pst}}
\def\QUE{\textsc{que}}
\def\REL{\textsc{rel}}
\def\SEL{\textsc{sel}}
\def\SEM{\textsc{sem}}
\def\SIT{\textsc{arg0}}
\def\SPR{\textsc{spr}}
\def\SRC{\textsc{src}}
\def\SUG{\textsc{sug}}
\def\SYN{\textsc{syn}}
\def\TPC{\textsc{top}}
\def\VAL{\textsc{val}}
\def\acc{\textsc{acc}}
\def\agt{\textsc{agt}}
\def\cop{\textsc{cop}}
\def\dat{\textsc{dat}}
\def\foc{\textsc{focus}}
\def\FOC{\textsc{focus}}
\def\fut{\textsc{fut}}
\def\hon{\textsc{hon}}
\def\imp{\textsc{imp}}
\def\kes{\textsc{kes}}
\def\lex{\textsc{lex}}
\def\loc{\textsc{loc}}
\def\mrk{{\nr MRK}}
\def\nom{\textsc{nom}}
\def\num{\textsc{num}}
\def\plu{\textsc{plu}}
\def\pne{\textsc{pne}}
\def\pst{\textsc{pst}}
\def\pur{\textsc{pur}}
\def\que{\textsc{que}}
\def\src{\textsc{src}}
\def\sug{\textsc{sug}}
\def\tpc{\textsc{top}}
\def\utt{\textsc{utt}}
\def\val{\textsc{val}}
\def\LITE{\textsc{lex}}
\def\PAST{\textsc{pst}}
\def\POSP{\textsc{pos}}
\def\PRS{\textsc{pres}}
\def\mod{\textsc{mod}}%
\def\newuse{{`kes'}}
\def\posp{\textsc{pos}}
\def\prs{\textsc{pres}}
\def\psp{{\it en\/}}
\def\skes{\textsc{kes}}
\def\CASE{\textsc{case}}
\def\CASE{\textsc{case}}
\def\COMP{\textsc{comp}}
\def\CONJ{\textsc{conj}}
\def\CONN{\textsc{conn}}
\def\CONT{\textsc{cont}}
\def\DECL{\textsc{decl}}
\def\FOCUS{\textsc{focus}}
\def\FORM{\textsc{form}}
\def\FREL{\textsc{frel}}
\def\GOAL{\textsc{goal}}
\def\HEAD{\textsc{head}}
\def\INDEX{\textsc{ind}}
\def\INST{\textsc{inst}}
\def\MODE{\textsc{mode}}
\def\MOOD{\textsc{mood}}
\def\NMLZ{\textsc{nmlz}}
\def\PHON{\textsc{phon}}
\def\PRED{\textsc{pred}}
%\def\PRES{\textsc{pres}}
\def\PROM{\textsc{prom}}
\def\RELN{\textsc{pred}}
\def\RELS{\textsc{rels}}
\def\STEM{\textsc{stem}}
\def\SUBJ{\textsc{subj}}
\def\XARG{\textsc{xarg}}
\def\bse{{\it bse\/}}
\def\case{\textsc{case}}
\def\caus{\textsc{caus}}
\def\comp{\textsc{comp}}
\def\conj{\textsc{conj}}
\def\conn{\textsc{conn}}
\def\decl{\textsc{decl}}
\def\fin{{\it fin\/}}
\def\form{\textsc{form}}
\def\gend{\textsc{gend}}
\def\inf{{\it inf\/}}
\def\mood{\textsc{mood}}
\def\nmlz{\textsc{nmlz}}
\def\pass{\textsc{pass}}
\def\past{\textsc{past}}
\def\perf{\textsc{perf}}
\def\pln{{\it pln\/}}
\def\pred{\textsc{pred}}


%\def\pres{\textsc{pres}}
\def\proc{\textsc{proc}}
\def\nonfin{{\it nonfin\/}}
\def\AGENT{\textsc{agent}}
\def\CFORM{\textsc{cform}}
%\def\COMPS{\textsc{comps}}
\def\COORD{\textsc{coord}}
\def\COUNT{\textsc{count}}
\def\EXTRA{\textsc{extra}}
\def\GCASE{\textsc{gcase}}
\def\GIVEN{\textsc{given}}
\def\LOCAL{\textsc{local}}
\def\NFORM{\textsc{nform}}
\def\PFORM{\textsc{pform}}
\def\SCASE{\textsc{scase}}
\def\SLASH{\textsc{slash}}
\def\SLASH{\textsc{slash}}
\def\THEME{\textsc{theme}}
\def\TOPIC{\textsc{topic}}
\def\VFORM{\textsc{vform}}
\def\cause{\textsc{cause}}
%\def\comps{\textsc{comps}}
\def\gcase{\textsc{gcase}}
\def\itkes{{\it kes\/}}
\def\pass{{\it pass\/}}
\def\vform{\textsc{vform}}
\def\CCONT{\textsc{c-cont}}
\def\GN{\textsc{given-new}}
\def\INFO{\textsc{info-st}}
\def\ARG-ST{\textsc{arg-st}}
\def\SUBCAT{\textsc{subcat}}
\def\SYNSEM{\textsc{synsem}}
\def\VERBAL{\textsc{verbal}}
\def\arg-st{\textsc{arg-st}}
\def\plain{{\it plain}\/}
\def\propos{\textsc{propos}}
\def\ADVERBIAL{\textsc{advl}}
\def\HIGHLIGHT{\textsc{prom}}
\def\NOMINAL{\textsc{nominal}}

\newenvironment{myavm}{\begingroup\avmvskip{.1ex}
  \selectfont\begin{avm}}%
{\end{avm}\endgroup\medskip}
\def\pfix{\vspace{-5pt}}


\def\jbsub#1{\lower4pt\hbox{\small #1}}
\def\jbssub#1{\lower4pt\hbox{\small #1}}
\def\jbtr{\underbar{\ \ \ }\ }


%\fi

  %% hyphenation points for line breaks
%% Normally, automatic hyphenation in LaTeX is very good
%% If a word is mis-hyphenated, add it to this file
%%
%% add information to TeX file before \begin{document} with:
%% %% hyphenation points for line breaks
%% Normally, automatic hyphenation in LaTeX is very good
%% If a word is mis-hyphenated, add it to this file
%%
%% add information to TeX file before \begin{document} with:
%% %% hyphenation points for line breaks
%% Normally, automatic hyphenation in LaTeX is very good
%% If a word is mis-hyphenated, add it to this file
%%
%% add information to TeX file before \begin{document} with:
%% \include{localhyphenation}
\hyphenation{
A-la-hver-dzhie-va
anaph-o-ra
affri-ca-te
affri-ca-tes
Atha-bas-kan
Chi-che-ŵa
com-ple-ments
Da-ge-stan
Dor-drecht
er-klä-ren-de
Ginz-burg
Gro-ning-en
Jon-a-than
Ka-tho-lie-ke
Ko-bon
krie-gen
Le-Sourd
moth-er
Mül-ler
Nie-mey-er
Prze-piór-kow-ski
phe-nom-e-non
re-nowned
Rie-he-mann
un-bound-ed
}

% why has "erklärende" be listed here? I specified langid in bibtex item. Something is still not working with hyphenation.


% to do: check
%  Alahverdzhieva

\hyphenation{
A-la-hver-dzhie-va
anaph-o-ra
affri-ca-te
affri-ca-tes
Atha-bas-kan
Chi-che-ŵa
com-ple-ments
Da-ge-stan
Dor-drecht
er-klä-ren-de
Ginz-burg
Gro-ning-en
Jon-a-than
Ka-tho-lie-ke
Ko-bon
krie-gen
Le-Sourd
moth-er
Mül-ler
Nie-mey-er
Prze-piór-kow-ski
phe-nom-e-non
re-nowned
Rie-he-mann
un-bound-ed
}

% why has "erklärende" be listed here? I specified langid in bibtex item. Something is still not working with hyphenation.


% to do: check
%  Alahverdzhieva

\hyphenation{
A-la-hver-dzhie-va
anaph-o-ra
affri-ca-te
affri-ca-tes
Atha-bas-kan
Chi-che-ŵa
com-ple-ments
Da-ge-stan
Dor-drecht
er-klä-ren-de
Ginz-burg
Gro-ning-en
Jon-a-than
Ka-tho-lie-ke
Ko-bon
krie-gen
Le-Sourd
moth-er
Mül-ler
Nie-mey-er
Prze-piór-kow-ski
phe-nom-e-non
re-nowned
Rie-he-mann
un-bound-ed
}

% why has "erklärende" be listed here? I specified langid in bibtex item. Something is still not working with hyphenation.


% to do: check
%  Alahverdzhieva

  \bibliography{../Bibliographies/stmue,
                ../localbibliography,
../Bibliographies/formal-background,
../Bibliographies/understudied-languages,
../Bibliographies/phonology,
../Bibliographies/case,
../Bibliographies/evolution,
../Bibliographies/agreement,
../Bibliographies/lexicon,
../Bibliographies/np,
../Bibliographies/negation,
../Bibliographies/argst,
../Bibliographies/binding,
../Bibliographies/complex-predicates,
../Bibliographies/coordination,
../Bibliographies/relative-clauses,
../Bibliographies/udc,
../Bibliographies/processing,
../Bibliographies/cl,
../Bibliographies/dg,
../Bibliographies/islands,
../Bibliographies/diachronic,
../Bibliographies/gesture,
../Bibliographies/semantics,
../Bibliographies/pragmatics,
../Bibliographies/information-structure,
../Bibliographies/idioms,
../Bibliographies/cg,
../Bibliographies/udc}

  \togglepaper[14]
}{}



\title{Anaphoric Binding} 
\author{%
Stefan Müller\affiliation{Humboldt-Universität zu Berlin} \lastand António Branco\affiliation{University of Lisbon}
}
% \chapterDOI{} %will be filled in at production

% \epigram{}

\abstract{
This chapter is an introduction into the Binding Theory assumed within HPSG. While it was inspired
by work on Government \& Binding in the beginning, it turned out that reference to tree structures
are not necessary and that relations that are required for interpreting the reference of personal pronouns
and reflexives can be established with respect to lexical properties of heads namely the argument
structure list, a list containing descriptions of arguments of a head.
}


\begin{document}
\maketitle

\label{chap-binding}

\section{Introduction} 

Binding Theories deal with questions of coreference and correspondence of forms. For example, the
reflexives in (\mex{1}) have to refer to the referent the NP in the same clause refers to and they
have to have the same gender as the NP they are coreferent with:
\eal
\ex[]{
Peter$_i$ thinks that Mary$_j$ likes herself$_{*i/j/*k}$.
}
\ex[*]{
Peter$_i$ thinks that Mary$_j$ likes himself$_{*i/*j/*k}$.
}
\ex[*]{
Mary$_i$ thinks that Peter$_j$ likes herself$_{*i/*j/*k}$.
}
\ex[]{
Mary$_i$ thinks that Peter$_j$ likes himself$_{*i/j/*k}$.
}
\zl
The indices show what bindings are possible and which ones are ruled out. For example, in
(\mex{0}a), \emph{herself} cannot refer to \emph{Peter}, it can refer to \emph{Mary} and it cannot
refer to some discourse referent that is not mentioned in the sentence. Coreference of
\emph{himself} and \emph{Mary} is ruled out in (\mex{0}b) since \emph{himself} has an incompatible gender.

Personal pronouns can not refer to an antecedent within the same clause:
\eal
\ex[]{
Peter$_i$ thinks Mary$_j$ that likes her$_{*i/*j/k}$.
}
\ex[]{
Peter$_i$ thinks Mary$_j$ that likes him$_{i/*j/k}$.
}
\ex[]{
Mary$_i$ thinks Peter$_j$ that likes her$_{i/*j/k}$.
}
\ex[]{
Mary$_i$ thinks Peter$_j$ that likes him$_{*i/*j/k}$.
}
\zl
As the examples show, the pronouns \emph{her} and \emph{him} cannot be coreferent with the subject
of \emph{likes}. If a speaker wants to express coreference he or she has to use a reflexive pronoun
as in (\mex{-1}). 

Interestingly, the binding of pronouns is less restricted than the one of reflexives, but this does
not mean that anything goes. For example, a pronoun cannot bind a full referential NP if the NP is
embedded in a clause and the pronoun is in the matrix clause:
\eal
\ex[]{
He$_{*i/*j/k}$ thinks that Mary$_i$ likes Peter$_j$.
}
\ex[]{
He$_{*i/*j/k}$ thinks that Peter$_i$ likes Mary$_j$.
}
\zl  

The sentences discussed so far can be assigned a structure like the one in Figure~\ref{fig-binding-gb}.
\begin{figure}
\begin{forest}
sm edges without translation
[S
  [NP [John\\John\\he]]
  [VP
    [V [thinks\\thinks\\thinks]]
    [CP 
      [C [that\\that\\that]]
      [S
        [NP [Paul\\Paul\\Mary]]
        [VP
         [V [likes\\likes\\likes]]
         [NP [him\\himself\\Peter]]]]]]]
\end{forest}

\caption{\label{fig-binding-gb}Tree configuration of examples for binding}
\end{figure}
\citet{Chomsky81a,Chomsky86a} suggested accounting for the facts by referring to the hierarchical
structure in Figure~\ref{fig-binding-gb}. He uses the notion of c(onstituent)-command going back to
work by \citegen{Reinhart76a-u}. \isi{c-command} is a relation that holds between nodes in a
tree. Accoring to one definition, a node Y is said to c-command another node Z, Y and Z
are sisters or if a sister of Y dominates Z.\footnote{
``Node A c(onstituent)-commands node B if neither A nor B dominates the other and the first
  branching node which dominates A dominates B.'' \citet[\page 32]{Reinhart76a-u}

\citet{Chomsky86a} uses another definition that allows one to go up to the next maximal projection
dominating A. As of 25/02/2020 the English and German Wikipedia pages for c-command have two
conflicting definitions of c-command. The English version follows \citet{SKS2013a-u}, whose
definition excludes c-command between sisters: ``Node X c-commands node Y if a sister of X dominates Y.''
}\todostefan{add page number}

To take an example, the NP node of
\emph{John} c-commands all other nodes dominated by S. The V of \emph{thinks} c-commands everything
within the CP including the CP node, the C of \emph{that} c-commands all nodes in S including also S
and so on. The CP c-commands the \emph{think}-V, and the \emph{likes him}-VP c-commands the
\emph{Paul}-NP. Per definition, a Y binds Z just in case Y and Z are coindexed and Y c-commands
Z. One precondition for being coindexed (in English) is that the person, number, and gender features
of the involved items are compatible.

Now, the goal is to find restrictions that ensure that reflexives are bound locally, personal
pronouns are not bound locally and that referential expressions like proper names and full NPs do
not refer to pronouns or fully referential expressions. The conditions that were developed for
Binding Theory are complex. They also account for the binding of traces that are the result of
moving elements by transformations. While it is elegant to subsume the filler-gap relations under a
general Binding Theory, proponents of HPSG think that coreferential semantic indices and filler-gap
dependencies are crucially different. The places of occurrence of gaps (if they are assumed at all)
is restricted by other components of the theory. For an overview of the treatment of nonlocal
dependencies in HPSG see \crossrefchapterw{udc}.

We will not go into the details of the Binding Theory in Mainstream Generative Grammar
(MGG)\footnote{
We follow \citet[\page 3]{CJ2005a} in using the term \emph{Mainstream Generative Grammar} when
referring to work in Government \& Binding \citep{Chomsky81a} or Minimalism \citep{Chomsky95a-u}.}, but we
give a verbatim description of the ABC of Binding Theory for overt elements. Chomsky distinguishes between
so-called R-expressions (referential expressions like proper nouns or full NPs/DPs), personal
pronouns and reflexives and reciprocals. The latter two are subsumed under the term anaphor. 
Principle A says that an anaphor must be bound within the least maximal projection containing a
subject. Principle B says that a pronoun that is governed by some element G has to be A-free in the
least maximal projection M containing G and a subject. Principle C says that a referential
expression Z heading its own chain has to be A-free in the domain of the head of the chain of Z.


\section{A non-configural Binding Theory}

HPSG's Binding Theory differs from GB's Binding Theory in referring less to tree structures but
rather to the notion of obliqueness of arguments of ahead. The arguments of a head are represented
in a list called the argument structure list. The list is the value of the feature \argst. The
\argst elements are descriptions of arguments of a head containing syntactic and semantic properties
of the selected arguments but not their daughters. So they are not complete signs but \type{synsem}
objects. See \crossrefchaptert{properties} for more on the general setup of HPSG
theories. The list elements are ordered with respect to their obliqueness, the least oblique element
being the first element:

\ea
\label{def-obliqueness-hierarchy}
\oneline{%
\is{object!indirect|(}\is{object!direct|(}\is{subject}%
\begin{tabular}[t]{@{}l@{\hspace{1ex}}l@{\hspace{1ex}}l@{\hspace{1ex}}l@{\hspace{1ex}}l@{\hspace{1ex}}l@{}}
SUBJECT $>$ & DIRECT $>$ & INDIRECT $>$ & OBLIQUES $>$ & GENITIVES $>$  & OBJECTS OF\\
             & OBJECT      & OBJECT        &               &                 & COMPARISON 
\end{tabular}%
}
\z
This order was suggested by \citet{KC77a}. It corresponds to the level of syntactic activity of grammatical functions\is{grammatical function}. Elements
higher in this hierarchy are less oblique and can participate more easily in syntactic constructions, like for instance,
reductions in coordinated structures\is{coordination} \citep[\page 15]{Klein85}\iaright{Klein, Wolfgang},
topic drop\is{topic drop}\is{Vorfeldellipse@{\it Vorfeldellipse}} \citep{Fries88b}\ia{Fries},
non-matching free relative clauses\is{relative clause!free} 
\citep{Bausewein90,Pittner95b,Mueller99b}, 
passive\is{passive} and relativization\is{relativization} \citep{KC77a}, and
depictive predicates\is{predicate!depictive secondary} \citep{Mueller2008a}.
In addition, \citet{Pullum77a} argued that this hierarchy plays a role in constituent order\is{scrambling}\is{serialization} (but see Section~\ref{sec-argst-order}.)
And, of course, it was claimed to have play an important role in Binding Theory\is{Binding Theory} 
(Grewendorf, \citeyear[\page 160]{Grewendorf85a} \citeyear[\page 60]{Grewendorf88a}; \citealp[Chapter~6]{ps2}\ia{Pollard}\ia{Sag}).


Figure~\ref{fig-binding-argst} shows a version of Figure~\ref{fig-binding-gb} including \argst information.
\begin{figure}
\begin{forest}
sm edges without translation
[S
  [\ibox{1} NP$_i$ [John\\John\\he]]
  [VP
    [V \sliste{ \ibox{1} NP$_i$, \ibox{2} CP } [thinks\\thinks\\thinks]]
    [\ibox{2} CP 
      [C [that\\that\\that]]
      [S
        [\ibox{3} NP$_j$ [Paul\\Paul\\Mary]]
        [VP
         [V \sliste{ \ibox{3} NP$_j$, \ibox{4} NP$_k$ } [likes\\likes\\likes]]
         [\ibox{4} NP$_k$ [him\\himself\\Peter]]]]]]]
\end{forest}

\caption{\label{fig-binding-argst}Tree configuration of examples for binding with \argst lists}
\end{figure}
The main points of HPSG's Binding Theory can be discussed with respect to this simple figure:
anaphors have to be bound locally. The definition of the domain of locality is rather simple. One
does not have to refer to tree configurations since all arguments of a head are represented locally
in a list. Simplifying a bit, reflexives and reciprocals may be bound to elements preceding them in
the \argstl and a pronoun like \emph{him} must not be bound by a preceding element in the same \argstl.

To be able to specif the conditions on binding of anaphors, pronouns and non-pronouns some further
definitions are necessary. The following definitions are definitions of local \emph{o-command}, \emph{o-command}
and \emph{o-bind}. The terms are reminiscent of \emph{c-command} and so on but we have an ``o''
rather than a ``c'' here, which is supposed to indicate the important role of the obliqueness
hierarchy. The definitions are as follows:

\ea
\label{def-local-o-command}
Let Y and Z be \type{synsem} objects with distinct \localvs, Y referential. Then Y \emph{locally
o-commands} Z just in case Y is less oblique than Z.
\z

\ea
\label{def-o-command}
Let Y and Z be \type{synsem} objects with distinct \localvs, Y referential. Then Y \emph{o-commands} Z just
in case Y locally o-commands X dominating Z.
\z

\ea
\label{def-o-bind}
Y (\emph{locally}) \emph{o-binds} Z just in case Y and Z are coindexed and Y (locally) o"=commands Z. If Z
is not (locally) o-bound, then it is said to be (\emph{locally}) \emph{o"=free}.
\z

(\ref{def-local-o-command}) says that an \argst element locally o-commands any other \argst element
further to the right of it. The condition of non-identity of the two elements under consideration in
(\ref{def-local-o-command}) and (\ref{def-o-command}) is necessary to deal with cases of raising, in
which one element may appear in various different \argstls. See Section~\ref{sec-binding-raising} below
and \crossrefchaptert{control-raising} for discussion of raising in HPSG. The condition that Y has
to be referential excludes expletive pronouns like \emph{it} in \emph{it rains} from entering
o-command relations. Such expletives are part of \argst and valence lists but they are entirely
irrelevant for Binding Theory, which is the reason for their exclusion in the definition.

The definition of o-command uses the relations of locally o-command and dominate. With respect to
Figure~\ref{fig-binding-argst}, we can say that NP$_i$ o-commands all nodes below the CP node since
NP$_i$ locally o-commands the CP and the CP node dominates everything below it. So NP$_i$ o-commands
C, NP$_j$, VP, V, and NP$_k$.

The definition of \emph{o-bind} in (\ref{def-o-bind}) says that two elements have to be coindexed
and there has to be a (local) o-command relation between them. The indices include person, number
and gender information (in English), so that \emph{Mary} can bind \emph{herself} but not
\emph{themselve} or \emph{himself}.

\begin{principle-break}[HPSG Binding Theory]
\begin{description}
\item [Prinzip A] A locally o-commanded anaphor must be locally o-bound.
\item [Prinzip B] A personal pronoun must be locally o-free.
\item [Prinzip C] A nonpronoun must be o-free.
\end{description}
\end{principle-break}

\noindent
Principle A accounts for the ungrammaticality of sentences like (\mex{1}):
\ea[*]{
Mary likes himself.
}
\z
Since both \emph{Mary} and \emph{himself} are members of the \argstl of \emph{likes}, there is an NP
that locally o-commands \emph{himself}. Therefore there should be a local o-binder. But since the
indices are incompatible because of incompatible gender values, \emph{Mary} cannot o-bind
\emph{himself}, \emph{himslef} is locally o-free and hence in conflict to Principle A.

Similarly, the binding in (\mex{1}) is excluded, since \emph{Mary} locally o-binds the pronoun \emph{her}
and hence Principle B is violated.
\ea[]{
Mary$_i$ likes her$_{*i}$.
}
\z

\noindent
Finally, Principle C accounts for the ungrammaticality of (\mex{1}):
\ea
He$_i$ thinks that Mary likes Peter$_{*i}$.
\z
Since \emph{he} and \emph{Peter} are coindexed and since \emph{he} o-commands \emph{Peter},
\emph{he} also o-binds \emph{Peter}. According to Principle C, this is forbidden and hence bindings
like the one in (\mex{0}) are ruled out.

This list-based Binding Theory seems very simple. So far we explained binding relations between
coarguments of a head where the coarguments are NPs or pronouns. But there are also prepositional
objects, which have an internal structure with the referential NPs embedded within a PP.

\citet[\page 246,255]{ps2} discuss examples like (\mex{1}):
\eal
\ex{
John$_i$ depends [on him$_{*i}$].
}
\ex{
\label{ex-mary-talked-to-john-about-himself}
Mary talked [to John$_i$] [about himself$_i$].
}
\zl
As noted by \citet[\page 246]{ps2}, the second example is a problem for the GB Binding Theory since
\emph{John} is inside the PP and does not c-command \emph{himself}. 
\begin{figure}
\begin{forest}
sm edges without translation
[S
  [\ibox{1} NP [Mary]]
  [VP
    [V \sliste{ \ibox{1}, \ibox{2}, \ibox{3} } [talked]]
    [\ibox{2} PP$_i$
       [P [to]]
       [NP$_i$ [John]]]
    [\ibox{3} PP$_i$
       [P [about]]
       [NP$_i$ [himself]]]]]
\end{forest}
\caption{Binding within prepositional objects poses a challenge for GB's Binding Theory}
\end{figure}
Examples involving case-marking
prepositions are no problem for HPSG however, since it is assumed that the semantic content of
propositions is identified with the semantic content of the NP they are selecting. Hence, the PP
\emph{to John} has the same referential index as the NP \emph{John} and the PP \emph{about himself}
has the same index as \emph{himself}. The \argstl of \emph{talked} is shown in (\mex{1}):
\ea
\sliste{ NP, PP, PP }
\z
The Binding Theory applies as it would apply to ditransitive verbs. Since the first PP is less
oblique than the second one, it can bind an anaphor in the second one. The same is true for the
example in (\mex{-1}a): since the subject is less oblique than the PP object it locally o-commands
it and even though the pronoun \emph{him} is embedded in a PP and not a direct argument of the verb
the pronoun cannot be bound by \emph{him}. An anaphor would be possible within the PP object though.
Of course the subject NP can bind NPs within both PPs: both \emph{to herself} and \emph{about
  herself} would be possible as well.


%\ea
%It was herself that Mary loved.
%\z

\section{Exempt anaphors}

The statement of Principle A has interesting consequences: if an anaphor is not locally o-commanded,
Principle A does not say anything about requirements fro binding. This means that anaphors that are
initial in an \argstl may be bound outside of their local environment.



\section{i within i condition}

\ea
Karl heiratet eine nur sich$_i$ selbst liebdende Frau$_i$.
\z



\section{Reconstruction}

\citet{ps2} point out an interesting consequence of the treatment of nonlocal dependencies in HPSG:
since nonlocal dependencies are introduced by traces that are lexical elements rather then by
transforming one structure into another one as is common in Transformational Grammar, there is no
way to reconstruct some phrase into the position of the trace. Since traces do not have daughters,
$\__j$ in (\mex{1}a) has the same local properties (part of speech, case, referential index) as \emph{which of Claire's$_i$ friends}
without having its internal structure.\footnote{
  Some of the more recent theories of nonlocal dependencies even do without traces. See
  \crossrefchaptert{udc} for details.
}
\eal
\ex I wonder [which of Claire's$_i$ friends]$_j$ [we should let her$_i$ invite $\__j$ to the party]?
\ex {}[Which picture of herself$_i$]$_j$ does Mary$_i$ think John likes $\__j$?
\zl
Since extracted elements are not reconstructed into the position where they would be usually
located, (\mex{0}a) is not related to (\mex{1}):
\ea
We should let her$_i$ invite which of Claire's$_i$ friends to the party.
\z
\emph{Claire} would be o-bound by \emph{her} in (\mex{0}), but since traces do not have daughters,
no problem arises.

This is an interesting feature of the Binding Theory introduced so far, but as
\citet[\page]{Mueller99a} pointed out, it makes wrong predictions as far as German (and English) are
concerned. German is a V2 language and the placement of one constituent infront of the finite verb
is usually accounted for by assuming a nonlocal dependency. If the fronted phrase is not
reconstructed into the position of the trace, it is predicted that bindings like the following are
acceptable, but they are not:
\ea
\gll [Karls$_i$ Freund]$_j$ kennt er$_{*i}$  $\__j$.\\
     \spacebr{}Karl's    friend knows he\\
\glt `He knows Karl's friend.'
\z
The situation is similar in English:
\ea
{}[Karl$_i$'s friend]$_j$, he$_{*i}$ knows $\__j$.
\z
According to the definition of o-command, \emph{he} locally o-commands the object of
\emph{knows}. This object is realized as a trace. Therefore the local properties of \emph{Karl's
  friend} are in relation to \emph{he} but since trace do not have daughters, there is no o-command
relation between \emph{he} and \emph{Karl}, hence \emph{Karl} is o-free and Principle C is not
violated. Hence there is no explanation for the impossibility to bind \emph{Karl} to \emph{he}.


\section{A general Binding Theory with reference to obliqueness or language-specific binding conditions}


\subsection{Obliqueness and constituent order}

As was explained above the order of the elements in the \argstl is seen as crucial for the
determination of possible bindings and reflexivization. Anaphors may refer to elements further to
the left on the \argstl. If one assumes a nom, acc, dat order on the \argstl, \citegen{Grewendorf88a}
binding examples are correctly predicted.\todostefan{add page numbers}
\eal
\label{bsp-der-arzt-zeigte-den-dem}
\ex
\label{bsp-der-arzt-zeigte-den} 
\gll Der Arzt   zeigte den        Patienten$_j$ sich$_j$ / ihm$_{*j}$ im Spiegel.\\
     the doctor showed the.\acc{} patient       self    {} him       in.the mirror\\
\glt `The doctor showed the patient himself in the mirror.'
\ex
\label{bsp-der-arzt-zeigte-dem} 
\gll Der Arzt zeigte dem Patienten$_j$ ihn$_j$ / sich$_{*j}$ im Spiegel.\\
     the doctor showed the.\dat{} patient     him    {} self       in.the mirror\\
\zl
But, as \citet[\page 184]{Eisenberg86} points out, bindings like those in (\mex{1}) exist as well:
\eal
\label{ex-dat-acc-verbs}
\ex Ich empfehle ihm$_j$ sich$_j$.
\ex Du ersparst ihm$_j$ sich$_j$.
\ex Du verleidest ihm$_j$ sich$_j$.
\zl
The examples in (\mex{0}) show that datives may bind accusatives. As (\mex{1}) shows,
\emph{empfehlen} `recommend' allows for passivization, so the accusative object is a direct object in
the sense of the obliqueness hierarchy and should be seen as less oblique than the dative.
\ea
\gll Dieser Stoff wurde ihm empfohlen.\\
     this cloth was him.\dat{} recommended\\
\glt `This cloth was recommended to him.'
\z
It is an open issue how this situation can be resolved. One way is to make binding principles verb
(class) dependent and independent of the obliqueness hierarchy or to assume that it is verb (class)
dependent and that verbs like those in (\ref{ex-dat-acc-verbs}) have a different order of elements
in the \argstl. Of course this could have consequences for other parts of the grammar relying on the
order of elements in the \argstl (see Section~\ref{sec-argst-order} for further discussion).

Note also that (\ref{bsp-der-arzt-zeigte-dem}) causes a Principle C violation. Since accusative
(direct object) is less oblique than dative (indirect object), \emph{ihn} `him' (locally) o-binds \emph{dem
  Patienten} `the patient', which is prohibited by Principle C. This seems to indicate that linear
order play a role in binding. Since the order on the \argstl determines the constituent order in
English, similar problems do not arise in English. In a configurational Binding Theory involving
movement and c-command, the dative would bei higher in the tree and hence c-command the accusative
but due to the analysis of scrambling in HPSG \crossrefchapterp{order} this is not the
case. \citet{Mueller2004b} discusses various alternative analyses for constituent
order. Chapter~\ref{chap-order} of this book presents the one that is commonly assumed: there is a
fixed order of element of the \argstl and heads may be combined with their arguments in any
order. The alternative would be to assume multiple lexical items with different \argstls, each
corresponding to one possible ordering of the arguments \citep{Uszkoreit86b}. This would fix the problem with
(\ref{bsp-der-arzt-zeigte-dem}) but it would cause new problems since subjects may be ordered after
objects:
\eal
\ex
\gll dass der Mann sich vorstellt\\
     that the man  self introduces\\
\glt `that the man introduces himself'
\ex
\gll dass sich der Mann vorstellt\\
     that self the man  introduces\\
\glt `that the man introduces himself'
\zl
The reflexive has to be bound to the subject independent of the relative order of subject and
accusative object. If constituent order were connected to the order of elements in the \argstl, one
would have to reverse the order of elements to be able to analyze sentences like (\mex{0}b), but
then \emph{sich} would be unbound and \emph{der Mann} would be bound (Principle C violation).

\subsection{Binding and prepositional objects}

We already discussed the English examples in (\ref{ex-mary-talked-to-john-about-himself}) with two
prepositional objects and showed that the second PP can contain a reflexive referring to a preceding
PP and that HPSG's Binding Theory explains this nicely. However, the situation in German is
different, as the following data from \citet[\page 58]{Grewendorf88a} show:
\eal
\ex 
\gll Ich sprach mit Maria$_i$ über sie$_i$ / *sich.\\
     I talked   with Maria about her {} self\\
\glt `I talked with Maria about herself.'
\ex 
\gll Ich beklagte mich bei Maria$_i$ über sie$_i$ / *sich.\\
     I complained myself at Maria    about her {} self\\
\glt `I complained to Maria about herself.'
\zl


The conclusion of the discussion in the previous two sections is that a Binding Theory that is entirely based on
obliqueness seems to be not possible and that language-specific binding rules referring to specific
situation involving case and part of speech are necessary (see \citew{Grewendorf88a} for such rules
for German in GB).



\section{Coindexing vs.\ coreference}

\citet[\page 75]{ps2} distinguish between coindexing and coreference. They explicitly mention this
possibility in the discussion of examples like the ones in (\mex{1}):\footnote{
  The following sentence is an attested example of a sentence in which one would expect a reflexive
  rather than a pronoun:
        \ea
        I saw me but I thought I was my dad! (J.\,K.\,Rowling \emph{Harry Potter and the Prisoner of Azbakan}, London: Bloomsbury, 1999, p.\,301)
        \z
  The sentence is uttered as part of a description of a time travel. Harry Potter traveled back three hours in
  time and could see the visitor from the future (himself).
}
\eal
\ex It isn't true that nobody voted for John$_{i}$. JOHN$_{j}$ voted for him$_{i}$. 
        (in a context where both uses of {\em John\/} refer to the same person)
\ex He$_{i}$ [pointing to Richard Nixon] voted for Nixon$_{j}$.
\zl

The definition of o-binding requires coindexing. Indices include person, number and gender
information. Since the anaphors in (\mex{1}) are objects and the subjects o-command them locally,
Principle A requires the subject to bind the anaphor and since this is impossible in (\mex{1}b) due
to gender mismatches, the sentence is ungrammatical.
\eal
\ex[]{
John knows himself.
}
\ex[*]{
John knows herself.
}
\zl
However, the inclusion of coindexing into the definition of o-binding causes a problem since nothing
rules out a coreference like the one in (\mex{1}):
\ea
\label{bsp-john-likes-her}
John$_{i'}$ likes her$_{*i'}$.
\z
The apostrophes show coreference rather than coindexation. According to the definition, \emph{John}
does not o-bind \emph{her} since the two NPs cannot be coindexed. Hence, \emph{her} is o-free and
Principle B is not violated even though the coreference in (\mex{0}) is impossible. One could now
stipulate that coreference is excluded in case of gender mismatches, but the problem is more general
and not restricted to gender. \citet[\page 197]{Eisenberg94a} discusses a German example with number mismatches:
\ea
\gll Auf der Brücke stand ein Paar. Sie stritten sich heftig.\\
     on the bridge stood a couple   they argued  self fiercely\\
\glt `There was a couple on the bridge. They argued fiercely.'
\z
The pronoun \emph{sie} `they' refers to \emph{ein Paar} although \emph{sie} is plural and \emph{ein
  Paar} is singular. HPSG's Binding Theory does not have anything to say about coreferences in
texts, but it is easy to create similar examples with pronoun binding within a single sentence:
\eal
\ex 
\gll Das Paar$_{i'}$ behauptet, dass sie$_{i'}$ sich lieben.\\
     the couple.\sg{} claims that they.\pl{} self love.\pl\\
\ex 
\gll Das Paar$_{i}$ behauptet, dass es$_{i}$ sich liebt.\\
     the couple.\sg{} claims that it.\sg{} self love.\sg\\
\zl
The two NPs cannot be coindexed in (\mex{0}a) since the number of the two NPs is
different.\footnote{
  The number is also shown by the agreeing verbs. One way to model agreement is to assume that the
  verb selects a subject with an index with person and number features corresponding to the
  agreement features of the verb. See \citew{WZ2003a} and \crossrefchapterw{agreement} on
  agreement. The alternative would be to have separate purely syntactic agreement features for
  NPs. \emph{das Paar} would be singular as far as agreement is concerned but could have a
  referential index that can be singular or plural.%
}
As in (\ref{bsp-john-likes-her}), \emph{das Paar} does not o-bind \emph{sie} in (\mex{1}), since it cannot be coindexed:
\ea
\gll Das Paar$_{i'}$ kennt sie$_{*i'}$.\\
     the couple.\sg{} knows they.\pl\\
\glt `The couple knows them.'
\z
Hence, the coindexing in (\mex{0}) does not violate any binding principles but it should be excluded
by something like Principle B requiring that a pronoun must not be bound by/coreferential with
something local.

Similarly Principle C does not apply in (\mex{1}):
\ea
\gll Sie$_{*i'}$ behaupten, dass das Paar$_{i'}$ sich liebt.\\
     they.\pl{} claim      that the couple.\sg{} self loves\\
\glt `They claim that the couple loves each other.'
\z

One could assume that \emph{Paar} is underspecified with respect to number, but this would require
an approach to agreement that does not refer to the index.
\eal
\ex[]{
\gll Das Paar schläft.\\
     the couple sleeps\\
\glt `The couple sleeps.'
}
\ex[*]{
\gll Das Paar schlafen.\\
     the couple sleeps\\
\glt Intended: `The couple sleeps.'
}
\zl
Similarly the match of the relative pronoun and the noun it refers to is usually established by
sharing the index (\citealp{ps2}; \citealp{Sag97a}; \citealp{MuellerLehrbuch1};
\crossrefchapteralp{relative-clauses}). While \citet[\page 417--418]{Mueller99a} has data for neuter nouns in German
like \emph{Mädchen} `girl' and \emph{Weib} archaic for `woman', showing that the relative pronoun may be both the neuter
relative pronoun \emph{das} and the female pronoun \emph{die}, the plural relative pronoun with
coreference to \emph{Paar} is strictly ungrammatical:  
\eal
\ex[]{
\gll das Paar, das sich liebt,\\
     the couple that.\sg{} self loves\\
}
\ex[*]{
\label{ex-paar-die-sich-lieben}
\gll das Paar, die sich lieben,\\
     the couple that.\pl{} self love\\
}
\zl\is{number|)}

Furthermore, as pointed out by \citet[]{Mueller99a}, underspecifying the number feature would run into problems with sentences like
(\mex{1}) containing two different pronouns bound by the same NP:
\ea
\label{ex-paar-es-sie}
\gll Das Paar$_{i}$ behauptet, dass es$_{i}$ sich liebt und dass sie$_{i}$ sich nie streiten.\\
     the couple    claims     that it       self loves and that they self never argue\\
\glt `The couple claims that they love each other and that they never argue with each other.'
\z
Since the number value of \emph{Paar} is underspecified, both \emph{es} and \emph{sie} would be
compatible with the index of \emph{Paar}, but identifying the index of \emph{Paar} with one pronoun
makes it incompatible with the other one. 

Now, HPSG developed some new and interesting techniques to cope with conflicting demands in
coordinate structures (see \citealp[\page 207]{LHC2001a-u} and
\crossrefchapteralp[Figure~\ref{qwsa}]{coordination}) and these seem to be applicable here as well:
Figure~\ref{fig-type-number} shows a type hierarchy that has a common subtype of \type{sg} and
\type{pl}. 
\begin{figure}
\begin{forest}
     [\type{number}
        [\type{sg}
          [\type{strict-sg} ]
          [\type{sg-pl}, name = sgpl ]] 
        [\type{pl}, name=pl,
          [\type{strict-pl} ] ]]
\draw  (pl.south) --(sgpl.north);
\end{forest}
\caption{Type hierarchy making the types \type{sg} and \type{pl} compatible}\label{fig-type-number}
\end{figure}
(\ref{ex-paar-es-sie}) can be analyzed now since \emph{Paar} can be specified to be of type
\type{sg} and this would be compatible with \emph{es} (\type{sg}) and \emph{sie} (\type{pl}): the
result of identifying all indices would result in an index with number value \type{sg-pl}. 

While this general compatibility of singular and plural looks frightening at first sight, one can avoid collapsing all occurrences of
singular and plural into \type{sg-pl} by specifying the number value of linguistic objects that are
strictly singular by assigning the type \type{strict-sg} to them. So \emph{Haus} `house' would have
the number value \type{strict-sg} and would be incompatible with \type{pl}. 

This can also be used for the  


For example the relative pronoun \emph{die} for antecedent nouns in the plural could be
\type{strict-pl}. Since \type{sg} and \type{strict-pl} are incompatible, (\ref{ex-paar-die-sich-lieben}) would be correctly
predicted to be ungrammatical.

\iw{Paar|)}
Dieses Problem tritt auch beim Satz (\ref{bsp-goethe-altes-weib}) auf: Der Index des Nomens
{\em Weib\/}, des Relativpronomens {\em das\/} und des Possessivpronomens {\em ihr\/}
können nicht identisch sein, da {\em das\/} ein Neutrum und {\em ihr\/} ein Femininum ist.


Man beachte, daß das hier diskutierte Problem nicht dadurch gelöst werden kann, daß man
in der Definition von O-binden `koindiziert' durch `koreferent' ersetzt.
\citet[S.\,75]{ps2}\ia{Pollard}\ia{Sag} erlauben nämlich explizit, 
daß zwei verschiedene Indizes auf dasselbe Individuum referieren können.\footnote{
        Ein echter Beleg für einen Satz, in dem man ein Reflexivum erwarten würde, ist
        (i).
        \ea
        I saw me but I thought I was my dad! (J.\,K.\,Rowling {\em Harry Potter and the Prisoner of Azbakan\/}, London: Bloomsbury, 1999, S.\,301)
        \z
        Harry Potter äußert den Satz in einer Beschreibung einer Zeitreise. Er ist
        drei Stunden zurückgereist und und konnte in der Vergangenheit den Besucher
        aus der Zukunft (sich selbst) sehen.
}
\eal
\ex It isn't true that nobody voted for John$_{i}$. JOHN$_{j}$ voted for him$_{i}$. 
        (in a context where both uses of {\em John\/} refer to the same person)
\ex He$_{i}$ [pointing to Richard Nixon] voted for Nixon$_{j}$.
\zl
Auf diese Weise wären dann auch die folgenden Sätze von \citet[S.\,153]{Grewendorf85a}\ia{Grewendorf} wegzuerklären.
\eal
\ex Wenn jeder Wolfgangs Mutter liebt, dann liebt auch Wolfgang$_{i}$ Wolfgangs$_{j}$ Mutter.
\ex Nur Adenauer$_{i}$ stimmte\iw{stimmen für} für Adenauer$_{j}$.
\zl
Im Prinzip macht die Existenz solcher Beispiele und die Erklärung, die für sie angenommen
wird, Prinzip B und Prinzip C überflüssig, da diese Prinzipien 
durch pragmatische\is{Pragmatik} Faktoren jederzeit
wieder außer Kraft gesetzt werden können.

Eine Lösung des Problems könnte die Unterscheidung zwischen {\sc index}"=Kongruenz und
pragmatischer Kongruenz\is{Kongrunez!pragmatische} sein, die \citet{WZ2003a} vornehmen.
%Im Serbokroatischen\il{Serbokroatisch} gibt es Genus"=Kongruenz zwischen Subjekten
%und nicht"=finiten Prädikaten. Diese 
{\sc index}"=Kongruenz liegt vor, wenn das Pronomen ein Neutrum ist, und pragmatische
Kongruenz liegt vor, wenn das Pronomen mit dem natürlichen Geschlecht übereinstimmt,
\dh, wenn ein Femininum verwendet wird.
 

\section{Raising and o-command}
\label{sec-binding-raising}

A further problem has to do with predicate complex constructions in languages like
German. Researcher working on SOV languages like German, Dutch or Korean assume that the verbs form
a verbal complex. The arguments of the embedded verb are attracted by the governing verb. This
technique was developed in the framework of Categorial Grammar and taken over to HPSG by
\citet{HN89a,HN94a}. See also
\crossrefchapterw{complex-predicates}. Figure~\ref{fig-verbal-complex-German} shows the analysis of
the following example:

\ea
\gll dass der Sheriff den Dieb  sich überlassen wird\\
     that the sheriff the thief self leave      will\\
\glt `The sheriff will leave the thief to himself.'
\z


\begin{figure}
\begin{forest}
sm edges
[CP
  [C [dass;that]]
  [S
     [\ibox{1} NP [der Sheriff;the sheriff]]
     [V$'$
       [\ibox{2} NP [den Dieb;the thief]]
       [V$'$
         [\ibox{3} NP [sich;self]]
         [V
           [\ibox{4} V \sliste{ \ibox{1}, \ibox{2}, \ibox{3} } [überlassen;leave]]
           [V \sliste{ \ibox{1}, \ibox{2}, \ibox{3}, \ibox{4} } [wird;will]]]]]]]
%% [S
%%   [\ibox{1} NP [Kim]]
%%   [VP
%%     [V \sliste{ \ibox{1}, \ibox{2}, \ibox{3} } [believes]]
%%     [\ibox{2} NP [her]]
%%     [\ibox{3} VP
%%       [V [to]]
%%       [VP
%%         [V \sliste{ \ibox{2}, \ibox{4} } [like]]
%%         [\ibox{4} NP [Sandy]]]]]]
\end{forest}
\caption{Analysis of a German sentence with a verbal complex}\label{fig-verbal-complex-German}
\end{figure}
The verb \emph{überlassen} `to leave' is ditransitive and takes a nominative \iboxb{1}, a dative \iboxb{2}, and an
accusative argument \iboxb{3}. A verb selecting another verb for verbal complex formation takes over
the argument of the embedded verb. The auxiliary \emph{wird} `will' selects \emph{überlassen} `to
leave' \iboxb{4} and the arguments of \emph{überlassen} (\ibox{1}, \ibox{2}, \ibox{3}). The \argstl
of \emph{wird} contains \emph{den Dieb} and \emph{sich} and hence \emph{den Dieb} locally o-binds
\emph{sich}, but \emph{sich} also binds \emph{den Dieb} since \emph{sich} \iboxb{3} is
less-oblique than the verbal complement \ibox{4} and \ibox{4} selects for \emph{den Dieb} \iboxb{2}. For the latter reason, Principle C is violated.  



Kiss95a:33
Der Junge ließ das Mädchen das Boot für sich reparieren.

\section{Linking, order, scope and binding}
\label{sec-argst-order}

While \citet{KC77a} showed that the obliqueness hierarchy is relevant for activeness of grammatical
functions crosslinguistically, it is an open question whether this hierarchy should be assumed to
hold for all lexemes in all languages and if so, whether it plays a role in the same phenomena
universally. As was discussed by \crossrefchaptert{arg-st}, the \argstl plays an important role in
linking theories. In an analysis of (\mex{1}a), \emph{the dog} is the direct object, while \emph{dem
  Hund} bearing dative case is the indirect object in (\mex{1}b): 
\eal
\ex The elephant gave the dog a ball.
\ex
\gll Der Elephant gab  dem Hund einen Ball.\\
     the elephant gave the dog  a ball\\\jambox*{(German)}
\zl
While \citet{Mueller99a} ordered arguments according to the obliqueness hierarchy in (\ref{def-obliqueness-hierarchy}),
\citet{MuellerCoreGram} decided to keep the \argstls and hence also the linking patterns constant
across languages. \citet{MuellerGermanic} analyzes the Germanic languages with \argstls having the
same order of elements and linking patterns, the differences resulting in a different distribution
of lexical and structural case and a different mapping from \argst to \spr and \comps. See
\crossrefchaptert{case} for more on case assignment and \argst in HPSG.

\citet{Kiss2005a} develops an account of quantifier scope determination for German arguing that cope
is determined with respect to an unmarked order.\footnote{
  See \citew{Hoehle82a} for a definition of \emph{normal order}.
}
The unmarked order is nominative, dative, accusative for most German verbs. This does not correspond
to universal tendencies, according to which the direct object precedes the secondary object
\citep{Pullum77a}. Kiss uses the \argstl to represent the unmarked constitutent order. The consequence is that
German seems to require a nom, dat, acc order for (uniform) linking, constituent order, and scope
and nom, acc, dat for binding. If this really is the case, one seems to need two separate \argstls
to be able to represent both orders.



\section*{Abbreviations}


\section*{Acknowledgements}

The research reported in this chapter was partially supported by the 
Research Infrastructure for the Science and Technology of Language (\mbox{PORTULAN CLARIN}).


{\sloppy
\printbibliography[heading=subbibliography,notkeyword=this]
}
\end{document}


%      <!-- Local IspellDict: en_US-w_accents -->
