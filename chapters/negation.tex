%% -*- coding:utf-8 -*-

\documentclass[output=paper
                ,modfonts
                ,nonflat
	        ,collection
	        ,collectionchapter
	        ,collectiontoclongg
 	        ,biblatex
                ,babelshorthands
                ,newtxmath
                ,draftmode
                ,colorlinks, citecolor=brown
]{./langsci/langscibook}

\IfFileExists{../localcommands.tex}{%hack to check whether this is being compiled as part of a collection or standalone
  % add all extra packages you need to load to this file 

\usepackage{graphicx}
\usepackage{tabularx}
\usepackage{amsmath} 
\usepackage{tipa}      % Davis Koenig
\usepackage{multicol}
\usepackage{lipsum}


\usepackage{./langsci/styles/langsci-optional} 
\usepackage{./langsci/styles/langsci-lgr}
%\usepackage{./styles/forest/forest}
\usepackage{./langsci/styles/langsci-forest-setup}
\usepackage{morewrites}

\usepackage{tikz-cd}

\usepackage{./styles/tikz-grid}
\usetikzlibrary{shadows}


%\usepackage{pgfplots} % for data/theory figure in minimalism.tex
% fix some issue with Mod https://tex.stackexchange.com/a/330076
\makeatletter
\let\pgfmathModX=\pgfmathMod@
\usepackage{pgfplots}%
\let\pgfmathMod@=\pgfmathModX
\makeatother

\usepackage{subcaption}

% Stefan Müller's styles
\usepackage{./styles/merkmalstruktur,german,./styles/makros.2e,./styles/my-xspace,./styles/article-ex,
./styles/eng-date}

\selectlanguage{USenglish}

\usepackage{./styles/abbrev}

\usepackage{./langsci/styles/jambox}

% Has to be loaded late since otherwise footnotes will not work

%%%%%%%%%%%%%%%%%%%%%%%%%%%%%%%%%%%%%%%%%%%%%%%%%%%%
%%%                                              %%%
%%%           Examples                           %%%
%%%                                              %%%
%%%%%%%%%%%%%%%%%%%%%%%%%%%%%%%%%%%%%%%%%%%%%%%%%%%%
% remove the percentage signs in the following lines
% if your book makes use of linguistic examples
\usepackage{./langsci/styles/langsci-gb4e} 

% Crossing out text
% uncomment when needed
%\usepackage{ulem}

\usepackage{./styles/additional-langsci-index-shortcuts}

%\usepackage{./langsci/styles/langsci-avm}
\usepackage{./styles/avm+}


\renewcommand{\tpv}[1]{{\avmjvalfont\itshape #1}}

% no small caps please
\renewcommand{\phonshape}[0]{\normalfont\itshape}

\regAvmFonts

\usepackage{theorem}

\newtheorem{mydefinition}{Def.}
\newtheorem{principle}{Principle}

{\theoremstyle{break}
%\newtheorem{schema}{Schema}
\newtheorem{mydefinition-break}[mydefinition]{Def.}
\newtheorem{principle-break}[principle]{Principle}
}

% This avoids linebreaks in the Schema
\newcounter{schema}
\newenvironment{schema}[1][]
  {% \begin{Beispiel}[<title>]
  \goodbreak%
  \refstepcounter{schema}%
  \begin{list}{}{\setlength{\labelwidth}{0pt}\setlength{\labelsep}{0pt}\setlength{\rightmargin}{0pt}\setlength{\leftmargin}{0pt}}%
    \item[{\textbf{Schema~\theschema}}]\hspace{.5em}\textbf{(#1)}\nopagebreak[4]\par\nobreak}%
  {\end{list}}% \end{Beispiel}

%% \newcommand{schema}[2]{
%% \begin{minipage}{\textwidth}
%% {\textbf{Schema~\theschema}}]\hspace{.5em}\textbf{(#1)}\\
%% #2
%% \end{minipage}}

%\usepackage{subfig}





% Davis Koenig Lexikon

\usepackage{tikz-qtree,tikz-qtree-compat} % Davis Koenig remove

\usepackage{shadow}




\usepackage[english]{isodate} % Andy Lücking
\usepackage[autostyle]{csquotes} % Andy
%\usepackage[autolanguage]{numprint}

%\defaultfontfeatures{
%    Path = /usr/local/texlive/2017/texmf-dist/fonts/opentype/public/fontawesome/ }

%% https://tex.stackexchange.com/a/316948/18561
%\defaultfontfeatures{Extension = .otf}% adds .otf to end of path when font loaded without ext parameter e.g. \newfontfamily{\FA}{FontAwesome} > \newfontfamily{\FA}{FontAwesome.otf}
%\usepackage{fontawesome} % Andy Lücking
\usepackage{pifont} % Andy Lücking -> hand

\usetikzlibrary{decorations.pathreplacing} % Andy Lücking
\usetikzlibrary{matrix} % Andy 
\usetikzlibrary{positioning} % Andy
\usepackage{tikz-3dplot} % Andy

% pragmatics
\usepackage{eqparbox} % Andy
\usepackage{enumitem} % Andy
\usepackage{longtable} % Andy
\usepackage{tabu} % Andy


% Manfred's packages

%\usepackage{shadow}

\usepackage{tabularx}
\newcolumntype{L}[1]{>{\raggedright\arraybackslash}p{#1}} % linksbündig mit Breitenangabe


% Jong-Bok

%\usepackage{xytree}

\newcommand{\xytree}[2][dummy]{Let's do the tree!}

% seems evil, get rid of it
% defines \ex is incompatible with gb4e
%\usepackage{lingmacros}

% taken from lingmacros:
\makeatletter
% \evnup is used to line up the enumsentence number and an entry along
% the top.  It can take an argument to improve lining up.
\def\evnup{\@ifnextchar[{\@evnup}{\@evnup[0pt]}}

\def\@evnup[#1]#2{\setbox1=\hbox{#2}%
\dimen1=\ht1 \advance\dimen1 by -.5\baselineskip%
\advance\dimen1 by -#1%
\leavevmode\lower\dimen1\box1}
\makeatother


% YK -- CG chapter

%\usepackage{xspace}
\usepackage{bm}
\usepackage{bussproofs}


% Antonio Branco, remove this
\usepackage{epsfig}

% now unicode
%\usepackage{alphabeta}



% Berthold udc
%\usepackage{qtree}
%\usepackage{rtrees}

\usepackage{pst-node}

  %add all your local new commands to this file

\makeatletter
\def\blx@maxline{77}
\makeatother


\newcommand{\page}{}



\newcommand{\todostefan}[1]{\todo[color=orange!80]{\footnotesize #1}\xspace}
\newcommand{\todosatz}[1]{\todo[color=red!40]{\footnotesize #1}\xspace}

\newcommand{\inlinetodostefan}[1]{\todo[color=green!40,inline]{\footnotesize #1}\xspace}


\newcommand{\spacebr}{\hspaceThis{[}}

\newcommand{\danish}{\jambox{(\ili{Danish})}}
\newcommand{\english}{\jambox{(\ili{English})}}
\newcommand{\german}{\jambox{(\ili{German})}}
\newcommand{\yiddish}{\jambox{(\ili{Yiddish})}}
\newcommand{\welsh}{\jambox{(\ili{Welsh})}}

% Cite and cross-reference other chapters
\newcommand{\crossrefchaptert}[2][]{\citet*[#1]{chapters/#2}, Chapter~\ref{chap-#2} of this volume} 
\newcommand{\crossrefchapterp}[2][]{(\citealp*[#1][]{chapters/#2}, Chapter~\ref{chap-#2} of this volume)}
% example of optional argument:
% \crossrefchapterp[for something, see:]{name}
% gives: (for something, see: Author 2018, Chapter~X of this volume)

\let\crossrefchapterw\crossrefchaptert



% Davis Koenig

\let\ig=\textsc
\let\tc=\textcolor

% evolution, Flickinger, Pollard, Wasow

\let\citeNP\citet

% Adam P

%\newcommand{\toappear}{Forthcoming}
\newcommand{\pg}[1]{p.#1}
\renewcommand{\implies}{\rightarrow}

\newcommand*{\rref}[1]{(\ref{#1})}
\newcommand*{\aref}[1]{(\ref{#1}a)}
\newcommand*{\bref}[1]{(\ref{#1}b)}
\newcommand*{\cref}[1]{(\ref{#1}c)}

\newcommand{\msadam}{.}
\newcommand{\morsyn}[1]{\textsc{#1}}

\newcommand{\nom}{\morsyn{nom}}
\newcommand{\acc}{\morsyn{acc}}
\newcommand{\dat}{\morsyn{dat}}
\newcommand{\gen}{\morsyn{gen}}
\newcommand{\ins}{\morsyn{ins}}
\newcommand{\loc}{\morsyn{loc}}
\newcommand{\voc}{\morsyn{voc}}
\newcommand{\ill}{\morsyn{ill}}
\renewcommand{\inf}{\morsyn{inf}}
\newcommand{\passprc}{\morsyn{passp}}

%\newcommand{\Nom}{\msadam\nom}
%\newcommand{\Acc}{\msadam\acc}
%\newcommand{\Dat}{\msadam\dat}
%\newcommand{\Gen}{\msadam\gen}
\newcommand{\Ins}{\msadam\ins}
\newcommand{\Loc}{\msadam\loc}
\newcommand{\Voc}{\msadam\voc}
\newcommand{\Ill}{\msadam\ill}
\newcommand{\INF}{\msadam\inf}
\newcommand{\PassP}{\msadam\passprc}

\newcommand{\Aux}{\textsc{aux}}

\newcommand{\princ}[1]{\textnormal{\textsc{#1}}} % for constraint names
\newcommand{\notion}[1]{\emph{#1}}
\renewcommand{\path}[1]{\textnormal{\textsc{#1}}}
\newcommand{\ftype}[1]{\textit{#1}}
\newcommand{\fftype}[1]{{\scriptsize\textit{#1}}}
\newcommand{\la}{$\langle$}
\newcommand{\ra}{$\rangle$}
%\newcommand{\argst}{\path{arg-st}}
\newcommand{\phtm}[1]{\setbox0=\hbox{#1}\hspace{\wd0}}
\newcommand{\prep}[1]{\setbox0=\hbox{#1}\hspace{-1\wd0}#1}

%%%%%%%%%%%%%%%%%%%%%%%%%%%%%%%%%%%%%%%%%%%%%%%%%%%%%%%%%%%%%%%%%%%%%%%%%%%

% FROM FS.STY:

%%%
%%% Feature structures
%%%

% \fs         To print a feature structure by itself, type for example
%             \fs{case:nom \\ person:P}
%             or (better, for true italics),
%             \fs{\it case:nom \\ \it person:P}
%
% \lfs        To print the same feature structure with the category
%             label N at the top, type:
%             \lfs{N}{\it case:nom \\ \it person:P}

%    Modified 1990 Dec 5 so that features are left aligned.
\newcommand{\fs}[1]%
{\mbox{\small%
$
\!
\left[
  \!\!
  \begin{tabular}{l}
    #1
  \end{tabular}
  \!\!
\right]
\!
$}}

%     Modified 1990 Dec 5 so that features are left aligned.
%\newcommand{\lfs}[2]
%   {
%     \mbox{$
%           \!\!
%           \begin{tabular}{c}
%           \it #1
%           \\
%           \mbox{\small%
%                 $
%                 \left[
%                 \!\!
%                 \it
%                 \begin{tabular}{l}
%                 #2
%                 \end{tabular}
%                 \!\!
%                 \right]
%                 $}
%           \end{tabular}
%           \!\!
%           $}
%   }

\newcommand{\ft}[2]{\path{#1}\hspace{1ex}\ftype{#2}}
\newcommand{\fsl}[2]{\fs{{\fftype{#1}} \\ #2}}

\newcommand{\fslt}[2]
 {\fst{
       {\fftype{#1}} \\
       #2 
     }
 }

\newcommand{\fsltt}[2]
 {\fstt{
       {\fftype{#1}} \\
       #2 
     }
 }

\newcommand{\fslttt}[2]
 {\fsttt{
       {\fftype{#1}} \\
       #2 
     }
 }


% jak \ft, \fs i \fsl tylko nieco ciasniejsze

\newcommand{\ftt}[2]
% {{\sc #1}\/{\rm #2}}
 {\textsc{#1}\/{\rm #2}}

\newcommand{\fst}[1]
  {
    \mbox{\small%
          $
          \left[
          \!\!\!
%          \sc
          \begin{tabular}{l} #1
          \end{tabular}
          \!\!\!\!\!\!\!
          \right]
          $
          }
   }

%\newcommand{\fslt}[2]
% {\fst{#2\\
%       {\scriptsize\it #1}
%      }
% }


% superciasne

\newcommand{\fstt}[1]
  {
    \mbox{\small%
          $
          \left[
          \!\!\!
%          \sc
          \begin{tabular}{l} #1
          \end{tabular}
          \!\!\!\!\!\!\!\!\!\!\!
          \right]
          $
          }
   }

%\newcommand{\fsltt}[2]
% {\fstt{#2\\
%       {\scriptsize\it #1}
%      }
% }

\newcommand{\fsttt}[1]
  {
    \mbox{\small%
          $
          \left[
          \!\!\!
%          \sc
          \begin{tabular}{l} #1
          \end{tabular}
          \!\!\!\!\!\!\!\!\!\!\!\!\!\!\!\!
          \right]
          $
          }
   }



% %add all your local new commands to this file

% \newcommand{\smiley}{:)}

% you are not supposed to mess with hardcore stuff, St.Mü. 22.08.2018
%% \renewbibmacro*{index:name}[5]{%
%%   \usebibmacro{index:entry}{#1}
%%     {\iffieldundef{usera}{}{\thefield{usera}\actualoperator}\mkbibindexname{#2}{#3}{#4}{#5}}}

% % \newcommand{\noop}[1]{}



% Rui

\newcommand{\spc}[0]{\hspace{-1pt}\underline{\hspace{6pt}}\,}
\newcommand{\spcs}[0]{\hspace{-1pt}\underline{\hspace{6pt}}\,\,}
\newcommand{\bad}[1]{\leavevmode\llap{#1}}
\newcommand{\COMMENT}[1]{}


% Andy Lücking gesture.tex
\newcommand{\Pointing}{\ding{43}}
% Giotto: "Meeting of Joachim and Anne at the Golden Gate" - 1305-10 
\definecolor{GoldenGate1}{rgb}{.13,.09,.13} % Dress of woman in black
\definecolor{GoldenGate2}{rgb}{.94,.94,.91} % Bridge
\definecolor{GoldenGate3}{rgb}{.06,.09,.22} % Blue sky
\definecolor{GoldenGate4}{rgb}{.94,.91,.87} % Dress of woman with shawl
\definecolor{GoldenGate5}{rgb}{.52,.26,.26} % Joachim's robe
\definecolor{GoldenGate6}{rgb}{.65,.35,.16} % Anne's robe
\definecolor{GoldenGate7}{rgb}{.91,.84,.42} % Joachim's halo

\makeatletter
\newcommand{\@Depth}{1} % x-dimension, to front
\newcommand{\@Height}{1} % z-dimension, up
\newcommand{\@Width}{1} % y-dimension, rightwards
%\GGS{<x-start>}{<y-start>}{<z-top>}{<z-bottom>}{<Farbe>}{<x-width>}{<y-depth>}{<opacity>}
\newcommand{\GGS}[9][]{%
\coordinate (O) at (#2-1,#3-1,#5);
\coordinate (A) at (#2-1,#3-1+#7,#5);
\coordinate (B) at (#2-1,#3-1+#7,#4);
\coordinate (C) at (#2-1,#3-1,#4);
\coordinate (D) at (#2-1+#8,#3-1,#5);
\coordinate (E) at (#2-1+#8,#3-1+#7,#5);
\coordinate (F) at (#2-1+#8,#3-1+#7,#4);
\coordinate (G) at (#2-1+#8,#3-1,#4);
\draw[draw=black, fill=#6, fill opacity=#9] (D) -- (E) -- (F) -- (G) -- cycle;% Front
\draw[draw=black, fill=#6, fill opacity=#9] (C) -- (B) -- (F) -- (G) -- cycle;% Top
\draw[draw=black, fill=#6, fill opacity=#9] (A) -- (B) -- (F) -- (E) -- cycle;% Right
}
\makeatother


% pragmatics
\newcommand{\speaking}[1]{\eqparbox{name}{\textsc{\lowercase{#1}\space}}}
\newcommand{\name}[1]{\eqparbox{name}{\textsc{\lowercase{#1}}}}
\newcommand{\HPSGTTR}{HPSG$_{\text{TTR}}$\xspace}

\newcommand{\ttrtype}[1]{\textit{#1}}
% \newcommand{\avmel}{\q<\quad\q>} %% shortcut for empty lists in AVM
\newcommand{\ttrmerge}{\ensuremath{\wedge_{\textit{merge}}}}
\newcommand{\Cat}[2][0.1pt]{%
  \begin{scope}[y=#1,x=#1,yscale=-1, inner sep=0pt, outer sep=0pt]
   \path[fill=#2,line join=miter,line cap=butt,even odd rule,line width=0.8pt]
  (151.3490,307.2045) -- (264.3490,307.2045) .. controls (264.3490,291.1410) and (263.2021,287.9545) .. (236.5990,287.9545) .. controls (240.8490,275.2045) and (258.1242,244.3581) .. (267.7240,244.3581) .. controls (276.2171,244.3581) and (286.3490,244.8259) .. (286.3490,264.2045) .. controls (286.3490,286.2045) and (323.3717,321.6755) .. (332.3490,307.2045) .. controls (345.7277,285.6390) and (309.3490,292.2151) .. (309.3490,240.2046) .. controls (309.3490,169.0514) and (350.8742,179.1807) .. (350.8742,139.2046) .. controls (350.8742,119.2045) and (345.3490,116.5037) .. (345.3490,102.2045) .. controls (345.3490,83.3070) and (361.9972,84.4036) .. (358.7581,68.7349) .. controls (356.5206,57.9117) and (354.7696,49.2320) .. (353.4652,36.1439) .. controls (352.5396,26.8573) and (352.2445,16.9594) .. (342.5985,17.3574) .. controls (331.2650,17.8250) and (326.9655,37.7742) .. (309.3490,39.2045) .. controls (291.7685,40.6320) and (276.7783,24.2380) .. (269.9740,26.5795) .. controls (263.2271,28.9013) and (265.3490,47.2045) .. (269.3490,60.2045) .. controls (275.6359,80.6368) and (289.3490,107.2045) .. (264.3490,111.2045) .. controls (239.3490,115.2045) and (196.3490,119.2045) .. (165.3490,160.2046) .. controls (134.3490,201.2046) and (135.4934,249.3212) .. (123.3490,264.2045) .. controls (82.5907,314.1553) and (40.8239,293.6463) .. (40.8239,335.2045) .. controls (40.8239,353.8102) and (72.3490,367.2045) .. (77.3490,361.2045) .. controls (82.3490,355.2045) and (34.8638,337.3259) .. (87.9955,316.2045) .. controls (133.3871,298.1601) and   (137.4391,294.4766) .. (151.3490,307.2045) -- cycle;
\end{scope}%
}


% KdK
\newcommand{\smiley}{:)}

\renewbibmacro*{index:name}[5]{%
  \usebibmacro{index:entry}{#1}
    {\iffieldundef{usera}{}{\thefield{usera}\actualoperator}\mkbibindexname{#2}{#3}{#4}{#5}}}

% \newcommand{\noop}[1]{}

% chngcntr.sty otherwise gives error that these are already defined
%\let\counterwithin\relax
%\let\counterwithout\relax

% the space of a left bracket for glossings
\newcommand{\LB}{\hspaceThis{[}}

\newcommand{\LF}{\mbox{$[\![$}}

\newcommand{\RF}{\mbox{$]\!]_F$}}

\newcommand{\RT}{\mbox{$]\!]_T$}}





% Manfred's

\newcommand{\kommentar}[1]{}

\newcommand{\bsp}[1]{\emph{#1}}
\newcommand{\bspT}[2]{\bsp{#1} `#2'}
\newcommand{\bspTL}[3]{\bsp{#1} (lit.: #2) `#3'}

\newcommand{\noidi}{§}

\newcommand{\refer}[1]{(\ref{#1})}

%\newcommand{\avmtype}[1]{\multicolumn{2}{l}{\type{#1}}}
\newcommand{\attr}[1]{\textsc{#1}}

\newcommand{\srdefault}{\mbox{\begin{tabular}{c}{\large <}\\[-1.5ex]$\sqcap$\end{tabular}}}

%% \newcommand{\myappcolumn}[2]{
%% \begin{minipage}[t]{#1}#2\end{minipage}
%% }

%% \newcommand{\appc}[1]{\myappcolumn{3.7cm}{#1}}


% Jong-Bok


% clean that up and do not use \def (killing other stuff defined before)
%\if 0
\def\DEL{\textsc{del}}
\def\del{\textsc{del}}

\def\conn{\textsc{conn}}
\def\CONN{\textsc{conn}}
\def\CONJ{\textsc{conj}}
\def\LITE{\textsc{lex}}
\def\lite{\textsc{lex}}
\def\HON{\textsc{hon}}

\def\CAUS{\textsc{caus}}
\def\PASS{\textsc{pass}}
\def\NPST{\textsc{npst}}
\def\COND{\textsc{cond}}



\def\hd-lite{\textsc{head-lex construction}}
\def\NFORM{\textsc{nform}}

\def\RELS{\textsc{rels}}
\def\TENSE{\textsc{tense}}


%\def\ARG{\textsc{arg}}
\def\ARGs{\textsc{arg0}}
\def\ARGa{\textsc{arg}}
\def\ARGb{\textsc{arg2}}
\def\TPC{\textsc{top}}
\def\PROG{\textsc{prog}}

\def\pst{\textsc{pst}}
\def\PAST{\textsc{pst}}
\def\DAT{\textsc{dat}}
\def\CONJ{\textsc{conj}}
\def\nominal{\textsc{nominal}}
\def\NOMINAL{\textsc{nominal}}
\def\VAL{\textsc{val}}
\def\val{\textsc{val}}
\def\MODE{\textsc{mode}}
\def\RESTR{\textsc{restr}}
\def\SIT{\textsc{sit}}
\def\ARG{\textsc{arg}}
\def\RELN{\textsc{rel}}
\def\REL{\textsc{rel}}
\def\RELS{\textsc{rels}}
\def\arg-st{\textsc{arg-st}}
\def\xdel{\textsc{xdel}}
\def\zdel{\textsc{zdel}}
\def\sug{\textsc{sug}}
\def\IMP{\textsc{imp}}
\def\conn{\textsc{conn}}
\def\CONJ{\textsc{conj}}
\def\HON{\textsc{hon}}
\def\BN{\textsc{bn}}
\def\bn{\textsc{bn}}
\def\pres{\textsc{pres}}
\def\PRES{\textsc{pres}}
\def\prs{\textsc{pres}}
\def\PRS{\textsc{pres}}
\def\agt{\textsc{agt}}
\def\DEL{\textsc{del}}
\def\PRED{\textsc{pred}}
\def\AGENT{\textsc{agent}}
\def\THEME{\textsc{theme}}
\def\AUX{\textsc{aux}}
\def\THEME{\textsc{theme}}
\def\PL{\textsc{pl}}
\def\SRC{\textsc{src}}
\def\src{\textsc{src}}
\def\FORM{\textsc{form}}
\def\form{\textsc{form}}
\def\GCASE{\textsc{gcase}}
\def\gcase{\textsc{gcase}}
\def\SCASE{\textsc{scase}}
\def\PHON{\textsc{phon}}
\def\SS{\textsc{ss}}
\def\SYN{\textsc{syn}}
\def\LOC{\textsc{loc}}
\def\MOD{\textsc{mod}}
\def\INV{\textsc{inv}}
\def\L{\textsc{l}}
\def\CASE{\textsc{case}}
\def\SPR{\textsc{spr}}
\def\COMPS{\textsc{comps}}
%\def\comps{\textsc{comps}}
\def\SEM{\textsc{sem}}
\def\CONT{\textsc{cont}}
\def\SUBCAT{\textsc{subcat}}
\def\CAT{\textsc{cat}}
\def\C{\textsc{c}}
\def\SUBJ{\textsc{subj}}
\def\subj{\textsc{subj}}
\def\SLASH{\textsc{slash}}
\def\LOCAL{\textsc{local}}
\def\ARG-ST{\textsc{arg-st}}
\def\AGR{\textsc{agr}}
\def\PER{\textsc{per}}
\def\NUM{\textsc{num}}
\def\IND{\textsc{ind}}
\def\VFORM{\textsc{vform}}
\def\PFORM{\textsc{pform}}
\def\decl{\textsc{decl}}
\def\loc{\textsc{loc   }}
% \def\   {\textsc{  }}

\def\NEG{\textsc{neg}}
\def\FRAMES{\textsc{frames}}
\def\REFL{\textsc{refl}}

\def\MKG{\textsc{mkg}}

\def\BN{\textsc{bn}}
\def\HD{\textsc{hd}}
\def\NP{\textsc{np}}
\def\PF{\textsc{pf}}
\def\PL{\textsc{pl}}
\def\PP{\textsc{pp}}
\def\SS{\textsc{ss}}
\def\VF{\textsc{vf}}
\def\VP{\textsc{vp}}
\def\bn{\textsc{bn}}
\def\cl{\textsc{cl}}
\def\pl{\textsc{pl}}
\def\Wh{\ital{Wh}}
\def\ng{\textsc{neg}}
\def\wh{\ital{wh}}
\def\ACC{\textsc{acc}}
\def\AGR{\textsc{agr}}
\def\AGT{\textsc{agt}}
\def\ARC{\textsc{arc}}
\def\ARG{\textsc{arg}}
\def\ARP{\textsc{arc}}
\def\AUX{\textsc{aux}}
\def\CAT{\textsc{cat}}
\def\COP{\textsc{cop}}
\def\DAT{\textsc{dat}}
\def\DEF{\textsc{def}}
\def\DEL{\textsc{del}}
\def\DOM{\textsc{dom}}
\def\DTR{\textsc{dtr}}
\def\FUT{\textsc{fut}}
\def\GAP{\textsc{gap}}
\def\GEN{\textsc{gen}}
\def\HON{\textsc{hon}}
\def\IMP{\textsc{imp}}
\def\IND{\textsc{ind}}
\def\INV{\textsc{inv}}
\def\LEX{\textsc{lex}}
\def\Lex{\textsc{lex}}
\def\LOC{\textsc{loc}}
\def\MOD{\textsc{mod}}
\def\MRK{{\nr MRK}}
\def\NEG{\textsc{neg}}
\def\NEW{\textsc{new}}
\def\NOM{\textsc{nom}}
\def\NUM{\textsc{num}}
\def\PER{\textsc{per}}
\def\PST{\textsc{pst}}
\def\QUE{\textsc{que}}
\def\REL{\textsc{rel}}
\def\SEL{\textsc{sel}}
\def\SEM{\textsc{sem}}
\def\SIT{\textsc{arg0}}
\def\SPR{\textsc{spr}}
\def\SRC{\textsc{src}}
\def\SUG{\textsc{sug}}
\def\SYN{\textsc{syn}}
\def\TPC{\textsc{top}}
\def\VAL{\textsc{val}}
\def\acc{\textsc{acc}}
\def\agt{\textsc{agt}}
\def\cop{\textsc{cop}}
\def\dat{\textsc{dat}}
\def\foc{\textsc{focus}}
\def\FOC{\textsc{focus}}
\def\fut{\textsc{fut}}
\def\hon{\textsc{hon}}
\def\imp{\textsc{imp}}
\def\kes{\textsc{kes}}
\def\lex{\textsc{lex}}
\def\loc{\textsc{loc}}
\def\mrk{{\nr MRK}}
\def\nom{\textsc{nom}}
\def\num{\textsc{num}}
\def\plu{\textsc{plu}}
\def\pne{\textsc{pne}}
\def\pst{\textsc{pst}}
\def\pur{\textsc{pur}}
\def\que{\textsc{que}}
\def\src{\textsc{src}}
\def\sug{\textsc{sug}}
\def\tpc{\textsc{top}}
\def\utt{\textsc{utt}}
\def\val{\textsc{val}}
\def\LITE{\textsc{lex}}
\def\PAST{\textsc{pst}}
\def\POSP{\textsc{pos}}
\def\PRS{\textsc{pres}}
\def\mod{\textsc{mod}}%
\def\newuse{{`kes'}}
\def\posp{\textsc{pos}}
\def\prs{\textsc{pres}}
\def\psp{{\it en\/}}
\def\skes{\textsc{kes}}
\def\CASE{\textsc{case}}
\def\CASE{\textsc{case}}
\def\COMP{\textsc{comp}}
\def\CONJ{\textsc{conj}}
\def\CONN{\textsc{conn}}
\def\CONT{\textsc{cont}}
\def\DECL{\textsc{decl}}
\def\FOCUS{\textsc{focus}}
\def\FORM{\textsc{form}}
\def\FREL{\textsc{frel}}
\def\GOAL{\textsc{goal}}
\def\HEAD{\textsc{head}}
\def\INDEX{\textsc{ind}}
\def\INST{\textsc{inst}}
\def\MODE{\textsc{mode}}
\def\MOOD{\textsc{mood}}
\def\NMLZ{\textsc{nmlz}}
\def\PHON{\textsc{phon}}
\def\PRED{\textsc{pred}}
%\def\PRES{\textsc{pres}}
\def\PROM{\textsc{prom}}
\def\RELN{\textsc{pred}}
\def\RELS{\textsc{rels}}
\def\STEM{\textsc{stem}}
\def\SUBJ{\textsc{subj}}
\def\XARG{\textsc{xarg}}
\def\bse{{\it bse\/}}
\def\case{\textsc{case}}
\def\caus{\textsc{caus}}
\def\comp{\textsc{comp}}
\def\conj{\textsc{conj}}
\def\conn{\textsc{conn}}
\def\decl{\textsc{decl}}
\def\fin{{\it fin\/}}
\def\form{\textsc{form}}
\def\gend{\textsc{gend}}
\def\inf{{\it inf\/}}
\def\mood{\textsc{mood}}
\def\nmlz{\textsc{nmlz}}
\def\pass{\textsc{pass}}
\def\past{\textsc{past}}
\def\perf{\textsc{perf}}
\def\pln{{\it pln\/}}
\def\pred{\textsc{pred}}


%\def\pres{\textsc{pres}}
\def\proc{\textsc{proc}}
\def\nonfin{{\it nonfin\/}}
\def\AGENT{\textsc{agent}}
\def\CFORM{\textsc{cform}}
%\def\COMPS{\textsc{comps}}
\def\COORD{\textsc{coord}}
\def\COUNT{\textsc{count}}
\def\EXTRA{\textsc{extra}}
\def\GCASE{\textsc{gcase}}
\def\GIVEN{\textsc{given}}
\def\LOCAL{\textsc{local}}
\def\NFORM{\textsc{nform}}
\def\PFORM{\textsc{pform}}
\def\SCASE{\textsc{scase}}
\def\SLASH{\textsc{slash}}
\def\SLASH{\textsc{slash}}
\def\THEME{\textsc{theme}}
\def\TOPIC{\textsc{topic}}
\def\VFORM{\textsc{vform}}
\def\cause{\textsc{cause}}
%\def\comps{\textsc{comps}}
\def\gcase{\textsc{gcase}}
\def\itkes{{\it kes\/}}
\def\pass{{\it pass\/}}
\def\vform{\textsc{vform}}
\def\CCONT{\textsc{c-cont}}
\def\GN{\textsc{given-new}}
\def\INFO{\textsc{info-st}}
\def\ARG-ST{\textsc{arg-st}}
\def\SUBCAT{\textsc{subcat}}
\def\SYNSEM{\textsc{synsem}}
\def\VERBAL{\textsc{verbal}}
\def\arg-st{\textsc{arg-st}}
\def\plain{{\it plain}\/}
\def\propos{\textsc{propos}}
\def\ADVERBIAL{\textsc{advl}}
\def\HIGHLIGHT{\textsc{prom}}
\def\NOMINAL{\textsc{nominal}}

\newenvironment{myavm}{\begingroup\avmvskip{.1ex}
  \selectfont\begin{avm}}%
{\end{avm}\endgroup\medskip}
\def\pfix{\vspace{-5pt}}


\def\jbsub#1{\lower4pt\hbox{\small #1}}
\def\jbssub#1{\lower4pt\hbox{\small #1}}
\def\jbtr{\underbar{\ \ \ }\ }


%\fi

  %% hyphenation points for line breaks
%% Normally, automatic hyphenation in LaTeX is very good
%% If a word is mis-hyphenated, add it to this file
%%
%% add information to TeX file before \begin{document} with:
%% %% hyphenation points for line breaks
%% Normally, automatic hyphenation in LaTeX is very good
%% If a word is mis-hyphenated, add it to this file
%%
%% add information to TeX file before \begin{document} with:
%% %% hyphenation points for line breaks
%% Normally, automatic hyphenation in LaTeX is very good
%% If a word is mis-hyphenated, add it to this file
%%
%% add information to TeX file before \begin{document} with:
%% \include{localhyphenation}
\hyphenation{
A-la-hver-dzhie-va
anaph-o-ra
affri-ca-te
affri-ca-tes
Atha-bas-kan
Chi-che-ŵa
com-ple-ments
Da-ge-stan
Dor-drecht
er-klä-ren-de
Ginz-burg
Gro-ning-en
Jon-a-than
Ka-tho-lie-ke
Ko-bon
krie-gen
Le-Sourd
moth-er
Mül-ler
Nie-mey-er
Prze-piór-kow-ski
phe-nom-e-non
re-nowned
Rie-he-mann
un-bound-ed
}

% why has "erklärende" be listed here? I specified langid in bibtex item. Something is still not working with hyphenation.


% to do: check
%  Alahverdzhieva

\hyphenation{
A-la-hver-dzhie-va
anaph-o-ra
affri-ca-te
affri-ca-tes
Atha-bas-kan
Chi-che-ŵa
com-ple-ments
Da-ge-stan
Dor-drecht
er-klä-ren-de
Ginz-burg
Gro-ning-en
Jon-a-than
Ka-tho-lie-ke
Ko-bon
krie-gen
Le-Sourd
moth-er
Mül-ler
Nie-mey-er
Prze-piór-kow-ski
phe-nom-e-non
re-nowned
Rie-he-mann
un-bound-ed
}

% why has "erklärende" be listed here? I specified langid in bibtex item. Something is still not working with hyphenation.


% to do: check
%  Alahverdzhieva

\hyphenation{
A-la-hver-dzhie-va
anaph-o-ra
affri-ca-te
affri-ca-tes
Atha-bas-kan
Chi-che-ŵa
com-ple-ments
Da-ge-stan
Dor-drecht
er-klä-ren-de
Ginz-burg
Gro-ning-en
Jon-a-than
Ka-tho-lie-ke
Ko-bon
krie-gen
Le-Sourd
moth-er
Mül-ler
Nie-mey-er
Prze-piór-kow-ski
phe-nom-e-non
re-nowned
Rie-he-mann
un-bound-ed
}

% why has "erklärende" be listed here? I specified langid in bibtex item. Something is still not working with hyphenation.


% to do: check
%  Alahverdzhieva

  \bibliography{../Bibliographies/stmue,
                ../localbibliography,
../Bibliographies/formal-background,
../Bibliographies/understudied-languages,
../Bibliographies/phonology,
../Bibliographies/case,
../Bibliographies/evolution,
../Bibliographies/agreement,
../Bibliographies/lexicon,
../Bibliographies/np,
../Bibliographies/negation,
../Bibliographies/argst,
../Bibliographies/binding,
../Bibliographies/complex-predicates,
../Bibliographies/coordination,
../Bibliographies/relative-clauses,
../Bibliographies/udc,
../Bibliographies/processing,
../Bibliographies/cl,
../Bibliographies/dg,
../Bibliographies/islands,
../Bibliographies/gesture,
../Bibliographies/semantics,
../Bibliographies/pragmatics,
../Bibliographies/information-structure,
../Bibliographies/idioms,
../Bibliographies/cg,
../Bibliographies/udc,
collection.bib}

  \togglepaper[14]
}{}

\author{Jong-Bok Kim\affiliation{Kyung Hee University, Seoul}}
\title{Negation}


%\usepackage{chrisbib}


%move the following commands to the "local..." files of the master project when integrating this chapter
%\usepackage{tabularx}
%\usepackage{langsci-gb4e}
%\usepackage{langsci-optional}
%\usepackage{graphicx}
%\usepackage[sectionbib]{natbib}
%\usepackage{chrisbib}
%\usepackage{avm}
%\usepackage[backend=biber]{biblatex}
%\usepackage[linguistics]{forest}


%\usepackage{biblatex}
%\addbibresource{negation-mine-bib}
%\printbibliography
%\bibliography{negation-mine-bib}


%% \newcommand\LIGHT{\textsc{light}}
%% %\newcommand\NEG{\textsc{neg}}
%% %\newcommand\FUT{\textsc{fut}}
%% %\def\jbsub#1{\lower4pt\hbox{\small #1}}
%% %\def\ssub#1{\lower4pt\hbox{\small #1}}
%% %\newcommand\trace{\underbar{\ \ \ }\ }
%% \newcommand\FORM{\textsc{form}}
%% \newcommand\hdlight{\textsc{head-light construction}}
%% \newcommand\POL{\textsc{pol}}




\title{Negation}

\abstract{Each language has a way to express (sentential) negation
that reverses the truth value of a certain sentence, but employs
language-particular expressions as well as grammatical strategies. There are
four main types of negative
in expressing sentential negation: adverbial negative, morphological negative,
auxiliary negative verb, and
preverbal negative. This chapter discusses HPSG analyses for these four strategies
in marking sentential negation.}


\begin{document}
\maketitle
\label{chap-negation}

{\avmoptions{center}

\section{Modes of expressing negation}

%In a typological study of sentential negation,  Dahl (1979) has
%identified three major ways of expressing negation in natural
%%languages as a morphological category on verbs, as an auxiliary
%verb, and as an adverb-like particle.

There are four main types of negative markers
in expressing negation in languages: morphological negative,
auxiliary negative verb, adverbial negative, and clitic-like
 preverbal negative (see \citealt{Dahl:79, Payne:85, Dryer:05}).
 %\addpages
 % these are papers (except Dryer) and hard to pin down the page numbers since the typology
 % is discussed across the papers.
Each of these types is illustrated in the following:

\eal
\ex\label{negation-1a}
\gll Ali  elmalar-i  ser-me-di-$\emptyset$. \\
Ali apples-\textsc{acc}  like-\textsc{neg}-\textsc{pst}-\textsc{3sg} \\ \hfill (\ili{Turkish})
\glt `Ali didn't like apples.'

\ex\label{negation-1b}
\gll sensayng-nim-i o-ci anh-usi-ess-ta. \\
teacher-\textsc{hon}-\textsc{nom} come-\textsc{conn} \textsc{neg}-\textsc{hon}-\textsc{pst}-\textsc{decl} \\  \hfill (\ili{Korean})
\glt `The teacher didn't come.'

\ex \label{negation-1c}
\gll Dominique (n')\'{e}crivait pas de lettre.\\
     Dominique ne.wrote \textsc{neg} of letter \\ \hfill (\ili{French})
\glt `Dominique did not write a letter.' %\todostefan{check gloss: what about the (n)} jb: fixed

\ex \label{negation-1d} %Italian:\\
\gll Gianni non legge articoli di sintassi. \\
Gianni \textsc{neg} reads articles of syntax \\ \hfill (\ili{Italian})
\glt `Gianni doesn't read syntax articles.'
\zl

\noindent
As shown in (\ref{negation-1a}), languages like Turkish
have typical examples of morphological negatives where
negation is expressed by an inflectional category realized on the
verb by affixation. Meanwhile, languages like Korean
 employ a negative auxiliary verb as in (\ref{negation-1b}).\footnote{Korean
 is peculiar in that it has two ways to
 express sentential negation: a negative auxiliary (a long form
 negation)  and a morphological negative (a short form negation)
 for sentential negation. See \citet{Kim:00,Kim:16} and references therein for detail.}
  The negative auxiliary
 verb here is marked with basic verbal categories such as agreement, tense, aspect, and mood, while the lexical, main verb remains in an invariant, participle form. The third major way of expressing negation is to use an adverbial
negative. This type of negation, forming an independent word, is found in
languages like English and French, as given in (\ref{negation-1c}). In these languages, negatives behave like adverbs in their ordering with respect to the verb.\footnote{In French, the negator \emph{pas}
often accompanies the optional preverb clitic \emph{ne}.} The fourth
type is to introduce a preverbal negative. The negative marker in Italian in (\ref{negation-1d}), preceding a finite verb like other types of clitics in the language,
belongs to this type.


In analyzing these four main types of sentential negation, there have been two main strands: derivational and nonderivational views. The derivational view has claimed that the positioning of all of the
four types of negatives is basically determined by the interaction of movement
operations, a rather large set of functional projections including NegP,
and their hierarchically fixed organization.\footnote{The term ``negator'' or ``negative marker'' is a cover term for any linguistic expression functioning as sentential negation.}
In particular, to account for the
fact that unlike English, only French allows main or lexical verb inversion
as in (\ref{negation-1c}), \citet{Pollock:89,Pollock:94} and a number of subsequent researchers
have interpreted these contrasts as providing critical motivation for
the process of \isi{head movement} and the existence of functional
categories such as MoodP, TP, AgrP, and NegP (see \citealt{Belletti:90, Zanuttini:91, Zanuttini:97,Zanuttini:01, Chomsky:91,Chomsky:93,Chomsky:95, Lasnik:95, Haegeman:95,Haegeman:97, Vikner97a-u, Zeijlstra:15}).
Within the derivational view, it has thus been widely
accepted that the variation between French and English can be explained only in terms of the respective properties of verb movement and its interaction with a view of clause
structure organized around functional projections.


Departing from the derivational view, the non-derivational, lexicalist view
introduces no uniform syntactic category (e.g., Neg or NegP) for the different types of negatives. This view allows negation to be realized in different grammatical categories, e.g., a morphological suffix, an auxiliary verb, or an adverbial expression. For instance, the negative \emph{not} in English is taken to be an adverb like other negative expressions in English (e.g., \textit{never, barely, hardly}). This view has been suggested by \citet{Jackendoff:72}, \citet{Baker:91}, \citet{Ernst:92}, \citet{AG:97}, \citet{Kim:00}, and \citet{Warner2000a-u}. In particular,
\citet{KS:96}, \citet{AG:97}, \citet{Kim:00}, and \citet{KS:02} develop analyses of sentential negation in English, French, Korean, and Italian within the framework of HPSG, showing that the postulation of Neg and its projection NegP creates more empirical and theoretical problems than it solves (see \citealt{Newmeyer:2006} for this point).
%For the account of English negation, \citet{Warner2000a-u},
%further developing the analyses of \citet{KS:96} and \citet{Kim:00},
%characterizes negation within the English auxiliary system without the use of lexical rules, %explores inheritance hierarchies in interpreting the distributional possibilities of %negation in various environments. For instance, \citet{Warner2000a-u} classifies auxiliaries %into two subtypes with respect to negation and inversion, each of which is again %subclassified in terms of being negated and inverted.
In addition, there has been substantial work on negation in other languages within the HPSG framework, which
does not resort to the postulation of functional projections or movement operations to account for the various distributional possibilities
of negation (see \citealt{PK:99, BJ:00, Prz:00, Kupsc:02, Swart:02, Borsley:05, Crysmann:10, Bender:13}).

This chapter reviews the HPSG analyses of these four main types of negation,
focusing on the distributional possibilities of these four types of negatives in
relation to other main constituents of the sentence. When
necessary, the chapter also discusses implications for
the theory of grammar.\footnote{This chapter grows out of \citet{Kim:00,kim:18}.} The
chapter starts with the HPSG analyses of adverbial negatives in English and French, which have been most extensively studied in transformational grammars.
The chapter then moves to the discussion of morphological
negatives, negative auxiliary verbs, and preverbal negatives. The chapter
also reviews the HPSG analyses of phenomena like genitive negation and
negative concord which are sensitive to the presence of negative expressions. The
final section concludes this chapter.


%\iffalse{}
\section{Adverbial negative}

\subsection{Two key factors}


The most extensively studied type of negation is the adverbial negative, which
we find in English and French.
There are two main factors
that determine the position of an adverbial negative: the finiteness of
the verb and its intrinsic properties, namely whether it is an auxiliary
or a lexical  verb (see \citealt{Kim:00, KS:02}).\footnote{German also
employs an adverbial negative \textit{nicht}, which behaves quite
differently from the negative in English and French. See \citet{MuellerGT-Eng1}
for a detailed review of the previous theoretical analyses of German negation.}

%\jbsubsubsection{Finiteness vs.\ Non-finiteness}

First consider the finiteness of the lexical verb that affects
the position of adverbial
negatives in English and French.
English shows us how the finiteness of a verb influences the
surface position of the adverbial negative \textit{not}:

\begin{exe}
\ex\label{negation-eng-fin-neg} \begin{xlist}
\ex[]{
Kim does not like Lee.
}
\ex[*]{
Kim not likes Lee.
}
\ex[*]{Kim likes not Lee.
}
\zl


\begin{exe}
\ex\label{negation-fr-fin-neg} \begin{xlist}
\ex[]{
Kim is believed [not [to like Mary]].
}
\ex[*]{
Kim is believed to [like not Mary].
}
\zl
%
\noindent As seen from the data above, the negation \textit{not} precedes an infinitive verb, but cannot follow
a finite lexical  verb (see \citealt{Baker:89,Baker:91,Ernst:92}).
French is not exceptional in this respect. The finiteness also affects the distributional possibilities of the French negative \emph{pas} (see \citealt{AG:97, KS:02, Zeijlstra:07}):

%\inlinetodostefan{Changed the optionality to be in the source rather than in the gloss. Check, Gloss
%jb: ok
%b. everything has to be glossed! Please provide all glosses and languages in brackets as in initial examples}
\eal
\ex[]{
\gll Robin  (n')aime   pas  Stacy. \\
     Robin  ne.likes \NEG{} Stacy \\
\glt`Robin does not like Stacy.'
}
\ex[*]{
\gll Robin ne pas aime Stacy.\\
     Robin ne \textsc{neg} likes Stacy \\
}
\zl

\eal\ex[]{
\gll Ne  pas         parler    Fran\c{c}ais  est  un  grand d\'{e}savantage  en ce cas. \\
     ne \textsc{neg} to.speak  French  is  a great disadvantage  in this case \\
\glt `Not speaking French is a great disadvantage in this case.'
}
\ex[*]{
\gll Ne  parler   pas          Fran\c{c}ais  est un  grand d\'{e}savantage en ce   cas.\\ %\addglosses (isn't this too obvious?)
     ne  to.speak \textsc{neg} French        is  a   great disadvantage    in this case \\
}
\zl

\noindent
The data illustrate that the negator \textit{pas} cannot precede a finite verb
but must follow it. But its placement with respect to
the nonfinite verb is the reverse image. The negator \textit{pas}
should precede an infinitive verb.

The second important factor that determines the position of adverbial
negatives concerns the presence of an auxiliary or lexical  verb.
Modern English displays a clear example where this
intrinsic property of the verb influences the position of
the English negator \textit{not}: the negator cannot follow
a finite lexical  verb, but when the finite verb is an auxiliary verb,
this ordering is possible.

\eal
\ex[*]{
Kim left not the town.
}
\ex[]{
Kim has not left the town.
}
\ex[]{
Kim is not leaving the town.
}
\zl

\noindent
The placement of \textit{pas} in French infinitival
clauses is also affected by the intrinsic property of
the verb:
% affects the position of the adverbial negative \textit{pas}:

\eal
\ex[]{
\gll Ne pas          avoir de voiture dans cette ville rend la vie difficile. \\
ne \textsc{neg} have a car      in   this city  make the life difficult\\
\glt `Not having a car in this city makes life difficult.'
}
\ex[] {
\gll N'avoir pas de voiture dans cette ville rend la vie difficile.\\
ne.have \textsc{neg} a car      in   this city  make the life difficult\\
} \label{negation-28b}
\zl

\eal
\ex[]{
\gll Ne pas \^{e}tre triste est une condition pour chanter des chansons. \\
     ne \textsc{neg}  be     sad     is   a condition for  singing  of songs\\
\glt `Not being sad is a condition for singing songs.'
}
\ex[]{
\gll N'\^{e}tre pas triste est   une condition pour chanter des chansons.\\
     ne.be  \textsc{neg}   sad     is   a  condition for  singing  of songs\\
\glt `Not being sad is a condition for singing songs.'
} \label{negation-29b}
\zl

\noindent
The negator \textit{pas} can either follow or precede an infinitive
auxiliary verb, although the acceptability of the
ordering in (\ref{negation-28b}) and (\ref{negation-29b}) is restricted to certain conservative
varieties.

In capturing the distributional behavior of such adverbial negatives
in English and French, as noted earlier, the derivational view (exemplified by \citealt{Pollock:89} and \citealt{Chomsky:91})
has relied on the notion of verb
movement and functional projections.  The most appealing aspect of this
view (initially at least) is that it can provide an analysis of the
systematic variation
between English and French. By simply assuming that the
two languages have different scopes of verb movement -- in English
only auxiliary verbs move to a higher functional projection, whereas
all French verbs undergo the same process -- the derivational
view could explain why the French negator \textit{pas} follows
a finite verb, unlike the English negator.  In order for this system to succeed,
nontrivial complications are required in the basic components of the
grammar, e.g., rather questionable subtheories (see \citealt{Kim:00} and \citealt{KS:02}
for detailed discussion).

Meanwhile, the nonderivational, lexicalist analyses of HPSG
license all surface structures by the system of phrase types
and constraints.  That is, the position of
adverbial negatives is taken to be determined not by
the respective properties of verb movement, but by their lexical
properties, the morphosyntactic (finiteness) features of the verbal head,
and independently motivated Linear Precedence (LP) constraints, as
we will see in the following discussion.

\iffalse{
For example, the introduction of Pollock's theta and quantification
theories has been necessary to account for the obligatory verb
movement.
%\footnote{His theta theory says only nonthematic verbs move
%up to the higher functional position, whereas his quantification
%theory says [$+$fin] is an operator that must bind a variable.}
However, when these subtheories interact with each other,
they bring about a `desperate' situation, as \citet{Pollock:89} himself concedes: his quantification theory forces
all lexical  verbs in English to undergo verb movement, but his
theory blocks this. This contradictory outcome has forced him to adopt
an otherwise unmotivated mechanism, a dummy
nonlexical counterpart of \textit{do} in English (which \citep{Chomsky:89} tries
to avoid by adopting the notion of LF re-raising).
Leaving the plausibility of this mechanism aside,  as
discussed by \citet{Kim:00} and \citet{KS:02},
a derivational analysis such as that of \citet{Pollock:89}
fails to allow for all the distributional possibilities of
English and French negators as well as adverb positioning in
various environments.

In capturing the interaction with auxiliary verbs, derivational analyses have chosen the direction of generating
auxiliaries and lexical  verbs in different positions. For example,
\citet{Pollock:89}'s system for English auxiliaries posits
various different positions for different verbs: lexical
verbs and \textit{have} and \textit{be} under V within the VP,
\textit{do} under Agr, modals such as \textit{will, may}, and \textit{can}
under T.\footnote{See \citep{Ouhalla:91}'s system in which
all auxiliaries are generated under the head of AspP.} But for French,
all verbs, whether
auxiliary or lexical  verbs, are generated under V.
This contrast does not seem to be unreasonable, considering that in
modern French no syntactic phenomenon clearly distinguishes auxiliary
verbs and lexical  verbs. Leaving aside the question of why the two
typologically related languages have such different ways of generating
verbs including auxiliaries, Pollock's system has suffered
from problems in capturing the distribution of \textit{not} and
\textit{pas} in \textit{have}/\textit{avoir} and modal constructions.}
\fi
%as well as the properties of \textit{have/avoir}.
%This has led the system to introduce rather weakly motivated
%and questionable assumptions, e.g.\ an exotic structure for the main
%verb usage of \textit{have/avoir}
%\footnote{This
%distinction has been required
%in Pollock's theory since [+fin] tense requires verb
%movement to Tense, prohibits affix movement in French and turns {\it
%not} into a block for affix movement
%in English, but [-fin] does not require verb movement, does not prohibit %affix
%movement and allows \textit{not} not to count as a block for affix
%movement. See Pollock (1989:391--395) for further details
%}.

\subsection{Constituent negation}

When English \textit{not} negates an embedded constituent, it behaves
much like the negative adverb \textit{never}. The similarity between {\it
not} and \textit{never} is particularly clear in nonfinite verbal
constructions (participle, infinitival, and bare verb phrases), as
illustrated in (\ref{negation-30}) and (\ref{negation-31}) \citep{Klima:64, Baker:89,Baker:91}.

\eal\label{negation-30}
\ex[]{
Kim regrets [never [having read the book]].
}
\ex[]{
We asked him [never [to try to read the book]].
}
\ex[]{
Duty made them [never [miss the weekly meeting]].
}
\zl

\eal\label{negation-31}
\ex[]{
Kim regrets [not [having read the book]].
}
\ex[]{
We asked him [not [to try to read the book]].
}
\ex[]{
Duty made them [not [miss the weekly meeting]].
}
\zl

\noindent
French \textit{ne-pas} is no different in this regard.  \textit{Ne-pas} and
certain other adverbs precede an infinitival VP:

\eal
\ex[]{
\gll [Ne           pas  [repeindre      sa    maison]] est une n\'{e}gligence. \\
     \spacebr{}ne  not  \spacebr{}paint one's house    is  a   negligence \\
\glt `Not painting one's house is negligent.'
}
\ex[]{
\gll
[R\'{e}guli\`{e}rement   [repeindre    sa   maison]]   est  une  n\'{e}cessit\'{e}. \\
regularly        to.paint    one's   house    is    a   necessity \\
\glt `Regularly painting one's house is a necessity.'
}
\zl

To capture such distributional possibilities, \citet{Kim:00} and \citet{KS:02} regard \textit{not} and \textit{ne-pas} as adverbs that modify
nonfinite VPs, not as  heads of their own functional projection as in the derivational view. The
analyses view the lexical entries for \textit{ne-pas} and \textit{not} to include at
least the
information shown in (\ref{negation-c-neg}).\footnote{Here we assume that both languages
distinguish \textit{fin(ite)} and \textit{nonfin(ite)} verb forms, but that
certain differences exist regarding lower levels of organization. For example,
\textit{prp} (\textit{present participle}) is a subtype of \textit{fin} in French,
whereas it is a subtype of \textit{nonfin} in English.}

%In \ex{1}, VP[\textit{nonfin}]:\lower4pt\hbox{\begin{avm}\@2\end{avm}}
%abbreviates a nonfinite VP whose CONTENT value is
%\lower4pt\hbox{\begin{avm}\@2\end{avm}}.  Similar abbreviations are
%used throughout.
% For ease of exposition, we will not treat cases where the negation modifies
%something other than VP, e.g.\ adverbs (\textit{not surprisingly}), NPs (\textit{not many %students}), or PPs (\textit{not in a million years}). Our analysis can accommodate such %cases by generalizing the \textsc{mod} specification in the lexical entry for \textit{not}. %In addition, we need to
%have a construction-based account for the pattern `not X but Y' (as
%similar to `both X and Y' or `neither X nor Y') in which
%\textit{not} must cooccur with the particular coordinator \textit{but} (e.g., *\textit{Kim likes %not beer}. vs.\ \emph{Kim likes not beer but wine.})}
%

%\iffalse{}
%\inlinetodostefan{check AVM changed Sag97 and numbering} jb: ok
\ea
\label{negation-c-neg}
\localvs of \emph{not} and \emph{ne-pas}:\\
\avmtmp{
[ cat|head & [\type*{adv}
               mod & !\upshape VP[\type{nonfin}]!:\1 ]\\
  cont     & [ restr \{ [ pred & neg-rel\\
                         arg1 & \1 ] \} ] ]}
%% \begin{avm} \avml
%%  %\textit{not}/\textit{ne-pas} \\
%%  \[form\ \; \q<\normalfont\textit{not}/\textit{ne-pas}\q>\\
%% \SYN|head\ \;  \[\type{adv}\\
%%                \MOD\ \; \<VP[\type{nonfin}]: \@2\>\]\\
%%   \SEM\ \; \[\RESTR\ \<\[\PRED\ \; \ \ \type{neg-rel}\\
%%                         \textsc{arg1}  \; \@2\]\>\]
%%   \]\avmr\end{avm}
\z
%\fi


\noindent %[JB: begins] CONT is added
The lexical information in (\ref{negation-c-neg}) specifies that
\textit{not} and \textit{ne-pas} modify a nonfinite VP and that this
modified VP serves as the semantic argument of the negation.
%[JB: ends]
This simple lexical specification correctly describes the
distributional similarities between English \textit{not} and French
\textit{ne-pas}, as seen from the structure in Figure~\ref{negation-not-vp-mod}.

\begin{figure}
	\begin{forest}
		sm edges
		[VP
			[V\\
			\avmtmp{
		          [head [\type*{adv}\\
                                 mod & \1 \textnormal{VP} ]]
	                }
					[not/ne-pas]]
			[\ibox{1}\,VP\\
\avmtmp{
[vform \type{nonfin}]}
					[\ldots]]]
	\end{forest}
\caption{Structure of constituent negation}\label{negation-not-vp-mod}
\end{figure}
%\fi
\noindent
This structure implies that
both \textit{ne-pas} and \textit{not} need to precede the VPs that they modify.\footnote{
This is constrained by the LP (linear precedence) Rule that specifies
that a modifier precedes the head that it modifies.}  It also
indicates that the negator does not separate an infinitival verb
from its complements, as observed from the following data:
%\footnote{The exception to this
%generalization, namely cases where \textit{pas} follows an auxiliary
%infinitive (\textit{n'avoir pas d'argent}), is discussed in section
%5.2 below.}

\eal
\ex[] {
[Not [speaking English]] is a disadvantage.
} \label{negation-35a}
\ex[*] {
[Speaking not English] is a disadvantage.
} \label{negation-35b}
\zl

\eal
\ex[]{
\gll [Ne           pas  [\sub{VP} parler  fran\c{c}ais]]  est  un grand d\'{e}savantage  en ce cas. \\
     \spacebr{}ne  not  {}        to.speak French  is  a great disadvantage  in this case \\
} \label{negation-34a}
\ex[*]{
\gll [Ne  parler  pas  fran\c{c}ais]  est  un  grand d\'{e}savantage en ce cas.\\
\spacebr{}ne  to.speak not French   is  a great disadvantage  in this case\\
} \label{negation-34b}
\zl



%\item{\bad Lee likes not Kim.}}
%
%\eenumsentence{
%\item{Lee is believed [not $_{VP[inf]}$[to like Kim]].}
%\item{\bad Lee is believed to $_{VP[inf]}$[like not Kim].}}

\noindent
Interacting with the LP constraints, the lexical specification
in (\ref{negation-c-neg}) ensures that the constituent negation
precede the VP it modifies. This predicts the
grammaticality of (\ref{negation-35a}) and (\ref{negation-34a}), where \textit{ne-pas} and \textit{not} are used as VP[\textit{nonfin}] modifiers.
(\ref{negation-35b}) and (\ref{negation-34b}) are ungrammatical, since
the modifier fails to appear in the required position -- i.e.,
before all elements of the nonfinite VP.

\iffalse{}
The lexical properties of \textit{not} thus ensures that it cannot
modify a finite VP, as shown  in (\ref{negation-36}), but it can modify any
nonfinite VP:
%, as is clear from the examples in \ex{2}:

\eal\label{negation-36}
\ex[*]{
Pat [not \jbssub{VP[fin]}[left]].
}
\ex[*]{
Pat certainly [not \jbssub{VP[fin]}[talked to me]].
}
\ex[*]{
Pat [not \jbssub{VP[fin]}[always agreed with me]].
}
\zl
%\eenumsentence{
%\item{I saw Pat acting rude and [not $_{VP[prp]}$[saying hello]].}
%\item{I asked him to [not $_{VP[\textit{bse}]}$[leave the bar]].}
%\item{Their having [not $_{VP[\textit{psp}]}$[told the truth]] was upsetting.}}

\noindent And much the same is true for French, as the
following contrast illustrates:

\eal
\ex[*]{
\gll Robin  [(ne) pas \jbssub{VP[\textit{fin}]}[aime  Stacy]]. \\
Robin  [(ne) not {\jbssub{VP[\textit{fin}]}[}likes Stacy] \\
}
\ex[]{
Il veut [ne pas publier dans ce journal]. \\
`He wants not to publish in this journal.'
}
\zl


Note that head-movement transformational analyses stipulate: (1) that negation
is generated freely, even in preverbal position in finite clauses and (2) that
a post-negation verb must move leftward because otherwise some need would be
unfulfilled---the need to bind a tense variable, the need to overcome some
morphological deficiency with respect to theta assignment, etc. On our
account, no such semantic or morphosyntactic requirements are stipulated;
instead, what is specified is a lexical selection property. There is no a
priori reason, as far as we are aware, to prefer one kind of stipulation over
the other. It should be noted, however, that our proposal only makes reference
to selectional properties that are utilized elsewhere in the grammar.
\fi


The HPSG analyses sketched here have recognized
the fact that finiteness plays a crucial role in
determining the distributional possibilities of negative
adverbs. Its main explanatory resource
has basically come from the proper lexical specification of these negative
adverbs. The lexical specification that \textit{pas} and
\textit{not} both modify nonfinite VPs has sufficed to predict their
occurrences in nonfinite environments.



\subsection{Sentential negation}
\label{sec-sentential-negation}

%As just
%illustrated, the analysis of \textit{not} and \textit{ne-pas} as nonfinite VP modifiers %provides a straightforward explanation for much of their distribution.
With respect to negation in finite clauses, there are important differences between English and French.
As we have noted earlier, it is a general fact of French that \textit{pas} must follow a finite verb, in which case the verb optionally bears negative morphology (\textit{ne}-marking):

\eal
\ex[]{
\gll Dominique (n')aime pas Alex.\\
     Dominique ne.like  \textsc{neg} Alex\\
\glt `Dominique does not like Alex.'
}
\ex[*]{
\gll Dominique pas aime Alex.\\
     Dominique \textsc{neg} like Alex\\
}
\zl
\noindent
In English, \textit{not} must follow a finite
auxiliary verb, not a lexical (or main) verb:

\eal
\ex[]{
Dominique does not like Alex.
}
\ex[*]{ Dominique not does like Alex.
}
\ex[*]{ Dominique likes not Alex.
}
\zl

In contrast to its distribution
in nonfinite clauses, the distribution
of \textit{not}
 in finite clauses concerns sentential
 negation.
  The need to distinguish between constituent and sentential negation can be
  observed from
  many grammatical environments including scope
possibilities that we can observe in an example like (\ref{negation-not-two}) (see \citealt{Klima:64, Baker:89, Warner2000a-u}).\footnote{\citet{Warner2000a-u} and \citet{BL:13}
discuss scopal interactions of negation with auxiliaries (modals) and quantifiers
within the system of Minimal Recursion Semantics (MRS).}

\ea[]{\label{negation-not-two} The president could not approve the bill.
}
\z
%
Negation here could have two different scope readings
paraphrased in the following:


\eal
\ex[]{
It would be possible for the president not to approve the bill.
}
\ex[]{
It would not be possible for the president to approve the bill.
}
\zl
%
The first interpretation is constituent negation; the second is
sentential negation. Another distinction comes from
distributional possibilities. As noted, the sentential
negation  cannot not modify a finite
VP, different from the adverb \textit{never}:

\eal
\ex[]{
Lee never/*not left.\ \ \ \ (cf.\ Lee did not leave.)
}
\ex[]{
Lee will never/not leave.
}
\zl
%
The contrast in these two sentences
shows one clear difference between \textit{never}
and \textit{not}:  The negator \textit{not} cannot
precede a finite VP, though it can freely occur
as a nonfinite VP modifier.

%, a
%property further illustrated by the following examples:
%
%\ees{\item John could [not [leave town]].
%
%\item John wants [not [to leave town]].}
%
%\ees{\item \bad John [not [left town]].
%
%\item \bad John [not [could
%leave town]].}

Another key difference between \textit{never} and \textit{not} can be found in
the VP ellipsis construction.  Observe the following
contrast \citep{KS:02}:

\eal
\label{negation-vpe-not-ex}\ex[]{
Mary sang a song, but Lee never could \trace.
}
\ex[*]{
Mary sang a song, but Lee could never \trace.
}
\ex[]{
Mary sang a song, but Lee could not \trace.
}
\zl
%
\noindent The data here indicate that \textit{not} can appear
after the VPE auxiliary, but this is not possible with
the true adverb \emph{never}.
% behaves differently from
%adverbs like \textit{never} in finite contexts, even though the two
%behave alike in nonfinite contexts.
The adverb \textit{never} is a true
diagnostic of a VP-modifier, and we use
these observed contrasts
between \textit{never} and \textit{not} to reason about what the properties of the negator \textit{not} must be.
%
%\iffalse{
%

We saw the lexical representation for constituent negation
\textit{not} in (\ref{negation-c-neg}) above. Unlike the
constituent negator, the sentential negator \textit{not} typically
 follows a finite auxiliary verb. In this respect,
   \textit{too}, \textit{so}, and \textit{indeed} also behave alike:
\eal
\ex[]{ Kim will not read it.
}
\ex[]{
Kim will too/so/indeed read it.
}
\zl
%
These expressions are used to
reaffirm the truth of the sentence in question and
follow a finite auxiliary verb.  This implies
that the sentential \emph{not} in English
form a group of adverbs (which we call Adv\jbsub{\textsc{i}}) that combine with a
preceding auxiliary verb (see \citealt{Kim:00}).
%The negator and these reaffirming expressions form
%a unit with the finite auxiliary, resulting in a lexical-level construction
 %, but shows
%different syntactic properties (while
% constituent negation need not follow an auxiliary
% as in \textit{Not eating gluten is dumb}).

Noting the properties of \emph{not} that we have discussed so far,
 the HPSG analyses of \citet{AG:97},
\citet{Kim:00}, and \citet{Warner2000a-u}
have taken this group of adverbs (Adv\jbsub{\textsc{i}}) including the sentential negation \emph{not} to function as the complement of a finite auxiliary verb via the following lexical rule:\footnote{The symbol $\oplus$ stands for the function \emph{append}, i.e., a relation that concatenates two lists.}

%\inlinetodostefan{Is the lowered i a capital or a small i?}  jb: small cap (figure 2 too)
\ea
Negative adverb-complement lexical rule:\\
\avmtmp{
[\type*{fin-aux}
  synsem|loc|cat & [ head & [aux   & $+$\\
                             vform & \type{fin}]\\
                     comps & \1 ]] } $\mapsto$
\avmtmp{
[\type*{neg-fin-aux}
  synsem|loc|cat & [ head & [aux   & $+$\\
                             vform & \type{fin}]\\
                     comps & < Adv\sub{I} > \+ \1 ]] }
%% 			\begin{avm}
%% 				\[\type{fin-aux}\\
%% 				syn \; \[head \; \ \[aux \; $+$\\
%%                                vform \; \type{fin}\]\\
%% 				     val|comps  \; \textit{L}\]\]
%% 			\end{avm}
%% \ \  $\mapsto$\  \
%% 	\begin{avm}
%% 			\[\type{neg-fin-aux}\\
%% 			syn \ \; \[head \ \;  \[aux \; $+$\\
%%                              vform \; \type{fin}\\
%% 	            		     neg \; $+$\]\\
%% 				     val|comps \ \; \<Adv\jbsub{\textsc{i}}\> \; \ $\oplus$ \; \ \textit{L}\]\]
%% 		\end{avm}
\z
%
This lexical rule specifies that when the input is a finite auxiliary verb,
the output is a neg-finite auxiliary (\textit{fin-aux} $\rightarrow$ \textit{neg-fin-aux})
that selects Adv\jbsub{\textsc{i}} as an additional complement. This would then
license a structure like in Figure~\ref{negation-fig:6}.


%\iffalse{}
\begin{figure}
	\begin{forest}
		sm edges
		[VP\\
		\avmtmp{
	          [head|vform  \type{fin}\\
		   subj  < \1 NP >\\
		   comps < > ]}, l sep*=1.5
			[V\\
			\avmtmp{
			[\type{neg-fin-v}\\
			head [ aux & $+$\\
                               vform & \type{fin}]\\
		        subj  < \1 NP >\\
		        comps < \2 \textnormal{Adv}\sub{I}, \3 VP > ]}
%				arg-st \; \< \@{1}\,NP{,} \@{2}\,$\textnormal{Adv}_\textnormal{I}${,} %\@{3}\,VP\>
				[could]]
			[\ibox{2} Adv\sub{I}
				[not]]
			[\ibox{3} VP
				[approve the bill,roof]]]
	\end{forest}
\caption{Structure of sentential negation}\label{negation-fig:6}
\end{figure}

%\fi
As shown in Figure~\ref{negation-fig:6}, the negative finite auxiliary
verb \textit{could} combines with two complements, the negator
\textit{not} (Adv\jbsub{\textsc{i}}) and the VP \textit{approve the bill}.
This combination results in a well"=formed head"=complement construct.
By treating \textit{not} as both a modifier (constituent negation)
and a lexical complement of a finite auxiliary (sentential negation), we thus can
account for the scope differences in (\ref{negation-not-two}) with the
following two possible structures:

\eal
\label{negation-two-int}
\ex[]{
The president could [not [approve the bill]].
}
\ex[]{
The president [could] [not] [approve the bill].
}
\zl
%
In (\ref{negation-two-int}a), \textit{not} functions as a modifier to
the base VP, while  in (\ref{negation-two-int}b), whose partial structure is
given in Figure~\ref{negation-fig:6}, it is a sentential
negation serving as the complement of \emph{could}.

The present analysis allows us to have a simple account for other related phenomena,
including the VP ellipsis discussed in (\ref{negation-vpe-not-ex}). The key point
was that unlike \textit{never}, the sentential negation can
host a VP ellipsis.  The VP ellipsis after \textit{not} is
possible, given that any VP complement of an auxiliary
verb can be unexpressed, as
specified by the following lexical rule \citep{Kim:00}:


\ea
\label{negation-vpe-cxt}
Predicate ellipsis lexical rule:\\
\avmtmp{
[\type*{aux-v-lxm}
 arg-st &  < \1 XP, \2 YP > ]}  $\mapsto$
\avmtmp{
[\type*{aux-ellipsis-wd}
 comps  & < >\\
 arg-st & < \1 XP, \2 ! YP[\type{pro}]! >]}
\z
%
%
What the rule in (\ref{negation-vpe-cxt}) tells us is that an auxiliary verb selecting two arguments
can be projected into an elided auxiliary verb (\type{aux-ellipsis-wd}) whose second argument
is realized as a small \emph{pro}, which in definition
behaves like a slashed expression not mapping into the syntactic grammatical
 function \COMPS. This analysis would then license
the following structure in Figure~\ref{negation-could-not}:
%
%following structure:
%, leaving
%the sentential complement intact.
%
%
%
%\todostefan{rephrase, reference to figure}
%\iffalse{}
\begin{figure}
	\begin{forest}
		sm edges
		[VP
			[V\\
			\avmtmp{
			[head|aux $+$\\
			 subj  < \1 >\\
		         comps < \2\textnormal{Adv}\sub{I} >\\
			 arg-st < \1, \2, VP![\type{bse}]! > ]}
					[could]]
			[\ibox{2} Adv\sub{I}\\
					[not]]]
	\end{forest}
\caption{A licensed VP ellipsis structure}\label{negation-could-not}
\end{figure}
%\fi

As represented in Figure~\ref{negation-could-not}, the auxiliary verb \textit{could} forms a well-formed head"=complement construct with \textit{not} while its
VP[\textit{bse}] is unrealized (see \citealt{Kim:00, KS:08} for
detail). The sentential negator \textit{not} can ``survive'' VP ellipsis because it can be
licensed in the syntax as the complement of an auxiliary, independent
of the following VP.  However, an adverb like \textit{never} is only
licensed as a modifier of VP. Thus if the VP were elided, we would have the hypothetical
structure like the one in Figure~\ref{negation-fig-could-never}:
%\iffalse{}
\begin{figure}
	\begin{forest}
		sm edges
		[VP
			[V{[\aux $+$]}
				[could]]
			[*VP
				[Adv{[\textsc{mod} VP]}
					[never]]]]
	\end{forest}
\caption{Ill-formed structure of the head-mod construction}\label{negation-fig-could-never}
\end{figure}
%\fi
the adverb \textit{never} modifies a VP through the feature \textsc{MOD},
which guarantees that the adverb requires the head VP that it
modifies. In an ellipsis structure, the absence of such a VP means
that there is no VP for the adverb to modify.  In other words, there
is no rule licensing such a combination -- predicting the
ungrammaticality of *\textit{has never}\is{adverb},  as opposed to \textit{has not}.


The HPSG analysis just sketched here can be easily extended to French negation, whose
data we repeat here.

\eal
\ex[*]
{
\gll Robin  ne pas aime  Stacy. \\
Robin  ne  \textsc{neg}  likes  Stacy \\
\glt {`Robin does not like Stacy.'}\label{negation-pas-good-a}}
 \ex[ ]
 {
 \gll Robin  (n')aime  pas  Stacy. \\
Robin  ne.likes  \textsc{neg}  Stacy \\
\glt{`Robin does not like Stacy.'}\label{negation-pas-good-b}}
\zl

\noindent
Unlike the English negator \textit{not}, \textit{pas} must follow a
finite verb. Such a distributional contrast has motivated verb
movement analyses (see \citealt{Pollock:89,Zanuttini:01}).
By contrast, the present HPSG analysis is cast
in terms of a lexical rule that maps a finite verb into a verb
with a certain adverb like \textit{pas} as an additional complement.
 The idea of converting modifiers in French into
complements has been independently proposed by \citet{Miller92d-u} and
\citet{AG:94} for French adverbs including
\textit{pas}.  Building upon this
previous work, \citet{AG:97} and \citet{Kim:00}
allow the adverb \textit{pas} to function
as a syntactic complement of a finite verb in French.\footnote{Following \citet{AG:94}, we could assume \textit{ne} to
be an inflectional affix which can be optionally realized
in the output of the lexical rule in Modern French.}
This output verb \textit{neg-fin-v} then allows the negator \textit{pas} to function
as the complement of the verb \textit{n'aime}, as represented in Figure~\ref{negation-pas-st}.

%\iffalse{}
\begin{figure}
\begin{forest}
sm edges
[VP\\
 \avmtmp{
 [head|form \type{fin}\\
  subj < \1 NP >\\
  comps <> ]}, l sep*=1.5
  [V\\
   \avmtmp{
   [\type*{neg-fin-v}\\
    head|vform \type{fin}\\
    subj  < \1 NP > & \\
    comps < \2\textnormal{Adv}\sub{I}, \3 NP > ]}
				%arg-st \; \< \@{1}\,NP{,} \@{2}\,$\textnormal{Adv}_\textnormal{I}${,} %\@{3}\,NP\>
   [n'aime;ne.likes]]
 [\ibox{2} Adv\sub{I}
	[pas;\textsc{neg}]]
 [\ibox{3} NP
	[Stacy;Stacy]]]
\end{forest}
\caption{Partial structure of (\ref{negation-pas-good-b})}\label{negation-pas-st}
\end{figure}
%\fi

%\noindent
The analysis also explains the position of \textit{pas} in
finite clauses. The placement of \textit{pas} before a finite verb
in (\ref{negation-pas-good-a})
 is unacceptable, since
\textit{pas} here is used not as a nonfinite VP modifier, but as
a finite VP modifier. But due to the
present analysis which allows \textit{pas}-type negative adverbs
to serve as the complement of a finite verb,
\textit{pas} in (\ref{negation-pas-good-b}) can be the sister of the finite verb
\textit{aime}.
%\footnote{Of course, this
%word ordering
%conforms to the independent LP rule that a lexical head precedes
%all complements.}

Given that the conditional, imperative, subjunctive,
and even present participle verb forms in French are finite, we
can expect that \textit{pas} cannot precede any of these verb
forms:


\eal
\ex[]{
\gll Si j'avais de l'argent, je n'ach\`{e}terais pas de voiture. \\
     if I.had   of money      I ne.buy \textsc{neg} a car\\
\glt `If I had money, I would not buy a car.'
}
\ex[*]{
\gll Si j'avais de l'argent, je ne pas   ach\`{e}terais de voiture.\\
     if I.had   of money      I ne \textsc{neg} buy     a car\\
}
\zl

\eal
\ex[]{
\gll Ne mange pas ta soupe.  \\
     ne  eat  \textsc{neg} your soup\\
\glt `Don't eat your soup!'
}
\ex[*]{
\gll Ne pas mange ta soupe.\\
     ne \textsc{neg} eat your soup\\
%\glt
}
\zl

\eal
\ex[]{
\gll Il est important que vous ne r\'{e}pondiez pas. \\
     it  is important  that you ne answer \textsc{neg} \\
\glt `It is important that you not answer.'
}
\ex[*]{
\gll Il est important que vous ne pas r\'{e}pondiez.\\
     it is   important that you ne \textsc{neg} answer \\
}
\zl

\eal
\ex[]{
\gll Ne parlant pas Fran\c{c}ais, Stacy avait des difficult\'{e}s. \\
     ne speaking  \textsc{neg} French    Stacy   had  of  difficulties\\
\glt `Not speaking French,  Stacy had difficulties.'
}
\ex[*]{
\gll Ne pas parlant Fran\c{c}ais, Stacy avait des difficult\'{e}s.\\
     ne \textsc{neg} speaking   French    Stacy   had  of  difficulties\\
}
\zl

Note that this non-derivational analysis reduces the differences between
French and English negation to a matter of lexical properties.
The negators \textit{not} and \textit{pas} are identical in that they both are
VP[\textit{nonfin}]-modifying adverbs. But they are different with respect to
which verbs can select them as complements:  \textit{not} can be the
complement of a finite auxiliary verb, whereas \textit{pas} can be the
complement of any finite verb.  So the only difference between \emph{not}
and \emph{pas} is the morphosyntactic value [\AUX\ $+$], and this induces
the difference in positioning the negators in English and French.



%
%In the nonderivational analysis sketched here, the required
%notion was the independently motivated morphosyntactic feature AUX
%(motivated from NICE constructions in English and possibly from
%AUX-to-COMP and clitic climbing in old French).
%Interacting with the notion of conversion, this elementary
%morphosyntactic feature has been able to capture the
%effects of the verb's intrinsic property in determining
%the positioning of the negative markers \textit{pas} and \textit{not}.



%The key fact is that the English negative adverb \textit{not} leads a double life: one as a
%nonfinite VP modifier, marking constituent negation, and the other
%as a complement of a finite auxiliary verb, marking sentential
%negation.\is{nonfinite}\is{negation} Constituent negation
%is the name for a construction where negation combines with some
%constituent to its right, and negates exactly that constituent (see Kim and Sag 2002, Kim and Sells %2008):

%
%The English negative adverb \textit{not} leads a double life: one as a
%nonfinite VP modifier, marking constituent negation, and the other
%as a complement of a finite auxiliary verb, marking sentential
%negation.\is{VP!nonfinite}\is{negation} Constituent negation
%is the name for a construction in which negation combines with some
%constituent to its right, and negates exactly that constituent.

\section{Morphological negative}

As noted earlier, languages like Turkish and Japanese employ morphological negation where the negative marker behaves like a suffix. Consider
a Turkish and a Japanese example:
% again:

\eal
\ex
\label{negation-turkish-jap}
\gll Git-me-yece\~{g}-$\varnothing$-im \\
     go-\textsc{neg-fut-cop}-\textsc{1sg} \\
\glt `(I) will not come.'
\ex
\gll kare-wa kinoo kuruma-de ko-na-katta. \\
     he-\textsc{top} yesterday car-\textsc{inst} come-\textsc{neg}-\textsc{pst} \\
\glt `He did not come by car yesterday.'
\zl

\noindent
As shown by the examples, the sentential negation of Turkish
and Japanese employ
morphological suffixes \textit{-me} and \textit{-na},
respectively.
It is possible to state the ordering
of these morphological negative markers in configurational
terms by assigning an independent syntactic status to them.
But it is too strong a claim to
take the negative suffix \textit{-me} or \textit{-na}  to be an independent syntactic element,
and to attribute its positional possibilities to syntactic constraints
such as verb movement and other configurational notions (see \citealt{kelepir} for
Turkish and \citealt{Kato:97,Kato:00} for Japanese).
%Kelepir 2001
%Japanese and Turkish show other clear examples of morphological negation.
%
In these languages, the negative affix acts just like
other verbal inflections in numerous respects.
%
%\enumsentence{
%\shortex{4}
%
%{T\"{u}rk-les-tir-il-me-mis-ler-den-siniz.}
%{turk-become-CAUS-PASS-NEG-PSP-PLUR-ABL-COP}
%{`You are of those who didn't have themselves Turkified.' (van
%Schaaik 1994:39)}}
%
%\noindent
%
%
The morphological status of
these negative markers comes from their morphophonemic alternation.
For example, the vowel of the Turkish negative suffix \textit{-me} shifts from open to closed when followed by the
future suffix, as in \textit{gel-mi-yecke} `come-\NEG-\FUT'.  Their
strictly fixed position also indicates their morphological
constituenthood. Though these languages allow a rather free permutation of
syntactic elements (scrambling), there exist strict ordering restrictions among
verbal suffixes including the negative suffix, as observed from the following:

\eal
\ex
\gll tabe-sase-na-i/*tabe-na-sase-i \\
     eat-\textsc{caus}-\textsc{neg}-\textsc{npst}/eat--\textsc{neg}-\textsc{caus}-\textsc{npst} \\

\ex
\gll tabe-rare-na-katta/*tabe-na-rare-katta \\
     eat-\textsc{pass}-\textsc{neg}-\textsc{pst}/eat-\textsc{neg}-\textsc{pass}-\textsc{pst} \\

\ex
\gll tabe-sase-rare-na-katta/*tabe-sase-na-rare-katta \\
     eat-\textsc{caus}-\textsc{pass}-\textsc{neg}-\textsc{pst}/eat-\textsc{caus}-\textsc{neg}-\textsc{pass}-\textsc{pst}\\
\zl

\noindent
The strict ordering of the negative affix here is a matter of morphology.
If it were a syntactic concern, then
the question would arise as to why
there is an obvious contrast in the ordering principles
of morphological and syntactic constituents, i.e., why the ordering
rules of morphology are distinct from the ordering rules of syntax. The
simplest explanation for this contrast is to accept
the view that morphological constituents including the negative marker
are formed in the lexical component and hence have no syntactic
status (see \citealt{Kim:00} for detailed discussion).

This being noted, it is more reasonable to assume that the placement of a
negative affix is regulated by morphological principles, i.e., by
the properties of the morphological negative affix itself.
  %
 %\citet{Prz:00} focuses on the non-local genitive of negation in Polish, where the object argument is not accusative but genitive-%marked with the presence of negative marker as in (\ref{negation-genitive-1a}). The assignment of genitive case to the object is also effective %in
 % the unbounded relation as shown in (\ref{negation-genitive-1b}):
%
%\eal
%\ex \label{negation-genitive-1a}
%\gll Nie lubi\c{e} Marii/*Mari\c{e}. \\
%     not like-1st.\textsc{sg} Mary-\textsc{gen}/Mary-\textsc{acc}\\
%\glt `I don't like Mary.'
%\ex \label{negation-genitive-1b}
%\gll Mog\c{e} nie chcie\'{c} tego napisa\'{c}. \\
%     may-1st.\textsc{sg} not want-\textsc{inf} this-\textsc{gen} write-\textsc{inf}\\
%\glt `I may not want to write this.'
%\zl
  %
%To account for this kind of phenomena, \citet{Prz:00} develops
%an HPSG-based analysis with the assumption that the combination of the
%negative morpheme with the verb stem introduces the feature \textsc{neg}. This feature
%tightly interacts with the mechanism of argument composition and construction-based case assignment (or satisfaction).
%
The process of adding a negative morpheme to a lexeme can be modeled
straightforwardly by a lexical rule given in the following (see \citealt{Kim:00,Crowgey:12}):

\ea
\label{lr-neg-word-formation}
Negative word formation lexical rule:\\
\avmtmp{
[\type*{v-lxm}
 phon < \1 >\\
 synsem|loc|cont \2 &]} $\mapsto$
\avmtmp{
[\type*{neg-v-lxm}
 phon < !\textbf{F}\jbsub{\type{neg}}(\1)! >\\
 synsem|loc [ cat|head|pol \type{neg}\\
              cont|restr \{ [ pred & neg-rel\\
                              arg1 & \2 ] \} ] & ]
}
\z
%
%
As shown here, any verb lexeme can be turned into a verb with the negative
morpheme attached. That is, the language-particular definition for
\textbf{F}\jbsub{{\type{neg}}} will ensure that an appropriate
negative morpheme is attached to the lexeme. For instance, the
suffix \suffix{ma} for Turkish
and \suffix{na} for Japanese will be attached to the
 verb lexeme, generating
the verb forms in (\ref{negation-turkish-jap}).\footnote{In a similar
manner, \citet{PK:99} and \citet{Prz:00, Prz:01}
discuss aspects of Polish negation which is realized as the prefix
  \emph{nie} to a verbal expression.}
%
%\citet{PK:99} and \citet{Prz:00, Prz:01}
%discuss aspects of Polish negation which is realized as the prefix
 % \emph{nie} to a verbal expression.

This morphological analysis can be extended to the negation of languages
like Libyan Arabic, as discussed in \citet{BK:12}. The language
 has a bipartite realization of negation, the proclitic \emph{ma-} and the enclitic -\u{s}:

%\eal
\ea
\gll la-wlaad ma-m\u{s}uu-\u{s} li-l-madrsa. \\
     the-boys \NEG-go.\pst.3.\pl-\NEG{} to-the-school\\
\glt `The boys didn't go to the school.'
\z
 %
 As \citet{BK:12} did, we could take these clitics as affixes and generate
 a negative word. Given that the function
 f\sub{\type{neg}} in Libyan Arabic allows the attachment of the negative prefix
 \textit{ma-} and the suffix -\textit{\u{s}} to the verb
 stem \emph{m\u{s}uu}, we would have the following output:\footnote{\citet{BK:12} note that
 the suffix -\textit{\u{s}} is not realized when a negative clause
 includes an n-word or an NPI (negative polarity item). See
 \citet{BK:12} for further details.}
 % The formulation given in
 %\citet{BK:12} is slightly different from the one given here, but both
 % have the same effects.}


\ea
\avmtmp{
[\type*{v-lxm}
 phon <\normalfont{m\u{s}uu}>\\
 synsem|loc|cont \2 &]} $\mapsto$
\avmtmp{
[\type*{neg-v-lxm}
 phon <\normalfont{ma-m\u{s}uu-\u{s}}>\\
 synsem|loc [ cat|head|pol \type{neg}\\
              cont|restr \{ [ pred & neg-rel\\
                              arg1 & \2 ] \} ] & ]
}
\z
%\inlinetodostefan{This does not work. The left side shows an input form. It looks like a %lexical
%  rule but it is two forms. This is very confusing. I split this into two forms and related %them. Please turn it into \citet{Sag97a} notation.}
%jb: this part is revised to make it  simple
%
%
%
%
%\ea
%\begin{avm}
%\small
%\[%\type{v-lxm}\\
%  form\ \q<\normalfont{m\u{s}uu}\q>\\
%  \SEM\| \@2\]  \ \  $\mapsto$\  \ \  \[form\ \q<\normalfont{ma-m\u{s}uu-\u{s}}\q>\\
%                                     \SYN\|head\|pol\ \type{neg}\\
%                                    \SEM\ \[\RESTR\<\[\PRED\ \ \ \type{neg-rel}\\
%                                            \textsc{arg1} \ \ \@2\]\>\]\]
%\end{avm}
%\z
%One could take these clitics as affixes and generate a negative word by applying the lexical %rule in (\ref{lr-neg-word-formation}) to (\mex{1}a), as was suggested by %\citet{BK:12}.\footnote{The formulation given in \citet{BK:12} is slightly different from %the one given here, but both
% have the same effects.}
%The result is shown in (\mex{1}b):
%
%\eal
%\ex add your avms here, please
%\avmtmp{
%}
%\ex add your avms here, please
%\zl
%
%
% The only thing we need to define here is the function
% f\sub{\type{neg}}
% in the language that allows the attachment of the prefix
% \textit{ma-} and the suffix -\textit{\u{s}} to the verb
% stem \emph{m\u{s}uu}.\footnote{\citet{BK:12} note that
% the suffix -\textit{\u{s}} is not realized when a negative clause
% includes an n-word or an NPI (negative polarity item). See
% \citet{BK:12} for further details.}
%
%
%
 %
 %
 %
 % explores a morphology-based analysis with the introduction
 % of the feature \textsc{pol} (polarity) to identify strong and weak negative
 % words in the language.

\iffalse{
In the
construction-based HPSG, we could take this as an inflectional
construction.  The negative marker, as we have seen in Turkish and Japanese, is realized as a suffix
attached to the verb root. The resulting combination is not
a word-level entity but a verb stem to which an aspectual or tense marker can be attached. We could thus take such a morphological process as an inflectional
one. For instance, Figure~\ref{negation-fig:2} could be a morphological
construction in Turkish.\footnote{See \citet{Sag:12} and \citet{Hilpert:16} for a construction-based approach to
inflectional as well as derivational processes.}


This inflectional construction ($\uparrow$\textit{infl-cxt}) allows us to generate a Turkish inflection construct like \textit{ser-me} `like-\NEG' (in (\ref{negation-1a})) from the v-lexeme \textit{ser-} with the change in the root's meaning into a sentential negation. The morphological function \textit{F}\jbsub{\textsc{neg}} could ensure that the vowel of the negative affix \textit{me} is subject to phonological changes depending on its environment. If it is followed by a consonant-initial morpheme, it undergoes vowel harmony with the vowel in the preceding syllable (e.g., \textit{yika-n-ma-di} `wash-\REFL-\NEG-\PST'). If it is followed by a vowel-initial morpheme, its vowel drops (gel-m-iyor `come-\NEG-\PROG')  \citep[see]{kelepir}.\footnote{As
 for a way of capturing the ordering of suffixes within this kind of system,
 see \citet{Kim:16}.}
}\fi

The lexicalist HPSG analyses sketched here
 have been built upon the
thesis that autonomous (i.e., non-syntactic) principles govern the
distribution of morphological elements \citep{BM:95}.
The position of the morphological negation is simply
defined in relation to
the verb stem it attaches to. There are no syntactic operations such
as head-movement or multiple functional projections in forming
a verb with the negative marker.\footnote{The lexical rule-based
approach here can be extended to a construction-based HPSG
approach or a constructionist approach. See
\citet{Sag:12} and \citet{Hilpert:16} for a construction-based
approach to morphological processes.}



\section{Negative auxiliary verb}

Another way of expressing sentential negation, as noted earlier, is to employ
a negative auxiliary
verb. Some head-final languages like Korean and Hindi employ
negative auxiliary verbs. Consider a Korean example:

\ea
\gll John-un ku chayk-ul ilk-ci anh-ass-ta. \\
     John-\textsc{top} that book-\textsc{acc} read-\textsc{conn} \textsc{neg}-\textsc{pst}-\textsc{decl}  \\
\glt `John did not read the book.'
\z
%
%\ex
%\gll anil  kitaab\~{e}  nah\~{\i}\~{\i}  becegaa.\\
%     Anil-\textsc{nom} book-\textsc{pl}  not sell-\textsc{fu}
%\glt `Anil will not sell the books.'
%\zl
%
\noindent
The negative auxiliary in head-final languages like Korean
typically appears clause"=finally, following the invariant form of the lexical verb.
In head"=initial SVO languages, however, the negative auxiliary
almost invariably occurs immediately before the lexical verb
\citep{Payne:85}. Finnish also exhibits this property \citep{Mitchell:91}:

\ea
\gll Min\"{a} e-n puhu-isi. \\
     I.\textsc{nom} \textsc{neg}-\textsc{1sg} speak-\textsc{cond} \\
\glt `I would not speak.'
\z

\noindent
These negative auxiliaries have syntactic status: they can be
inflected, above all. Like other verbs, they can also be marked
with verbal inflections such as agreement, tense, and mood.

In dealing with auxiliary negative constructions,
most of the derivational approaches have
followed Pollock's and Chomsky's analyses in factoring out grammatical
information (such as tense, agreement, and mood) carried by lexical items into various different phrase-structure nodes (see, among others, \citealt{Hagstrom:02}, \citealt{Han:07} for Korean, and \citealt{Vasishth:00} for Hindi).
This derivational view has
been appealing in that the configurational structure for English-type
languages could be applied even for languages with different types
of negation. However, issues arise about how to address the grammatical
properties of auxiliary negatives, which are quite different from the
other negative forms.
%
%
%could explain
%different types of negation.   However,
%questions arise with respect to how the
%
%
%issues arise from the fact that it misses the basic properties
%of this type of negation which, for example, differentiate it from
%morphological negation (i.e., double negation, lexical
%idiosyncrasies, phonological restriction, etc).

%
%\footnote{See \citep{Nino:94} for arguments against a derivational analysis
%for Finnish negative auxiliary such as that of \citep{Mitchell:91}.}


%
%and an independently
%motivated construction for other types of auxiliary verbs.
The Korean negative auxiliary displays all the key properties of auxiliary verbs in the language. For instance, both the canonical auxiliary verbs and 
the negative auxiliary alike require the preceding lexical verb to be marked with a specific verb form (\vform), as illustrated
in the following:

\eal
\ex\label{negation-14a}
\gll ilk-ko/*-ci siph-ta. \\
     read-\textsc{conn}/\textsc{conn} would.like-\textsc{decl} \\
\glt `(I) would like to read.'

\ex\label{negation-14b}
\gll ilk-ci anh-ass-ta. \\
     read-\textsc{conn} \textsc{neg}-\textsc{pst}-\textsc{decl} \\
\glt `(I) did not read.'
\zl
\noindent
The auxiliary verb \textit{siph-} in (\ref{negation-14a}) requires a
\textit{-ko}-marked lexical verb while the negative auxiliary
 verb \textit{anh-} in (\ref{negation-14b}) asks for a \textit{-ci}-marked lexical 
 verb. This shows that the negative is also an auxiliary verb in the language. 

In terms of syntactic structure, there
are two possible analyses.  One is to assume that the negative auxiliary takes a VP complement and the other is to claim that it forms a verb complex with
an immediately preceding lexical verb, as represented in Figures~\ref{negation-fig:3a} and~\ref{negation-fig:3b}, respectively
\citep{Chung98a-u, Kim:16}.
%\footnote{Another possibility is to assume that the
%auxiliary verb are in the sisterhood relationship with the following
%lexical verb and its putative complement(s). For this option, see ???}
%
%\iffalse{}
\begin{figure}
	\begin{subfigure}[b]{0.48\textwidth}
\centering
		\begin{forest}
%		sm edges
			[VP
				[VP
					[ \dots\ ]
					[V {[\textsc{vform} \textit{ci}]}
					]
					]
				[V {[\textsc{aux $+$}]}
					[anh-ta\\ \textsc{neg-decl}]
				]
			]	
		\end{forest}
	\caption{VP structure}\label{negation-fig:3a}
		\end{subfigure}	
\hfill
	\begin{subfigure}[b]{0.48\textwidth}
\centering
		\begin{forest}
%		sm edges
			[VP, s sep=1cm
				[ \dots\ ]
				[V
					[V {[\textsc{vform} \textit{ci}]}
						[\dots]]
					[V {[\textsc{aux $+$}]}
						[anh-ta\\ \textsc{neg-decl}]]]]
		\end{forest}
	\caption{Verb-complex structure}\label{negation-fig:3b}	
		\end{subfigure}
	\caption{Two possible structures with the negative auxiliary}
\end{figure}
%\fi

%\noindent
The distributional properties of the negative auxiliary in the language, however, support
 a complex predicate structure (cf.\ Figure~\ref{negation-fig:3b}) in which the negative auxiliary verb
forms a syntactic/semantic unit with the preceding lexical verb.
For instance, no adverbial expression, including
a parenthetical adverb, can intervene between
the main and the auxiliary verb, as illustrated by the
following:

\ea
\gll Mimi-nun          (yehathun)           tosi-lul          (yehathun)           ttena-ci            (*yehathun) anh-ass-ta. \\
     Mimi-\textsc{top} \hspaceThis{(}anyway city-\textsc{acc} \hspaceThis{(}anyway leave-\textsc{conn} \hspaceThis{(*}anyway \textsc{neg}-\textsc{pst}-\textsc{decl} \\
\glt `Anyway, Mimi didn't leave the city.'
\z
%
Further, in an elliptical construction, the elements of a verb complex
 always occur together. Neither the lexical  verb nor the auxiliary verb alone can serve
as a fragment answer to the given polar question:
% The two verbs
%must occur together.

%\inlinetodostefan{Does \textsc{del} stand for \textsc{decl}?}
%jb:  it means delimiter. in the earlier version of footnote 1, these kinds of nonstandard % abbreviations were given. as an non-standard abbreviations.
%
%
\eal
\label{negation-fragment}
\ex[]{
\gll Kim-i hakkyo-eyse pelsse tolawa-ss-ni? \\
     Kim-\textsc{nom} school-\textsc{src} already return-\textsc{pst}-\textsc{que} \\
\glt `Did Kim return from school already?'
}
\ex[]{
\gll ka-ci-to anh-ass-e.\\
     go-\textsc{conn}-\textsc{del} \textsc{not}-\textsc{pst}-\textsc{decl} \\
\glt `(He) didn't even go.'
}
\ex[*]{
\gll ka-ci-to.\\
     go-\textsc{conn}-\textsc{del} \\
}
\ex[*]{
\gll anh-ass-e. \\
\textsc{neg}-\textsc{pst}-\textsc{decl}\\
}
\zl

% \end{xlist}
%\end{exe}
%
%
%\ex[*] \gll ka-ci-to.\\
%   go-\textsc{conn}-\textsc{del}\\
%   \glt `(int.) not even go'%
%
%\ex[*] \gll anh-ass-e.\\
%   \textsc{neg}-\textsc{pst}-\textsc{decl}
%  \glt `(int.) not even go'
%
As shown in (\ref{negation-fragment}c) and (\ref{negation-fragment}d) here, neither the lexical verb nor the auxiliary verb alone can serve as an independent fragment answer. The two
verbs must appear together as given in (\ref{negation-fragment}b). These constituenthood
properties again indicate that the negative auxiliary forms
a syntactic unit with a preceding lexical  verb in Korean.

To address these complex verb properties, we could assume that
an auxiliary verb forms a complex predicate, licensed by
the following schema (see \citet{Bratt:96}, \citet{Chung:98}, \citet{Kim:16}):


%\iffalse{}
\ea
\label{negation-hd-lex-cxt}
\head-\LIGHT Schema:\\
\avmtmp{
[\type*{hd-light-phrase}
 comps & \1\\
 light & $+$\\
 head-dtr & \2\\
 dtrs     & < \3 [light & $+$\\
                  comps & \1 ], \2 [light & $+$\\
                                    comps & < \3 > \+ \1 ] > ]}
%% \begin{avm}
%% \[\tp{hd-light-cxt}\\
%% comps & \textit{L}\\
%% light & $+$\]    $\rightarrow$ \@1 \[light & $+$\\
%%                                   comps & \textit{L}\], H \[light & $+$\\
%%                                                             comps & \q<\@1\q> $\oplus$ \textit{L}\]
%% \end{avm}
\z
%\fi

\noindent   This construction rule means that a \LIGHT\ head
expression combines with a \LIGHT\ complement, yielding
a light, quasi-lexical constituent \citep{BW:13}.
When this combination happens,
there is a kind of argument composition: the \COMPS\  value of this
lexical complement is passed up to the resulting mother.
The constructional constraint thus induces the effect of argument composition in syntax,
as illustrated by Figure~\ref{kor-v-complex}. 
%
%\footnote{The V$'$ is just a
%notational variant to indicate that it is a syntactic complex predicate.}
%\iffalse{}
\begin{figure}
	\begin{forest}
		[V\\
		\begin{avm}
			\[\type{hd-light-cxt}\\
			head \; \@{3}\\
			light \; $+$\\
			comps \; \@{2}\,\<NP\>\]
		\end{avm}, l sep*=3
			[\ibox{1}\,V\\
			\begin{avm}
				\[head \|vform & ci\\
				  light \; $+$\\
				  comps \; \@{2}\,\<NP\>\]
			\end{avm}, edge label={node[midway,left,outer sep=1.5mm,]{Lexical arg.}}
				[ilk-ci\\read-\textsc{conn},tier=word]]
			[V\\
			\begin{avm}
				\[head \; \@{3}\\
				comps \; \<\@{1}\,V\> \; $\oplus$ \; \@{2}\,\<NP\>\]
			\end{avm}, edge label={node[midway,right,outer sep=1.5mm,]{H}}
					[anh-ass-ta\\ \textsc{neg-pst-decl},tier=word]]]
	\end{forest}
\caption{An instance (construct) of the \hdlight}\label{kor-v-complex}

\end{figure}
%\fi
%
The auxiliary verb \textit{anh-ass-ta} `\NEG-\PST-\DECL' combines with the matrix verb \textit{ilk-ci} `read-\conn',
creating a well-formed head-light construct.
%\footnote{The negative auxiliary
%verb selects two arguments, a subject and a lexical  verb. See \citet{Kim:16} for
%a detailed analysis.}
Note that the resulting construction metaphorically inherits the
\COMPS\ value from that of the lexical complement \textit{ilk-ci} `read-\conn' in accordance with the structure-sharing
imposed on by the \hdlight\
in (\ref{negation-hd-lex-cxt}). That is, the \hdlight\ licenses
the combination of an auxiliary verb with its lexical verb, while
inheriting the lexical  verb's complement value as argument composition.
The present system thus allows the argument composition at the syntax level, rather than in the lexicon.
%\footnote{??? With respect
%to the argument composition, we could it happens at the lexical
%level. Bender and Adam}

The HPSG analyses have taken the negative auxiliary in Korean
to select a lexical verb, whose combination forms a verb complex
structure. The present analysis implies that there is no upper limit for the
number of auxiliary verbs to
occur in sequence, as long as each combination observes
the morphosyntactic constraint on the preceding auxiliary expression. Consider
the following:

\eal\ex \gll sakwa-lul [mek-ci anh-ta]. \\
apple-\textsc{acc} eat-\textsc{conn} \textsc{neg}-\textsc{decl} \\
\glt`do not eat the apple'

\ex \gll sakwa-lul [[mek-ko siph-ci] anh-ta]. \\
apple-\textsc{acc} eat-\textsc{conn} wish-\textsc{conn} \textsc{neg}-\textsc{decl} \\
\glt`would not like to eat the apple'

\ex \label{negation-20c} \gll sakwa-lul [[[mek-ko siph-e] ha-ci] anh-ta]. \\
apple-\textsc{acc} eat-\textsc{conn} wish-\textsc{conn} do-\textsc{conn} \textsc{neg}-\textsc{decl} \\
\glt`do not like to eat the apple'

\ex \gll sakwa-lul [[[[mek-ko siph-e] ha-key] toy-ci] anh-ta]. \\
apple-\textsc{acc} eat-\textsc{conn} wish-\textsc{conn} do-\textsc{conn} become-\textsc{conn} \textsc{neg}-\textsc{decl} \\
\glt`do not become to like to eat the apple.'
\end{xlist} \end{exe}
%
As seen from the bracketed structures, we can add one more auxiliary verb to
an existing head-light construct with the final auxiliary bearing an appropriate
  connective marker. There is no upper limit to the possible number  of auxiliary
  verbs we can add (see \citealt{Kim:16} for detailed discussion).

The present analysis in which the auxiliary negative forms a complex
predicate structure with a lexical verb can be applied for languages
like Basque, as suggested by \citet{CB:11}. They explore the interplay of sentential
negation and word order in Basque. Consider their example:

%\inlinetodostefan{What is \textsc{sgs}? Is it supposed to be \sg? If not it should be in the %list of abbreviations. Changed \perf to \PRF}
%jb: I tried to stick the original source.
%plO=plural-object, 3sgS=3rd singular subject. I added these in footnote 1.
\ea
\label{negation-basque-ex}
\gll ez-ditu irakurri liburuak. \\
     \NEG-3\textsc{plo}.\textsc{pres}.\textsc{3sgs} read.\PRF{} book.\ABS.\pl\\
\glt `has not read books'
\z
%
%
Unlike Korean, the negative auxiliary \textit{ez-ditu} precedes
the main verb. Other than this ordering difference, just
like Korean, the two form a verb complex structure, as represented in
Figure~\ref{negation-basque} (adopted from \citealt{CB:11}):

%\iffalse{}
\begin{figure}
	\begin{forest}
sm edges without translation
		[{V\\
		\begin{avm}
			\[\type{hd-light-cxt}\\
			%head \; \@{3}\\
			light \; $+$\\
			comps \; \@{2}\,\<NP\>\]
		\end{avm}}, l sep*=3
			[{V\\
			\begin{avm}
				\[head|aux  +\\
				  light \; $+$\\
				  comps \;  \<\@{1}\> $\oplus$ \@{2}\,\<NP\>\]
			\end{avm}}%, edge label={node[midway,left,outer sep=1.5mm,]}
				[ez-ditu\\ \textsc{neg}-3\textsc{plo.pres.3sgs}]]
			[{\ibox{1}\,V\\
			\begin{avm}
				\[comps \; \@{2}\,\<NP\>\]
			\end{avm}}%, edge label={node[midway,right,outer sep=1.5mm,]{H}}
					[irakurri\\read.\textsc{perf}]]]
	\end{forest}
\caption{Partial structure of (\ref{negation-basque-ex})}\label{negation-basque}
\end{figure}
%
%
%
In the treatment of negative auxiliary verbs, the HPSG analyses
have taken the negative auxiliary to be an independent lexical
verb whose grammatical (syntactic) information is not distributed
over different phrase structure nodes, but rather is incorporated into
its precise lexical specifications. In particular, the negative
auxiliary forms in many languages a verb complex structure, whose
constituenthood is motivated by other independent phenomena.
%%, and provides a simple and straightforward explanation for
%phenomena such as aspect selection.
%The conclusion we can draw from here is that the
% distribution of a negative auxiliary verb is determined by
%independent constructional constraints
%that regulate the placement of other
%similar verbs.



\section{Preverbal negative}

The final type of sentence negation is preverbal negatives, which
we can observe in languages like Italian and Welsh:

\eal
\ex \label{negation-position-1a}
\gll Gianni non telefona a nessuno.\\
     Gianni \textsc{neg} telephones to nobody\\ \hfill (Italian)
\glt`Gianni does not call anyone.'
%\ex \label{negation-position-1b}
%\gll Jag har inte gett boken till henne. (Swedish)
%I have \textsc{neg} given the.book to her
%\glt `I hae not given the book to her.'
\ex \label{negation-position-1c}
\gll Dw i ddim wedi gweld neb.\\
     am I \textsc{neg} \textsc{perf} see nobody\\ \hfill  (Welsh)
\glt `I haven't seen anybody.'
\zl
%
%
As seen here, in Italian,
the preverbal negative \textit{non}, also called negative particle or
clitic, always precedes a lexical  verb, whether finite or
non-finite, as further attested by the following
examples:
%
%\ex[]{
%\gll
%Gianni non legge articoli di sintassi. \\
%Gianni   \textsc{neg}   reads  articles  of  syntax \\
%\glt`Gianni doesn't read syntax articles.'
%}
\eal
\ex[]{
\gll Gianni  vuole  che  io  non  legga  articoli  di  sintassi. \\
     Gianni  wants  that  I   \textsc{neg}   read  articles  of  syntax. \\
\glt `Gianni hopes that I do not read syntax articles.'
}
\ex[]{
\gll Non   leggere  articoli di sintassi   \`{e}  un vero peccato. \\
     \textsc{neg}  to.read  articles of syntax   is  a real shame \\
\glt `Not to read syntax articles is a real shame.'}
%
%
\ex[]{
\gll Non    leggendo  articoli di sintassi,  Gianni  trova  la linguistica  noiosa.\\
\textsc{neg}   reading articles of syntax  Gianni  finds  {} linguistics  boring\\
\glt `Not reading syntax articles, Gianni finds linguistics boring.'}
\zl
%
%
The derivational view again attributes the distribution of such
a preverbal negative to the reflex of verb movement and functional
projections (see \citealt{Belletti:90, Zanuttini:91}). This line of analysis also appears to be persuasive
in that the different scope of verb movement application could explain
the observed variations among typologically and genetically related
languages. Such an analysis, however,
  fails to capture unique properties of preverbal negative
  in contrast to the morphological negative, the auxiliary negative, and the adverb negative.

\citet{Kim:00} offers an HPSG analysis of Italian and Spanish negation.
His anlaysis takes \textit{non}
to be an independent lexical head element, even though it is a clitic.
This claim follows the  analyses sketched by \citet{Monachesi:93} and \citet{Monachesi:98}
assuming that there are two types of clitics, affix-like
clitics and word-like clitics: pronominal clitics belong to the
former, whereas the bisyllabic clitic \textit{loro} `to-them' to the
latter. Kim's analysis suggests that \textit{non} also belongs
to the latter group.\footnote{But one main difference between
\textit{non} and \textit{loro} is that \textit{non} is a head
element, whereas \textit{loro} is a complement XP. See
\citet{Monachesi:98} for further discussion of the
behavior of \textit{loro} and its treatment.} One key difference from
pronominal clitics is thus that \textit{non} functions as an independent word.
Treating \textit{non} as
a word-like element, as given in the following, will allow us to capture its word-like
properties such as the possibility of stress on the negator and
its separation from the first verbal element. However, it is not a
phrasal modifier, but an independent particle (or clitic) which combines with
the following lexical  verb.\footnote{See \citet{Kim:00} for
detailed discussion.}

%\newpage
%\inlinetodostefan{The embedding of the semantics was missing. I added it. I hope htis was %correct. I modeled it after the lexical rule.}
%jb: yes, this is correct.
\ea
\label{negation-non}
Lexical specifications for \textit{non} in Italian:\\
\avmtmp{
[ phon \phonliste{ non }\\
  synsem|loc [ cat [ head \1\\
                     comps < V [ head & \1\\
                                 comps & \2\\
                                 cont  & \3] > \+ \2 ]\\
               cont [ restr \{ [ pred & \type{neg-rel}\\
                                 arg1 & \3 ] \} ]]]
}
%% \begin{avm}
%% \[form\ \q<\normalfont non\q>\\
%%   \SYN\ \; \[head\ \@1\\
%%          \COMPS\ \; \<V\[head\ \ \; \@1\\
%%                      \COMPS\ \ \; \textit{L}\]\> \ \; $\oplus$ \; \ \textit{L}
%%          \]\\
%%   \SEM\ \; \[\textsc{restr} \; \<\[\PRED\ \ \; \type{neg-rel}\]\>\]\]
%%   \end{avm}
\z
%
\noindent
This lexical entry roughly corresponds to the entry for
Italian auxiliary verbs (and restructuring verbs with clitic climbing),
in that the negator \textit{non} selects a verbal complement and, further, the
complement list. One key property of \textit{non}
is its \textsc{head} value: this value is in a sense undetermined, but structure-shared with the \textsc{head} value of its verbal complement.
The value is thus
determined by what it combines with. When \textit{non} combines with a finite
verb, it will be a finite verb. When it combines with an
infinitive verb, it will be an infinitive verb.

In order to see how
this system works, let us consider an Italian example where
the negator combines with a transitive verb as in the
following:

\begin{exe}
\ex
\label{negation-read-it}
\gll Gianni non legge articoli di sintassi.\\
Gianni \textsc{neg} reads articles of syntax\\
\glt `Gianni doesn't read syntax articles.'
\end{exe}

\noindent
When the negator \textit{non} combines with the finite verb \textit{legge} that
selects an NP object, the resulting combination will form
a verb complex structure given in Figure~\ref{negation-read-it-st}.

%
%\inlinetodostefan{The figure you provided contained a mistake. It is not the element in the %comps list that is shared but the complete \compsl. Please check.}
%jb: the one you have here % is a correct one.
%
%\iffalse{}
\begin{figure}
	\begin{forest}
sm edges without translation
		[V\\
		\avmtmp{
		[\type*{hd-light-phrase}\\
			head & \1\\
			light & $+$\\
			comps & \2 < NP > ]}, l sep*=3
			[V\\
			\avmtmp{
			[head  & \1\\
			 light & $+$\\
			 comps & < \3 > \+ \2 < NP > ]}
				[non\\ \textsc\textsc{neg}]]
			[\ibox{3}\,V\\
			 \avmtmp{
			  [head  & \1\\
			   comps & \2 < NP > ]}
					[legge\\ reads]]]
	\end{forest}
%% 	\begin{forest}
%% sm edges without translation
%% 		[V\\
%% 		\begin{avm}
%% 			\[\type{hd-light-cxt}\\
%% 			head \; \@{3}\\
%% 			light \; $+$\\
%% 			comps \; \<\@{2}\,NP\>\]
%% 		\end{avm}, l sep*=3
%% 			[V\\
%% 			\begin{avm}
%% 				\[head \; \@{3}\\
%% 				  light \; $+$\\
%% 				  comps \;  \<\@{1}\> $\oplus$ \<\@{2}\,NP\>\]
%% 			\end{avm}%, edge label={node[midway,left,outer sep=1.5mm,]}
%% 				[non\\ \textsc\textsc{neg}]]
%% 			[\ibox{1}\,V\\
%% 			\begin{avm}
%% 				\[head \; \@{3}\\
%% 				comps \; \<\@{2}\,NP\>\]
%% 			\end{avm}%, edge label={node[midway,right,outer sep=1.5mm,]{H}}
%% 					[legge\\ reads]]]
%% 	\end{forest}
\caption{Verb complex structure of (\ref{negation-read-it})}\label{negation-read-it-st}
\end{figure}


%In the nonderivational, lexicalist analysis I propose,
%the negator is taken to be a negative verb
%bearing a clitic feature. This analysis
%not only allows us to capture its dual properties -- clitic-like and
%word-like properties, but also
%correctly predicts the positioning\index{position!of \emph{non}} of \emph{non} in various %contexts,
%and its behavior in clitic climbing and AUX-to-COMP constructions.

\citet{Borsley:06}, adopting \citegen{Kathol2000a} topological approach,
 provides a linearization"=based HPSG approach to capturing the distributional possibilities of negation in Italian and Welsh, which we have seen
 in (\ref{negation-position-1a}) and in (\ref{negation-position-1c}), respectively.
%
% and The VSO language Welsh also has a preverbal negation but allows the multiple %realizations of negation, as in  (data from \citep{Borsley:06}).
%
  Different from \citegen{Borsley:05} selectional
  approach where a negative expression selects its own complement,
  Borsley's linearization"=based approach allows the negative expression to
   have a specified topological field.
   % with the assumption that
%constituents have an order domain to which ordering constituents
%apply. 
For instance, \citet{Borsley:06}, accepting the analysis of \citet{Kim:00}
where \textit{non} is taken to be a type of clitic-auxiliary, posits the following
order domain:


\ea
\avmtmp{
[ dom < [\type*{first}
         \phonliste{ Gianni } & ],
        [\type*{second}
          neg $+$\\
          \phonliste{ non } & ],
        [\type*{third}
         \phonliste{ telephona } & ],
        [\type*{third}
         neg $+$\\
         \phonliste{ a nessuno } & ] >]}
\z
%
With this ordering domain, \citet{Borsley:06} postulates
that the Italian sentential negator \emph{non} bearing the positive \textsc{neg} feature is in the second field.\footnote{
\citet{Borsley:06} also notes that Italian negative expressions like \emph{nessuno} `nobody' also bear the feature \textsc{neg}
but are defined to be in the third field.}
The analysis then can attribute the distributional differences between Italian and Welsh negators
by referring to the difference in their domain value. That is,
within the analysis, the Welsh \textsc{neg} expression \emph{ddim}, unlike Italian \emph{non},
is defined to be in the third field as illustrated in the following domain for the sentence (\ref{negation-position-1c}) (from
\citealt{Borsley:06}):\footnote{Different from \citet{Borsley:06}, \citet{BJ:00} offer  a selectional analysis of Welsh negation.
That is, the finite negative verb selects
two complements (e.g., subject and object) while
the nonfinite negative verb selects a VP. See \citet{BJ:00} for details.}

\ea
\avmtmp{
[ dom < [\type*{second}
         \phonliste{ dw } ],
        [\type*{third}
         \phonliste{ i } ],
        [\type*{third}
         neg $+$\\
         \phonliste{ ddim } & ],
        [\type*{third}
         \phonliste{ wedi gweld neb } & ] > ]}
\z
As such,  with the assumption that 
constituents have an order domain to which ordering constituents
apply, the topological approach enables to capture the complex distributional 
behavior of the negators in Italian and Welsh. 
%
 % with the assumption that
%constituents have an order domain to which ordering constituents
%apply. 

\iffalse{
The VSO language Welsh also employs a preverbal
negative, as illustrated by \citet{BJ:00}:

\ea
\ex \label{negation-NC-1a}
\gll Ni chaf sefyll yma \\
     \textsc{neg} can stand here\\
\glt `I can't stand here.'
\z
%
%  \citet{BJ:00} offer an HPSG-based analysis of negation
%in Welsh.
As given in (\ref{negation-NC-1a}), Welsh allows the preverbal
particle \type{ni} to occur in the preverbal position.
%
With treating the negative particle to form a negative word
with the following stem, \citet{BJ:00} offer a selectional
analysis of negation.
That is, the finite negative verb selects
two complements (e.g., subject and object) while
the nonfinite negative verb selects a VP.}\fi

\section{Other related phenomena}

In addition to this work focusing on the distributional possibilities
 of negation, there has also been work on genitive negation and negative concord.

 \citet{Prz:00} offers a HPSG analysis for the non-local genitive of negation in Polish.
 Polish allows the object argument to be genitive-marked with the presence of negative marker as in (\ref{negation-genitive}).
 %In Polish,  negation is realized as the prefix
  %\type{nie} to a verbal expression (\citep{PK:99, Prz:00, Prz:01}).
  The assignment of genitive case to the object is also effective in
  the unbounded relation as shown in (\ref{negation-genitive-1}) (data from \citealt{Prz:00}):

\eal
\ex  \label{negation-genitive}
\gll Lubi\c{e} Mari\c{e} \\
     like.1st.\textsc{sg} Mary.\textsc{acc}\\
\glt `I like Mary.'
\ex
\gll Nie lubi\c{e} Marii / * Mari\c{e} \\
     not like.1st.\textsc{sg} Mary.\textsc{gen} {} {} Mary.\textsc{acc}\\
\glt `I don't like Mary.'
\zl

%\inlinetodostefan{what is \textsc{rm}? Explain and add in list of abbreviations. \textsc{rm} %= \REFL?}
% jb: added in footnote 1.
\eal
\ex \label{negation-genitive-1}
\gll  Janek wydawa\l{} si\c{e} lubi\'{c} Mari\c{e}.\\
      John seemed \textsc{rm}     like.\textsc{inf} Mary.\textsc{acc}\\
\glt `John seemed to like Mary.'
\ex
\gll  Janek nie wydawa\l{} si\c{e} lubi\'{c} Marii / Mari\c{e}.\\
      John not seemed \textsc{rm} like.\textsc{inf}      Mary.\textsc{gen} {} Mary.\textsc{acc}\\
\glt `John did not seem to like Mary.'
\zl

\iffalse{
\eal
\ex \label{negation-genitive-1}
\gll Pisz\c{e} listy /*list\'{o}w.
     write.1\textsc{sg} letters.\textsc{acc}/letters.\textsc{gen}\\
\glt `I am writing letters.'

\ex
\gll Nie chcialem   pisa\'{c} list\'{o}w. \\
     not wanted.1\sg.\mas{} write.\textsc{inf} letters.\textsc{gen}\\
\glt `I didn't want to write letters.'
\zl}\fi
  %
To account for this kind of phenomenon, \citet{Prz:00} 
suggests that the combination of the
negative morpheme \type{nie} with a verb stem introduces the
feature \textsc{neg}.\footnote{In Polish,  negation is realized as the prefix
  \type{nie} to a verbal expression (see \citealt{PK:99, Prz:00, Prz:01}).}
  With this lexical specification, his analysis introduces 
 the following principle (adopted from \citealt[\page 143]{Prz:00}):
  
%\iffalse{}
%\inlinetodostefan{ The rule contained an error. I fixed it and adapted notation. The rule is %not a lexical rule $\mapsto$ but an implication \impl so it should be named the case %priniple, I checked Adam's paper. You misquoted him. I fixed this and replaced it by a %correct version.}
% jb: ok -- i meant to be simpler. anyway this is ok.
\ea
\label{negation-polish-gen-case}
Case Principle for Polish:\\
\avmtmp{
[ head & [\type*{verb}
          neg & + ]\\
  arg-st & \1 \type{nelist} \+ < ![case \type{str}]! > \+ \2  ]}  \impl
\avmtmp{
[arg-st \1 \+ < ![case \type{gen}]! > \+ \2  ]}
\z
%\ea
%old formulation
%\begin{avm}
%\[%\textit{hd-light-cxt }\\
%head\ \; \[\type{verb}\\
%       \textsc{neg} \ +\]\\
%arg-st\ \; \<XP, YP[case\ \type{str}]\>\] \; \; $\mapsto$ \; \;
%\[arg-st\ \; \<XP\> $\oplus$ \<NP[case\ \type{gen}]\> $\oplus$ \textit{L} \]
%\end{avm}
%\z
%\fi
The principle allows a \textsc{neg} verbal expression to assign
 the \textsc{gen} to its non-initial argument. 
 % ensures that
 % the object NP in (\ref{negation-genitive}b) is \textsc{gen}-marked
%
This is why the negative word \textit{nie} triggers
 the object complement of
(\ref{negation-genitive}) to be \GEN-marked. 
As for the long distance \textsc{gen} in (\ref{negation-genitive-1}), \citet[\page 145]{Prz:00}
allows the VP complement of a raising verb like \type{seem} to optionally undergo the lexical
argument composition. This process yields the following output for the
matrix verb in (\ref{negation-genitive-1}):

%\inlinetodostefan{the structure sharing was wrong. It is the whole \compsl that is shared %not the element within it.}
%jb: the adoptation here meant to be a simple one. anyway this is also fine.
\ea
\label{negation-polish-case}
%\textsc{polish case assignment Rule:\\
\avmtmp{
[ phon & \phonliste{ nie wydawa\l{} si\c{e} }\\
  head & [\type*{verb}
          neg & + ]\\
  arg-st & < NP, VP[ comps \1 < NP > ] > \+ \1 < NP![\type{str}]! > ]
}
%% \begin{avm}
%% \[phon\ \ \<\normalfont{nie wydawa\l{} si\c{e}}\>\\
%%  head\ \ \[\type{verb}\\
%%            \textsc{neg} \ +\]\\
%% arg-st\ \  \<NP, VP\[comps\ \<\@1NP\>\]\> $\oplus$ \<\@1NP[\type{str}]\>\]
%% \end{avm}
\z
%
This lexical specification allows the object NP of the verb to get
\GEN-marked in accordance with the constraint in (\ref{negation-polish-gen-case}).\footnote{When
there is no argument composition, the positive verb \type{lubi\'{c}}
assigns \textsc{acc} to the object NP.} In Przepiórkowski's analysis, the feature
\textsc{neg} thus tightly interacts with the mechanism of argument composition and construction-based case assignment (or satisfaction).

Negative concord also concerns negation that
we often find in languages like French, Italian, Polish, and so forth.  \citet{Swart:02} investigates  negative concord in French, where multiple occurrences
of phonologically negative constituents express either
double negation or single negation:

%\eal
\ea \label{negation-nc-ex}
\gll Personne (n')aime personne.\\
     no.one ne.likes no.one\\
\glt `No one is such that they love no one.' \hfill (double negation)
\glt `No one likes anyone.' \hfill  (negative concord)
\z
%
%With the semantic assumption that the contribution of negation in a concord context is %semantically empty, they formulate an HPSG analysis for negative concord.
%
The double negation reading in (\ref{negation-nc-ex}) has two quantifiers while the single
negation reading is an instance of negative concord, where the two
quantifiers merge into one. \citet{Swart:02} assume that the information of
each quantifier is stored in \textsc{qstore} and retrieved at the
lexical level in accordance with constraints on the verb's arguments and semantic
content. For instance, the verb \textit{n'aime} in (\ref{negation-nc-ex}) will have two different ways of retrieving the
\textsc{qstore} value as given in the following:\footnote{The
\textsc{qstore} value contains information
roughly equivalent to first order logic expressions like \textit{NO}x[Person(x)]. See \citet{Swart:02}.}

\eal
\ex
\avmtmp{
[phon   &  \phonliste{ n'aime }\\
 arg-st & < NP![\textsc{store} \{\ibox{1}\}], NP[\textsc{store} \{\ibox{2}\}]! >\\
 quants & < \ibox{1}, \ibox{2} > ]
}
\ex
\avmtmp{
[phon   & \phonliste{ n'aime } \\
 arg-st & < NP![\textsc{store} \{\ibox{1}\}], NP[\textsc{store} \{\ibox{2}\}]! >\\
 quants & < \1 > ]
}
\label{negation-quant}
\zl
%
%\; \; :
%\; \; :  $\neg\exists$x$\exists$y Love(x,y)
%
%\inlinetodostefan{changed Love to love} jb: ok
\noindent
In the AVM (\ref{negation-quant}), the two quantifiers are retrieved, inducing double negation ($\neg\exists$x$\neg\exists$y[love(x,y)]) while in (\ref{negation-quant}), the two have a resumptive interpretation in which the two are merged into one ($\neg\exists$x$\exists$y[love(x,y)]).\footnote{See \citet{Swart:02} for detailed formulation of the retrieval condition of stored value.} This analysis, coupled with the complement treatment of \textit{pas} as a lexically stored quantifier, can account
for why \emph{pas} does not induce a resumptive interpretation with a quantifier (from \citealt{Swart:02}):


\ea
\gll Il ne va pas nulle part,         il va \'{a} son travail.\\
     he ne goes \textsc{neg} no where he goes at  his work\\
\glt `He does not go nowhere, he goes to work.'
\z
%
In this standard French example, \citet{Swart:02}, accepting
the analysis of \citet{Kim:00} as \textit{pas} as a complement,
specify the adverbial complement \emph{pas} to be included the negative quantifier in the \textsc{quants} value.
 This means there would be no resumptive
reading for standard French, inducing double negation as in (\ref{negation-qstore}):\footnote{See \citet{Swart:02} for cases where \textit{pas} induces negative concord.}

\ea
\label{negation-qstore}
\avmtmp{
[phon   & \phonliste{ ne va }\\
 arg-st & < Adv\sub{i}![\textsc{store} \{\ibox{1}\}], NP[\textsc{store} \{\ibox{2}\}]! >\\
 quants & < \ibox{1}, \ibox{2} > ]}
\z

\citet{PK:99} and \citet{BJ:00} also  investigate negative concord in Polish and Welsh
and offer HPSG analyses. Consider a Welsh example from \citet{BJ:00}:

\ea
\gll Nid         oes neb yn yr ystafell\\
\textsc{not}    is no.one in the room\\
\glt `There is no one in the room.'
\z
\noindent \citet{BJ:00}, identifying n-words with the feature
\textsc{nc} (negative
concord),  takes the verb \emph{nid oes} to bear the positive \textsc{neg} value,
and specifies the subject \emph{neb yn} to carry the positive \textsc{nc} (negative
concord) feature. This selectional approach, interacting with
well-defined features, tries to capture how more than one
negative element corresponds to a single semantic negation.\footnote{See
\citet{BJ:00} for detailed discussion.}



\section{Conclusion}

%In the previous two sections, I have provided my
%answers to the two main questions in this study.
%Then, the remaining question is what these answers
%imply for the theory of grammar.

%The types of negation we have seen are identical
%in that they negate a sentence or clause in the given language.
%Does this entail that there is a universal functional category Neg
%that, interacting with other
%grammatical constraints such as movement operations, allows all their
%distributional possibilities?  My answer to this question is no.


One of the most attractive consequences of the
derivational perspective has been that one uniform category,
given other syntactic operations and constraints,
explains the derivational properties of all types of negation
in natural languages, and can further provide a surprisingly
close and parallel structure among languages, whether typologically
related or not. However, this line of thinking, first of all, runs the risk of
missing the peculiar properties of each type of
negation. Each individual language has its own
way of expressing negation, and moreover has
its own restrictions in the surface realizations of negation which
can hardly be reduced to one uniform category.


In the nonderivational HPSG analyses for the four main
types of sentential negation that I have reviewed
in this chapter,  there is no uniform
syntactic element, though a certain universal aspect of
negation does exist, viz.\ its semantic contribution.
%Each type of negation exists as a distinct category.
Languages appear to employ various possible
ways of negating a clause or sentence. Negation can
be realized as different morphological and syntactic categories.
By admitting morphological and syntactic categories,
we have been able to capture their idiosyncratic properties in a
simple and natural manner. Furthermore, this theory has been built upon
the lexical integrity principle, the thesis that the principles that govern the
composition
of morphological
constituents are fundamentally different from the principles that
govern sentence structures. The obvious advantage of
this perspective is that it can capture the distinct properties of
morphological and syntactic negation, and also of their distribution,
in a much more complete and satisfactory way.

%When compared with the derivational analyses put forth so far, it seems to
%be far more economical to discard the uniform functional category
%Neg, deep structure, and transformational component, and predict most of
%the positional possibilities of each type of negation from
%its own lexical properties and `surface structure constraints'.

\iffalse{
One can view the difference between the derivational view
and the nonderivational, lexicalist view as a matter
of a different division of labor. In the derivational view
the syntactic components of grammars bear almost all the
burden of descriptive as well as explanatory resources.
But in the nonderivational view,  it is both the morphological
and syntactic components that carry the
burden.  It is true that a derivational
grammar whose chief explanatory resources are functional projections
including NegP and syntactic movement, also has
furthered our understanding of negation and
relevant phenomena in certain respects.
But in so doing it has also brought other complexities into the basic
components of the grammar. The present research strongly suggests
that a more conservative division of labor between morphology and syntax is
far more economical and feasible.
\fi

\section*{Abbreviations}

\begin{tabularx}{.99\textwidth}{@{}lX}
\textsc{3sgs} & 3rd singular subject\\
\textsc{3plo} & 3rd plural object\\
\textsc{conn} & connective\\
\textsc{del}  & delimiter\\
\textsc{hon}  & honorific\\
\textsc{npst} & nonpast\\
\textsc{perf} & perfective\\
\textsc{pres} & present\\
\textsc{rm}   & reflexive marker\\
\end{tabularx}
%\textsc{acc} (accusative), \textsc{caus} (causative), \textsc{cond} (conditional), \textsc{conn} (connective),
%\textsc{cop} (copula),  \textsc{decl} (declarative), \textsc{del} (delimiter), \textsc{fut} (future), \textsc{gen} (genitive),
%\textsc{hon} (honorific), \textsc{inf} (infinitival), \textsc{neg} (negative), \textsc{nom} (nominative),
%\textsc{npst} (nonpast), \textsc{pass} (passive), \textsc{perf} (perfective), \textsc{pres} (present), \textsc{pst} (past), \textsc{prog} (progressive),  %\textsc{sg} (singular), and so forth.}

\section*{Acknowledgements}

I thank reviewers of this chapter for comments and suggestions, which
helped improve the quality of this chapter. I also thank
Bob Borsley and Stefan M\"{u}ller for detailed comments on
the earlier version. My thanks also go to Rok Sim
and Jungsoo Kim for
helpful feedback.
%
%
%
%
} %\avmoptions{center}

%\fi
%
{\sloppy
\printbibliography[heading=subbibliography,notkeyword=this]
}

\end{document}
