\documentclass[output=paper]{langsci/langscibook} 
\author{Jesse Tseng\affiliation{Université Paris Diderot}}
\title{Phonology}

% \chapterDOI{} %will be filled in at production

%\epigram{Change epigram in chapters/03.tex or remove it there }
%\abstract{}
\maketitle

\begin{document}
\label{chap-phonology}

\section{Introduction: \textsc{phonology} in the HPSG sign} 

The \textsc{phonology} attribute in \citep{ps} and \citep{ps2}: 
\begin{itemize}
\item rudimentary \textsc{phon} value 
\item basic Phonology Principle constrained by Linear Precedence
  rules: corresponds to simple terminal spell-out of the phrase structure tree
\item ``Phonology-Free Syntax'' \citep{MPZ97a-u-platte}: \textsc{phon}
  information inaccessible for selection via SYNSEM
\end{itemize}

There has been relatively little work within HPSG on phonological
representation and the analysis of phonological phenomena. Most
references to the \textsc{phon} attribute use it simply as a lexical
identifier, or they are dealing with phenomena at the phonology-syntax
interface (e.g.\ constituent order, ellipsis). For such applications,
the actual content of the \textsc{phon} value is unimportant. These
topics are covered in other chapters.

\section{Phonological representations in HPSG}

Proposals for the detailed content of \textsc{phon} values:
\begin{itemize}
\item encoding of phonological constituents \citep{BK94b,Klein2000a,Hoehle99a-u}

\item syllable structure \cite{TsengHPSG08}

\item metrical phonology \citep{Klein2000a,BonamiDelais06}
\end{itemize}

\section{Phonological analysis in HPSG}

\begin{itemize}
\item principles of constraint-based phonology vs derivational phonology
\citep{BK94b}: compositionality, monotonicity

\item compositional construction of prosodic structure in parallel
  with phrase structure \citep{Klein2000a}
\end{itemize}

But HPSG is formally compatible with many approaches, and there is as
yet no emerging consensus among practitioners.
\begin{itemize}
\item Finite state phonology \citep{Bird92a,Bird95a}

\item need for abstract underlying forms \citep{Skwarski09};
  phonologically empty categories 

\item OT in HPSG \citep{Orgun96a}

\end{itemize}

\section{Specific phenomena and case studies}

\begin{itemize}
\item shape conditions \citep{AKlein2002a}
\item French \citep{TsengLiaison,BBT04}
\item phonological idioms [already covered in Manfred's chapter]
\item \ldots
\end{itemize}
 
\section*{Abbreviations}
\section*{Acknowledgements}

\printbibliography[heading=subbibliography,notkeyword=this] 
\end{document}
